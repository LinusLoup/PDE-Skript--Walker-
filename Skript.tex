\documentclass[a4paper,11pt,oneside]{book}

\usepackage[utf8]{inputenc}
%\usepackage{pslatex}
\usepackage[intlimits]{amsmath}
\usepackage{amsfonts}
\usepackage{amssymb}
\usepackage{amsthm}
\usepackage{enumerate}
\usepackage[ngerman]{babel}
%\usepackage{ps4pdf}
%\PSforPDF{
	\usepackage{pstricks}
	\usepackage{pst-all}
	\usepackage{multido}
	\usepackage{pst-plot}
	\usepackage{pst-3dplot}
%}
%\usepackage{pdftricks}
\usepackage{hyperref}
\usepackage[mathscr]{euscript}
\usepackage{makeidx}
\usepackage{fancyhdr}
\usepackage{mathtools}
\usepackage[cmtip,arrow]{xy}
\usepackage{pb-diagram,pb-xy}
\usepackage{breqn}
\usepackage{dsfont}
%\usepackage{bbold}
%\usepackage[active]{preview}
\usepackage{stackrel}

% Theoreme
%\theoremstyle{plain}
\newtheorem{satz}{Satz}[chapter]
\newtheorem{lemma}[satz]{Lemma}
\newtheorem{theorem}[satz]{Theorem}
\newtheorem{kor}[satz]{Korollar}
\theoremstyle{definition}
\newtheorem*{defi}{Definition}
\newtheorem{bemdef}[satz]{Bemerkungen und Definitionen}
\theoremstyle{remark}
\newtheorem{bem}[satz]{Bemerkung}
\newtheorem*{bem*}{Bemerkung}
\newtheorem*{folge}{Folgerung}
\newtheorem*{einschr}{Einschränkung}
\newtheorem{bsp}[satz]{Beispiel}
\newtheorem*{bsp*}{Beispiel}
\newtheorem*{erinnerung}{Erinnerung}
\newtheorem*{notation}{Notation}
\newtheorem*{vor}{Voraussetzungen}

% kleine Helferlein
\newcommand{\R}{\mathbb{R}}
\newcommand{\N}{\mathbb{N}}
\newcommand{\K}{\mathbb{K}}
\newcommand{\C}{\mathbb{C}}
\newcommand{\B}{\mathbb{B}}
\newcommand{\D}{\mathscr{D}}
\newcommand{\E}{\mathscr{E}}
\newcommand{\F}{\mathscr{F}}
\renewcommand{\H}{\mathbb{H}}
\newcommand{\Ext}{\text{Ext}}
\newcommand{\id}{\text{id}}
\renewcommand{\L}{\mathcal{L}}
\newcommand{\Lloc}{L_{1,\scriptsize{\textnormal{loc}}}}
\newcommand{\diff}[1]{\frac{d}{d #1}}
\newcommand{\pdiff}[1]{\frac{\partial}{\partial #1}}
\renewcommand{\div}{\operatorname{div}}
\newcommand{\spur}{\operatorname{spur}}
\DeclareMathOperator{\graph}{graph}
\DeclareMathOperator{\supp}{supp}
\DeclareMathOperator{\vol}{vol}
\DeclareMathOperator{\dist}{dist}
\DeclareMathOperator{\im}{im}
\newcommand{\spn}{\operatorname{span}}
\renewcommand{\Re}{\operatorname{Re}}
\renewcommand{\Im}{\operatorname{Im}}
\renewcommand{\d}{\,\mathrm{d}}
\newcommand{\Ra}{\Longrightarrow}
\newcommand{\La}{\Leftarrow}
\newcommand{\ra}{\rightarrow}
\newcommand{\Lra}{\Leftrightarrow}
\newcommand{\hhookrightarrow}{\lhook\mkern-3mu\relbar\mkern-12mu\hookrightarrow}
\newcommand{\abs}[1]{\lvert #1\rvert}
\newcommand{\Abs}[1]{\left\lvert #1\right\rvert}
\newcommand{\norm}[1]{\lVert #1\rVert}
\newcommand{\Norm}[1]{\left\lVert #1\right\rVert}
\newcommand{\fa}{\forall}
\newcommand{\with}{\;\vert\,}
\newcommand{\<}{\langle}
\renewcommand{\>}{\rangle}
\renewcommand{\epsilon}{\varepsilon}
\newcommand{\dom}{\operatorname{dom}}


% Gleichung nummerieren
\newcommand{\eq}[1]{\
  \addtocounter{equation}{1}
  \tag{\theequation}
  \label{#1}
}

% Index
\newcommand{\idx}[1]{\index{#1}#1}
\makeindex

\newcommand{\nameofchapter}{}
\newcommand{\newchapter}[1]{\chapter{#1}
	\renewcommand{\nameofchapter}{#1}
	\setcounter{satz}{0}
	\setcounter{equation}{0}}


% Titel etc.
\title{Partielle Differentialgleichungen}
\author{Verfasser: Marc Hauptmann \\
	erweitert durch: Cornelius Rüther}

% Dokument
\begin{document}
\frontmatter

\maketitle
\tableofcontents
\thispagestyle{fancy}{
	\rhead{}
	\lhead{\sl Inhaltsverzeichnis}
	%\renewcommand{\headrulewidth}{0.4pt}
	\renewcommand{\headheight}{14pt}
	\renewcommand{\footrulewidth}{0.4pt}
	\cfoot{\thepage}
}

\mainmatter

\pagestyle{fancy}{
	\rhead{}
	\lhead{\sl\thechapter. \nameofchapter}
	%\renewcommand{\headrulewidth}{0.4pt}
	\renewcommand{\headheight}{14pt}
	\renewcommand{\footrulewidth}{0.4pt}
	\cfoot{\thepage}
}


\newchapter{Einleitung}
\label{sec:Einleitung}

\section{Motivation}

\begin{align*}
  u(t,x)\in\R: \, & \text{\underline{\idx{Dichte}} (z.B. Wärme)} \\
  t: \, & \underline{\text{Zeit}}, x\in\Omega\subset\R^n: \underline{\text{Raumvariable}} \\
  J(t,x)\in\R^n:\, &\text{\underline{\idx{Fluss}}} \\
  f(t,x)\in\R : \, &\text{\underline{\idx{Quelle}} (\underline{\idx{Senke}})}
\end{align*}
\begin{figure}[h!]
  \begin{center}
  %\PSforPDF{
    \begin{pspicture}(-1.5,-1.5)(2,2)
      % Omega
      \psccurve(-2,1)(0,2)(2,1)(2,-1)(1,-2)(-.5,-.5)
      \rput[bl](-1,-1){$\Omega$}

      % Umgebung von x
      \pscircle(0,1){.4} \psdot(0,1) \rput[tl](.1,1){$x$}
      \rput[r](-.5,1){$V$}

      % Normalenvektor
      \psline{->}(0,1.4)(0,1.9) \rput[tl](.1,1.7){$\nu$}

      % Vektorfeld J
      \psline{->}(.2,.6)(.6,.2) \psline{->}(.5,1)(.9,.6)
      \psline{->}(.7,.1)(1,-.3) \psline{->}(1,.5)(1.4,-.1)
      \psline{->}(1.45,-.2)(1.6,-.7) \rput[bl](1.2,.4){$J$}
    \end{pspicture}
    %}
  \end{center}
  \caption{Vergleichsgebiet $V$ mit Fluss $J$}
  \label{abbildung:1}
\end{figure}

Sei $V\subset\Omega$ \idx{Vergleichsgebiet} mit Rand $\partial V$ und äußerer Einheitsnormale $\nu(x)\in\R^n, x\in\partial\Omega$ wie in Abbildung~\ref{abbildung:1} dargestellt, sowie $\tau$ die Oberfläche von $V$.

\begin{description}  
\item[\idx{Bilanzgleichung}:] 

  \[
  \underbrace{
    \diff{t} \underbrace{\int\limits_V u(t,x)\d x}_{\text{Änderungsrate in $V$}}
  }_{=\int\limits_V \partial_t u\d x}
  =
  \underbrace{
    - \underbrace{\int\limits_{\partial V} J\cdot\nu\d\tau}_{\text{Fluss nach Draußen}}
  }_{\underset{\scriptsize\text{Gauß}}{=} -\int\limits_V\div J\d x}
  +\underbrace{\int\limits_V f(t,x)\d x}_{\parbox{3.3cm}{\scriptsize was im Gebiet entsteht/ \\  verloren geht}}
  \]
  wobei $\div J:=\nabla\cdot J=\sum\limits_{j=1}^n\partial_jJ^j$
  mit $J=(J^1,\ldots,J^n)\in\R^n$ und $\partial_j:=\pdiff{x_j}$. Also gilt:
  \[ \int\limits_V\partial_t u\d x+\int\limits_V\div J\d x=\int\limits_V f\d x \]
  \[
  \underset{V\subset\Omega \scriptsize\text{ bel.}}\Ra \text{\underline{\idx{Kontinuitätsgleichung:}}} \quad  \partial_tu+\div J=f
  \]
  für $(t,x)\in(0,\infty)\times\Omega$.
\end{description}

Je nach physikalischer (biologischer, $\ldots$) Situation besteht ein Zusammenhang zwischen dem Fluss $J$ und der Dichte $u$.

\begin{enumerate}[(i)]
\item \underline{\textbf{\idx{Transportgleichung}:}} $J=ub, b\in\R^n$
\[ \Ra \boxed{\partial_t u+b\cdot \nabla u=f}\quad (n=1: u_t+bu_x=f)\]
\item \underline{\textbf{\idx{Wärmeleitungsgleichung}:}} Fluss ist proportional zu Dichtegefälle, d.h. $J=-d\nabla u$ ($d>0$: Geschwindigkeit).

Laplace-Operator: $\Delta := \div\cdot\nabla = \sum\limits_{j=1}^n\partial_j^2$
\begin{align*}
 \Ra\; & \boxed{\partial_tu-d\Delta u=f} \\
  & \partial_tu-d\sum\limits_{j=1}^n\partial_j^2 u=f\quad (n=1: u_t-du_{xx}=f)
\end{align*}
\item \underline{\textbf{(nicht viskose) \idx{Burgersgleichung}:}} $n=1, J=\frac 1 2 u^2$
\[ \Ra \boxed{u_t+uu_x=f} \]

\item \underline{\textbf{\idx{Laplace-Gleichung}:}} Stationäre beziehungsweise zeitunabhängige Lös\-ung\-en der Wär\-me\-lei\-tungs\-glei\-chung
\[ \Ra \boxed{-\Delta u=f} \]
\end{enumerate}

Andere Typen, die nicht auf der Kontinuitätsgleichung beruhen sind:

\begin{enumerate}
\item[(v)] \underline{\textbf{\idx{Wellengleichung}:}}
\[ \boxed{\partial_t^2u-c\Delta u=f} \]
Hierbei ist $c>0$ die Geschwindigkeit.

\item[(vi)] \underline{\textbf{\idx{Navier-Stokes}:}}
\[ \boxed{\partial_tu-\Delta u+\sum\limits_{j=1}^nu^j\partial_j u=f-\nabla p} \]
 mit $u=(u^1,\ldots,u^n)$.
\end{enumerate}

\section{Typische Fragen}

\begin{enumerate}[1.]
\item Wohldefiniertheit (lokale/globale Existenz, Eindeutigkeit, Abhängigkeit bzgl. Daten)
\item Regularität
\item Darstellungsformeln (z.B. explizite Berechnungen)
\item Qualitatives Verhalten
\item Approximation
\item[$\vdots$]
\end{enumerate}

\begin{bem}
  \begin{enumerate}[(a)]
  \item \underline{Eindeutigkeit} benötigt meist zusätzliche
    \underline{Bedingungen}; z.B.
    \begin{itemize}
    \item \underline{\textbf{\idx{Anfangsbedingung}:}} (z.B. für $t=0$)
      \[ \boxed{ u(0,x) \overset{!}= u^0(x),\;x\in\Omega} \] wobei $u^0$
      gegeben ist.
    \item \underline{\textbf{\idx{Randbedingungen}:}} (wie sieht $u$ auf
      $\partial\Omega$ aus?)
      \begin{itemize}
      \item \underline{\idx{Dirichlet-Randbedingung}:}
        \index{Randbedingung!Dirichlet}
        \[ \boxed{u(t,x)=g(x),\;x\in\partial\Omega} \]
      \item \underline{\idx{Neumann-Randbedingung}:}
        \index{Randbedingung!Neumann}
        \[ \boxed{\partial_\nu u=\nabla u\cdot\nu=g(x),\;
          x\in\partial\Omega} \]
          wobei $\nu$ der Einheitsnormalenvektor von $\partial \Omega$ ist.
      \end{itemize}
    \end{itemize}

  \item Lösungen existieren nicht immer "`klassich"' (d.h. $u\in C^k$)
    $\Ra$ klassische Lösung/schwache Lösung/distributionelle Lösung.
  \end{enumerate}
\end{bem}

\section{Notation und Begriffe}

\begin{itemize}

\item \underline{\textbf{\idx{Multiindex}:}} $\alpha=(\alpha_1,\ldots,\alpha_n)\in\N^n$
\begin{align*}
  \abs{\alpha} := & \alpha_1+\ldots+\alpha_n \\
  \alpha! := & (\alpha_1)!\cdots (\alpha_n)! \\
  x^\alpha := & x_1^{\alpha_1}\ldots x_n^{\alpha_n} \\
  \partial^\alpha f := & \frac{\partial^{\alpha_1}}{\partial x_1^{\alpha_1}} \frac{\partial^{\alpha_2}}{\partial x_2^{\alpha_2}} \ldots \frac{\partial^{\alpha_n}}{\partial x_n^{\alpha_n}} \quad  \text{für } f\in C^{\abs\alpha} \\
\end{align*}
$\alpha,\beta\in\N^n:(\alpha\geq\beta: \Lra \alpha_j\geq\beta_j\fa j=1,\ldots,n)$

\item \underline{\textbf{\idx{Leibniz}:}} Sei $X\subset\R^n$ offen, $f,g\in C^k(X)$, $\alpha\in\N^n$, $\abs\alpha <k$, dann gilt
\[ \partial^\alpha(fg)=\sum\limits_{\beta\leq\alpha} \binom{\alpha}{\beta}\partial^{\alpha-\beta}f \, \partial^\beta g \]
mit $\binom{\alpha}{\beta}:=\frac{\alpha!}{\beta!(\alpha-\beta)!}$.

\item Eine partielle Differentialgleichung ist von der Form
  \[ F\left(x, (\partial^\alpha u(x))_{0\leq\abs\alpha\leq k}\right)=0, \]
  wobei $X\subset\R^n$ offen,
  \[ F:X\times(\R^m\times\cdots\times\R^m)\ra\R^l \]
  gegeben und $u:X\ra\R^m$ ist.
  \begin{itemize}
  \item $k$: \textbf{\idx{Ordnung}} der Gleichung
  \item $l=1$: eine Gleichung
  \item $l>1$: System von Gleichungen 
  \end{itemize}
\end{itemize}

\section{Spezialfälle}
\begin{enumerate}[(a)]
\item \index{Gleichung!lineare} \underline{\textbf{lineare PDGl:}}
  \[ \sum\limits_{\abs\alpha\leq k}a_\alpha(x)\partial^\alpha u(x)=f(x) \]
  mit $a_\alpha:X\ra\R, f:X\ra\R^m$ gegeben. Ist $f=0$, so heißt die PDE \textbf{homogen}, für $f\neq 0$ \textbf{inhomogen}.
\item \index{Gleichung!semilineare} \underline{\textbf{Semilineare PDGl:}}
  \[ \sum\limits_{\abs\alpha\leq k}a_\alpha(x)\partial^\alpha u(x)=f\left(x,(\partial^\beta u(x))_{0\leq\abs\beta\leq k-1}\right) \]
\item \index{Gleichung!quasilineare} \underline{\textbf{Quasilineare PDGl:}}
  \[ \sum\limits_{\abs\alpha=k}a_\alpha\left( x, (\partial^\beta u(x))_{0\leq\abs\beta\leq k-1} \right)\partial^\alpha u(x) = f\left( x, (\partial^\beta u(x))_{0\leq\abs\beta\leq k-1}\right) \]
\item \index{Gleichung!voll-nicht-lineare} Gilt keiner der Fälle (a)-(c), so heißt die PDE \textbf{voll-nicht-lineare} Gleichung.
\end{enumerate}


%%% Local Variables: 
%%% mode: latex
%%% TeX-master: "Skript"
%%% End: 


\newchapter{Gleichungen 1. Ordnung: Methode der Charakteristiken}

\section{Motivation}

Für $(x,y)\in\Omega\subset\R^2$ offen betrachten wir
\[
  \left.
  \begin{aligned}
      a(x,y)\cdot u_x+b(x,y)\cdot u_y & = c(x,y,u), &(x,y)\in\Omega \\
    \text{mit}\qquad  u(x,y) &= \varphi(x,y), &(x,y)\in\Gamma
  \end{aligned}
  \quad
  \right\}
  \eq{eq:charakteristik}
\]
wobei $a,b:\Omega\ra\R$, $c:\Omega\times\R\ra\R$, $\Gamma$ eine 1-dimensionale Kurve in $\Omega$ und $\varphi:\Gamma\ra\R$ gegeben sind. 

Wir suchen nun ein $u=u(x,y)$, das \eqref{eq:charakteristik} erfüllt.
Die linke Seite von \eqref{eq:charakteristik} ist die Richtungsableitung von diesem $u(x,y)$ in Richtung $(a(x,y),b(x,y))$. Es ist nun naheliegend, dass wir diejenigen Kurven $(x,y)=(x(t),y(t))$ durch einen Punkt $(x_0,y_0)\in\Gamma$ betrachen, deren Tangentialvektor $(\dot x,\dot y)$ in Richtung von $\left(a(x(t),y(t)), b(x(t),y(t))\right)$ zeigt. Also erhalten wir
\begin{eqnarray*}
  \dot x = a(x,y), & x(0)=x_0 \\
  \dot y = b(x,y), & y(0)=y_0\;.
\end{eqnarray*}
Entlang einer charakteristischen Kurve genügt $u=u(x(t),y(t))$ einer gewöhn\-lichen Differentialgleichung, denn:
\[
\left.
  \begin{aligned}
    \diff t u(t) &= \dot x u_x+\dot y u_y=a(x,y)u_x+b(x,y)u_y \\
    &= c(x(t),y(t),u(t))
  \end{aligned}
  \quad
\right\vert
\eq{eq:gewDgl_charakteristik}
\]
mit $u(0)=u(x_0,y_0)=\varphi (x_0,y_0)$. Damit haben wir eine Reduktion des Problems auf gewöhnliche Differentialgleichungen erreicht, und $u$ ist durch $\varphi(x_0,y_0)$ entlang der gesamten charakteristischen Kurve durch $(x_0,y_0)\in\Gamma$ eindeutig bestimmt.
\begin{itemize}
\item[$\Ra$] $u(x,y)$: Projektionen der charakteristischen Kurven durch $(x_0,y_0)\in\Gamma$
\end{itemize}

Es liegt nun folgende Vermutung nahe: Mittels Gleichung \eqref{eq:gewDgl_charakteristik} lässt sich eine eindeutige Lösung von \eqref{eq:charakteristik} in dem Gebiet bestimmen, das durch die charakteristischen Kurven abgedeckt wird.

Allerdings darf der Vektor  $(a(x_0,y_0),b(x_0,y_0))$ in keinem Punkt $(x_0,y_0)\in\Gamma$ tangential an $\Gamma$ stehen! Genau dann nennen wir $\Gamma$ nicht-charakteristisch.

\begin{satz}
  \label{satz:2.1}
  Sei $f\in C^1(\R\times\R^n,\R^n)$, $x_0\in\R^n$. Dann existiert eine Umgebung $(\alpha,\beta)\times\Omega$ von $(0,x_0) \in \R\times\R^n$, sodass das Anfangswertproblem
  \[ \dot x=f(t,x),\quad x(0)=\xi \]
  für jedes $\xi\in\Omega$ eine eindeutige Lösung
  \[ w=w(\;\cdot\;;\xi)\in C^1((\alpha,\beta),\R^n) \]
  hat. 

  Ferner gilt: $w\in C^1((\alpha,\beta)\times\Omega,\R^n)$.
\end{satz}

\begin{proof}
  Siehe Analysis II (Picard-Lindelöf).
\end{proof}

\section{Quasilineare Gleichungen 1. Ordnung}

Wir betrachten
\begin{align*}
\left.
  \begin{aligned}
    a(x,y,u)u_x+b(x,y,u)u_y & =c(x,y,u), &(x,y)\in\Omega \\
    u(x,y) & =\varphi(x,y), &(x,y)\in\Gamma
  \end{aligned}
  \quad
\right\rvert
\eq{eq:3}
\end{align*}
wobei $\Gamma$ eine $C^1$-Kurve in $\Omega\subset\R^2$ mit Parameterdarstellung $\gamma:[a,b]\ra\R^2$ ist, und $\varphi\in C^1(\Gamma)$, $a,b,c\in C^1(\Omega\times\R)$. Ist $u\in C^1(\Omega)$ eine Lösung von \eqref{eq:3}, so gilt
\[
\vec A :=
\begin{pmatrix}
  a(x,y,u(x,y)) \\
  b(x,y,u(x,y)) \\
  c(x,y,u(x,y))
\end{pmatrix}
\perp
\begin{pmatrix}
  u_x(x,y) \\
  u_y(x,y) \\
  -1
\end{pmatrix}
=: \vec u.
\]
Das heißt, $\vec A=\vec A(x,y,u)$ liegt tangential auf dem Graphen $\{(x,y,z)\with z=u(x,y)\}$.

\begin{figure}[ht!]
  \centering
  \begin{pspicture}(-1,-2)(5,6)
    \psset{Alpha=155,Beta=12}
    \pstThreeDCoor[linewidth=.8pt,linecolor=black,xMax=6, yMax=8]
    
    % Weg am Boden
    \listplotThreeD[plotstyle=curve]{
      1.5  3  0
      2  4  0
      3  3  0
      4  4  0
    }

    % Weg auf Fläche
    \listplotThreeD[plotstyle=curve]{
      1.5 3 2
      2 4 2.1
      3 3 2.2
      4 4 1.5
    }
    \pstThreeDLine[linestyle=dotted](1.5, 3, 0)(1.5, 3, 2)
    \pstThreeDLine[linestyle=dotted](4, 4, 0)(4, 4, 1.5)

    % Weg Beschriftungen
    \pstThreeDPut[origin=lt](4.3, 4, 0){$\Gamma$}
    \pstThreeDPut[origin=lb](4.3, 4.2, 1.9){$\graph(\varphi)$}
   
    % A und u
    \pstThreeDLine[arrows=->](2, 4, 2.1)(3, 4, 2.1)
    \pstThreeDPut[origin=b](2, 4, 3.3){$\vec u$}
    \pstThreeDLine[arrows=->](2, 4, 2.1)(2, 4,3.1)
    \pstThreeDPut(3.1, 4, 2.3){$\vec A$}

    % Graph von u
    \listplotThreeD[plotstyle=curve]{
      1 1 2
      3 1 2.2
      5 1 1.5
    }
    \listplotThreeD[plotstyle=curve]{
      1 6 2
      3 6 2.2
      5 6 1.5
    }
    \pstThreeDLine(1, 1, 2)(1, 6, 2)
    \pstThreeDLine(5, 1, 1.5)(5, 6, 1.5)

    % Graph Beschriftung
    \pstThreeDPut[origin=lb](3, 6, 2.6){$\graph(u)$}
  \end{pspicture}
  \caption{Weg $\Gamma$}
\end{figure}

\begin{defi}
  Der Graph $\Sigma$ einer Funktion $u\in C^1(\Omega)$ heißt \idx{Lösungsfläche} für Gleichung \eqref{eq:3}, falls in jedem $P=(x,y,z)\in\Sigma$ der Vektor $$
\vec A =
\begin{pmatrix}
  a(P) \\ b(P) \\ c(P)
\end{pmatrix}
$$ tangential in $\Sigma$ liegt.
\end{defi}

\begin{defi}
  Lösungen $(t\mapsto (x(t),y(t),z(t))$ von
  \begin{eqnarray*}
    \dot x=a(x,y,z), & x(0)=x_0 \\
    \dot y=b(x,y,z), & y(0)=y_0 \\
    \dot z=c(x,y,z), & z(0)=z_0
  \end{eqnarray*}
  heißen Charakteristiken\index{Charakteristik} von \eqref{eq:3} durch $(x_0,y_0,z_0)$.
\end{defi}

\begin{lemma}
  Sei $u\in C^1(\Omega)$, $\Sigma=\graph(u)$. $\Sigma$ ist genau dann eine Lösungs-fläche wenn es die Vereinigung von Charakteristiken ist.
\end{lemma}

\begin{proof}
  Es sei $\Sigma:=\{(\bar x, \bar y, \bar z)\in\Omega\times\R\with \bar z=u(\bar x,\bar y)\}$, $t\mapsto(x(t),y(t),z(t))$ Charakteristik durch einen beliebigen Punkt $(x_0,y_0,z_0)\in\Sigma$ und $w(t):=z(t)-u(x(t),y(t))$. Damit ist
  \[
  \eq{eq:5}
  w(0)=z_0-u(x_0,y_0)=0,  
  \]
  da $(x_0,y_0,z_0)\in\Sigma$.

  Ferner ist 
  \[
  \left.
    \begin{aligned}
      \dot w(t)=\;&\dot z-\dot xu_x-\dot yu_y=c(x,y,z)-a(x,y,z)u_x-b(x,y,z)u_y \\
      =\;& c(x,y,w+u(x,y))-a(x,y,w+u(x,y))u_x \\
      &-b(x,y,w+u(x,y))u_y
    \end{aligned}
    \quad
  \right\vert
  \eq{eq:6}
  \]
  und $w$ ist durch \eqref{eq:5}, \eqref{eq:6} eindeutig bestimmt.
  
  \begin{itemize}
  \item["`$\Rightarrow$"']  Sei $\Sigma$ eine Lösungsfläche. Damit gilt $au_x+bu_y=c$. Aus \eqref{eq:5}, \eqref{eq:6} folgt $\dot w\equiv 0$, und da $w(0)=0$ ist, gilt $w\equiv 0$. Die Charakteristik $(t\mapsto (x(t),y(t),z(t)))$ liegt also ganz in $\Sigma$. Nun ist aber $(x_0,y_0,z_0)\in\Sigma$ beliebig gewählt. Somit ist $\Sigma$ eine Vereinigung von Charakteristiken.

  \item["`$\La$"'] Es sei $\Sigma$ Vereinigung von Charakteristiken. Es ist also $w\equiv 0$ und mit Gleichung \eqref{eq:6} folgt 
    \[
    c(x_0,y_0,\underbrace{u(x_0,y_0)}_{z_0})-a(x_0,y_0,z_0)u_x
    -b(x_0,y_0,z_0)u_y=0.
    \]
    Das heißt, $
      \begin{pmatrix}
        a(x_0,y_0,z_0) ,
        b(x_0,y_0,z_0) ,
        c(x_0,y_0,z_0)
      \end{pmatrix}^T
      $ steht tangential an $\Sigma$. Da $(x_0,y_0,z_0)$ beliebig ist, ist $\Sigma$ eine Lösungsfläche.
  \end{itemize}
  Damit folgt die Behauptung.
\end{proof}

\begin{folge}
  Lösungsflächen lassen sich durch Vereinigung von Charakteristiken durch jeden $(\gamma(s),\varphi(\gamma(s))), s\in [\alpha,\beta]$ (mit $\gamma : \R \ra \R^2$) konstruieren.
\end{folge}

\begin{einschr}
  Der Vektor $(a(\gamma(s),\varphi(\gamma(s))), b(\gamma(s),\varphi(\gamma(s))))$ darf in keinem Punkt $\gamma(s)=(\gamma_1(s),\gamma_2(s))\in\Gamma$ tangential sein an $\Gamma$. Der Weg $\Gamma$ ist also genau dann nicht-charakteristisch, wenn
  \begin{align*} \tag{$\ast$}
    a\left(\gamma(s),\varphi(\gamma(s))\right)\dot\gamma_2(s)
    -b\left(\gamma(s),\varphi(\gamma(s))\right)\dot\gamma_1(s)\neq 0\quad \fa s\in
    [\alpha,\beta]\;
  \end{align*}
  gilt.
  \index{Weg!nicht-charakteristisch}
\end{einschr}

\begin{theorem}
  Seien $a,b,c,\varphi$ reellwertige Charakteristiken und $\,\Gamma$ eine nicht-charakteristische $C^1$-Kurve. Dann besitzt das quasilineare Problem 1. Ordnung
  \[
  a(x,y,u)u_x+b(x,y,u)u_y=c(x,y,u),\quad u\vert_\Gamma=\varphi
  \]
  in einer Umgebung von $\Gamma$ eine eindeutige $C^1$-Lösung $u$.
\end{theorem}

\begin{proof}
  \begin{description}
  \item[Existenz:] Für $s\in [\alpha,\beta]$ berechnet $x(t,s),y(t,s),z(t,s)$ durch:
    \[
    \left.
     \begin{aligned}
       \pdiff t x(t,s)=a(x,y,z), &\quad x(0,s)=\gamma_1(s) \\
       \pdiff t y(t,s)=b(x,y,z), &\quad y(0,s)=\gamma_2(s) \\
       \pdiff t z(t,s)=c(x,y,z), &\quad z(0,s)=\varphi(\gamma(s))
    \end{aligned}
    \quad
    \right\rbrace
    \eq{eq:star1}
    \]
    (Charakteristik durch $(\gamma(s), \varphi(\gamma(s)))$)

    Nach Satz~\ref{satz:2.1} exitiert eine eindeutige Lösung $(x,y,z)$ und hängt $C^1$ von $s$ ab.

    Für $t=0$:
    \begin{align*}
      \det
      \begin{pmatrix}
        \partial_t x(0,s) & \partial_sx(0,s) \\
        \partial_t y(0,s) & \partial_sy(0,s)
      \end{pmatrix}
      =\det
      \begin{pmatrix}
        a(\gamma(s),\varphi(\gamma(s))) & \dot\gamma_1(s) \\
        b(\gamma(s),\varphi(\gamma(s))) & \dot\gamma_2(s)
      \end{pmatrix}
      \stackrel{(\ast)}{\neq} 0
     \end{align*}
     Nach dem Satz über Umkehrabbildungen ist die Abbildung $((t,s)\mapsto(x,y) = (x(s,t),y(s,t)))$ umkehrbar in einer Umgebung der Anfangskurve $t=0$. Also existieren  $t=t(x,y)$ und $s=s(x,y)$ in einer Umgebung $D\subset\R^2$ von $\Gamma$ und sind dort $C^1$. Setze $$u(x,y):=z(t(x,y),s(x,y)),  \quad(x,y)\in D.$$
Nun ist einerseits
\begin{align*}
  \pdiff t u(x,y)= \ &u_x(x,y)\cdot \pdiff t x+u_y(x,y)\pdiff t y \\
  \overset{\scriptsize\eqref{eq:star1}}=& a(x,y,u(x,y))u_x(x,y)+b(x,y,u(x,y))u_y(x,y) \eq{eq:7} \\
  \intertext{und andererseits}
  \pdiff t u(x,y)=&\pdiff t z(t,s)\overset{\scriptsize\eqref{eq:star1}}=
  c(x,y,z)=c(x,y,u(x,y)). \eq{eq:8}
\end{align*}
         
Aus den Gleichungen \eqref{eq:7} und \eqref{eq:8} folgt, dass $u$ eine Lösungsfläche über $D$ definiert.

       \item[Eindeutigkeit:] Es sei nun $\bar u$ eine weitere Lösung. Wir setzen $\bar z(t,s):=\bar u(\bar x(t,s),\bar y(t,s))$, wobei
    \begin{eqnarray*}
      \partial_t\bar x(t,s)=a(\bar x,\bar y,\bar u(\bar x,\bar y)), &
      \bar x(0,s)=\gamma_1(s), \\
      \partial_t\bar y(t,s)=b(\bar x,\bar y,\bar u(\bar x,\bar y)), &
      \bar y(0,s)=\gamma_2(s).
    \end{eqnarray*}
Dann ist
    \begin{align*}
      \pdiff t \bar z(t,s) = \ & a(\bar x,\bar y,\bar u)\bar u_x
      + b(\bar x,\bar y,\bar u)\bar u_y = c(\bar x,\bar y,\bar u) \\
      \underset{\scriptsize\text{Def.}}=&c(\bar x,\bar y,\bar z).
    \end{align*}
Nun ist
\[
\bar z(0,s)=\bar u(\gamma_1(s),\gamma_2(s))\overset{\bar u\; \scriptsize\text{Lsg.}}=\varphi(\gamma(s))
\]
und  $(\bar x,\bar y,\bar z)$ löst \eqref{eq:star1}. Der Satz von Picard-Lindelöf liefert die Eindeutigkeit der Lösung, d.h. $(x,y,z)=(\bar x,\bar y,\bar z)$. Also ist $u=\bar u$.
  \end{description}
  
  Damit folgt komplett die Behauptung.
\end{proof}

\begin{bem}
  \begin{enumerate}[(a)]
  \item Der Beweis ist konstruktiv!
  \item Der Beweis lässt sich auf $n>2$ verallgemeinern.
  \end{enumerate}
\end{bem}

\begin{bsp}
  $\Gamma:=\{(x,1)\with x\in\R\}$, $\alpha\in\R$, $\varphi\in C^1(\R)$. Wir betrachten die lineare Differentialgleichung erster Ordnung
  \begin{align*}
    \left.
      \begin{aligned}
        xu_x+yu_y=&\;\alpha u\\
        u\rvert_\Gamma=&\;\varphi
      \end{aligned}
      \quad\right\rvert
  \end{align*}
  Damit ist $
  \begin{pmatrix}
    a(x,y,u) \\
    b(x,y,u) \\
    c(x,y,u)
  \end{pmatrix}
  =
  \begin{pmatrix}
    x \\
    y \\
    \alpha u
  \end{pmatrix}$, und für $\Gamma$ gilt wegen $
  \begin{pmatrix}
    a(x,y,u) \\
    b(x,y,u)
  \end{pmatrix}
  =
  \begin{pmatrix}
    x \\
    1
  \end{pmatrix}$, dass es nicht-charakteristisch ist, da $(a,b)^T$ für $y = 0$ tangential wäre.

  \begin{figure}[ht!]
    \centering
    \begin{pspicture}(-4,-1)(4,3)
      \psaxes{->}(0,0)(-4,-1)(4,3)
      \psline[linewidth=1.6pt](-4,1)(4,1)
      \rput[tl](1.1,.9){$\Gamma$}
      \psline{->}(1,1)(2,2)
      \psline{->}(1.5,1)(3,2)
      \psline{->}(2,1)(4,2)
      \psline{->}(0.5,1)(1,2)
      \psline{->}(0,1)(0,2)
      \psline{->}(-0.5,1)(-1,2)
      \psline{->}(-1,1)(-2,2)
      \psline{->}(-1.5,1)(-3,2)
      \psline{->}(-2,1)(-4,2)
    \end{pspicture}
    \caption{Weg $\Gamma$ mit Vektorfeld $(x,1)$}
  \end{figure}
  Wir treffen die Vereinbarung, dass die zeitliche Ableitung durch einen Punkt dargestellt wird, also $\dot{\hspace{1em}} := \partial_t$. Mit dieser Notation ergibt sich aus den obigen Gleichungen
  \begin{eqnarray*}
    &\dot x = x, & x(0,s)=s \\
    &\dot y = y, & y(0,s)=1 \\
    &\dot z = \alpha \cdot z, & z(0,s)=\varphi(s).
  \end{eqnarray*}
  Wir erkennen sofort, dass $x=x(t,s)=s\cdot e^t$ und $y=y(t,s)=e^t$ sein muss. Zusätzlich ergibt sich $z=z(t,s)=\varphi(s)e^{\alpha t}$. Durch Auflösen der Gleichungen für $x$ und $y$ nach $t,s$, erhalten wir $t=\ln y$ und $s=\frac x y$. Schließlich erhalten wir      
  \[
  \underline{u(x,y)}=\varphi\left(\frac x y\right)e^{\alpha\ln
    y}=\underline{\varphi\left(\frac x y\right)y^\alpha}
  \]
  mit $x,y\in\R$ und $y>0$.
\end{bsp}

\begin{bsp}[nicht-viskose Burgersgleichung]
  \index{Burgersgleichung}
  \label{bsp:6}
  Wir betrachten die quasilineare Differentialgleichung
  \begin{align*}
    \left.
      \begin{aligned}
        u_x+uu_y&=0,  &&x>0,y\in\R \\
        u(0,y)&=h(y), && y\in\R
      \end{aligned}\quad
    \right\vert\tag{B}\label{eq:B}
  \end{align*}
  \[
  \begin{pmatrix}
    a(x,y,u) \\
    b(x,y,u)
  \end{pmatrix}
  =
  \begin{pmatrix}
    1 \\
    u
  \end{pmatrix}
  \]
  \begin{figure}[ht!]
    \centering
    \begin{pspicture}(-1,-2)(3,2)
      \psaxes{->}(0,0)(-1,-2)(3,2)
      \psline[linewidth=1.6pt](0,-2)(0,2)
      \rput[bl](0.1,1.1){$\Gamma$}
    \end{pspicture}
    \caption{Weg $\Gamma$ entlang der y-Achse}
  \end{figure}
  Dies führt auf das Gleichungssystem
  \begin{eqnarray*}
    \dot x = 1, &x(0,s)=0 \\
    \dot y=z, &y(0,s)=s \\
    \dot z=0, &z(0,s)=h(s),
  \end{eqnarray*}
  mit folgender Lösung:
  \begin{align*}
    \left.
      \begin{aligned}
        x=x(t,s)=t \\
        y=y(t,s)=s+h(s)t \\
        z=z(t,s)=h(s)
      \end{aligned}\quad
    \right\}\eq{eq:star2}
  \end{align*}
  Wir erkennen, dass $u$ entlang der Geraden durch $(0,s)$ mit der Steigung $h(s)$ den konstanten Wert $h(s)$ hat. Also haben wir
  \[ u(t,s+h(s)t)=h(s).\eq{eq:star3} \]
  Wir wissen $(t,s)\mapsto (x,y)$ ist \underline{lokal} ein Diffeomorphismus. Es ergibt sich die implizite Gleichung 
  \[
  \boxed{
    u(x,y) \stackrel{\scriptsize\eqref{eq:star3}}= h(s)
    \stackrel{\scriptsize\eqref{eq:star2}}=h(y-h(s)x)=h(y-u(x,y)x)
  }\,.
  \]
  Global kann aber ein Problem entstehen: Die Geraden
  \[
  \{(t,s+h(s)t\with t\geq0\} \text{ und }
  \{(t,\bar s+h(\bar s)t\with t\geq 0\}
  \]
  können sich schneiden. Im Schnittpunkt müsste $u$ beide Werte
  $h(s)$ bzw. $h(\bar s)$ annehmen. Dies ist ein Widerspruch.
  \begin{figure}[ht!]
    \centering
    \begin{pspicture}(-1,-1)(4,3)
      \psaxes[labels=none,ticks=none]{->}(0,0)(-1,-1)(3,3)
      \psline(-1,0)(3,2)
      \rput[l](3.1,2){$u=h(s)$}
      \rput[br](-.1,.5){$s$}
      \psline(-1,1.5)(3,1.5)
      \rput[l](3.1,1.5){$u=h(\bar s)$}
      \rput[br](-.1,1.6){$\bar s$}
      \pscircle(2,1.5){.3}
    \end{pspicture}
    \caption{Schnitt der Geraden}
  \end{figure}

  Wir betrachten nun $u(x,y)=h(y-u(x,y)x)$. Ableiten nach $y$ ergibt
  \begin{align*}
    u_y(x,y)=&\underbrace{h'(y-u(x,y)x)}_{=h'(s)}\cdot(1-u_y(x,y) x)
    \end{align*}
    und somit
    \begin{align*}
    u_y(x,y)=&\frac{h'(s)}{1+h'(s)x} \, .
  \end{align*}
  Ist nun $h'(s)<0$, so geht $u_y(x,y) \ra -\infty$ für $x\ra\frac {-1}
  {h'(s)}$. Dieses Verhalten nennt man "`Schock"' (der Gradient "`explodiert"').
  \begin{figure}[ht!]
    \centering
    \begin{pspicture}(-1,-1)(4,4)
      \psaxes[labels=none,ticks=none]{->}(0,0)(-1,-1)(4,4)
      \psline(0,0.5)(1.5,1.25)
      \psline(0,1.5)(3,3)
      \psline[linestyle=dotted](1.5,0)(1.5,3.5)
      \psline[linestyle=dotted](3,0)(3,3.5)
      \psdot(3,3)
      \psdot(1.5,1.25)
      \rput[r](-.1,.5){$s_0$}
      \rput[t](1.5,-.1){$-\frac 1{h'(s_0)}$}
    \end{pspicture}
    \caption{Test}
  \end{figure}

  Ist also allgemeiner $h'(s_0)=\min h'<0$, so gibt es keine $C^1$-Lösung in einer beidseitigen Umgebung der Geraden $x=\frac{-1}{h'(s_0)}$. Dieser "`\idx{blow-up}"' ist "`typisch"' für nicht-lineare Gleichungen.
  Als Ausweg verwenden wir einen "`schwächeren"' Lösungsbegriff (nicht $C^1$). Seien $R,S\in C^1(\R),S'(w)=wR'(w)$, $R'(w)\neq0\, \fa\,  w\in\R$. Sei $u$ eine $C^1$-Lösung von \eqref{eq:B}. Wir betrachten dafür die Divergenzform
  \begin{align*}
    \partial_xR(u)+\partial_yS(u)=&R'(u)\cdot u_x+S'(u)u_y \\
    =& R'(u)(u_x+uu_y)=0.
  \end{align*}
  Integrieren wir diese bezüglich $y$, erhalten wir
  \[ \diff x
  \int\limits_a^bR(u(x,y))dy+S(u(x,b))-S(u(x,a))=0\eq{eq:I}
  \]
  $\quad\fa\, x>0,\fa\, a<b$. Jede Funktion $u$, die \eqref{eq:I} löst, heißt \idx{Integrallösung} von \eqref{eq:B}. Somit gilt \eqref{eq:I} für eine größere Klasse von Funktionen. Wenn allerdings $u$ eine Integrallösung und $C^1$ ist, dann differenziere \eqref{eq:I} nach \eqref{eq:B}, und wir erhalten
  \begin{align*}
    \underbrace{R'(u(x,b))}_{\neq 0}(u_x(x,b)+u(x,b)\cdot u_y(x,b))=0
    \end{align*}
    {und schließlich ist}
    \begin{align*}
    u_x(x,b)+u(x,b)\cdot u_y(x,b)=0
  \end{align*}
  mit $x>0$ und $b\in\R$. Also ist $u$ Integrallösung und $C^1$ bzw. $u$ ist $C^1$-Lösung von \eqref{eq:B}.
\end{bsp}

Betrachten wir nun folgende Situation: Ein Gebiet $\Omega\subset\R^2$ wird durch eine glatte Kurve $y=\xi(x)$ in zwei Gebiete $\Omega_+$ und $\Omega_-$ zerlegt und $u\in C^1$ sei eine auf $\Omega_+$ bzw. $\Omega_-$ von beiden Seiten bis zum Rand $y=\xi(x)$ stetige Integrallösung von \eqref{eq:I}. Allerdings sei $u$ einem Sprung auf $y=\xi(x)$ versehen. Wir setzen
\[ u^\pm(x):=\lim_{y\ra\xi^\pm(x)}u(x,y). \]

\begin{figure}[ht!]
  \centering
  \begin{pspicture}(-1,-1)(6,5)
    \psaxes[labels=none,ticks=none]{->}(0,0)(-.5,-.5)(5.5,4.5)
    \rput[tl](5.5,-.1){$x$}
    \rput[br](-.1,4.5){$y$}

    %Omega
    \psccurve(1,1)(2.5,.7)(4,.7)(4,3)(1,3)
    \rput[tl](4,.7){$\Omega$}
    \rput(3.5,1.5){$\Omega_-$}
    \rput(1.8,2.5){$\Omega_+$}

    % Kurve
    \pscurve(.5,.5)(2,1.5)(3,2.5)(4.5,3.5)
    \rput[tl](4.5,3.5){$y=\xi(x)$}
  \end{pspicture}
  \caption{Zerlegung von $\Omega$ in zwei Gebiete}
\end{figure}
Unser Ziel ist eine Charakterisierung der \idx{Schockkurve} $y=\xi(x)$. Für $a<\xi(x)<b$:
\begin{align*}
  0=&S(a(x,b))-S(u(x,a))+\diff x\Big( 
    \int_a^{\xi(x)}R(u(x,y))\d y+\int\limits_{\xi(x)}^bR(u(x,y))\d y 
  \Big) \\
  =& S(u(x,b))-S(u(x,a))+\xi'(x)R(u^-(x))-\xi'(x)R(u^+(x)) \\
  &+\Big(
    \int_b^{\xi(x)}\underbrace{\partial_x R(u(x,y))}_{=-\partial_yS(u(x,y))}\d y
    +\int_{\xi(x)}^b\underbrace{\partial_x R(u(x,y))}_{=-\partial_yS(u(x,y))}\d y
  \Big) \\
  =& \xi'(x)\left(
    R(u^-(x))-R(u^+(x))
  \right) - S(u^-(x))+S(u^+(x)).
\end{align*}

Falls $R(u^-)\neq R(u^+)$, so gilt:
\[
\left. 
  \xi'(x)=\frac{S(u^-(x))-S(u^+(x))}{R(u^-(x))-R(u^+(x))}\quad
\right\vert\qquad\text{"`Schockrelation"'}
\]

\begin{bsp}[vergleiche Beispiel~\ref{bsp:6}]
  Wir betrachten wieder die Burgersgleichung mit einem konkreten $h$
  \begin{align*}
    u_x+uu_y&=0,\quad x>0,y\in\R \\
    u(0,y)=h(y)&:=
    \begin{cases}
      1 &,y<0 \\
      1-y &,0\leq y\leq 1 \\
      0 &, y>1
    \end{cases}
  \end{align*}

  Aus Beispiel~\ref{bsp:6} wissen wir, dass $u(x,y)$ für $s\in\R$ entlang der Geraden $(x,y)=(t,s+h(s)t)$ den konstanten Wert $h(s)$ hat. Ist beispielsweise $s<0$ (und damit $h(s)=1)$, so ist $u(x,y)=1$. Entlang $(x,y)=(t,s+t)$ ist dann $u(x,y)=1$ für alle $0< x, y< x$. Analog lässt sich folgern, dass
  \[
  u(x,y)=
  \begin{cases}
    1 &,0< x\leq 1,y< x \\
    \frac{1-y}{1-x} &, 0<x\leq y\leq 1 \\
    0 &, 0< x\leq1,y>1
  \end{cases}
  \]
  eine stetige und stückweise $C^1$-Lösung auf $x \in (0,1]$ ist. Was ist jedoch für $x>1$?

  Wir wählen $R(u):=u$ und $S(u):=\frac 12 u^2$. Dann ist die Schockrelation zwischen $\Omega_-=[u=1]$ und $\Omega_+=[u=0]$ 
  \[ \frac{S(1)-S(0)}{R(1)-R(0)}=\frac 12. \]
  Die Schockkurve $y=\xi(x)$ beginnt im Punkt $(1,1)$ und erfüllt $\xi'(x)=\frac 12$ und $\xi(1)=1$, also $\xi(x)=\frac{x+1}2$. Daher definieren wir  mit $x>1$
  \[
  u(x,y):=
  \begin{cases}
    1 &, y<\xi(x) \\
    0 &, y>\xi(x)
  \end{cases}.
  \]
  
  \begin{figure}[ht!]
    \centering
    \begin{pspicture}(-1,-1)(5,4)
      \psaxes[labels=none,ticks=none]{->}(0,0)(-.5,-.5)(4.5,3.5)
      \rput[tl](4.5,-.1){$x$}
      \rput[br](-.1,3.5){$y$}
      \psline(0,0)(3,3)
      \psline(0,0.4)(2,2)
      \psline(0,0.8)(2,2)
      \psline(0,1.2)(2,2)
      \psline(0,1.6)(2,2)
      \psline(0,2)(2,2)

      \rput(3,1){$u=1$}
      \rput(1.5,3){$u=0$}
      \rput[tl](3.1,3){$y=\xi(x)$}
    \end{pspicture}
    \caption{Schockkurve}
  \end{figure}

  Die so definierte Funktion ist eine Integrallösung, denn
  \[
  \diff x\int\limits_a^bu(x,y)\d y+\frac 12(u(x,b)^2-u(x,a)^2)=0.
  \]
\end{bsp}


%%% Local Variables: 
%%% mode: latex
%%% TeX-master: "Skript"
%%% End: 

\newchapter{Distributionen}

Für dieses Kapitel treffen wir zwei Generalvoraussetzungen. Sofern nicht anders definiert, sei die nicht-leere Menge $\Omega\subset\R^n$ offen, und der Körper $\K$ stehe für einen Körper der Menge $\{\R, \C\}$.

\begin{defi}
  \begin{enumerate}[(a)]
  \item $K\Subset\Omega:\Lra\bar K\subset\Omega$ kompakt.
  \item Sei $f:\Omega\ra\K$. Dann heißt die Menge 
    \[ \supp(f):=\overline{\{x\in\Omega\with f(x)\neq 0\}} \]
    \idx{Träger} von $f$.
    \begin{figure}[ht!]
      \centering
      \begin{pspicture}(-3,-1)(3,3)
        \psccurve(-1,0)(-2.5,1.5)(-1,2)(1,1.5)(3,0)(2,-1)(0,-.5)
        \rput[tl](2.1,-1){$\Omega$}
        \psccurve(-.5,0)(-1.5,1)(2,0)(0,-.2)
        \rput(.3,.5){$\supp(f)$}
      \end{pspicture}
      \caption{Träger}
    \end{figure}
  \item Mit  $m\in\N$ definieren wir die Funktionen mit kompakten Träger als
    \begin{align*}
      C_c^m(\Omega)&:=\{f\in C^m(\Omega)\with\supp(f)\Subset\Omega\}, \\
      C_c(\Omega)&:=C_c^0(\Omega).
    \end{align*}
    
  \item Es sei $Y$ metrischer Raum mit $X\subset Y$. Wir schreiben dann $X\overset{d}\subset Y$, wenn $X$ \idx{dicht} in $Y$ liegt. In diesem Fall ist $\bar X=Y$.
  \item Für $1\leq p\leq\infty$ definieren wir die \idx{Lebesgue-Räume} als
    \[ L_p(\Omega):=(L_p(\Omega),\norm{\,\cdot\,}_p). \]
  \item Mit $k\in\N$ sei 
    \[ 
    BC^k(\Omega):=\left(
    \{f\in C^k(\Omega)\with\norm{f}_{BC^k}<\infty\},\norm{\,\cdot\,}_{BC^k}
    \right),
    \]
    wobei 
    \[ 
    \norm f_{BC^k}:=\max_{\abs\alpha\leq k}\norm{\partial^\alpha f}_\infty
    =\max_{\abs\alpha\leq k}\sup_{x\in\Omega}\abs{\partial^\alpha f(x)},
    \]
    d.h. alle partiellen Ableitungen in jedem Punkt $x \in \Omega$ sind endlich.
  \item Wie oben sei $k \in \N$, dann ist $$ BUC^k(\Omega):=\{f\in BC^k(\Omega)\with\partial^\alpha f
    \;\text{glm.\ stetig }\fa\,\abs\alpha\leq k\}.$$
  \end{enumerate}
   $L_p(\Omega)$, $BC^k(\Omega)$ und $BUC^k(\Omega)$ sind Banachräume.
\end{defi}

\begin{lemma}
  \label{lemma:3.1}
  Für $1\leq p\leq\infty$ gilt:
  \[ C_c(\R^n)\overset{d}\subset L_p(\R^n) \]
\end{lemma}

\begin{proof}
  Analysis III
\end{proof}

\begin{defi}
  Sei $\varphi\in C^\infty(\R^n)$ mit $\varphi\geq0$, sowie $\supp(\varphi)\subset\bar\B_{\R^n}(0,1)$ und 
  \[\int_{\R^n}\varphi \d x=1.\]
  Dann sei für alle $\epsilon>0$
  \[ \varphi_\epsilon(x):=\epsilon^{-n}
  \varphi\left(\frac x\epsilon\right), \quad x\in\R^n.\]
 Die Menge $\{\varphi_\epsilon\with\epsilon>0\}$ heißt \idx{Mollifier} (\idx{glättender Kern}).  
\end{defi}

\begin{bem} Es sei
  \[
  \varphi(x):=
  \begin{cases}
    0&,\abs x\geq 1 \\
    C_0e^{-\frac 1{1-\abs x^2}}&,\abs x<1
  \end{cases}
  \]
  mit
  \[
  C_0=\left(\;
    \int_{\B_n}e^{-\frac 1{1-\abs y^2}}\d y
  \right)^{-1}.
  \]
  $\{\varphi_\epsilon\with\epsilon>0\}$ ist ein Mollifier.
\end{bem}

\begin{proof}
  Nachrechnen.
\end{proof}

\begin{defi}
  Es seien $f,g:\R^n\ra\K$ messbar und
  \[ ( f\ast g)(x) :=\int_{\R^n}f(x-y)g(y)\d y \]
  für fast alle $x\in\R^n$, so dass das Integral existiert. Die daraus entstehende Funktion $f\ast g:\R^n\ra\K$ heißt \idx{Faltung} von $g$ und $f$.
\end{defi}

\begin{erinnerung}
  $E,F,G$ seien normierte Vektorräume und $T:E\times F\ra G$ sei eine bilineare und stetige Abbildung. Somit existert ein $c_T>0$ mit
  \[
  \norm{T(x,y)}_G\leq c_T\norm x_E\cdot\norm y_F,\quad x\in E,y\in F.
  \]
  Dann induziert $\norm T:=\inf{c_T}$ eine Norm auf $G$. 
\end{erinnerung}

\begin{satz}
  \label{satz:3.3}
  Seien $L_p:=L_p(\R^n)$, $BC^k:=BC^k(\R^n)$ und $BUC^k:=BUC^k(\R^n)$. Dann gelten folgende Aussagen:
  
  \begin{enumerate}[\rm(i)]
  \item \label{satz:3.3-1} Die Faltung ist bilinear und stetig als Abbildung
    \begin{itemize}
    \item $L_p\times L_q\ra L_r ,1\leq p,q,r\leq\infty$ mit $\frac 1p+\frac 1q=1+\frac 1r$,
    \item $L_1\times BC^k\ra BC^k, k\in\N$,
    \item $L_1\times BUC^k\ra BUC^k, k\in\N$,
    \end{itemize}
    jeweils mit Norm $1\, ($z.B. $\norm{f\ast g}_r\leq\norm{f}_p\norm{g}_q)$

  \item $f\ast g=g\ast f$, falls sinnvoll.
  \item $\partial^\alpha(f\ast g)=f\ast(\partial^\alpha g)\,\fa\,\alpha\in\N^n, f\in\L_1, g\in BC^{\abs\alpha}$.
  \item \label{satz:3.3-4} $\supp(f\ast g)\subset\supp(f)+\supp(g):=\{x+y\with x\in\supp(f),y\in\supp(g)\}$, falls $f$ oder $g$ einen kompakten Träger hat und $f\ast g$ existiert.
  \item \label{satz:3.3-5} ist $\{\varphi_\epsilon\with \epsilon>0\}$ Mollifier und $f\in L_p$ mit $1\leq p\leq\infty$, so gilt
    \[
    \varphi_\epsilon\ast f\xrightarrow[\epsilon\ra 0^+]{}f\text{ in }L_p.
    \]
  \end{enumerate}
\end{satz}

\begin{proof}
  %\hspace{.7em}
 % \begin{itemize}
  (\ref{satz:3.3-1})-(\ref{satz:3.3-4}) Übung. \newline
  (\ref{satz:3.3-5}) Sei $f\in L_p$. Dann folgt o.B.d.A. aus Lemma~\ref{lemma:3.1} $f\in C_c(\R^n)$.
    \begin{align*}
      \boxed{\begin{aligned}
          \delta>0 & \underset{\scriptsize\text{Lemma~\ref{lemma:3.1}}}\Ra
          \exists \, g\in C_c(\R^n): \norm{f-g}_p<\frac \delta 2 \\
          \Ra\norm{\varphi_\epsilon\ast f-f}_p & \stackrel[\triangle\scriptsize\text{-Ungl.}]{}\leq 
          \underbrace{\norm{\varphi_\epsilon\ast f-\varphi_\epsilon\ast g}_p}
          _{\underset{\scriptsize\eqref{satz:3.3-1}}\leq \norm{\varphi_\epsilon}_1\cdot\norm{f-g}_p}
          +\norm{\varphi_\epsilon\ast g-g}_p+\underbrace{\norm{g-f}_p}
          _{\leq\frac\delta 2} \\
          \int_{\R^n}\varphi_\epsilon(x&)  \d x  =\int_{\R^n}\epsilon^{-n}\cdot\varphi\left(\frac x\epsilon\right)\d x =\int_{\R^n}\varphi \d x=1
        \end{aligned}}
    \end{align*}
  %\end{itemize}
  $\varphi_\epsilon\ast f\underset{\scriptsize(\ref{satz:3.3-1}),(\ref{satz:3.3-4})}\in C_c(\R^n)$, setze $(\tau_a f)(x):=f(x-a), x,a\in\R^n$, dann:
  \begin{align*}
    f(x)-\varphi_\epsilon\ast f(x)\quad \,\,\,\,=\quad \,\,\,\,&\int_{\R^n}\left(
      f(x)-f(x-y)
    \right)\varphi_\epsilon(y)\d y \\
    \stackrel[z:=\frac x\epsilon]{\scriptsize\textnormal{Transf.-Satz}}=
    &\int_{\R^n}\left(
      f(x)-f(x-\epsilon z)
    \right)\varphi(z)\d z \\
    = \quad \,\,\,\, &\int_{\bar\B^n}\left(
      f(x)-(\tau_{\epsilon z}f)(x)
    \right)f(z)\d z
    \end{align*}
  \[
  \Ra\quad \abs{f(x)-\varphi_\epsilon\ast f(x)}\leq
  \sup_{\abs z\leq 1}\abs{f(x)-\tau_{\epsilon z}f(x)}:=h_\epsilon(x)\eq{eq:3.1}
  \]
  \begin{itemize}
  \item $f\in C_c(\R^n)\Ra\, \exists \, K=\bar K\Subset\R^n:\supp(\tau_{\epsilon z}f(\cdot))\subset K\;\fa \, 0<\epsilon\leq 1$, $\abs z\leq 1$
    \[
    \Ra\quad\fa\, x\in K^c:h_\epsilon(x)=0\eq{eq:3.2}
    \]
  \item $k\vert_K$ ist gleichmäßig stetig $\Ra$ $h_\epsilon\vert_K\xrightarrow[\epsilon\ra 0]{}0$ gleichmäßig
    \begin{align*}
      \Ra\ \, \, \,  &(h_\epsilon\vert_K)^p\xrightarrow[\epsilon\ra 0]{} 0 \;\text{gleichmäßig} \\
      \underset{\scriptsize\eqref{eq:3.1},\eqref{eq:3.2}}\Ra&\;\norm{\varphi_\epsilon\ast f-f}_p\xrightarrow[\epsilon\ra0^+]{}0
    \end{align*}
  \end{itemize}
  und damit folgt die Behauptung.
\end{proof}

\begin{defi}
  \begin{itemize}
  \item $\D(\Omega):=C_c^\infty=\{\varphi\in\C^\infty(\Omega)\with\supp(\varphi)\Subset\Omega\}$ heißt der Raum der Testfunktionen.
  \item $\D(\Omega)$ wird mit der induktiven Limes-Topologie versehen. $\varphi_j\xrightarrow[j\ra\infty]{}0$ in $\D(\Omega):\Lra\exists \, K=\bar K\Subset\Omega$ mit $\supp(\varphi_j)\subset K\, \fa \,j$ und $\fa\,\alpha\in\N^n$:
    \[ \sup_{x\in K}\abs{\partial^\alpha\varphi_j(x)}\xrightarrow[j\ra\infty]{}0. \]
  \item $\varphi_j\ra\varphi$ in $\D(\Omega):\Lra\varphi_j-\varphi\ra0$ in $\D(\Omega)$.
  \end{itemize}
\end{defi}

\begin{satz}\label{satz:3.4}
  Es sei $1\leq p\leq\infty$.
  \[ \Ra \D(\R^n)\overset{d}\subset L_p(\R^n) \]
\end{satz}

\begin{proof}
  Es sei $g\in L_p$ und $ \chi\in\D(\R^n)$ mit $ 0\leq\chi\leq 1, \chi\vert_{\bar\B^n}=1$, weiter definieren wir $g_\epsilon:=\chi(\epsilon)\cdot g\in L_p(\R^n)$, wobei $ \supp(g_\epsilon)\Subset\R^n$.
  \begin{align*}
    \Ra\quad \norm{g-g_\epsilon}_p=&\norm{g-g_\epsilon}_{L_p([\abs x>\frac 12])}
    \leq2\cdot\norm g_{L_p([\abs x>\frac 12])} \\
    =&2\cdot\left(
      \int_{[\abs x>\frac 1\epsilon]}\abs{g(x)}^p\d x
    \right)^{\frac 1p}\xrightarrow[\epsilon\ra0]{}0.
  \end{align*}
  Sei $\delta>0$ beliebig $\Ra\exists\,\epsilon_0>0:\norm{g-g_{\epsilon_0}}_p<\frac\delta2$. Sei $\{\varphi_\xi\with\xi>0\}$ Mollifier 
  $$\underset{\scriptsize\text{Satz}\ref{satz:3.3}}\Ra\varphi_\xi\ast g_{\epsilon_0}\in\D(\R^n) \text{ und }\varphi_\xi\ast g_{\epsilon_0}\xrightarrow[\xi\ra0^+]{}g_{\epsilon_0} \text{ in } L_p,$$
   d.h. $\exists\,\xi_0>0: \norm{\varphi_{\xi_0}\ast g_{\epsilon_0}-g_{\epsilon_0}}_p<\frac\delta2$
  \[ \Ra\quad\norm{\varphi_{\xi_0}\ast g_{\epsilon_0}-g}_p<\delta \]
  wegen der $\triangle$-Ungleichung.
\end{proof}

\begin{defi}
  \[\Lloc(\Omega):=\{f:\Omega\ra\K\text{ messbar}\with\; f\vert_K\in L_1(K)\, \fa\, K=\bar K\Subset\Omega\}\]
\end{defi}

\begin{bem}
  \label{bem:3.5}
  $f\in\Lloc(\Omega)$, $g\in\D(\R^n)\Ra f\ast g\in BUC^\infty(\R^n)$.
\end{bem}

\begin{theorem}
  \label{theorem:3.6}
  Es sei $f\in\Lloc(\Omega)$ mit $\int_\Omega f\varphi\d x=0\;\fa\,\varphi\in\D(\Omega)$
  \[ \Ra\quad f=0\quad(\text{fast überall}) \]
\end{theorem}

\begin{proof}
  Das Resultat ist lokal $\Ra\supp(f)\Subset\Omega$
  \begin{itemize}
  \item[$\Ra$] o.B.d.A $f\in L_1(\R^n)$, also $\Omega=\R^n$.
  \item Es sei $\{\varphi_\epsilon\with \epsilon>0\}$ Mollifier $\underset{\scriptsize \text{Satz}\, \ref{satz:3.3}\eqref{satz:3.3-5}}\Ra\varphi_\epsilon\ast f\longrightarrow f$ in $L_1(\R^n)$ für $\epsilon\ra 0^+. \,\fa\, x\in\R^n$, $\epsilon>0$ sei $\Psi_x(y):=\varphi_\epsilon(x-y)$ $\Ra\Psi_x\in\D(\R^n)$.
  \end{itemize}
   \begin{align*}
   	\Ra (\varphi_\epsilon\ast f)(x)&=\int_{\R^n}\varphi_\epsilon(x-y)f(y)\d y\overset{\scriptsize\text{Vor.}}=0 \\
	 &\underset{\scriptsize \text{Satz}\, \ref{satz:3.3}\eqref{satz:3.3-5}}\Ra  \quad  f=0,
\end{align*}
was zu zeigen war.
\end{proof}

\begin{defi}
  \index{Distribution}
  Eine lineare Abbildung $T:\D(\Omega)\ra\K$ ist stetig 
  \[
  	:\Longleftrightarrow T(\varphi_j)\xrightarrow[j\ra\infty]{}0
\]
in $\K$ für alle Folgen $\varphi_j\ra 0$ in $\D(\Omega).$
  \begin{itemize}
  \item $\D'(\Omega):=\{T:\D(\Omega)\ra\K\with T\text{ stetig, linear}\}$ heißt der \idx{Raum der Distributionen}.
  \item $\D'(\Omega)$ versehen wir mit $w^*$-Topologie, d.h. $T_j\xrightarrow{w^*}T$ in $\D'(\Omega):\Lra T_j(\varphi)\xrightarrow[j\ra\infty]{} T(\varphi)$ für alle $\varphi\in\D(\Omega)$.
  \end{itemize}
\end{defi}

\begin{notation}
  $
  \< T,\varphi \>:=\< T,\varphi\>_{\D(\Omega)}:=T(\varphi),
   T\in\D'(\Omega), \varphi\in\D(\Omega).
  $
\end{notation}

\begin{bsp}
  \begin{enumerate}[(a)]
  \item \textbf{\underline{reguläre Distributionen:}} \index{Distribution!reguläre} 
    \[
    \fa \,f\in \Lloc(\Omega):\<T_f,\varphi\>:=\int_\Omega f\varphi\d x
   , \quad\varphi\in\D(\Omega)\]
    \begin{itemize}
    \item[$\Ra$] $T_f:\D(\Omega)\ra\K$ linear.
    \end{itemize}
    Sei $\varphi_j\ra0$ in $\D(\Omega)\Ra\exists \,K\Subset\Omega:\supp(\varphi_j)\subset K\;\fa \, j$
    \[
      \Ra\quad\abs{\<T_f,\varphi_j\>}=\Abs{\int_K f\varphi_j\d x}
      \leq\sup_{x\in K}\abs{\varphi_j(x)}\cdot\norm{f}_{L_1(K)}
    \]
    $\Ra T_f:\D(\Omega)\ra\K$ ist stetig, d.h. $T_f\in\D'(\Omega)$. Aus Theorem~\ref{theorem:3.6} folgt $$
    	(T_f=T_g\Longrightarrow f=g)\;\fa\, f,g\in\Lloc(\Omega).
$$ 
Also ist $(f\mapsto T_f):\Lloc(\Omega)\ra\D'(\Omega)$
injektiv, d.h. $f\in\Lloc(\Omega)$ lässt sich mit der Distribution $T_f\in\D'(\Omega)$ identifizieren und als reguläre Distribution bezeichnen:
    \[ \Lloc(\Omega)\subset\D'(\Omega). \]
    Alle anderen Distributionen heißen singulär. \index{Distribution!singuläre}
  \item \textbf{\underline{\idx{Dirac-Distribution}:}} Sei $x_0\in\R^n$, dann ist
    \[ \<\delta_{x_0},\varphi\>:=\varphi(x_0), \quad\varphi\in\D(\Omega)\subset\D(\R^n) \]
    $\Ra\delta_{x_0}\in\D'(\Omega)$, denn ... (muss noch ausgerechnet werden...)

    \textbf{Anmerkung:} $\exists \, f\in\Lloc(\Omega): T_f=\delta_{x_0}$
    \begin{itemize}
    \item[$\Ra$] $\fa\,  \varphi\in\D(\Omega)\subset\D(\R^n)$ mit $\supp(\varphi)\subset X:=\R^n\setminus\{x_0\}$:
      \[ \<T_f,\varphi\>=\<\delta_{x_0},\varphi\>=\varphi(x_0)=0 \]
    \item[$\Ra$] $f\vert_X=0\, (\text{fast überall})\Ra f=0 \Ra f \neg$ injektiv. \hspace*{\fill}$\sharp$
    \end{itemize}
    Somit ist $\delta_{x_0}$ singulär (speziell: $\Lloc(\Omega)\subsetneq\D'(\Omega)$).

    \textbf{Notation:} $\delta:=\delta_0$ (d.h. $x_0=0$).

  \item $\{\varphi_\epsilon\with\epsilon>0\}$ Mollifier $\Ra\varphi_\epsilon\xrightarrow[\epsilon\ra0]{w^*}\delta$ in $\D'(\R^n)$
    \begin{proof}
      $\varphi_\epsilon\in\Lloc(\R^n)\subset\D'(\R^n)$. Sei $\Psi\in\D(\R^n)$.
      \begin{align*}
        \<\varphi_\epsilon,\Psi\> &\ = \
        \int_{\R^n}\underbrace{\varphi_\epsilon(x)}
        _{=\epsilon^{-n}\varphi\left(\frac x\epsilon\right)}
        \Psi(x)\d x \\
        &\underset{y=\frac x\epsilon}
        =\int_{\R^n}\varphi(y)\Psi(\epsilon y)\d y
        \xrightarrow[\epsilon\ra0]{\scriptsize\text{(Lebesgue)}}
        \Psi(0)\underbrace{\int\limits_{\R^n}\varphi dx}_{=1} \\
        &\ =\ \<\delta,\Psi\>
      \end{align*}
      Damit folgt die Behauptung.
    \end{proof}
  \end{enumerate}
\end{bsp}

\begin{bemdef}
  \begin{enumerate}[(a)]
  \item \textbf{\underline{Multiplikation:}} Es seien $a\in C^\infty(\Omega)$, $\varphi\in\D(\Omega)\Ra a\varphi\in\D(\Omega)$ und $f\in\Lloc\Ra af\in\Lloc(\Omega)$
    \[
    \<af,\varphi\>=\int_\Omega (af)\varphi dx=\int\limits_\Omega f(a\varphi)\d x
    =\<f,a\varphi\>
    \]
    \begin{defi} Sei
      $T\in\D'(\Omega)$, $a\in C^\infty(\Omega)$:
      \[ \<aT,\varphi\>:=\<T, a\varphi\>,\quad\varphi\in\D(\Omega) \]
      $\Ra aT\in\D'(\Omega)$ (punktweise Multiplikation).
    \end{defi}

  \item \textbf{\underline{Differentation:}} Sei $\alpha\in\N^n$, $\varphi\in\D(\Omega)\Ra\partial^\alpha\varphi\in\D(\Omega)$, sei weiter $f\in C^{\abs\alpha}\Ra\partial^\alpha f\in C(\Omega)\subset\Lloc(\Omega)$.
    \begin{align*}
      \<\partial^\alpha f,\varphi\>=&\int_\Omega\partial^\alpha f\varphi\d x
      \underset{\parbox{1.7cm}{\scriptsize\centering\text{part. Int.} \\
          supp$\,(\varphi)\Subset\Omega$}}=(-1)^{\abs\alpha}\int_\Omega f\partial^\alpha\varphi\d x \\
      =&(-1)^{\abs\alpha}\<f,\partial^\alpha\varphi\>
    \end{align*}
    \begin{defi}
     Es sei  $T\in\D'(\Omega)$, $\alpha\in\N^n$, wir definieren
      \begin{align*} \<\partial^\alpha T,\varphi\>&:=(-1)^{\abs\alpha}\<T,\partial^\alpha\varphi\>,\quad\varphi\in\D(\Omega) \\
      & \\
    &  \underset{\scriptsize\text{Übung}}\Ra\partial^\alpha T \in \D'(\Omega).
    \end{align*}
      \textbf{Beachte:} Für $f\in C^{\abs\alpha}(\Omega)$ stimmt $\partial^\alpha T_f$ (distributionelle Ableitung) mit $\partial^\alpha f$ (klassische Ableitung) überein.
    \end{defi}
  \end{enumerate}
\end{bemdef}

\begin{bem}
  \label{bem:3.9}
  \begin{enumerate}[(a)]
  \item $\fa\,  T\in\D'(\Omega)$, $1\leq j,k\leq n: \partial_k\partial_j T=\partial_j\partial_k T$
  \item $\varphi_j\ra\varphi$ in $\D(\Omega)\Ra\;\fa\,\alpha\in\N^n: \partial^\alpha\varphi_j\ra\partial^\alpha\varphi$ in $\D(\Omega)$
  \item $T_j\xrightarrow{w^*}T$ in $\D'(\Omega)\Ra\;\fa\,\alpha\in\N^n:\partial^\alpha T_j\xrightarrow{w^*}\partial^\alpha T$ in $\D'(\Omega)$
    \begin{proof}
      Übung.
    \end{proof}
  \end{enumerate}
\end{bem}

\begin{bsp}
  \begin{enumerate}[(a)]
  \item Es seien $x_0\in\R^n, \varphi\in\D(\Omega),\alpha\in\N^n$. Da ein $f \in \Lloc (\Omega)$ existiert, so dass $T_f = \delta_{x_0}$, gilt:
    \begin{itemize}
    \item $\<\partial^\alpha\delta_{x_0},\varphi\>=(-1)^{\abs\alpha}\<\delta_{x_0},\partial^\alpha\varphi\>=(-1)^{\abs\alpha}\partial^\alpha\varphi(x_0)$
    \item $a\in C^\infty(\R^n):\<a\delta_{x_0},\varphi\>=:\<\delta_{x_0},a\varphi\>=a(x_0)\varphi(x_0)$
    \end{itemize}
  \item Wir betrachten
    \[
    \Theta(x):=
    \begin{cases}
      1&,x\geq0 \\
      0&,x<0
    \end{cases}
    \qquad\text{(Heavyside-Funktion)}
    \]
    $\Ra\Theta\in\Lloc(\R)$ mit $\partial\Theta=\delta$ (im distributionellen Sinn).
    \begin{proof}
      Sei $\varphi\in\D(\R)$:
      \[
      \<\partial\Theta,\varphi\>=-\<\Theta,\varphi'\>
      =-\int_0^\infty\varphi'(x)\d x=\varphi(0)
      =\<\delta,\varphi\>,
      \]
      d.h. $\partial\Theta=\delta$.
    \end{proof}
  \item Es seien $x_0,x_1,\ldots,x_n\in\Omega\subset\R, u\in C^1(\Omega\setminus\{x_0,\ldots,x_n\})$, weiter sei $\partial_{\text{klass}}u\in\Lloc(\Omega)$.
    \begin{align*}
    \Ra \fa\, j\in\{0,\ldots,n\}: u(x_j\pm 0):=\lim_{h\ra 0^+}u(x_j\pm h)  
    \end{align*}
    existieren und es gilt
    \[
    	\partial u=\partial_{\text{klass}}u + \sum\limits_{j=0}^n[u](x_j)\delta_{x_j}
    \]
     mit $[u](x):=u(x+0)-u(x-0)$ (Sprung).

    \begin{proof}
      Sei o.B.d.A. $n=0$ ($n\geq 1$ analog).
 Sei $x_0<x<y$ ($[x_0,y]\subset\Omega$).
      \begin{align*}
        \Ra \! \quad \qquad& u(x)=u(y)-\int_x^y\partial_{\text{klass}} u(z)\d z \\
        \stackrel[x\searrow x_0]
        {\begin{array}{c}
            \scriptsize  \text{Lebesgue} \\
             \scriptsize \partial_{\text{klass}}u\in\Lloc 
            \end{array}}{\Ra} 
        & u(x_0+0)\;\text{existiert, analog } 
        u(x_0-0) \text{ mit } y > x > x_0\\
        \Ra \! \quad \qquad &  u\in\Lloc(\Omega)
      \end{align*}
      
      Sei $\varphi\in\D(\Omega), 0<\epsilon\ll 1$ (mit $(x_0-\epsilon,x_0+\epsilon)\subset\Omega$), dann ist
      \begin{align*}
        \<\partial u,\varphi\>=\quad \, &-\int_\Omega u\varphi'\d x
        =-\int_{\Omega\setminus(x_0-\epsilon,x_0+\epsilon)}u\varphi'\d x
        -\underbrace{\int_{x_0-\epsilon}^{x_0+\epsilon}u\varphi'\d x}
        _{\xrightarrow[\text{Lebesgue}]{\epsilon\ra0}0} \\
        \underset{\scriptsize\text{part. Int.}}=&\int_{\Omega\setminus(x_0-\epsilon,x_0+\epsilon)}
        \partial_{\text{klass}}u\varphi\d x+u(x_0+\epsilon)\varphi(x_0+\epsilon)\\
        & -u(x_0-\epsilon)\varphi(x_0-\epsilon) \\
        \xrightarrow[\scriptsize\text{Lebesgue}]{\epsilon\ra0}&\int_\Omega\partial_{\text{klass}}u\varphi\d x+[u](x_0)\varphi(x_0) \\
        =\quad \,&\<\partial_{\text{klass}}u,\varphi\>+\<[u](x_0)\delta_{x_0},\varphi\>,
      \end{align*}
      woraus die Behauptung für $n=0$ folgt.
    \end{proof}
  \end{enumerate}
\end{bsp}

\begin{bemdef}
  \label{bem:3.11}
  \begin{enumerate}[(a)]
  \item \textbf{\underline{\idx{Translation}:}} Wir definieren $(\tau_af)(x):=f(x-a)$, $x,a\in\R^n, f\in\Lloc(\R^n)$, damit gilt
    \begin{align*}
    \fa\, \varphi\in\D(\R^n):\<\tau_af,\varphi\>=
    &\int_{\R^n}f(x-a)\varphi(x)\d x \\
    =&\int_{\R^n}f(x)\varphi(x+a)\d x=\<f,\tau_{-a}\varphi\>.
    \end{align*}
    \begin{defi}
     Es sei $T\in\D'(\R^n), a\in\R^n$, dann ist
      \[ \<\tau_aT,\varphi\>:=\<T,\tau_{-a}\varphi\>,\quad\varphi\in\D(\R^n) \]
      $\Ra \tau_aT\in\D'(\R^n)$ (Translation).
    \end{defi}

  \item \textbf{\underline{\idx{Spiegelung}:}} Sei $\check f(x):=f(-x), x\in\R^n, f\in\Lloc(\R^n)$
    \[ \Ra\quad\<\check f,\varphi\>=\<f,\check\varphi\>,\quad\varphi\in\D(\R^n) \]
    \begin{defi}
      $T\in\D'(\Omega): \<\check T,\varphi\>:=\<T,\check\varphi\>,  \varphi\in\D(\R^n)$.
    \end{defi}

  \item \textbf{\underline{\idx{Faltung}:}} Es sei $f\in\Lloc(\R^n), \varphi\in\D(\R^n), x\in\R^n$
    \[ \Ra\quad\tau_x\check\varphi=\varphi(x-\cdot)\in\D(\R^n). \]
    \[( f\ast \varphi)(x)=\int_{\R^n}f(y)\varphi(x-y)\d y=\<f,\tau_x\check\varphi\> \]
    \begin{defi}
      Es seien $T\in\D'(\R^n), x\in\R^n, \varphi\in\D(\R^n)$, dann definieren wir
      \[ 
     ( T\ast \varphi)(x):=\<T,\tau_x\check\varphi\>\underset{\scriptsize\text{(a),(b)}}
      =\<(\tau_{-x}T\check),\varphi\>
      \]
    \end{defi}
    Man kann zeigen dass $T\ast\varphi\in C^\infty(\R^n) \, \fa \, T \in \D'(\Omega), \, \fa \, \varphi \in \D(\Omega)$ und $\partial^\alpha(T\ast\varphi)=(\partial^\alpha T)\ast\varphi=T\ast(\partial^\alpha\varphi), \alpha\in\N^n$.
  \item Für $\varphi\in\D(\R^n)$ gilt dann 
    \[
  (  \delta\ast\varphi)(x)=\<\delta,\tau_x\check\varphi\>=(\tau_x\check\varphi)(0)=\varphi(x)\quad\fa \,x\in\R^n,
    \]
    d.h. $\delta\ast\varphi=\varphi$.
  \item Der Träger einer Distribution $T\in\D'(\R^n)$ ist definiert als:
    \index{Distribution!Träger}
    \begin{align*}
      \supp(T):=&\text{ Komplement der größten offenen Teilmenge des $\R^n$,} \\
      &\text{ auf dem $T$ verschwindet.}
  \end{align*}
  Das heißt, es gilt:
  \[
  \<T,\varphi\>=0\quad\fa\,\varphi\in\D(\R^n)\text{ mit }\supp(\varphi)
  \subset(\supp(T))^c.
  \]
  \[
    \E'(\R^n):=\{ T\in\D'(\R^n)\with \supp(T)\Subset\R^n \} 
  \]
      sind die Distributionen mit kompakten Träger.
  \begin{bsp*}
    \begin{itemize}
    \item $\supp(\delta_{x_0})=\{x_0\}\Ra\delta_{x_0}\in\E'(\R^n)$
    \item $\D(\R^n)\subset\E'(\R^n)$
    \end{itemize}
  \end{bsp*}
\item \textbf{\underline{Erweiterung der Faltung auf Distributionen:}}
  \index{Distribution!Faltung}
  Man kann zeigen: \newline
   Für $T\in\E'(\R^n)$ und $S\in\D'(\R^n)$ lässt sich $T\ast S=S\ast T\in\D'(\R^n)$ definieren mit folgenden Eigenschaften:
  \begin{enumerate}[\rm(i)]
  \item $\partial^\alpha(T\ast S)=(\partial^\alpha T)\ast S=T\ast(\partial^\alpha S)$ für alle $\alpha\in\N^n$,
  \item $\delta\ast S=S$,
  \item $\partial^\alpha S=(\partial^\alpha\delta)\ast S, \alpha\in\N^n$,
  \item die so definierte Faltung erweitert die ursprüngliche Faltung zwischen $T \in \D'( \R^n)$ und $\varphi \in \D(\R^n)$,
  \item $\ast$ ist bilinear.
  \end{enumerate}
\end{enumerate}
\end{bemdef}
% -------- 21.04.2011 -------------------------
Wir betrachten die Differentialgleichung 
\[
\sum_{\abs\alpha\leq m}a_\alpha\cdot\partial^\alpha u=f \text{ in } \D'(\R^n).
\]

\begin{defi}
  $K\in\D'(\R^n)$ heißt \idx{Fundamentallösung} von
  \[
  A:=\sum_{\abs\alpha\leq m}a_\alpha\cdot\partial^\alpha
  \]
  mit $a_\alpha\in\C$, falls $AK=\delta$.
\end{defi}

\begin{bem}
  \label{bem:3.12}
  \begin{enumerate}[(a)]
  \item Fundamentallösungen sind im Allgemeinen nicht eindeutig.
    \begin{proof}
      Falls $T\in\D'(\R^n)$ mit $AT=0$ ist, so gilt für eine Fundamentallösung $K\in\D'(\R^n)$:
      \[
      A(T+K)=AK=\delta \, . \qedhere
      \]
    \end{proof} 

    \item \textbf{\idx{Malgrange-Ehrenpreis}:} Jeder Differentialoperator mit konstanten Koeffizienten besitzt eine Fundamentallösung.
  \end{enumerate}
\end{bem}

\begin{theorem}
  \label{theorem:3.13}
  Es sei
  \[ A:=\sum_{\abs\alpha\leq m}a_\alpha\partial^\alpha \]
  mit $a_\alpha\in\C$ und sei $K\in\D'(\R^n)$ eine Fundamentallösung von $A$. Es sei weiter $f\in\D'(\R^n)$ sowie $f\in\E'(\R^n)$ oder $K\in\E'(\R^n)$. Dann gilt:
  \begin{enumerate}[\rm (a)]
  \item $u:=K\ast f\in\D'(\R^n)$ löst $Au=f$ in $\D'(\R^n)$. Diese Lösung wird als "`distributionelle Lösung"' bezeichnet.
  \item $u=K\ast f$ ist die eindeutige Lösung von $Au=f$ in der Klasse aller Distributionen, die mit $K$ faltbar sind.
  \end{enumerate}
\end{theorem}

\begin{proof}
  \begin{enumerate}[(a)]
  \item Wegen Bemerkung~\ref{bem:3.11} ist $u=K\ast f\in\D'(\R^n)$ wohldefiniert. Bemerkung~\ref{bem:3.11} liefert ebenfalls
    \[ Au=A(K\ast f)=(AK)\ast f=\delta\ast f=f. \]
  \item Seien $u=K\ast f$ und $v\in\D'(\R^n)$ faltbar mit $K$ und $Av=f$. Mit $w:=u-v$ erhalten wir
    \[ Aw\underset{A\;\scriptsize\text{linear}}=Au-Av=f-f=0. \]
    Also ist
    \[ w=\delta\ast w=(AK)\ast w=A(K\ast w)=K\ast \underbrace{Aw}_{=0}=0, \]
    damit folgt $u = v$.\qedhere
  \end{enumerate}
\end{proof}

\begin{bsp}[Fundamentallösung des Laplace-Operators für $n=1$]
  \index{Fundamentallösung!Laplace-Operator}
  $(x\mapsto -x\Theta(x))\in\Lloc(\R)$ ist Fundamentallösung von $-\partial^2_x$.

\begin{proof}
  Für alle $\varphi\in\D(\R)$ gilt:
  \begin{align*}
    \<-\partial^2_x(-x\Theta),\varphi\>=&\<x\Theta,\varphi''\> \\
    =&\int_0^\infty x\varphi''(x)\d x=\underbrace{x\varphi'(x)\bigg|_0^\infty}_{=0}-\int_0^\infty\varphi'(x)\d x \\
    =&\varphi(0)=\< \delta,\varphi\>,
  \end{align*}
  also ist $-\partial^2_x (-x \Theta) = \delta$.
\end{proof}
\end{bsp}


%%% Local Variables: 
%%% mode: latex
%%% TeX-master: "Skript"
%%% End: 

\newchapter{Harmonische Funktionen}

Auch in diesem Kaptitel sei wieder $\Omega$ eine nicht-leere Teilmenge des $\R^n$. Wir betrachten die Laplace-Gleichung
\[ -\Delta u=f \]
mit dem Laplace-Operator
\[ -\Delta:=-\sum_{j=1}^n\partial^2_j. \]
Diese Gleichung ist eine elliptische Differentialgleichung. 

\begin{defi}
  \index{harmonische Funktion}
  \index{Funktion!harmonische}
  Eine Funktion $u\in C^2(\Omega)$ heißt harmonisch, falls $-\Delta u=0$ in $\Omega$ erfüllt wird.
\end{defi}

\begin{defi}
  Es sei $\Omega\subset\R^n$ ein Gebiet (d.h.\ offen und zusammenhängend), und $m\in\N\cup\{\infty\}$. $\Omega$ ist ein $C^m$-Gebiet (kurz: $\Omega\in C^m$) genau dann, wenn gilt, dass $\partial\Omega$ eine $(n-1)$-dimensionale $C^m$-\idx{Untermannigfaltigkeit} des $\R^n$ ist. Siehe dazu Abbildung~\ref{fig:4.1}.
  \begin{figure}[ht!]
    \centering
    \begin{pspicture}(-6,-1.5)(4,3)
      \psaxes[ticks=none,labels=none]{->}(0,0)(-.5,-1.5)(3.5,2.5)
      \rput[tl](3.5,-.1){$\R^{n-1}$}
      \rput[br](-.1,2.5){$\R$}

      % Kreis kartesisch
      \pscircle(1.5,0){.5cm}
      \pswedge[fillstyle=solid,fillcolor=lightgray](1.5,0){.5cm}{0}{180}
      \psdot(1.5,0)

      % Omega
      \psccurve(-3,0)(-3,1)(-3,2)(-5,2)(-5.5,1.5)(-4.5,0)
      \pscircle(-3,1){.5cm}
      \pswedge[fillstyle=solid,fillcolor=lightgray](-3,1){.5cm}{80}{280}
      \psdot(-3,1)
      \rput[tr](-4.6,-.1){$\Omega$}

      % Abbildung
      \pscurve{->}(-.3,.5)(-1.5,1.4)(-2.4,1.2)
      \rput[bl](-1.5,1.6){$\varphi\in C^m$}
    \end{pspicture}
    \caption{Untermannigfaltigkeit}
    \label{fig:4.1}
  \end{figure}
\end{defi}

\begin{defi}
  Es sei $u\in C^m(\bar\Omega)$. Genau dann ist $u\in C^m(\Omega)$ und alle Ableitungen der Ordnung $\leq m$ lassen sich auf $\bar\Omega$ stetig fortsetzen.
\end{defi}

\begin{satz}
  \label{satz:4.1}
  Es $\Omega\in C^1$ beschränkt und $\nu$ äußere Einheitsnormale an $\partial\Omega$. Außerdem seien $u,v\in C^2(\Omega)\cap C^1(\bar\Omega)$ und $w\in C^1(\Omega,\R^n)\cap C(\bar\Omega,\R^n)$. Dann gilt:
  \begin{enumerate}[\rm(i)]
  \item \label{satz:4.1-1} \textbf{Gauß:}
    \index{Satz von!Gauß}
    \[
    \int_\Omega \div(w(x))\d x=\int_{\partial\Omega}w(x)\cdot\nu(x)\d\sigma(x)
    \]
  \item \label{satz:4.1-2} \textbf{1. Greensche Formel:}
    \index{Satz von!Green}
    \index{Greensche Formeln}
    \[
    \int_\Omega v\Delta u\d x+\int_\Omega\nabla u\cdot\nabla v\d x
    =\int_{\partial\Omega}v\partial_\nu u\d\sigma(x)
    \]
    mit $\partial_\nu u=\nabla u\cdot\nu$.
  \item \label{satz:4.1-3} \textbf{2. Greensche Formel:}
    \[
    \int_\Omega (u\Delta v-v\Delta u)\d x
    = \int_{\partial\Omega}(u\partial_\nu v-v\partial_\nu u)\d\sigma(x)
    \]
  \end{enumerate}
\end{satz}

\begin{proof}
  \begin{itemize}
  \item[(\ref{satz:4.1-1})] Analysis III.
  \item[(\ref{satz:4.1-2})] Wende (\ref{satz:4.1-1}) auf $w=v\nabla u$ an.
  \item[(\ref{satz:4.1-3})] vertausche $u$ und $v$ in (\ref{satz:4.1-2}). \qedhere
  \end{itemize}
\end{proof}

\begin{bem}
  $u\in C^2(\Omega)\cap C^1(\bar\Omega)$ ist harmonisch und $\Omega$ wie in Satz~\ref{satz:4.1}. Dann ist mit $v=1$
  \[ \int_{\partial\Omega}\partial_\nu u\d\sigma=0. \]
\end{bem}

\section{Fundamentallösung}

Sei $K\in\D'(\R^n)$ eine Fundamentallösung von $-\Delta$, also gelte $-\Delta K=\delta$. Dann ist $u=K\ast f$ für alle $f\in\E'(\R^n)$ eine Lösung von $-\Delta u=f$ in $\R^n$.

Betrachten wir die orthogonale Matrix $A\in\R^{n\times n}$ (d.h. $A^TA=1$). Gilt $-\Delta u(x)=0$ mit $x\in\R^n\setminus\{0\}$, so gilt ebenfalls $-\Delta u(Ax)=0$. Die Lösungen $u$ sind also rotationsinvariant. Nun ist es naheliegend rotationssymmetrische Lösungen, also Lösungen für die $u(x)=\varphi(r)$ mit $r=\abs x$ gilt, zu suchen. Wegen $\pdiff{x_j}r=\frac{x_j}{\abs x}$ gilt dann:
\[  
  \begin{split}
    0\overset!=&\Delta u(x)=\sum_{j=1}^n\partial_j^2\varphi(r)
  =\sum_{j=1}^n\partial_j\left(\varphi'(r)\frac{x_j}r\right) \\
  =&\sum_{j=1}^n\left(
    \varphi''(r)\frac{x_j^2}{r^2}+\varphi'(r)\frac 1r-\varphi'(r)\frac{x_j}{r^2}\frac{x_j}r
  \right) \\
  =&\varphi''(r)+\frac nr\varphi'(r)-\frac 1r\varphi'(r).
  \end{split}
\]
Also ist
\begin{align*}
  &&\varphi''(r)+\frac{n-1}r\varphi'(r)=&0 \\
  \Longleftrightarrow&& \diff r\ln(\varphi'(r)) =& \frac{1-n}r \\
  \Longleftrightarrow&& \ln(\varphi'(r))=&(1-n)\ln r+c\qquad \Longleftrightarrow \varphi'(r)=c_0r^{1-n} \\
  \Longleftrightarrow&& \varphi(r)=&
  \begin{cases}
    c_1r^{2-n}+c_2 &,n>2 \\
    c_1\ln r+c_2 &, n=2
  \end{cases}.
\end{align*}

\begin{lemma}
  \label{lemma:4.3}
  Ist $u\in C^2(\R^n\setminus\{0\})$ harmonisch und radial-symmetrisch, d.h. $-\Delta u=0$ in $\R^n\setminus\{0\}$ und $u(x)=\varphi(r)$ mit $r=\abs x$, so ist $\varphi$ von der Form
  \[
  \varphi(r)=\begin{cases}
    c_1 r^{2-n}+c_2 &,n>2 \\
    c_1\ln r+c_2 &, n=2
  \end{cases}
  \]
mit $r>0$. Umgekehrt ist jedes solches $\varphi$ harmonisch auf $\R^n\setminus\{0\}$.
\end{lemma}

\begin{defi}
  Es sei $r>0$ und $S_r^{n-1}:=\partial\B_{R^n}(0,r)=\{x\in\R^n\with \abs x=r\}$. Wir definieren $S^{n-1}:=S_1^{n-1}$ und
  \[
  \omega_n:=\lambda_{n-1}(S^{n-1})=\vol(S^{n-1})=\frac{2\pi^{\frac n2}}{\Gamma\left(\frac n2\right)}.
  \]
  Dabei ist $\Gamma$ die \idx{Gamma-Funktion}, die durch
  \[
  \Gamma(x):=\int_0^\infty t^{x-1}e^{-t}\d t,\quad x>0
  \]
  definiert ist. Somit ist $\vol(S_r^{n-1})=r^{n-1}\omega_n=r^{n-1}\vol(S^{n-1})$.
\end{defi}

\begin{defi}[\idx{Newton-Potential}]
  Das Newton-Potential ist definiert als
  \[
  \mathcal{N}_n(x):=\mathcal{N}(x):=\begin{cases}
    -\frac 1{2\pi}\ln\abs x &, n=2 \\
    \frac 1{\omega_n(n-2)}\abs x^{2-n} &, n\geq 3
  \end{cases}.
  \]
\end{defi}

\begin{theorem}
  \label{theorem:4.4}
  $\mathcal{N}\in\Lloc(\R^n)$ ist Fundamentallösung von $-\Delta$ auf $\R^n$.
\end{theorem}

\begin{proof}
  $n=2$: Übung

  $n\geq3$: Für alle $R>0$ ist
  \[
  \int_{\B(0,R)}\abs x^{2-n}\d x\underset{\scriptsize\text{Pol-Koord.}}=
  \omega_n\int_0^Rr^{2-n}r^{n-1}\d r.
  \]
  Damit ist $\mathcal{N}\in\Lloc(\R^n)\subset\D'(\R^n)$.

  Sei nun $\varphi\in\D(\R^n)$ mit $\supp(\varphi)\subset\B(0,R)$ und $R>0$. Dann gilt:
  \begin{align}
    \label{eq:4.1}
    \begin{aligned}
      \<\Delta\abs x^{2-n},\varphi\>
    \quad   =\quad\,&\int_{\R^n}\abs x^{2-n}\Delta\varphi(x)\d x \\
      \underset{\scriptsize\text{Lebesgue}}=&
      \lim_{\epsilon\ra 0}\int_{\epsilon\leq\abs x\leq R}\abs x^{2-n}\Delta\varphi(x)\d x\\
      \underset{\scriptsize\text{Green}}=\ \, &   \lim_{\epsilon\ra0}\left\{
        \int_{\epsilon\leq\abs x\leq R}\underbrace{\Delta\abs x^{2-n}}_{=0}\varphi(x)\d x \right. \\
        & +\int_{\abs x=R}\underbrace{(\abs x^{2-n}\partial_\nu\varphi-\varphi\partial_\nu\abs x^{2-n})}_{=0}\d\sigma \\
      &  + \left.\int_{\abs x=\epsilon}(\abs x^{2-n}\partial_\nu\varphi-\varphi\partial_\nu\abs x^{2-n})\d\sigma
      \right\}.
      \end{aligned}
  \end{align}
  Auf $[\abs x=\epsilon]$ ist $\nu(x)=-\frac x{\abs x}=-\frac x\epsilon$ und deshalb
  \[ \partial_\nu\varphi(x)=\nabla\varphi(x)\cdot\nu(x)=-\frac{x\cdot\nabla\varphi(x)}\epsilon,\quad\abs x=\epsilon. \]
  Dann ist
  \[\begin{split}
    \Abs{\,
      \int_{\abs x=\epsilon}\abs x^{2-n}\partial_\nu\varphi\d\sigma
    }\leq&\epsilon^{2-n}\int_{\abs x=\epsilon}\underbrace{\frac{\abs x}\epsilon}_{=1}\abs{\nabla\varphi(x)}\d\sigma \\
    \leq&\epsilon^{2-n}\norm{\nabla \varphi}_\infty\underbrace{\vol(S^{n-1}_\epsilon)}_{=\omega_n\epsilon^{n-1}} \\
    =&c\epsilon\xrightarrow[\epsilon\ra0]{}0.
  \end{split}
  \eq{eq:4.2}
\]
Ferner gilt auf $[\abs x=\epsilon]$:
\[
\partial_\nu\abs x^{2-n}=\nabla\abs x^{2-n}\cdot\nu(x)=\frac{(2-n)x}{\abs x^n}\cdot\frac{-x}{\epsilon}=(n-2)\epsilon^{1-n}.
\]
Damit können wir folgern, dass
  \begin{align}
    \label{eq:4.3}\tag{4.3a}
    \begin{aligned}
    \lim_{\epsilon\ra0}\int_{\abs x=\epsilon}-\varphi\partial_\nu\abs
    x^{2-n}\d\sigma =&\lim_{\epsilon\ra0}(2-n)\epsilon^{1-n}\int_{\abs
      x=\epsilon}\varphi\d\sigma\\
       =&\lim_{\epsilon\ra0}(2-n)\omega_n
    \underbrace{\frac{1}{\vol(S^{n-1}_\epsilon)}\int_{S^{n-1}_\epsilon}\varphi\d\sigma}_{\ra\varphi(0)},
  \end{aligned}
  \end{align}
    denn
  \begin{align}\tag{4.3b}
  \begin{aligned}
    \Abs{\frac{1}{\vol(S^{n-1}_\epsilon)}\int_{S^{n-1}_\epsilon}\varphi\d\sigma-\varphi(0)}
    =&\Abs{\frac{1}{\vol(S^{n-1}_\epsilon)}\int_{S^{n-1}_\epsilon}\varphi(x)-\varphi(0)\d\sigma} \\
    \leq& \max_{\abs x\leq\epsilon}\abs{\varphi(x)-\varphi(0)}\xrightarrow[\epsilon\ra0]{}0.
  \end{aligned}
\end{align}
Aus \eqref{eq:4.1}, \eqref{eq:4.2} und (4.3) folgt dann
\[ \<\Delta\abs x^{2-m},\varphi\>=\<(2-n)\omega_n\delta,\varphi\> \]
für alle $\varphi\in\D(\R)$.
\end{proof}

\begin{kor}[Lösung der inhomogenen Laplace-Gleichung]
  \label{kor:4.5}
  \begin{enumerate}[\rm(a)]
  \item \label{kor:4.5-1} Für $f\in\E'(\R^n)$ löst $\mathcal{N}\ast f\in\D'(\R^n)$ die Laplace-Gleichung 
    \[ -\Delta u=f \] 
    in $\D'(\R^n)$.
  \item \label{kor:4.5-2} Es sei $f\in L_1(\R^n)$ mit
    \[
    \underbrace{\int_{\R^n}\abs{f(y)}\cdot\abs{\log\abs{y}}\d y<\infty\quad\text{falls }n=2.}_{\text{in }\D'(\R^n)}
    \]
    Dann ist $\mathcal{N}\ast f\in\Lloc(\R^n)$ und $-\Delta(\mathcal{N}\ast f)=f$ in $\D'(\R^n)$.
  \end{enumerate}
\end{kor}

\begin{proof}
  \begin{itemize}
  \item[(\ref{kor:4.5-1})] Theorem~\ref{theorem:3.13} (a).
  \item[(\ref{kor:4.5-2})] $n=2$: Übung.
    
    $n\geq3$: Für $R>0$ sei
    \[
    \chi_R(x):=\begin{cases}
      1 &,\abs x<R \\
      0 &,\abs x\geq R
    \end{cases}.
    \]
    Dann folgt aus Theorem~\ref{theorem:4.4}, dass $\chi_R\mathcal{N}\in L_1(\R^n)$ und $(1-\chi_R)\mathcal{N}\in L_\infty(\R^n)$. Damit ist
    \[
      \mathcal{N}\ast f=\underbrace{(\chi_R\mathcal{N})\ast f}_{
        \begin{subarray}{c}
        \in  L_1(\R^n) \\
          \text{Satz~\ref{satz:3.3}~(\ref{satz:3.3-1})}
        \end{subarray}}
      +\underbrace{((1-\chi_R)\mathcal{N})\ast f}_{
        \begin{subarray}{c}
        \in  L_\infty(\R^n)\subset\Lloc(\R^n) \\
          \text{Satz~\ref{satz:3.3}~(\ref{satz:3.3-1})}
        \end{subarray}}
      \in\Lloc(\R^n).
      \]
    Ferner gilt für alle $\varphi\in\D(\R^n)$:
\begin{align*}
	\< -\Delta (N\ast f),\varphi\> \ \ =\ \   & - \int_{\R^n} (\mathcal{N}\ast f)(x) \Delta \varphi (x) \d x \\
	\stackrel{\scriptsize\text{Fubini}}= & - \int_{\R^n} f(y) \int_{\R^n}\mathcal{N}(x-y) \Delta \varphi(x) \d x \d y \\
	= \ \ & \int_{\R^n} \mathcal{N} (x) \Delta \varphi (x+y) \d x = \< \underbrace{\Delta \mathcal{N}}_{-\delta},\tau_{-y}\varphi\> = - \varphi (y) \, . 
\end{align*}
Insgesamt folgt damit
\[
	\< - \Delta (\mathcal{N} \ast f),\varphi\> = \int_{\R^n} f(y) \varphi(y) \d y = \< f,\varphi\> \quad \fa \, \varphi \in \D (\R^n).
\]
  \end{itemize}
\end{proof}

\begin{bem*}
  Korollar~\ref{kor:4.5} (b) spiegelt Theorem~\ref{theorem:3.13} wieder. Allerdings sind weder $\mathcal{N}\in\E'$ noch $f\in\E'$, sodass Theorem~\ref{theorem:3.13} nicht direkt angewendet werden kann.
\end{bem*}

\section{Darstellungsformeln und Folgerungen}

\begin{theorem}[Darstellungsformel]
  \label{theorem:4.6}
  Beschreibe $\Omega\in C^1$ ein Gebiet und sei $u\in C^2(\bar\Omega)$. Dann ist für alle $x\in\Omega$
  \[
  \begin{split}
    u(x)=& \overbrace{-\int_\Omega\Delta
      u(y)\mathcal{N}(x-y)\d y}^{\text{Newton-Potential}}\\
    &+\int_{\partial\Omega}(
    \underbrace{\mathcal{N}(x-y)\partial_\nu
      u(y)}_{\text{Einfachschicht-Potential}} -
    \underbrace{u(y)\partial_\nu\mathcal{N}(x-y)}_{\text{Doppelschicht-Potential}}
    )\d\sigma(y)
  \end{split}
  \]
\end{theorem}

\begin{proof}
  Sei $x\in\R$ beliebig und $\epsilon>0$ mit $\overline{\B}(x,\epsilon)\subset\Omega$. Nach Lemma~\ref{lemma:4.3} ist $\Delta_y\mathcal{N}(x-y)=0$ für $y\in\Omega\setminus\overline{\B}(x,\epsilon)$ beliebig. Wegen der Rotationssymmetrie ist außerdem $\mathcal{N}(x-y)=\mathcal{N}(y-x)$. Mit Hilfe der 2. Greenschen-Formel (Satz~\ref{satz:4.1}~(\ref{satz:4.1-3})) und der Feststellung, dass $\partial(\Omega\setminus\overline{\B}(x,\epsilon))=\partial\Omega\cup\partial\B(x,\epsilon)$, erhalten wir
    \begin{dmath*}
      \int_{\Omega\setminus\bar{\B}(x,\epsilon)}\mathcal{N}(x-y)\Delta u(y)\d y
      = \int_{\partial\Omega}(\mathcal{N}(y-x)\partial_\nu u(y)-u(y)\partial_\nu\mathcal{N}(y-x))\d\sigma(y) 
      +\int_{\partial\B(x,\epsilon)}(\mathcal{N}(y-x)\partial_\nu u(y)-u(y)\partial_\nu\mathcal{N}(y-x))\d\sigma(y).
    \end{dmath*}
    Dabei ist $\Delta u\in C(\bar\Omega)$ und $\mathcal{N}\in\Lloc(\R^n)$. Da $\bar\Omega$ kompakt ist, folgt aus dem Satz von Lebesgue
    \[
    \int_{\Omega\setminus\B(x,\epsilon)}\mathcal{N}(y-x)\Delta u(y)\d y
    \xrightarrow{\epsilon\searrow 0}
    \int_\Omega\mathcal{N}(y-x)\Delta u(y)\d y.
    \]
    Für $\abs{x-y}=\epsilon$ gilt:
    \[
    \mathcal{N}(y-x)=\left\{
      \begin{aligned}
        &c\cdot\epsilon^{2-n}\quad &,n\geq 3 \\
        &c\cdot\log\epsilon \quad&,n=2
      \end{aligned}
      \;
    \right\}
    =\mathcal{N}(\epsilon).
    \]
    Nun ist
    \[
    \Abs{\;
      \int_{\partial\B(x,\epsilon)}\mathcal{N}(y-x)\partial_\nu u(y)\d\sigma(y)
    }\leq \norm{\nabla u}_\infty\mathcal{N}(\epsilon)
    \underbrace{\vol(S^{n-1}_\epsilon)}_{=\omega_n\epsilon^{n-1}}\xrightarrow{\epsilon\searrow0}0.
    \]
    \addtocounter{equation}{1}
    Es ist $\nu(y)=\frac{y-x}\epsilon$ auf $\abs{x-y}=\epsilon$. $\nu(y)$ zeigt also nach Innen. Wir erhalten
    \begin{align}
     \begin{aligned}
      \label{eq:4.4}
     & \int_{\partial\B(x,\epsilon)}u(y)\partial_\nu\mathcal{N}(x-y)\d\sigma(y)
      \underset{\parbox{1.45cm}{\centering \scriptsize
          \text{Def. }$\mathcal{N}$ \\
          $\partial_\nu\mathcal{N}=\nabla\mathcal{N}\cdot\nu$
        }}=
      \frac{\epsilon^{1-n}}{\omega_n}\cdot\int_{\partial\B(x,\epsilon)}u(y)\d\sigma(y) \\
      = \, &\frac{1}{\vol(S^{n-1}_\epsilon)}\int_{\partial\B(x,\epsilon)}u(y)\d\sigma(y) 
      \xrightarrow[
      \begin{subarray}{c}
        \text{vgl. Beweis zu} \\
        \text{Theorem~\ref{theorem:4.4}}
      \end{subarray}
      ]{\epsilon\searrow0}  u(x).
      \end{aligned}
    \end{align}
    Aus \eqref{eq:4.1}-\eqref{eq:4.4} folgt die Behauptung.
\end{proof}

\begin{bem}
  \begin{enumerate}[(a)]
  \item Es sei $\supp(u)\Subset\Omega$. Dann ist
    \[
    u(x)=-\int_\Omega\Delta u(y)\mathcal{N}(x-y)\d y,
    \]
    d.h. $u$ ist ein reines Newton Potential. Ist $u$ harmonisch, so ist
    \[
    	u(x) = \int_{\partial\Omega}(\mathcal{N}(y-x)\partial_\nu u(y)-u(y)\partial_\nu\mathcal{N}(y-x))\d\sigma(y) .
    \]

  \item Es könnte der Eindruck entstehen, dass das Problem
    \begin{align*}
      -\Delta u&=0\quad\text{in } \Omega \\
      u\rvert_{\partial\Omega}&= f \\
      \partial_\nu u\rvert_{\partial\Omega}&=g
    \end{align*}
    durch
    \[
    \tag{$\ast$}
    \label{eq:4.star}
    u(x)=\int_{\partial\Omega}(\mathcal{N}(x-y)g(y)-f(y)\partial_\nu\mathcal{N}(x-y))\d\sigma(y),\quad x\in\Omega
    \]
    gelöst werden kann. Dies ist jedoch nicht richtig! Man kann im Allgemeinen nur eine beliebige Randbedingung vorgeben. Zum Beispiel hat das Dirichlet-Problem
    \[
          \Big\vert \quad -\Delta u=0,\quad u\rvert_{\partial\Omega}=f
    \]
    eine eindeutige Lösung (vgl.\ später). Das heißt, $g$ muss dann gewissen Bedingungen unterliegen.
  \end{enumerate}
\end{bem}

\begin{lemma}
  \label{lemma:4.8}
  Es sei $f\in C(\Omega), x\in\R, r>0$ und $\overline{\B(x,r)}\subset\Omega$. Dann ist
  \[
  \int_{\B(x,r)}f(y)\,dy=\int\limits_0^r\int\limits_{\partial\B(x,s)}f(y)\,d\sigma(y)\d s.
  \]
\end{lemma}

\begin{proof}
  Polarkoordinaten.
\end{proof}

\begin{theorem}[Mittelwerteigenschaft]
  \label{theorem:4.9}
  Sei $u\in C^2(\Omega)$ harmonisch auf $\Omega$, $x\in\Omega, r>0$ und $\overline{\B(x,r)}\subset\Omega$. Dann ist
  \[
  u(x)=\underbrace{
    \frac1{\vol(\partial\B(x,r))}\cdot\int_{\partial\B(x,r)} u(y)\d\sigma(y)
  }_{\text{Sphärisches Mittel}}
  =\underbrace{
    \frac1{\vol(\B(x,r))}\cdot\int_{\B(x,r)}u(y)\d y
  }_{\text{Kugelmittel}}.
  \]
\end{theorem}

\begin{proof}
  Wir wissen, dass
  \begin{align*}
    \vol(\partial\B(x,r))=\omega_nr^{n-1}\\
    \intertext{und}
    \vol(\B(x,r))=\frac1n\omega_nr^n
  \end{align*}
  ist. Theorem~\ref{theorem:4.6} mit $\Omega=\B(x,r)$ liefert uns
  \[
  \eq{eq:4.5}
  u(x)=\int_{\partial\B(x,r)}\mathcal{N}(x-y)\partial_\nu u(y)-u(y)\partial_\nu\mathcal{N}(x-y)\d\sigma(y).
  \]
  Damit erhalten wir
  \[
  \eq{eq:4.6}
  \int_{\partial\B(x,r)}\underbrace{\mathcal{N}(x-y)}_{=\mathcal{N}(r)}\partial_\nu u(y)\d\sigma(y)
  =\mathcal{N}(r)\cdot\int_{\partial\B(x,r)}\partial_\nu u(y)\d\sigma(y)
  \stackrel[\scriptsize\text{Gauß}]{\Delta u=0}=0.
  \]
  Auf $\partial\B(x,r)$ gilt $\nu(y)=\frac{y-x}{\abs{x-y}}$. Damit zeigt $\nu$ nach Außen.

  Für $n>2$ gilt:
  \[
  \eq{eq:4.7}
  \partial_\nu\mathcal{N}(x-y)=-\frac 1{\omega_n}\abs{y-x}^{1-n}=-\frac{r^{1-n}}{\omega_n},\quad y\in\partial\B(x,r).
  \]
  Gleichungen \eqref{eq:4.5}-\eqref{eq:4.7} liefert uns
  \[
  \eq{eq:4.8}
  u(x)=\frac1{\omega_n r^{n-1}}\int_{\partial\B(x,r)}u(y)\d\sigma(y).
  \]
  Schließlich erhalten wir
  \[
  \begin{split}
    \int_{\B(x,r)}u(y)\d y\overset{\scriptsize\text{Lemma~\ref{lemma:4.8}}}=&
    \int_0^r\int_{\partial\B(x,s)}u(y)\d\sigma(y)\d s
    \overset{\scriptsize\eqref{eq:4.8}}=
    \int_0^r\omega_ns^{n-1}u(x)\d s \\
    =\quad\,\, \, &\frac{\omega_nr^n}n u(x),
  \end{split}
  \]
  woraus die Behauptung folgt.
\end{proof}

\begin{satz}
  \label{satz:4.10}
  Sei $u\in C(\Omega)$ und für alle Kugeln $\overline{\B(x,r)}\subset\Omega$ gelte die Mittelwerteigenschaft
  \[
  u(x)=\frac{1}{\omega_nr^{n-1}}\cdot\int_{\partial\B(x,r)}u(y)\d\sigma(y).
  \]
  Dann ist $u\in C^\infty(\R^n)$ und $u$ harmonisch.
\end{satz}

\begin{proof}
  Sei $\varphi\in C_c^\infty(\B(0,1))$ mit $\int_{\B(0,1)}\varphi=1$ und $\varphi(x)=\Psi(\abs x)$ für $\Psi\in C_c^\infty(\R)$ und $\varphi_\epsilon(x)=\epsilon^{-n}\varphi\left(\frac x\epsilon\right)$. Sei außerdem $\Omega_\epsilon:=\{x\in\Omega\with\bar\B(x,\epsilon)\subset\Omega\}$. Mit $x\in\Omega_\epsilon$ ist dann $\supp(\varphi_\epsilon(x-\cdot{}))\Subset\Omega$. Nun ist
\[
\begin{split}
  (u\ast \varphi_\epsilon)(x) = \quad \, \ &\int_{\B(0,\epsilon)}u(x-y) \underbrace{\varphi \left(\frac y\epsilon\right)\epsilon^{-n}}_{= \varphi_\epsilon (y)} \d y 
    \overset{\bar y=\frac y\epsilon}=\int_{\B(0,1)}u(x-\epsilon y)\varphi(y)\d y \\
    \overset{\scriptsize\text{Lemma~\ref{lemma:4.8}}}= &\int_0^1\int_{\partial\B(0,s)}u(x-\epsilon y)\underbrace{\varphi(y)}_{=\Psi(s)}\d\sigma(y)ds \\
    = \quad \, \ &\int_0^1\Psi(s)\underbrace{\int_{\partial \B(0,s)}u(x-\epsilon y)\d\sigma(y)}_{\parbox{4.3cm}{\scriptsize$\overset{z=x-\epsilon y} 
        =\epsilon^{1-n}\int_{\partial\B(x,\epsilon s)}u(z)\d\sigma(z)$ \\
       \text{ }  \text{ }\,$ =\epsilon^{1-n}\omega_n(\epsilon s)^{n-1}u(x)
    $}} \d s \\
    = \quad \, \ &\omega_n\int_0^1\Psi(s)s^{n-1}\d s \, u(x)
\end{split}
\]
die \idx{triviale Fortsezung} von $u$. Es folgt nun aus
\[
\begin{split}
  1&=\int_{\B(0,1)}\varphi(y)\d y\overset{\scriptsize\text{Lemma~\ref{lemma:4.8}}}
  =\int_0^1\int_{\partial\B(0,s)}\underbrace{\varphi(y)}_{=\Psi(s)}\d\sigma(y)ds \\
  &=\int_0^1\Psi(s)\underbrace{\vol(\partial\B(0,s))}_{=\omega_ns^{n-1}}\d s,
\end{split}
\] 
dass $(u\ast\varphi_\epsilon)(x)=u(x)$ mit $x\in\Omega_\epsilon$ und $u\ast\varphi_\epsilon\in C^\infty(\Omega_\epsilon)$ ist. Für $\epsilon>0$ beliebig, ist $u\in C^\infty(\Omega)$.

Wir nehmen nun an, das $u$ nicht harmonisch ist. Somit existieren ein $x\in\Omega$ und ein $r_0>0$ mit $\Delta u(y)>0 \,  \forall \, y\in\B(x,r_0)$. Dann ist
\[
u(x)\overset{\scriptsize\text{MWE}}=\frac 1{\omega_n r^{n-1}}\int_{\partial\B(x,r)}u(y)\d\sigma(y)\overset{y=x+rz}=\frac 1{\omega_n}\int_{\partial\B(0,1)}u(x+rz)\d\sigma(z)
\]
unabhängig von $r\in(0,r_0)$. Differentation nach $r$ liefert uns
\[
\begin{split}
  0&\quad=\quad \frac 1{\omega_n}\int_{\partial\B(0,1)}\nabla u(x+rz)\cdot
  z\d\sigma(z) \\
  &\overset{y=x+rz}= \frac{r^{1-n}}{\omega_n}\int_{\partial\B(x,r)}\nabla
  u(y)\cdot\frac{y-x}{r}\d\sigma(y) \\
  &\ \, \overset{\scriptsize\text{Gauß}}= \ \, \frac{r^{1-n}}{\omega_n}\int_{\partial\B(x,r)}\Delta u(y)\d y>0. \qquad\qquad \sharp
\end{split}
\]
Also ist $u$ harmonisch. Daraus folgt die Behauptung.
\end{proof}

\begin{kor}
  \label{kor:4.11}
  Sei $u\in C^2(\Omega)$ harmonisch. Dann ist $u\in C^\infty(\Omega)$.
\end{kor}

\begin{proof}
  Aus Theorem~\ref{theorem:4.9} und Satz~\ref{satz:4.10} folgt die Behauptung.
\end{proof}

\begin{theorem}[\idx{Maximumprinzip}]
  \label{theorem:4.12}
  Sei $\Omega\subset\R^n$ ein Gebiet und $u\in C^2(\Omega)$ harmonisch und reellwertig.
  \begin{enumerate}[\rm(a)]
  \item Gilt $A:=\sup_{x\in\Omega}u(x)<\infty$, so ist entweder $u(x)<A$ für alle $x\in\Omega$ oder $A\equiv u$
  \item Sei $u\in C(\bar\Omega)$ und $\Omega$ beschränkt. Dann ist $\max_{\bar\Omega}u=\max_{\partial\Omega}u$. Das heißt, Maximum und Minimum werden bei harmonischen Funktionen immer auf dem Rand angenommen.
  \end{enumerate}
\end{theorem}

\begin{proof}
  \begin{enumerate}[(a)]
  \item Sei $\Omega_A=\{x\in\Omega\with u(x)=A\}=u^{-1}[\{A\}]$. Dann ist $\Omega_A$ abgeschlossen in $\Omega$. 

    Wir wählen nun ein $x_0\in\Omega_A$ und ein $r>0$ so, dass für $\bar\B(x_0,r)\subset\Omega$ ist. Aus der Mittelwerteigenschaft folgt $u(y)=A$ für alle $y\in\B(x_0,r)$. Somit ist $\Omega_A$ offen in $\Omega$. Da $\Omega$ zusammenhängend ist, muss $\Omega_A=\Omega$ oder $\Omega_A=\emptyset$ sein. Daraus folgt die Behauptung.
  \item Übung.\qedhere
  \end{enumerate}
\end{proof}

\begin{bem}
  \begin{enumerate}[(a)]
  \item Analoges gilt für das Minimum mit $u\mapsto -u$.
  \item Sei $\Omega$ beschränkt und $u\in C^2(\Omega)\cap C(\bar\Omega)$ eine Lösung von $\Delta u=0$ in $\Omega$ mit $u\rvert_{\partial\Omega}=g$ und $g\geq 0$. Dann ist $u(x)>0$ in $\Omega$, falls $g(x_0)>0$ für $x_0\in\partial\Omega$ ist. 
    \begin{proof}
      Sei $\min_{\bar\Omega}u=\min_{\partial\Omega}u\geq0$. Dann ist $u\geq 0$ in $\bar\Omega$. Wir nehmen an, es existiere ein $x_1\in\Omega$ mit $u(x_1)=0$. Dann wäre aber $u\equiv0$ in $\bar\Omega$ nach dem Maximumsprinzip. $\sharp$
    \end{proof}
  \end{enumerate}
\end{bem}

\begin{kor}[Identitätssatz]
  \label{kor:4.14}
  Sei $\Omega\subset\R^n$ ein beschränktes Gebiet und $u_j\in C^2(\Omega)\cap C(\bar\Omega), j = 1,2,$ mit
  \begin{align*}
    \Delta u_1&=\Delta u_2 \\
    u_1\rvert_{\partial\Omega}&=u_2\rvert_{\partial\Omega}.
  \end{align*}
  Dann ist $u_1\equiv u_2$ in $\bar\Omega$.

  Insbesondere hat $-\Delta u=f$ in $\Omega$ höchstens eine Lösung $u\in C^2(\Omega)\cap C(\bar\Omega)$ mit $u=g$ auf $\partial\Omega$.
\end{kor}

\begin{proof}
  Sei $w:=u_1-u_2$. Dann ist 
  \begin{align*}
    \Delta w=\Delta u_1&-\Delta u_2=0 \\
    w\rvert_{\partial\Omega}=u_1\rvert_{\partial\Omega}&-u_2\rvert_{\partial\Omega} = 0
  \end{align*}
  und $max_{\bar\Omega}w=\max_{\partial\Omega}w=0$. Analog verfahren wir für $-w$ und erhalten $w\equiv 0$ in $\bar\Omega$.
\end{proof}

\begin{satz}[Liouville]
  \label{satz:4.15}
  Sei $u\in C^2(\R^n)$ beschränkt und harmonisch. Dann ist $u\equiv c= \text{const}$.
\end{satz}

\begin{proof}
  Sei $x\in\R^n$ und $R>\abs x$. Dann ist
  \[
  \begin{split}
    \abs{u(x)-u(0)}&\overset{\scriptsize\text{MWE}}=\frac{n}{\omega_nR^n}\Abs{\;\int_{\B(x,R)}u(y)\d y-\int_{\B(0,R)}u(y)\d y} \\
    &\ \ \leq\ \  \frac{n\norm{u}_\infty}{\omega_nR^n}\left(
      \;\int_{\B(x,R)\setminus\B(0,R)}1\d y+\int_{\B(0,R)\setminus\B(x,R)}1\d y
    \right)\\
    &\ \ \leq\ \  \frac{n\norm{u}_\infty}{\omega_nR^n} \int_{R-\abs x< y<R+\abs x}1\d y
    =\frac{n\norm{u}_\infty\omega_n}{\omega_nR^n} \int_{R-\abs x}^{R+\abs x}r^{n-1}\d r \\
    &\ \ =\ \ \frac{\norm u_\infty}{R^n}((R+\abs x)^n-(R-\abs x)^n)\xrightarrow{R\ra\infty}0.
  \end{split}
  \]
  Also ist $u(x)=u(0)$ und somit konstant.
\end{proof}

\begin{theorem}[Harnacksche Ungleichung]
  \label{theorem:4.16}
  Für alle Teilgebiete $\Omega'\Subset\Omega$ existiert ein $c=c(\Omega',\Omega)>0$, für das gilt:
  \[
  \sup_{\Omega'}u\leq c\inf_{\Omega'}u
  \]
  für alle harmonischen $u\in C^2(\Omega)$ mit $u\geq0$.
\end{theorem}

\begin{proof}
  \begin{enumerate}[(i)]
  \item \label{proof:4.16-1} Sei $\Omega'=\B(x_0,r)\subset\B(x_0,4r)\subset\Omega$. Seien $y_1,y_2\in\B(x_0,r)$. Dann ist $\B(y_1,r)\subset\B(y_2,3r)$. Sei $u\in C^2(\Omega)$ harmonisch mit $u\geq0$. Aus der Mittelwerteigenschaft folgt
    \[
    \begin{split}
      u(y_1)&\ \, =\ \, \frac{n}{\omega_nr^n}\cdot\int_{\B(y_1,r)}u(y)\d y\overset{u\geq0}
      \leq\frac{n3^n}{\omega_n(3r)^n}\cdot\int_{\B(y_2,3r)}u(y)\d y \\
      &\overset{\scriptsize\text{MWE}}=3^n\cdot u(y_2).
    \end{split}
    \]
    Also ist
    \[\sup_{\B(x_0,r)}u\leq 3^n\inf_{\B(x_0,r)}u.\]
  \item Sei $\Omega'\Subset\Omega$ ein beliebiges Gebiet. Dann existiert ein $r>0$ mit $r<\frac 14\dist(\Omega',\partial\Omega)$. Da $\bar\Omega'$ kompakt ist, gilt für $x_1,\ldots,x_m\in\bar\Omega'$, dass
    \[
    \bar\Omega'\subset\bigcup_{j=1}^m\B(x_j,r).
    \]
    Da $\Omega'$ zusammenhängend ist, können $y_1$ und $y_2$ durch einen Weg über $m$ Punkte in $\Omega'$ verbunden werden. Deshalb ist mit Hilfe von (\ref{proof:4.16-1}) $u(y_1)\leq 3^{nm}u(y_2)$. Daraus folgt die Behauptung.\qedhere
  \end{enumerate}
\end{proof}

\begin{bem}
  Das Maximumprinzip und die Harnacksche Ungleichung gelten für allgemeine elliptische Differentialoperatoren $L$ der Form
  \[
  Lu=\sum_{j,k=1}^na_{jk}\partial_j\partial_ku+\sum_{i=1}^nb_i\partial_iu
  \]
  mit $\sum_{i,k}a_{ik}\xi_i\xi_k\geq\alpha\abs\xi^2$ für alle $\xi\in\R^n$.
\end{bem}



%%% Local Variables: 
%%% mode: latex
%%% TeX-master: "Skript"
%%% End: 

% ------- 03.05.2011 ------------------------------
\newchapter{Dirichletproblem für die Laplace-Gleichung}

Wir treffen wieder eine Generalvoraussetzung: $\emptyset\neq\Omega\subset\R^n$ sei ein $C^\infty$-Gebiet.

Dann betrachten wir das Problem
\[
\tag{DP}
\label{eq:5.DP}
\left\lvert\quad
\begin{split}
  -\Delta u=f &&\text{in }\Omega \\
  u=g && \text{auf }\partial\Omega
\end{split}
\right.
\]
wobei $f:\Omega\ra\K$ und $g:\partial\Omega\ra\K$ gegeben sind.

%\begin{figure}[ht!]
%  \centering
%  
%  \caption{Dirichletproblem}
%\end{figure}

\begin{bem}
  \begin{enumerate}[(a)]
  \item Ist $\Omega$ beschränkt, so hat \eqref{eq:5.DP} nach Korollar~\ref{kor:4.14} höchstens eine Lösung $u\in C^2(\Omega)\cap C(\bar\Omega)$. Jedoch bestitzt \eqref{eq:5.DP} nicht immer eine (klassische) Lösung.
  \item         \begin{figure}[h!]
      \centering
      \begin{pspicture}(-2,-0.6)(2,1.3)
      		\psccurve(-2,0)(-1,0.7)(0.8,1.5)(2,-0.5)(0,-0.5)
		\rput(-0.2,0.4){$\Omega$}
		\rput(-1.8,0.8){$\partial\Omega$}
		\psdot[dotsize=3pt](1.725,0.79)
		\psline{->}(1.725,0.79)(2.37,1.25)
		\rput(2.45,0.75){$\nu(x)$}
		\rput(1.55,0.6){$x$}
      \end{pspicture}
      \caption{Neumannproblem}
 \end{figure}
 Das Neumannproblem für die Laplace-Gleichung
    \[
    \label{eq:5.NP}
    \tag{NP}
    \left\lvert\quad
    \begin{split}
      -\Delta u&=f &&\text{in }\Omega \\
      \partial_\nu u&=g &&\text{auf }\partial\Omega
    \end{split}
    \right.
    \]
    hat entweder keine oder unendlich viele Lösungen, denn ist $u$ Lösung, dann ist auch $u+c$ Lösung für alle $c\in\K$. Ist $\Omega$ beschränkt und $u\in C^2(\bar\Omega)$ eine Lösung, so ist
    \[
    \int_\Omega f\d x=-\int_\Omega\Delta u\d x\overset{\scriptsize\text{Gauß}}=-\int_{\partial\Omega}\partial_\nu u\d\sigma(x)=-\int_{\partial\Omega}g\d\sigma(x).
    \]
   \item Abstrakte Lösungen für \eqref{eq:5.DP} und \eqref{eq:5.NP} folgen in Kapitel 7.
  \end{enumerate}
\end{bem}

\begin{bem}
  \label{bem:5.2}
  \begin{enumerate}[(a)]
  \item Um \eqref{eq:5.DP} zu lösen, genügt es die beiden Teilprobleme
    \begin{align}
      \tag{A}\label{eq:5.A}
      -\Delta v&=f &&\text{in }\Omega, v\rvert_{\partial\Omega}=0 \\
      \tag{B}\label{eq:5.B}
      -\Delta w&=0 &&\text{in }\Omega, w\rvert_{\partial\Omega}=g
    \end{align}
    zu lösen. Dann ist $u:=v+w$.
  \item Die Probleme \eqref{eq:5.A} und \eqref{eq:5.B} sind im wesentlichen äquivalent. Exemplarisch gilt:
    \begin{itemize}
    \item Wenn \eqref{eq:5.A} lösbar und $\tilde g\in C^2(\bar\Omega)$ existiert mit $\tilde g\rvert_{\partial\Omega}=g$, so löse $-\Delta v=-\Delta\tilde g$ mit $v\rvert_{\partial\Omega}=0$. Setze $w:=\tilde g-v$. Dann ist $-\Delta w=0$ in $\Omega$ mit $w\rvert_{\partial\Omega}=g$.
    \item Wenn \eqref{eq:5.B} lösbar und $\bar\Omega$ kompakt ist, so wähle für $f\in C(\bar\Omega)$ eine Forsetzung $\tilde f\in C_c(\R^n)$. Dann erfüllt nach Korollar~\ref{kor:4.5} $\tilde v:=\mathcal{N}\ast\tilde f\in C(\R^n)$ die Gleichung $-\Delta\tilde v=\tilde f$. Lösen wir nun $-\Delta w=0$ in $\Omega$ mit $w\rvert_{\partial\Omega}=\tilde v\rvert_{\partial\Omega}$ und setzen $v:=\tilde v-w$, so erhalten wir
      \[
      -\Delta v=f\quad\text{in }\Omega\, ,\qquad v\rvert_{\partial\Omega}=0\, .
      \]
    \end{itemize}
  \end{enumerate}
\end{bem}

Im Folgenden ist $\mathcal{N}$ immer das \idx{Newtonpotential}.

\begin{defi}
  Eine Funktion $G:\Omega\times\bar\Omega\ra\R$ heißt \idx{Greensche Funktion} für $\Omega$, falls
  \begin{enumerate}[(i)]
  \item Für alle $x\in\Omega$ ist $G(x,\cdot)-\mathcal{N}(x-\cdot)\in C(\bar\Omega)\cap C^2(\Omega)$ harmonisch.
  \item $G(x,y)=0$ für alle $x\in\Omega$ und $y\in\partial\Omega$.
  \end{enumerate}
\end{defi}

\begin{bem}
  \label{bem:5.3}
  \begin{enumerate}[(a)]
  \item \label{bem:5.3-1} Es ist $\mathcal{N}(x-\cdot)\in C^\infty(\R^n\setminus\{0\})$. Für alle $x\in\Omega$ sei $\Psi(x,\cdot):=G(x,\cdot)-\mathcal{N}(x-\cdot)\in C^2(\Omega)\cap C(\bar\Omega)$. Die Funktion $\Psi$ löst
    \[
    \left\lvert\quad
    \begin{split}
      -\Delta_y\Psi(x,\cdot)&=0&&\text{in }\Omega \\
      \Psi(x,\cdot)&=-\mathcal{N}(x-\cdot) && \text{auf } \partial\Omega.
    \end{split}
    \right.
    \]
  \item \label{bem:5.3-2} Für alle $x\in\Omega$ per Definition und Lemma~\ref{lemma:4.3} $-\Delta_yG(x,y)=0$ für alle $y\in\Omega\setminus\{x\}$, und $G(x,\cdot)$ hat an der Stelle $y=x$ dieselbe Singularität wie $\mathcal{N}(x-\cdot)$.
  \item  \label{bem:5.3-3} Ist $\Omega$ beschränkt, so gibt es höchstens eine Greensche Funktion.
    \begin{proof}
      (a) und Korollar~\ref{kor:4.14} (Identitätssatz).
    \end{proof}
  \item \label{bem:5.3-4} Falls $\Omega$ nicht beschränkt ist, so ist (c) im Allgemeinen falsch (z.B. bei $\Omega=\R^{n-1}\times(0,\infty)$.
  \item \label{bem:5.3-5} Wenn $\Omega$ beschränkt (und $C^\infty$) ist, so existiert eine (eindeutige) Greensche Funktion (ohne Beweis). Jedoch ist es oft schwierig, diese Greensche Funktion zu finden. Ausnahme ist $\Omega=\B(0,r)$.
  \end{enumerate}
\end{bem}

\begin{satz}[Symmetrie]
  \label{satz:5.4}
  Sei $G$ eine Greensche Funtkion für $\Omega$. Dann ist
  \[ G(x,y)=G(y,x) \]
  für alle $x,y\in\Omega$.
\end{satz}

\begin{proof}
  Seien $x_1,x_2\in\Omega$ mit $x_1\neq x_2$ und $B_j:=\B(x_j,\epsilon)$ mit $\epsilon>0$ und $B_1\cap B_2=\emptyset$. Weiter sei $\Omega'=\Omega\setminus(B_1\cup B_2)$. Dann ist $G(x_j,\cdot)\in C(\bar\Omega')\cap C^2(\Omega')$ und $-\Delta_yG(x_j,y)=0$ mit $y\in\Omega'$. Satz~\ref{satz:4.1} liefert uns 
  \begin{align}
    \label{eq:5.1}
    \begin{aligned}
    &0=\int_{\partial\Omega'}\Big(
      \underbrace{G(x_1,y)}_{=0,\;y\in\partial\Omega}\partial_{\nu(y)}G(x_2,y)
      -\underbrace{G(x_2,y)}_{=0,\;y\in\partial\Omega}\partial_{\nu(y)}G(x_1,y)
    \Big)\d\sigma(y) \\
    =&\left(
      \int\limits_{\partial B_1}+\int\limits_{\partial B_2}
    \right) \!
    (
      G(x_1,y)\partial_{\nu(y)}G(x_2,y)-G(x_2,y)\partial_{\nu(y)}G(x_1,y)
   )\!\d\sigma(y).
    \end{aligned}
  \end{align}
Dabei ist $G(x_2,\cdot)\in C^1(\B(x_1,\epsilon))$ und aus Bemerkung~\ref{bem:5.3}~(\ref{bem:5.3-2}) folgt für $n\geq3$, dass $\abs{G(x_1,y)}=\epsilon^{2-n}$ ist. Der Fall $n=2$ ist Übung.

Dann ist
\begin{dmath}
  \label{eq:5.2}
  \Abs{\; 
    \int_{\partial B_1}G(x_1,y)\partial_\nu G(x_2,y)\d\sigma(y)
  }\leq
  c\epsilon^{2-n}\overbrace{\max_{y\in\B(x_1,\epsilon_0)}\abs{\nabla_yG(x_2,y)}}^{\leq
    c} \underbrace{\vol(\partial B_1)}_{=\omega_n\epsilon^{n-1}}
  \leq c\epsilon\xrightarrow{\epsilon\ra0}0 \, .
\end{dmath}
Weiter folgt
\begin{dmath}
  \label{eq:5.3}
 \ \ \quad \lim_{\epsilon\ra0}-\int_{\partial\B_1}G(x_2,y)\partial_\nu\underbrace{G(x_1,y)}_{\parbox{1.2cm}{\scriptsize$
    G(x_1,y)-\mathcal{N}(x_1,y)$ \\$+\mathcal{N}(x_1,y)
 $}} \d\sigma(y)
  =\lim_{\epsilon\ra0}-\int_{\partial\B_1}G(x_2,y)\partial_\nu\mathcal{N}(x_1,y)\d\sigma(y)
  =\lim_{\epsilon\ra0}-\frac{1}{\vol(\B(x_1,\epsilon))}\int_{\partial\B(x_1,\epsilon)}G(x_2,y)\d\sigma(y)
  =-G(x_2,x_1)\, .
\end{dmath}
Analog gelten \eqref{eq:5.2} und \eqref{eq:5.3} auch für das Integral über $\B_2$. Schließlich folgt mit \eqref{eq:5.1}
\[
0=-G(x_2,x_1)+G(x_1,x_2) \, .\qedhere
\]
\end{proof}

\begin{bem}
  \label{bem:5.5}
  \begin{enumerate}[(a)]
  \item Aufgrund von Satz~\ref{satz:5.4} definiert man $G(x,y):=0$, falls $x\in\partial\Omega$ oder $y\in\partial\Omega$. Damit ist $G:\bar\Omega\times\bar\Omega\ra\R$ symmetrisch.
  \item $G(\cdot,y)-\mathcal{N}(\cdot-y)\in C^2(\Omega)\cap C(\bar\Omega)$ ist harmonisch.
  \end{enumerate}
\end{bem}

\begin{lemma}
  \label{lemma:5.6}
  Seien $\Omega$ und $\tilde\Omega$ zwei beschränkte $C^\infty$-Gebiete mit entsprechenden Greenschen Funktionen $G$ und $\tilde G$. Ferner sei $\tilde\Omega\subset\Omega$. Dann ist
  \[ 0\leq \tilde G(x,y)\leq G(x,y) \]
  für $x,y\in\tilde\Omega$.
\end{lemma}

\begin{proof}
  Sei $y_0\in\Omega$. Dann ist $\lim_{x\ra y_0}\mathcal{N}(x-y_0)=\infty$ und $G(\cdot,y_0)-\mathcal{N}(\cdot-y_0)\in C(\bar\Omega)$. Es existiert ein $\epsilon>0$ mit 
  \[ G(x,y_0)>0\quad\text{für alle }x\in\B(y_0,\epsilon)\, . \tag{$\ast$} \]
  Wir definieren $\omega:=\Omega\setminus\bar\B(y_0,\epsilon)$. Dann ist per Definition ($\ast$) $G(\cdot,y_0)\geq0$ auf $\partial\omega$ und $-\Delta_xG(\cdot,y_0)=0$ in $\omega$ (Bemerkung \ref{bem:5.5} (b)). Das Maximumprinzip (Theorem~\ref{theorem:4.12}) liefert $G(\cdot,y_0)\geq0$ in $\omega$ und damit $G(x,y_0)\geq0$ für alle $x,y_0\in\Omega$.
  
  Sei nun 
  \[ \Phi(x,y):=G(x,y)-\tilde G(x,y)=(G(x,y)-\mathcal{N}(x-y))-(\tilde G(x,y)-\mathcal{N}(x-y)) \]
  für $x,y\in\tilde\Omega$. Wir halten $x\in\tilde\Omega$ fest. Dann ist $\Phi(x, \cdot)$ harmonisch in $\tilde\Omega$. Für alle $y\in\partial\tilde\Omega$ gilt 
  \[
\Phi(x,y)=\underbrace{G(x,y)}_{\geq0}-\underbrace{\tilde G(x,y)}_{=0}\geq0 \, .
\]
Das Maximum-Prinzip liefert dann
\[
\Phi(x,\cdot)\geq0\quad\text{in}\;\tilde\Omega \, .\qedhere
\]
\end{proof}

\begin{theorem}[Lösung des Dirichletproblems]
  \label{theorem:5.7}
  Sei $\Omega\subset\R^n$ ein beschränktes Gebiet mit Greenscher Funktion $G$. Für $f\in C(\bar\Omega)$ setze 
  \[ v(x):=\int_\Omega G(x,y)f(y)\d y \]
  für $x\in\Omega$. Dann ist $v\in C^1(\bar\Omega)$. 

  Gilt sogar $f\in C^\alpha(\bar\Omega)$ mit $\alpha\in (0,1)$, so ist $v\in C^2(\Omega)\cap C^1(\bar\Omega)$ die eindeutige klassische Lösung von
  \[
  -\Delta v=f\text{ in }\Omega\, ,\quad v\rvert_{\partial\Omega}=0\, .
  \]
\end{theorem}

\begin{proof}
  Di Benedetto, Gilbarg-Trudinger.
\end{proof}

\begin{theorem}[\idx{Poisson-Formel}]
  \label{theorem:5.8}
  Sei $\Omega$ ein $C^\infty$-Gebiet mit Greenscher Funktion $G$. Für $g\in C(\partial\Omega)$ setzen wir
  \[
  w(x):=-\int_{\partial\Omega}\partial_{\nu(y)}G(x,y)g(y)\d\sigma(y)
  \]
  für $x\in\Omega$. Dann löst $w\in C^2(\Omega)\cap C(\bar\Omega)$ das Problem
  \[
    -\Delta w=0\quad\text{in }\Omega \, , \quad w\rvert_{\partial\Omega}=g \, .
  \]
  Der Ausdruck $\partial_{\nu(y)}G(x,y)$ heißt \idx{Poisson-Kern}.
\end{theorem}

\begin{proof}
 Di Benedetto, Gilbarg-Trudinger.
\end{proof}

% ---------------- 05.05.2011 --------------------

\section{Dirichletproblem auf einer Kugel}

Im folgenden Abschnitt sei immer $R>0$ und $\Omega:=\B(0,R)$. Wir definieren für $y\in\R^n$ die Ausdrücke
\begin{align*}
  \bar y&:=\frac {R^2} {\abs{y}^2}y\, ,  \quad y\neq 0 \, ,\\
  \bar y&:=\infty\, ,\qquad \ y=0\, .
\end{align*}

\begin{figure}[ht!]
  \centering
  \begin{pspicture}(-4,-2.2)(4,3)
    \pscircle(0,0){2cm}
    \psdot(0,0)
    \psline(0,0)(2.5,2.5)
    \psdot(.8,.8)
    \psdot(2,2)
    \rput[tl](.9,.7){$y$}
    \rput[tl](2.1,1.9){$\bar y$}
    \rput[tl](1.5,-1.4){$\B(0,R)$}
  \end{pspicture}
  \caption{Kugel im $\R^n$ mit $y$ und $\bar y$ für $R^2 > \abs{y}^2$.}
\end{figure}
\begin{satz}
  \label{satz:5.9} Sei $n\geq 3$ und
  \[
  G_R(x,y)=\mathcal{N}(x-y)-\left(
    \frac{R}{\abs{y}}
  \right)^{n-2}\mathcal{N}(x-\bar y)\, ,\qquad x,y\in\B(0,R)\, .
  \]
  Dann ist $G_R$ die $($eindeutige$)$ Greensche Funktion für $\Omega=\B(0,R)$.
\end{satz}

\begin{proof}
  Für $x,y\in\R^n$ gilt:
  \begin{align*}
    \abs{y}^2\abs{x-\bar y}^2&=\abs{y}^2(\abs x^2-2\frac{R^2}{\abs y^2}xy+\frac{R^4}{\abs y^4}\abs y^2) \\
    &=\abs y^2\abs x^2-2R^2xy+R^4,
    \intertext{also}
    \eq{eq:5.9-1}
    &\abs y\abs{x-\bar y}=\abs x\abs{y-\bar x}\, .
  \end{align*}
  Wir definieren
  
  \begin{align}
  \begin{aligned}
    \label{eq:5.9-2}
    \Psi(x,y)&:=G_R(x,y)-\mathcal{N}(x-y) \\
    &\ =-\left(\frac R {\abs
        y}\right)^{n-2}
    \underbrace{\mathcal{N}(x-\bar y)}_{\underset{\text{Def.}} =c\abs{x-\bar y}^{2-n}} \\
   & \overset{\scriptsize\eqref{eq:5.9-1}}=-\left(\frac R{\abs x}\right)^{n-2}\mathcal{N}(y-\bar x)\, .
  \end{aligned}
  \end{align}
  Sei weiter $x\in\B(0,R)$. Dann ist $\abs{\bar x}>R$ und wegen Gleichung \eqref{eq:5.9-2} ist $\Psi(x,\cdot)\in C^2(\bar\B(0,R))$ mit $\Delta_y\Psi(x,y)=0$ und $y\in\B(0,R)$. 

  Sei nun $y\in\partial\B(0,R)$. Dann ist $\abs y=R$ und $\bar y=y$. Aus der Definition von $G_R$ folgt, dass $\Psi(x,y)=0$ ist für alle $x\in\B(0,R)$ und $y\in\partial\B(0,R)$.
\end{proof}

\begin{bem}
  \label{bem:5.10} Für $n=2$ ist die Greensche Funktion für $\Omega=\B(0,R)$ definiert durch
  \[
  G_R(x,y)=
  \begin{cases}
    \frac 1{2\pi}\log\abs y &,x=0 \\
    \frac 1{2\pi}\left(
      \log\abs{x-y}-\log\Abs{\frac R{\abs s}x-\frac{\abs x}Ry}
    \right) &,x\neq 0
  \end{cases}.
  \]
\end{bem}

\begin{proof}
  Übung.
\end{proof}

\begin{satz}
  \label{satz:5.11} Sei $n\geq2$. Dann gilt für den Poissonkern auf $\B(0,R)$:
  \[
  P(x,y):=\partial_{\nu(y)}G_R(x,y)=\frac{\abs x^2-R^2}{\omega_nR}\cdot\frac 1{\abs{x-y}^n}
  \]
  für $x\in\B(0,R)$ und $y\in\partial\B(0,R)$.
\end{satz}

\begin{proof}
  $n=2$: Übung.

  $n\geq3$: Seien $x\in\B(0,R)$ und $y\in\partial\B(0,R)$. Dann ist $\nu(y)=\frac yR$ und $\bar y=y$.

  Nun ist
  \[
  \begin{split}
  &  \partial_{\nu(y)}G_R(x,y)=\frac yR\cdot\nabla_yG_R(x,y) \\
    =&\frac yR\nabla_y\left(
      \frac 1{\omega_n(n-2)}\abs{x-y}^{2-n}-\left(
        \frac R{\abs x}
      \right)^{n-2}
      \frac 1{\omega_n(n-2)}\abs{y-\bar x}^{2-n}
    \right) \\
    =&\frac 1{\omega_nR}y\left(
      \frac{x-y}{\abs{x-y}^n}+\left(
        \frac R{\abs x}
      \right)^{n-2}
    \right)
    \cdot \frac{y-\bar x}{\abs{y-\abs x}^n} \\
    =&\frac 1{\omega_nR}\cdot\frac 1{\abs{x-y}^n}\underbrace{y\cdot\left(
        x-y+\frac{\abs x^2}{R^2}(x-\bar x)
      \right)}_{\underset{\abs y=R}=\abs x^2-R^2} \\
    =&\frac{\abs x^2-R^2}{\omega_nR\abs{x-y}^n}. \qedhere
  \end{split}
  \]
\end{proof}

\begin{theorem}[Laplace-Dirichlet-Problem auf $\Omega=\B(0,R)$]
  \label{theorem:5.12} Seien $n\geq2, R>0$ und $\Omega=\B(0,R)$.
  \begin{enumerate}[\rm (a)]
  \item Für $\alpha\in(0,1)$ und $f\in C^\alpha(\bar\B(0,R))$ setze
    \[ v(x):=\int_{\B(0,R)}G_R(x,y)f(y)\d y \]
    für $x\in\B(0,R)$, wobei $G_R$ wie in Satz~\ref{satz:5.9} bzw. Bemerkung~\ref{bem:5.10}. Dann löst $v\in C^2(\B(0,R))\cap C(\bar\B(0,R))$ die Gleichung $-\Delta v=f$ in $\B(0,R)$ mit $v\rvert_{\partial\B(0,R)}=0$.
  \item Für $g\in C(\partial\B(0,R))$ sei
    \[ w(x):=\frac{R^2-\abs x^2}{\omega_nR}\int_{\partial\B(0,R)}\frac{g(y)}{\abs{x-y}^n}\d \sigma(y) \]
      für $x\in\B(0,R)$. Dann löst $w\in C^2(\B(0,R))\cap C(\bar\B(0,R))$ die Gleichung $-\Delta w=0$ in $\B(0,R)$ mit $w\rvert_{\partial\B(0,R)}=g$.
  \end{enumerate}  
\end{theorem}

\begin{proof}
  \begin{enumerate}[(a)]
  \item Theorem~\ref{theorem:5.7}, Satz~\ref{satz:5.9}, Bemerkung~\ref{bem:5.10}.
  \item Theorem~\ref{theorem:5.8}, Satz~\ref{satz:5.11}.\qedhere
  \end{enumerate}
\end{proof}

\begin{bem}
  \label{bem:5.13}
  \begin{enumerate}[(a)]
  \item $v$ und $w$ sind eindeutig. (vgl. Korollar~\ref{kor:4.14})
  \item Da $w$ harmonisch ist, ist $w\in C^\infty(\B(0,R))$. (vgl. Korollar~\ref{kor:4.11})
  \item $u:=v+w$ löst $-\Delta u=f$ in $\B(0,R)$ mit $u\rvert_{\partial\B(0,R)}=g$.
  \end{enumerate}
\end{bem}


%%% Local Variables: 
%%% mode: latex
%%% TeX-master: "Skript"
%%% End: 

\newchapter{Sobolevräume}

Auch in diesem Kapitel sei $\emptyset\neq\Omega\subset\R^n$ offen.

\begin{defi}
  Seien $1\leq p\leq\infty$ und $m\in\N$. Die Menge
  \[
  W_p^m(\Omega):=\left(
    \{u\in L_p(\Omega)\with\partial^\alpha u\in L_p(\Omega)\,\fa\,\abs a\leq m\}
    , \norm \cdot_{W_p^m}
  \right)
  \]
  heißt \idx{Sobolevraum} der Ordnung $m$. Dabei ist 
  \[
  \norm u_{W_p^m}:=\norm u_{W_p^m(\Omega)}:=
  \left(
    \sum_{\abs\alpha\leq m}\norm{\partial^\alpha u}_{L_p}^p
  \right)^{\frac 1p},
  \]
  wenn $1\leq p<\infty$. Im Fall $p=\infty$ ist $\norm u_{W_p^m}:=\max_{\abs\alpha\leq m}\norm{\partial^\alpha u}_\infty$.
\end{defi}

\begin{bem}
  \label{bem:6.1}
  \begin{enumerate}[\rm(a)]
  \item Es ist $W_p^0(\Omega)=L_p(\Omega)$.
  \item Seien $u\in W_p^m(\Omega)$, $\varphi\in\D(\Omega)$ und $\alpha\in\N^n$ mit $\abs\alpha\leq m$. Dann gilt:
    \[
    \begin{split}
      \int_\Omega\partial^\alpha u\cdot\varphi\d x
      &=\<\partial^\alpha u,\varphi\>_{\D(\Omega)}
      =(-1)^{\abs\alpha}\<u,\partial^\alpha\varphi\>_{\D(\Omega)} \\
      &=(-1)^{\abs\alpha}\int_\Omega u\cdot\partial^\alpha\varphi\d x\, .
    \end{split}
    \]
    $v = \partial^\alpha u$ bezeichnen wir dann als schwache Ableitung von $u$.
  \end{enumerate}
\end{bem}

\begin{defi}
  Seien $E,F$ normierte Vektorräume.
  \begin{enumerate}[\rm(a)]
  \item Dann ist $T\in\L(E,F)$ genau dann, wenn $T:E\ra F$
    linear und stetig ist und wir definieren
    \[
    \norm T_{\mathcal{L}(E,F)}:=\sup_{\norm e_E=1}\norm{T(e)}_F=\sup_{
      \begin{subarray}{c}
        e\in E \\
        e\neq 0
      \end{subarray}}\frac{\norm{T(e)}_F}{\norm e_E}\, .
    \]
    Damit ist $\norm{T(e)}_F\leq\norm T_{\mathcal{L}(E,F)}\cdot\norm
    e_E$, d.h. $T$ ist  ein beschränkter Operator.
  \item Wir schreiben $E\hookrightarrow F$, wenn $E$ \idx{Untervektorraum} von $F$ ist und $i:[e\mapsto e]\in\L(E,F)$ (d.h.\ es existiert ein $c>0$, so dass $\norm e_F\leq c\cdot\norm e_E$ für alle $e\in E$).
  \item Wir schreiben $E\xhookrightarrow{d}F$ für $E\hookrightarrow F$ und $E \overset d\subset F$.
  \item $E\hhookrightarrow F$ bedeutet $E\hookrightarrow F$ und beschränkte Mengen in $E$ sind relativ kompakt in $F$, d.h. wenn $\bar E$ in $F$ kompakt ist.
  \end{enumerate}
\end{defi}

\begin{theorem}
  \label{theorem:6.2} 
  Es seien $1\leq p\leq\infty$ und $m\in\N$. Dann gilt:
  \begin{enumerate}[\rm(a)]
  \item $W_p^m(\Omega)$ ist ein \idx{Banachraum}.
  \item $H^m:=W_2^m(\Omega)$ ist ein \idx{Hilbertraum} mit Skalarprodukt 
    \begin{align*}
      (u\vert v)_{H^m}&:=\sum_{\abs\alpha\leq m}(\partial^\alpha u\vert\partial^\alpha v)_{L_2},\quad u,v\in H^m,
      \intertext{wobei}
      (u\vert v)_{L_2}&:=\int_\Omega u\bar v\d x\, .
    \end{align*}
  \item $W_p^m(\Omega)\hookrightarrow W_p^k(\Omega)$ mit $k\in\N$ und $k\leq m$.
  \item Es ist $\partial^\alpha\in\L(W_p^m(\Omega),W_p^{m-\abs{\alpha}}(\Omega))$ für alle $\abs\alpha\leq m$.
  \end{enumerate}
\end{theorem}

\begin{proof}
  Übung.
\end{proof}

\begin{satz}
  \label{satz:6.3}
  Es seien $1\leq p\leq\infty$ und $m\in\N$. Dann gilt:
  \begin{enumerate}[\rm(a)]
  \item \label{satz:6.3-1} Ist $\{\varphi_\epsilon\with\epsilon>0\}$ ein Mollifier, so gilt
    \[ \varphi_\epsilon\ast u\xrightarrow[\epsilon\ra0]{}u\ \text{ in }\  W_p^m(\Omega)\, . \]

  \item Es ist $\D(\R^n)\overset d\subset W_p^m(\R^n)$.
  \end{enumerate}
\end{satz}

\begin{proof}
  \begin{enumerate}[\rm(a)]
  \item Satz~\ref{satz:3.4} liefert uns, dass $\varphi_\epsilon\ast u\ra u$ in $L_p(\Omega)$ gilt und
    \[ \partial^\alpha(\varphi_\epsilon\ast u)=\varphi_\epsilon\ast\partial^\alpha u \ra\partial^\alpha u\ \text{ in }\ L_p(\Omega) \]
    mit $\abs\alpha\leq m$. Damit folgt $\varphi_\epsilon\ast u\ra u$ in $W_p^m(\Omega)$.
  \item Sei $u\in W_p^m(\R^n)$ und $\chi\in\D(\R^n)$ mit $0\leq\chi\leq1$, $\chi\rvert_{\bar\B(0,1)}=1$. Wir definieren
    \[ u_\epsilon:=\chi(\epsilon\, \cdot)u\, , \]
    und aus den Sätzen von Lebesgue und Leibniz folgt $u_\epsilon\in W_p^m(\Omega)$ mit $u_\epsilon\xrightarrow[\epsilon\ra0^+]{}u$ in $W_p^m(\Omega)$ (vgl.\ Beweis zu Satz~\ref{satz:3.4}).

    Sei $\eta>0$ beliebig. Dann existiert ein $\epsilon>0$ mit $\norm{u_\epsilon-u}_{W_p^m}<\frac\eta2$. Aus Satz~\ref{satz:6.3} (a) folgt dann, dass ein $\delta>0$ mit 
    \[ \norm{\underbrace{\varphi_\epsilon}_{\in\D(\R^n)}\ast u_\epsilon-u}_{W_p^m}<\frac\eta2 \]
    existiert. \qedhere
  \end{enumerate}
\end{proof}

\begin{defi}
  Seien $m\in\N$, $u\in\D'(\R^n)$ und $\varphi\in\D(\Omega)$. Wir definieren
  \[ \<r_\Omega u,\varphi\>_{\D(\R^n)}:=\<u,\varphi\>_{\D(\Omega)} \]
  mit der \idx{Restriktion} $r_\Omega u:=u\rvert_\Omega$ für alle $u\in\Lloc(\Omega)$. Es ist $r_\Omega\in\L(W_p^m(\R^n),\allowbreak W_p^m(\Omega))$. Aus Theorem~\ref{theorem:3.6} folgt, dass $r_\Omega u\in\D'(\Omega)$.

  Wir schreiben $\Omega\in\Ext^m$, wenn $\Omega$ die $C^m$-Erweiterungseigenschaft hat. Dann existiert ein $e_\Omega\in\L(W_p^k(\Omega),W_p^k(\R^n))$ mit $r_\Omega\circ e_\Omega=\id_{W_p^m(\Omega)}$ für alle $k\in\{0,\ldots,m\}$ und $1\leq p<\infty$.
\end{defi}

\begin{bem}
  \label{bem:6.4} Es sei $\Omega\in\Ext^m$. Dann existiert ein $c=c(\Omega,m)>0$ mit
  \[ \norm{e_\Omega u}_{W_p^k(\R^n)}\leq c\norm{u}_{W_p^k(\Omega)} \]
  für alle $u\in W_p^k(\Omega)$ und
  \[ \norm{r_\Omega v}_{W_p^k(\Omega)}\leq\norm v_{W_p^k(\R^n)} \]
  mit $v\in W_p^k(\R^n)$.
\end{bem}

\begin{theorem}
  \label{theorem:6.5}
  Es sei $\R^n\in\Ext^m$ für alle $m\in\N$. Wir definieren
  \[ \H :=\R^{n-1}\times(0,\infty)\in\Ext^m \]
  für alle $m\in\N$. Sind nun $\Omega\in C^m$ und $\partial\Omega$ kompakt, dann ist $\Omega\in\Ext^m$.
\end{theorem}

\begin{proof}
  Literatur.
\end{proof}

\begin{kor}
  \label{kor:6.6} Sei $\Omega\in\Ext^m$. Dann ist $\D(\R^n)\overset{d}\subset W_p^k(\Omega)$ für $1\leq k\leq m$ und $1\leq p<\infty$.

  Speziell gilt: Ist $\Omega\in C^m$ beschränkt, so ist $C^\infty(\bar\Omega)\overset{d}\subset W_p^k(\Omega)$ für $0\leq k\leq m$ und $1\leq p<\infty$.
\end{kor}

\begin{proof}
  Seien $u\in W_p^k(\Omega)$ und $\epsilon>0$. Dann ist $e_\Omega u\in W_p^k(\R^n)$. Nach Satz~\ref{satz:6.3} existiert ein $v\in\D(\R^n)$ mit $\norm{v-e_\Omega u}_{W_p^k(\R^n)}<\epsilon$. Dann ist 
  \[
  \norm{r_\Omega v-\underbrace{r_\Omega e_\Omega u}_{=u}}_{W_p^k(\Omega)}\leq\norm{v-e_\Omega u}_{W_p^k(\R^n)}<\epsilon\, .
  \]
\end{proof}

% ----------------- 10.05.2011 ----------------------------

\begin{defi}
  Seien $m\in\N$ und $1\leq p\leq\infty$. Wir definieren
  \[ \mathring{W}_p^m(\Omega):=\operatorname{cl}_{W_p^m(\Omega)}\D(\Omega) \, , \]
  d.h. der Abschluss von $\D (\Omega)$ bzgl. $\norm{\cdot}_{W_p^m(\Omega)}$.
\end{defi}

\begin{bem}
  \begin{enumerate}[\rm(a)]
  \item $\mathring{W}_p^m(\Omega)$ ist ein abgeschlossener Untervektorraum von $W_p^m(\Omega)$, also ein Banachraum. 
  \item $\mathring{W}_p^m(\R^n)=W_p^m(\R^n)$ für $p<\infty$.
    \begin{proof}
      Satz~\ref{satz:6.3}.
    \end{proof}
  \end{enumerate}
\end{bem}

\begin{theorem}[\idx{Friedrichsche Ungleichung}]
  \label{theorem:6.8} Sei $\Omega$ beschränkt und $1<p<\infty$. Dann existiert ein $c:=c(\Omega,p)$ mit
  \[ \norm u_{L_p(\Omega)}\leq c\norm{\nabla u}_{L_p(\Omega)} \]
  für $u\in\mathring{W}_p^1(\Omega)$.
\end{theorem}

\begin{proof}
  Sei $R>0$ so, dass $\Omega\subset[-R,R]^n$ ist. Seien außerdem $u\in\D(\Omega)$ und $x=(y,t)\in\R^{n-1}\times\R$. Nach dem Mittelwertsatz ist
  \[ u(x)=\int_{-R}^t\partial_u u(y,\tau)\d\tau \]
  für $x\in\Omega$. Damit ist
  \[
  \abs{u(x)}^p\leq\left(
    \int_{-R}^t 1\cdot\abs{\partial_u u(y,\tau)}\d\tau
  \right)^p
  \stackrel[\scriptsize\text{Hölder}]{1=\frac 1p+\frac 1{p'}}\leq
  (2R)^{\frac p{p'}}\int_{-R}^t\abs{\partial_u u(y,\tau)}^p\d\tau\, .
  \]
  Schließlich erhalten wir
  \[
  \begin{split}
    \int_\Omega\abs{u(x)}^p\d x&\leq c\int_{[-R,R]^n}\int_{-R}^t\abs{\partial_u u(y,\tau)}^p\d\tau\d(y,t) \\
    &\stackrel[\d(y,t)=\d y\d\tau]{ \scriptsize\text{Fubini}}{\leq}
    c\cdot 2R\int_\Omega\underbrace{\abs{\partial_u u(x)}^p}_{\abs{\nabla u(x)}^p}\d x
    \leq c\int_\Omega\abs{\nabla u(x)}^p\d x\, .
  \end{split}
  \]
  Somit erhalten wir $\norm u_{L_p(\Omega)}\leq c\norm{\nabla u}_{L_p(\Omega)}$ für alle $u\in\D(\Omega)\overset d\subset\mathring{W}_p^1(\Omega)$.
\end{proof}

\begin{bem}
  \label{bem:6.9} Die Friedrichsche Ungleichung bleibt richtig, wenn $\Omega$ nur in einer Richtung (nicht notwendigerweise in Koordinatenrichtung) beschränkt ist.
\end{bem}

\begin{kor}
  \label{kor:6.10} Es sei $\Omega$ beschränkt $($zumindest in einer Richtung$)$ und $1<p<\infty$. Dann ist $[u\mapsto\norm{\nabla u}_{L_p(\Omega)}]$ eine äquivalente Norm auf $\mathring{W}_p^1(\Omega)$. Für $p=2$ definiert $(\nabla u\vert\nabla v)$ ein normerzeugendes Skalarprodukt auf $\mathring W_2^1(\Omega)=:\mathring H^1(\Omega)$.
\end{kor}

\begin{proof}
  Für alle $u\in\mathring W_p^1(\Omega)$ gilt
  \[
  \begin{split}
    \norm{\nabla u}_{L_p(\Omega)}^p&\ \ \quad\leq \ \ \quad\norm u^p_{L_p(\Omega)}+\norm{\nabla u}^p_{L_p(\Omega)}=\norm u^p_{\mathring W_p^1(\Omega)} \\
    &\underset{\text{\scriptsize Theorem~\ref{theorem:6.8}}}\leq(c(\Omega,p)+1)\norm{\nabla u}^p_{L_p(\Omega)}. \qedhere
  \end{split}
  \]
\end{proof}

\begin{lemma}[Sobolevsche Ungleichung]
  \label{lemma:6.11} Für $1\leq p<n$ sei
  \[ \frac 1{p\ast}:=\frac 1p-\frac 1n \]
  der \idx{Sobolev-Exponent} mit $p\ast>p$. Dann existiert ein $c:=c(n,p)>0$ mit
  \[ \norm u_{L_{p\ast}(\R^n)}\leq c\norm{\nabla u}_{L_p(\R^n)} \]
  für alle $u\in W_p^n(\R^n)$.
\end{lemma}

\begin{proof}
  Übung.
\end{proof}

\begin{theorem}[Sobolevscher Einbettungssatz]
  \label{theorem:6.12} Seien $k,m\in\N^\ast$ mit $k \leq m$ und $  1\leq p,q<\infty$ mit
  \[
  \eq{eq:6.star}\frac 1p\geq\frac 1q\geq\frac 1p-\frac{m-k}n>0.
  \]
  Dann gilt:
  \begin{enumerate}[\rm(a)]
  \item \label{theorem:6.12-1} $\mathring W_p^m(\Omega)\overset d\hookrightarrow\mathring W_q^k(\Omega)$
  \item \label{theorem:6.12-2} Ist $\Omega\in\Ext^m$, so ist $W_p^m(\Omega)\overset d\hookrightarrow W_q^k(\Omega)$.
  \end{enumerate}
\end{theorem}

\begin{proof} Wir beweisen die Aussagen in umgekehrter Reihenfolge.
  \mbox{}
  \begin{enumerate}[(b)]
  \item Sei $\Omega\in\Ext^m$.
    \begin{enumerate}[\rm(i)]
    \item \label{proof:6.12-1} Sei $n>p$ und $\frac 1p\geq\frac 1q\geq\frac 1{p\ast}$. Wir erhalten das Diagramm
      \[
      \begin{diagram}
        \node{W_p^1(\Omega)} \arrow{s,l}{e_\Omega} 
        \arrow[2]{e,J} \node{} \node{L_q(\Omega)} \\
        \node{W_p^1(\R^n)} 
        \arrow{e,b,J}{\text{Lemma~\ref{lemma:6.11}}} 
        \node{L_p(\R^n)\cap L_{p\ast}(\R^n)}
        \arrow{e,b,J}{p\leq q\leq p\ast}
        \node{L_q(\R^n)} \arrow{n,r}{r_\Omega}
      \end{diagram}
      \]
    \item \label{proof:6.12-2} Sei $k=m-1$ und $u\in W_p^m(\Omega)$. Für alle $\alpha\in\N^n$ mit $\abs\alpha\leq m-1=k$ ist $\partial^\alpha u\in W_p^1(\Omega)$. Nach (i) ist $\partial^\alpha u\in L_q(\Omega)$ und es existiert ein $c>0$ mit
      \[
      \norm{\partial^\alpha u}_{L_q(\Omega)}\leq c\norm{\partial^\alpha u}_{W_p^1(\Omega)}\leq c\norm u_{W_p^m(\Omega)}
      \]
      für alle $\abs\alpha\leq m-1$. Also ist $W_p^m\hookrightarrow W_p^{m-1}$ für die Bedingungen aus \eqref{eq:6.star}.
    \item \label{proof:6.12-3} Sei $k=m-2$ und damit wegen \eqref{eq:6.star} $p<\frac n2$. Wegen (ii) ist dann
      \[
      \eq{eq:6.12-1}
      \left.
      \begin{split}
        W_p^m&\hookrightarrow W_r^{m-1} &&, \frac 1p\geq\frac 1r\geq\frac 1p-\frac 1n \\
        W_r^{m-1}&\hookrightarrow W_p^{m-2} &&, \frac 1r\geq\frac 1q\geq\frac 1r-\frac 1n
      \end{split}\qquad
      \right\}
      \]
      Nach \eqref{eq:6.star} existiert ein $r$ mit \eqref{eq:6.12-1}. Es gilt also $W_p^m\hookrightarrow W_q^{m-2}$ für $\frac 1p\geq\frac 1q\geq\frac 1p-\frac 2n=\frac 1p-\frac{m-k}n$.
    \end{enumerate}
    Der Rest ergibt sich durch Induktion. Die Dichtheit folgt aus Korollar~\ref{kor:6.6}.

  \item[(a)] Sei $\Omega$ beliebig und offen in $\R^n$. Es sei
    \[ 
    \tilde e_\Omega:=
    \begin{cases}
      u(x) &, x\in\Omega \\
      0 &, x\in\R^n\setminus\Omega
    \end{cases}
    \]
    die \idx{triviale Fortsetzung} von $u$ in $\R^n$. Es ist $\tilde e_\Omega\in\L(W_p^m(\Omega), W_p^m(\R^n))$ und $r_\Omega\circ\tilde e_\Omega=\id_{W_p^m(\Omega)}$. Wir nun haben die folgenden Beziehungen:
    \[
    \begin{diagram}
      \node{\mathring W_p^m(\Omega)}
      \arrow{s,l}{\tilde e_\Omega} \arrow{e,J} 
      \node{W_q^n(\Omega)} \\
      \node{W_p^m(\R^n)} \arrow{e,b,J}{\text{(\ref{theorem:6.12-2})}}
      \node{W_q^n(\R^n)} \arrow{n,r}{r_\Omega}
    \end{diagram}
    \]
    Damit erhalten wir $\mathring W_p^m(\Omega)\overset d\hookrightarrow\mathring W_q^k(\Omega)$.\qedhere
  \end{enumerate}
\end{proof}

\begin{theorem}[Rellich-Kondrachov]
  \index{Theorem~von!Rellich-Kondrachov}
  \label{theorem:6.13}
  Sei $\Omega$ beschränkt und $C^1$. Dann gilt
  \[ W_p^1(\Omega)\hhookrightarrow L_q(\Omega) \]
  für $1\leq p<n$ und $1\leq q<p\ast$.
\end{theorem}

\begin{proof}
  Theorem~\ref{theorem:6.12} und Theorem~\ref{theorem:6.5} liefern uns $W_p^1(\Omega)\hookrightarrow L_q(\Omega)$.

  Sei $B\subset W_p^1(\Omega)$ beschränkt. Dann genügt zu zeigen, dass $B$ relativ kompakt in $L_q(\Omega)$ ist. Mit Theorem~\ref{theorem:6.5} können wir o.B.d.A. annehmen, dass jedes $u\in B$ auf ganz $\R^n$ definiert ist und einen Träger in $V$ mit $U\Subset V=\mathring V\Subset\R^n$ hat. Außerdem ist $B$ beschränkt in $W_p^1(\Omega)$.

  Sei $\{\varphi_\epsilon\with\epsilon>0\}$ ein Mollifier und $u\in B$, $x\in\Omega$. Dann ist
  \begin{dmath}
    %\eq{eq:6.13-1}
    \label{eq:6.13-1}
    \abs{\varphi_\epsilon\ast u(x)}\leq \int_{\epsilon\B^n}\abs{u(x-y)}\varphi_\epsilon(y)\d y 
    \underset{z=\frac y\epsilon}{=}\int_{\B^n}\abs{u(x-\epsilon z)}\varphi(z)\d z 
    \leq\norm\varphi_\infty\norm u_{L_1(\Omega)}\leq c(B)\, ,
  \end{dmath}
  da $W_p^1(V)\hookrightarrow L_p(V)\hookrightarrow L_1(V)$. Weiter ist
  \begin{dmath}
    % \eq{eq:6.13-2}
    \label{eq:6.13-2}
    \abs{\partial_j(\varphi_\epsilon\ast u)(x)}
    \leq\int_{\epsilon\B^n}\abs{u(x-y)}\abs{\underbrace{\partial_j\varphi_\epsilon(y)}_{\frac 1\epsilon(\partial_j\varphi)(\frac y\epsilon)}}\d y
    \leq\frac 1\epsilon\norm{\nabla\varphi}_\infty\norm u_{L_1(V)}\leq c(B)\epsilon^{-1}.
  \end{dmath}
  Der Mittelwertsatz liefert uns
  \[
  \eq{eq:6.13-3}
  \abs{\varphi_\epsilon\ast u(x)-\varphi_\epsilon\ast u(y)}
  \underset{\scriptsize ~\eqref{eq:6.13-2}}\leq\epsilon^{-1}c(B)\abs{x-y}
  \]
  mit $x,y\in\Omega$, $\epsilon>0$ und $u\in B$. 

  Wir definieren $B_\epsilon:=\{\varphi_\epsilon\ast u\with u\in B\}$. Nach \eqref{eq:6.13-1}, \eqref{eq:6.13-3} und dem Satz von Arzela-Ascoli ist $B_\epsilon$ kompakt in $C(\bar\Omega)\hookrightarrow L_1(\Omega)$, d.h.
  \[
  \eq{eq:6.13-4}
 B_\epsilon\;\text{ist relativ kompakt in}\;L_1(\Omega).
 \]

  Sei $u\in\D(\R^n)$. Dann ist
  \begin{equation}
  \begin{split}
    \abs{u(x)-\varphi_\epsilon\ast u(x)}&\underset{z=\frac y\epsilon}\leq
    \int_{\B^n}\abs{u(x)-u(x-\epsilon z)}\varphi(z)\d z \\
    &\underset{\scriptsize\text{MWS}}=\int_{\B^n}\Abs{\int_0^1\nabla u(x-t\epsilon z)\cdot z\d t}\varphi(z)\d z \\
    &\stackrel[\scriptsize\text{Schwartz}]{\scriptsize\text{Cauchy-}}\leq
    \epsilon\norm\varphi_\infty\int_{\B^n}\int_0^1
    \abs{\
      \nabla u(\underbrace{x\;-\;t\;\epsilon\;z}_{\parbox{1.89cm}{\centering\scriptsize$\in V\;\text{für}\;\epsilon\;\text{klein},$ \\ $\; x\in\Omega$}})
    }\d t\d z\, .
  \end{split}
  \end{equation}
  Also ist
  \[
  \norm{u-\varphi_\epsilon\ast u}_{L_1(\Omega)}\leq c(\varphi)\epsilon\norm{\nabla u}_{L_1(V)}
  \]
  mit $u\in\D(\R^n)$. Aus Korollar~\ref{kor:6.6} und der Tatsache, dass $r_\Omega\D(\R^n)\overset d\subset W_p^1(\Omega)$ ist, folgt
  \[
  \eq{eq:6.13-5}
  \norm{u-\varphi_\epsilon\ast u}_{L_1(\Omega)}\leq c(\varphi, B)\epsilon
  \]
  für alle $u\in B$.

  Sei nun $r>0$ beliebig. Dann existiert ein $\epsilon_0>0$ mit $c_0(\varphi, B)\epsilon_0<\frac r2$. Aus \eqref{eq:6.13-4} folgt dann, dass $w_1,\ldots,w_l\in B_{\epsilon_0}$ existieren mit
  \[
  \eq{eq:6.13-6}
  B_{\epsilon_0}\subset\bigcup_{j=1}^l\B_{L_1(\Omega)}\left(w_j,\frac r2\right).
  \]
  Für ein beliebiges $u\in B$ ist wegen \eqref{eq:6.13-5} $\norm{u-\varphi_{\epsilon_0}\ast u}_{L_1(\Omega)}<\frac r2$. Mit \eqref{eq:6.13-6} folgt die Existenz eines $j\in\{1,\ldots,l\}$ mit $\varphi_{\epsilon_0}\ast u\in\B_{L_1(\Omega)}(w_j,\frac r2)$. Also ist auch $u\in\B_{L_1(\Omega)}(w_j,\frac r2)$. Da $u$ beliebig ist, ist 
  \[ B\subset\bigcup_{j=1}^l\B_{L_1(\Omega)}(w_j,r)\, . \]
  Da $r>0$ beliebig war, ist $B$ total beschränkt, also relativ kompakt. Somit ist 
  \[ \eq{eq:6.13-7} W_p^1(\Omega)\hhookrightarrow L_1(\Omega)\, . \]
  
  Seien nun $1<q<p\ast$ und $p<n$. Dann existiert ein $\Theta\in(0,1)$ mit
  \[ \frac 1q=\frac \Theta1+\frac{1-\Theta}{p\ast}. \]
  Die Hölder-Ungleichung liefert damit
  \begin{dmath}
    %\eq{eq:6.13-8}
    \label{eq:6.13-8}
    \norm u_{L_q(\Omega)}\leq\norm u_{L_1(\Omega)}^\Theta\norm u_{L_{p\ast}(\Omega)}^{1-\Theta}
    \overset{\scriptsize\text{Lem.~\ref{lemma:6.11}}}\leq c\norm u_{L_1(\Omega)}^\Theta
    \underbrace{\norm u_{L_q(V)}^{1-\Theta}}_{\leq c(B)}
    \leq c(B)\norm u_{L_1(\Omega)}^\Theta\, .
  \end{dmath}
  Aus \eqref{eq:6.13-7}, \eqref{eq:6.13-8} und der Cauchy-Schwartzschen Ungleichung folgt dann $W_p^1(\Omega)\hhookrightarrow L_q(\Omega)$.
\end{proof}

\begin{bem}
  \label{bem:6.14}
  \begin{enumerate}[\rm(a)]
  \item \label{bem:6.14-1} Es sei $\Omega$ beschränkt und $C^1$. Dann ist $W_p^1(\Omega)\hhookrightarrow L_q(\Omega)$ für $1\leq p<\infty$.
    \begin{proof}
      Aus $p<n$ folgt $p<p\ast$. $p\geq n(n-\epsilon)\ast\nearrow_{\epsilon\ra0}\infty$
    \end{proof}
  \item \label{bem:6.14-2} Sei $\Omega\subset\R^n$ offen und beschränkt. Dann ist $\mathring W_p^1(\Omega)\hhookrightarrow L_p(\Omega)$.
    \begin{proof}
      (\ref{bem:6.14-1}) $+$ triviale Fortsetzung.
    \end{proof}
  \end{enumerate}
\end{bem}

% -------------- 12.05.2011 --------------------------

\begin{defi}
  Es sind für $0<\nu<1$
  \begin{align*} 
      BUC^\nu(\Omega):=\left\{ u:\Omega\ra\K \with \norm
        u_{BUC^\nu(\Omega)}< \infty \right\}
        \intertext{mit}
        \norm
        u_{BUC^\nu(\Omega)}:=\sup_{x\in\Omega}\abs{u(x)} +\sup_{\
          \begin{subarray}{c}
            x,y\in\Omega \\
            x\neq y
          \end{subarray}
        } \frac{\abs{u(x)-u(y)}}{\abs{x-y}^\nu} 
\end{align*}
und
\begin{align*}
      BUC^{k+\nu}(\Omega):=\left\{ 
        u\in BUC^k(\Omega)\with \partial^\alpha u\in BUC^\nu(\Omega)\, \fa\, \abs\alpha \leq k \right 
      \} .
  \end{align*}
\end{defi}

\begin{theorem}[Sobolev-Morrey]
  \label{theorem:6.15}
  \index{Theorem~von!Sobolev-Morrey}
  Es seien $m\in\N$, $1\leq p<\infty$ und $0\leq\nu\leq m-\frac np$ mit $\nu<m-\frac np$, falls $m-\frac np\in\N$. Dann gilt:
  \begin{enumerate}[\rm(i)]
  \item \label{theorem:6.15-1} $\mathring W_p^m(\Omega)\hookrightarrow BUC^\nu(\Omega)$
  \item Ist $\Omega\in\Ext^m$, dann ist $W_p^m(\Omega)\hookrightarrow BUC^\nu(\Omega)=BUC^\nu(\bar\Omega)$.
  \end{enumerate}
\end{theorem}

\begin{proof}
  Adams ("`Sobolev Spaces"'), Triebel.
\end{proof}

\begin{bem*}
  $-\Delta u=f$ in $\Omega$, $u\rvert_{\partial\Omega}=0$ und für $f\in C(\bar\Omega)$ gibt es im Allgemeinen keine Lösung $u\in C^2(\Omega)\cap C(\bar\Omega)$. Deshalb betrachten wir $u\in W_p^2(\Omega)$. Es ergibt sich dabei allerdings die Frage, wie wir $u\rvert_{\partial\Omega}=0$ interpretieren.
\end{bem*}

\begin{theorem}[\idx{Spursatz}]
  \label{theorem:6.16}
  Es sei $\Omega\subset\R^n$ und $C^2$. Dann gilt für $1<p<\infty$, dass ein $c>0$ existiert mit
  \[ \norm{u\rvert_{\partial\Omega}}_{L_p(\partial\Omega)}\leq c\norm u_{W_p^1(\Omega)} \]
  für alle $u\in C^\infty(\bar\Omega)$.

  Die Restriktion $[u\mapsto u\rvert_{\partial\Omega}]$ lässt sich also eindeutig stetig zum Spuroperator $\gamma_0\in\L(W_p^1(\Omega),L_p(\partial\Omega))$ fortsetzen.
\end{theorem}

\begin{proof}
  Sei $\nu$ die äußere Einheitsnormale an $\partial\Omega$ mit $\norm{\nu(x)}=1$ mit $x\in\partial\Omega$. $\nu$ kann zu einem $C^2$-Vektorfeld auf $\bar\Omega$ erweitert werden. Für $u\in C^\infty(\bar\Omega)$ ist
  \begin{dmath*}
    \int_{\partial\Omega}\abs u^p\d\sigma
    =\int_{\partial\Omega}(\abs u^p\nu)\cdot\nu\d\sigma
    \hiderel{\underset{\scriptsize\text{Gauß}}=}\int_\Omega\div(\abs u^p\nu)\d x
    \leq\int_\Omega\sum_{j=1}^n\left(
      p\abs u^{p-1}\cdot\abs{\partial_j u}\abs{\nu^j}
      +\abs u^p\cdot\abs{\partial_j\nu^j}
    \right)\d x 
    \leq c\cdot\left(\norm\nu_{C^1(\bar\Omega)}\int_\Omega\sum_{j=1}^n\left(
      p\abs u^{p-1}\cdot\abs{\partial_j u}+\abs u^p
    \right) \d x\right)
    \mkern-14mu\underset{\scriptsize\text{Young}}\leq c\cdot\int_\Omega\sum_{j=1}^n\left(
      \abs u^p+\abs{\partial_j u}+\abs u^p
    \right) \d x\, .
  \end{dmath*}
  Damit ist 
  \[
  \norm {u\rvert_{\partial\Omega}}_{L_p(\Omega)}\leq c\cdot\norm u_{W_p^1(\Omega)}
  \]
  für alle $u\in C^\infty(\bar\Omega)$. Mit Korollar~\ref{kor:6.6} folgt damit $C^\infty(\bar\Omega)\overset d\subset W_p^1(\Omega)$.

  In den Übungen werden wir zeigen, dass eine Erweiterung $\gamma_0\in \L(W_p^1(\Omega),\allowbreak L_p(\partial\Omega))$ existiert mit $\gamma_0(u)=u\rvert_{\partial\Omega}$ für $u\in C\infty(\bar\Omega)$. Damit folgt die Behauptung.
\end{proof}

\begin{bem}
  \label{bem:6.17}
  Es seien $1<p<\infty$, $m\in\N^*$ und $\Omega\in C^2$ beschränkt. Dann gelten:
  \begin{enumerate}[\rm(i)]
  \item \label{bem:6.17-1} Sei
    \begin{dseries*}
      \begin{math}
        \mathring W_p^m(\Omega)=\operatorname{cl}_{W_p^m(\Omega)}\D(\Omega)
      \end{math},
      \begin{math}
        \partial^\alpha u\vert_{\partial\Omega}=0
      \end{math}
    \end{dseries*}
    für alle $u\in\D(\Omega)$ und $\alpha\in\N^m$. Dann ist
    \begin{dmath*}
      \gamma_0(\partial^\alpha u)=0
      \condition{für alle $u\in\mathring W_p^m(\Omega)$, $\abs\alpha\leq m-1$}.
    \end{dmath*}
  \item \label{bem:6.17-2} Gauß:
    \begin{dmath*}
      \int_\Omega\div(u)\d x = \int_{\partial\Omega}\gamma_0(u)\cdot \nu \d\sigma
      \condition{für alle $u\in W_p^1(\Omega,\K^n)=(W_p^1(\Omega))^n$}.
    \end{dmath*}
  \end{enumerate}
\end{bem}

\begin{erinnerung}
\[ -\Delta u = f \quad \text{in } \Omega \, , \quad u\vert_{\partial \Omega} = 0\]
hat im Allgemeinen keine Lösung $u \in C^2(\Omega)$, falls $f \in C(\Omega)$. Nun muss also $u \in W_p^2(\Omega)$ sein, jedoch stellt sich dann die Frage, wie $u|_{\partial\Omega}$ zu interpretieren ist.

Die Idee ist nun
\[
	u \stackrel{!}\in \mathring W^2_p = \{u \in W^2_p \with \gamma_0 u = 0\}
\]
und wenn $f\in L_p(\Omega)$, dann existiert genau ein $u \in \mathring W^2_p(\Omega)$ für $-\Delta u = f(u)$.
\end{erinnerung}

\begin{defi}
Es sei $m\in \N, 1\leq p < \infty, 1 = \frac 1 p + \frac 1 {p'}$.
\begin{enumerate}[(a)]
\item Eine Folge $(u_j)$ in $L_p$ konvergiert schwach gegen $u \in L_p(\Omega)$
\begin{align*}
	&: \Longleftrightarrow u_j \rightharpoonup u \text{ in } L_p(\Omega) \\
	& : \Longleftrightarrow \, \fa \, v \in L_{p'}(\Omega) : \int_\Omega u_j v \d x \longrightarrow \int_\Omega uv \d x \text{ in } \K \, .
\end{align*}
\item Eine Folge $(u_j) \in W^m_p(\Omega)$ konvergiert schwach gegen $u \in W_p^m (\Omega)$
\begin{align*}
	& : \Longleftrightarrow u_j \rightharpoonup u \text{ in } W_p^m(\Omega) \\
	& : \Longleftrightarrow \partial^\alpha u_j \rightharpoonup \partial^\alpha u \text{ in } L_p(\Omega) \, \forall \, \abs\alpha \leq m \, .
\end{align*}
\end{enumerate}
\end{defi}

\begin{bem}
Man kann "`schwache Konvergenz"' (bzw. "`schwache Topologie"') für allgemeine (z.B. Banach-/lokal konvexe Räume) definieren (s. z.B. Rudin). Wir haben in der Definition benutzt, dass gilt
\[
	(L_p(\Omega))' := \mathcal L(L_p(\Omega),\K) \, , \quad (L_p(\Omega))' \cong L_{p'} (\Omega) \, ,
\]
denn ist $v \in L_{p'} (\Omega)$, so ist
\[
	\langle v ,u\rangle := \int_\Omega vu \d x \, , \quad u \in L_p(\Omega) 
\]
und damit folgt mit Hölder, dass $\langle v, \, \cdot \, \rangle \in \mathcal L(L_p(\Omega), \K)$ gilt.

Speziell gilt also $(L_2(\Omega))' \cong L_2(\Omega)$.
\end{bem}

\begin{bem}
\label{bem:6.19}
Sei $1\leq p < \infty, m \in \N$, dann ist:
\begin{enumerate}[(a)]
\item Ist $u_j \ra u$ in $W^m_p(\Omega)$, dann folgt $u_j \rightharpoonup u$ in $W^m_p(\Omega)$, d.h. "`starke Konvergenz ist stärker als schwache Konvergenz"'.
\begin{proof}
$\fa \, v \in L_{p'}(\Omega), \abs \alpha \leq m$ gilt
\[
	\Abs{\int_\Omega (\partial^\alpha u_j - \partial^\alpha u) v \d x} \stackrel[\scriptsize\text{Hölder}]{}\leq \norm v_{L_{p'}(\Omega)} \norm{\partial^\alpha u_j -\partial^\alpha u }_{L_p(\Omega)} \longrightarrow 0 \, .
\] 
\end{proof}
\item Sei $1 < p < \infty, (u_j)\subset W^m_p(\Omega)$ beschränkt (bzgl. $\norm{\, \cdot\, }_{W^m_p}$), dann folgt, dass eine Teilfolge $(u_{j'})$ und ein $u \in W^m_p(\Omega)$ existiert, so dass $u_j \rightharpoonup u$ in $W^m_p(\Omega)$, d.h. beschränkte Folgen sind relativ schwach kompakt.
\begin{proof}
Vgl. Rudin.
\end{proof}
\item Sei $M\subset W^m_p(\Omega)$ konvex und abgeschlossen (bzgl. $\norm{\, \cdot\, }_{W^m_p}$), sowie $(u_j) \in M$ mit $u_j \rightharpoonup u $ in $W^m_p(\Omega)$, dann ist $u \in M$, d.h. "`abgeschlossene konvexe Mengen sind schwach abgeschlossen"' (Theorem von Mazun; ohne Beweis, vgl. Rudin).
\item Es sei $u_j \rightharpoonup u $ in $W^p_m (\Omega)$, dann folgt $(u_j)$ ist beschränkt in $W^m_p(\Omega)$ (bzgl. $\norm{\, \cdot\, }_{W^m_p}$), d.h. "`schwach konvergente Folgen sind beschränkt"'.
\begin{proof}
Theorem von Mackey, vgl. Rudin.
\end{proof}
\item $u_ j \rightharpoonup u $ in $W^m_p(\Omega), u_j \rightharpoonup v$ in $W^m_p(\Omega)$, dann gilt $u=v$, d.h. "`Grenzwerte von schwach konvergenten Folgen sind eindeutig"'.
\begin{proof}
Aus Theorem~\ref{theorem:3.6} folgt die Behauptung.
\end{proof}
\item Sei $u_j \rightharpoonup u $ in $W^m_p(\Omega)$, dann folgt $\norm u_{W^m_p(\Omega)} \leq \lim \inf \norm{u_j}_{W^m_p(\Omega)}$.
\end{enumerate}
\end{bem}

%%% Local Variables: 
%%% mode: latex
%%% TeX-master: "Skript"
%%% End: 

\newchapter{Hilbertraummethoden und schwache Lösungen}


Unser Ziel ist es, elliptische Gleichungen wie
\begin{align*}
    -\Delta u&=f\condition{in $\Omega$} \\
    u\vert_{\partial\Omega}&=0
\end{align*}
für nicht-reguläre $f$ in allgemeinen Gebieten zu lösen.

\section{Hilberträume und die Sätze von Riesz und Lax-Milgram}

Als Generalvoraussetzung für dieses Kapitel sei $H=(H,(\cdot\vert\cdot))$ ein $\K$-Hilbert\-raum mit $\norm x^2=(x\vert x)$.

\begin{bem}
  \label{bem:7.1} Für alle $x,y\in H$ gilt die \idx{Parallelogrammgleichung}
  \[ 2(\norm x^2+\norm y^2)=\norm{x-y}^2+\norm{x+y}^2 \]
  und die \idx{Cauchy-Schwartzsche~Ungleichung}
  \[ \abs{(x\vert y)}\leq\norm x\cdot\norm y \, . \]
\end{bem}

\begin{satz}[Approximationssatz]
  \label{satz:7.2} Es sei $\emptyset\neq M\subset H$ konvex und abgeschlossen. Dann existiert für alle $x\in H$ ein $m_x\in M$ mit
  \[ \norm{x-m_x}=\dist(x,M)\coloneqq \inf_{y\in M}\norm{x-y}\, .  \]
  Wir nennen $P_M:H\ra M$ mit $x\mapsto m_x$ die \idx{Projektionen} auf $M$.
\end{satz}

\begin{proof}
  Seien $x\in H$ und $\alpha\coloneqq\dist(x,M)$. Dann existiert für alle $j\in\N$ ein $m_j\in M$ mit $\norm{x-m_j}\leq\alpha+\frac 1j$. Bemerkung~\ref{bem:7.1} liefert uns dann
  \begin{dmath*}
    2(\norm{x-m_j}^2+\norm{x-m_k}^2)=4\Norm{\frac{m_j+m_k}2-x}^2+\norm{m_j-m_k}^2
    \geq4\alpha^2+\norm{m_j-m_k}^2.
  \end{dmath*}
  Weiter ist
  \[ \norm{m_j-m_k}^2\leq2\norm{x-m_j}^2+2\norm{x-m_k}^2-4\alpha^2\xrightarrow{j,k\ra\infty}0\, . \]
  Also existiert wegen der Chauchy-Schwartzschen Ungleichung und der Abgeschlossenheit von $M$ ein $m\in M$ mit $m_j\ra m$.

  Zum Beweis der Eindeutigkeit seien $m,n\in M$ mit $\norm{x-m}=\alpha=\norm{x-n}$. Weil $M\cap\bar\B(x,\alpha)$ konvex ist, liegt $\frac{m+n}2 \in M\cap\bar\B(x,\alpha)$. Damit ist $\Norm{\frac{m+n}2-x}=\alpha$. Bemerkung~\ref{bem:7.1} liefert uns
%  \begin{dmath*}
\[
    \norm{m-n}^2=2\cdot\norm{x-m}^2+2\cdot\norm{x-n}^2-4\cdot\Norm{\frac{m+n}2-x}^2=0\, ,
 \]
 % \end{dmath*}
  und damit $m=n$.
\end{proof}

\begin{bem*}
  Satz~\ref{satz:7.2} ist falsch in allgemeinen Banachräumen.
\end{bem*}

\begin{satz}[Charakterisierung der Projektionen]
  \label{satz:7.3} $\emptyset\neq M\subset H$ sei abgeschlossen und konvex und $x\in H$. Dann gilt:
  \[ m_0=P_M(x)\quad\Longleftrightarrow\quad \Re(m-m_0\vert x-m_0)\leq0 \]
  für alle $m\in M$.
\end{satz}

\begin{proof}
  O.B.d.A. sei $0\in M$ und $m_0=0$.
 
  "`$\Rightarrow$"' Wegen $0=P_M(x)$ muss $\norm{x-tm}\geq\norm x$ für $m\in M$ und $0\leq t\leq1$ sein. Dann ist
    \[ \norm x^2\leq\norm x^2-2t\cdot\Re(x\vert m)+t^2\norm m^2. \]
    Damit ist $2\Re(x\vert m)\leq0$.
 
 "`$\La$"' Für alle $m\in M$ ist $\Re(x\vert m)\leq0$. Es folgt
    \[ \norm x^2\leq\norm x^2+\norm m^2-2\cdot\Re(m\vert x)=\norm{x-m}^2. \]
    Wegen $0\in M$ ist $\dist(x,M)=\norm x^2$ und damit $0=P_M(x)$.
\end{proof}

\begin{bem}
  \label{bem:7.4}
  \begin{enumerate}[(a)]
  \item   Das \idx{Minimierungsproblem} auf einer konvexen Menge ist
   \[ \norm{x-P_M(x)}\overset!=\inf_{m\in M}\norm{x-m} \, .\]
  \item      Die \idx{Variationsgleichung} (vgl. später) ist
  \begin{dmath*}
      \Re(m-P_M(x)\vert x-P_M(x))\leq0\condition{für alle $m\in M$} \, .
    \end{dmath*}
  \end{enumerate}
\end{bem}

\begin{kor}
  \label{kor:7.5} Es sei $M$ ein abgeschlossener Untervektorraum von $H$ und $x\in H$. Dann gilt:
    \[ m_0=P_M(x)\quad\Longleftrightarrow\quad x-m_0\perp M\quad\Longleftrightarrow\quad(x-m_0\vert m)=0 \]
    für alle $m\in M$.
\end{kor}

\begin{proof}
  Für alle $m\in M$ ist $m_0=P_M(x)$ genau dann, wenn $\Re(m-m_0\vert x-m_0)\leq0$. Da $M$ ein Untervektorraum ist, ist auch $\Re(m\vert x-m_0)\leq0$. Da dies auch für $-m$ gilt, ist $\Re(m\vert x-m_0)=0$. Also ist die Behauptung für $\K=\R$ bewiesen.

  Wir betrachten nun den Fall $\K=\C$. Dafür ersetzen wir $m$ durch $im\in M$ und erhalten
  \begin{dseries*}
    \begin{math}
      \Re(x\vert x-m_0)=0\condition{$\fa\,  m\in M$}
    \end{math}
    $\Longleftrightarrow$
    \begin{math}
      \Im(m\vert x-m_0)=0\condition{$\fa \, m\in M$}
    \end{math}.
  \end{dseries*}
  Also ist $(m\vert x-m_0)=0$ für alle $m\in M$ und die Behauptung ist bewiesen.

\end{proof}

\begin{defi}
  Es sei $\emptyset\neq A\subset H$ und wir definieren das orthogonale Komplement\index{orthogonales~Komplement} von A durch
  \[ A^\perp\coloneqq\{x\in H\with x\perp A\}\coloneqq\{x\in H\with (x\vert a)=0\;\fa\,  a\in A\}\, . \]
\end{defi}

\begin{theorem}
  \label{theorem:7.6} Es sei $M$ ein abgeschlossener Untervektorraum von $H$. Dann ist
  \[ H=M\oplus M^\perp, \]
  d.h.\ jedes $x\in H$ hat eine eindeutige Zerlegung $x=x_M+x_{M^\perp}$ mit $x_M\in M$ und $x_{M^\perp}\in M^\perp$.
\end{theorem}

\begin{proof}
  Für alle $x\in H$ gilt
  \[ 
  x=\underbrace{P_M(x)}_{\underset{\text{Satz~\ref{satz:7.2}}}\in M}
  +\underbrace{(x-P_M(x))}_{\underset{\text{Korollar~\ref{kor:7.5}}}\in M^\perp}\in M+M^\perp.
  \]

  Um die Eindeutigkeit zu zeigen sei $x=a+b$ mit $a\in M$ und $b\in M^\perp$. Dann ist
  \[ 0=a-P_M(x)+b-(x-P_M(x))\, . \]
  Sei nun $c\coloneqq a-P_M(x)\in M$. Es ist dann aber auch $c=(x-P_M(x))-b\in M^\perp$, und somit $c\in M\cap M^\perp$. Dann ist $(c\vert c)=0=\norm c^2$, also $c=0$.
\end{proof}

\newpage

\begin{kor}
  \label{kor:7.7} Für $0\neq A\subset H$ gilt:
  \begin{enumerate}[\rm(i)]
  \item \label{kor:7.7-1} $A^\perp$ ist abgeschlossener Untervektorraum von $H$.
  \item \label{kor:7.7-2} $\overline{\spn A}=(A^\perp)^\perp=:A^{\perp\perp}$.
  \item \label{kor:7.7-3} Es ist $A$ ein Untervektorraum. Dann ist $\bar A=H$ genau dann wenn $A^\perp=\{0\}$.
  \end{enumerate}
\end{kor}

\begin{proof}
  \begin{enumerate}
  \item[(\ref{kor:7.7-1})] \[ A^\perp=\bigcap_{a\in A}\ker(\, \cdot \, \vert a) \]
    ist abgeschlossener Untervektorraum.
  \item[(\ref{kor:7.7-2})] Es ist klar, dass $A\subset A^{\perp\perp}$, da $(a\vert b)=0$ für alle $b\in A^\perp$, $a\in A$. Wegen (\ref{kor:7.7-1}) ist $A^{\perp\perp}$ ein abgeschlossener Untervektorraum von $H$, ist also ein Hilbertraum.

    Sei $B:=\overline{\spn A}$. Dann ist wegen $A\subset A^{\perp\perp}$, $B$ auch abgeschlossener Untervektorraum von $A^{\perp\perp}$. Nach Theorem~\ref{theorem:7.6} ist
    \[ A^{\perp\perp}=B\oplus B^\perp. \]
    Außerdem ist
    \[ B^\perp\subset A^\perp\cap A^{\perp\perp}=\{0\}\, , \]
    denn $A\subset B$. Diese beiden Aussagen führen zu $A^{\perp\perp}=B$.
  \item[(\ref{kor:7.7-3})] $A$ sei Untervektorraum. Dann gilt laut Theorem~\ref{theorem:7.6}:
    \[ A^\perp\oplus A^{\perp\perp}=A^\perp\oplus \bar A\, . \qedhere \]
  \end{enumerate}
\end{proof}

% ----------------- 17.05.2011 ------------------------------

\begin{theorem}[Riesz]
  \label{theorem:7.8} \index{Satz~von!Riesz}
  Für alle $f\in\L(H,\K)=:H'$ existiert genau ein $J(f)\in H$ mit
  \begin{dmath*}
    f(x)=(x\vert J(f))\condition{$x\in H$}.
  \end{dmath*}
  Ferner gilt $\norm{J(f)}=\norm f_{H'}$.  
\end{theorem}

\begin{proof}
  O.B.d.A. sei $f\neq 0$. Dann ist $\ker(f)\neq H$ und dach Korollar~\ref{kor:7.7} (\ref{kor:7.7-3}) existiert ein $y\in\ker(f)^\perp$ mit $\norm y=1$. Für alle $x\in H$ ist $f(x)y-f(y)x\in\ker(f)$. Damit ist $(f(x)y-f(y)x|y)=0$. Somit ist
  \[ f(x)=f(x)(y\vert y)=(f(y)x\vert y)=(x\vert \overline{f(y)}y) \]
  für alle $x\in H$.

  Wir setzen $J(f):=\overline{f(y)}y\in H\setminus\{0\}$. Es ist
  \[ 
  \norm f_{H'}=\sup_{x\neq0}\frac{\abs{f(x)}}{\norm x}
  \geq\frac{\abs{f(J(f))}}{\norm{J(f)}}=\frac{\abs{(J(f)\vert J(f))}}{\norm{J(f)}}
  =\norm{J(f)}
  \]
  und
  \begin{dmath*}
    \abs{f(x)}=\abs{(x\vert J(f))}\leq\norm x\norm{J(f)}
    \condition{$x\in H$}.
  \end{dmath*}
  Damit ist $\norm f_{H'}\leq\norm{J(f)}$.
  
  Um die Eindeutigkeit zu zeigen sei $f=(\, \cdot\, \vert y_1)=(\, \cdot\, \vert y_2)$. Dann ist $(x\vert y_1-y_2)=0$ für alle $x\in H$. Mit $x=y_1-y_2$ ist $\norm{y_1-y_2}^2=0$, d.h.\ $y_1=y_2$.
\end{proof}

\begin{bem}
  \label{bem:7.9}
  \begin{enumerate}[(a)]
  \item \label{bem:7.9-1} $E$ sei ein normierter Vektorraum. Dann bezeichnen wir mit $E':=\L(E,\K)$ den (topologischen) \idx{Dualraum} von $E$.
  \item \label{bem:7.9-2} $J:H'\ra H$ mit $f\mapsto J(f)$ ist ein konjugiert linearer (d.h.\ $J(f+\alpha g)=J(f)+\alpha J(g)$ für $\alpha\in\K$, $f,g\in H'$) isometrischer Isomorphismus. Damit lässt sich $H'$ mit $H$ identifizieren.
  \item \label{bem:7.9-3} Ist $\Omega$ eine offene Teilmenge des $\R^n$, dann ist wegen (b) $(L_2(\Omega))'\cong L_2(\Omega)$.
  \end{enumerate}
\end{bem}

\begin{theorem}[Lax-Milgram]
  \index{Theorem~von!Max-Milgram}
  \label{theorem:7.10}
  Sei $a:H\times H\ra\K$ eine stetige, koerzive \idx{Sesquilinearform}, d.h.\ für alle $x\in H$ sind $a(\, \cdot\, , x)$ und $\overline{a(x,\, \cdot\, )}$ linear $($\idx{sesquilinear}$)$, es existiert ein $c>0$ mit
{\rm  \begin{dmath}[number={\textit{stetig}}]
    a(x,y)\leq c\norm x\norm y\condition{\textit{für alle} $x,y\in H$}
  \end{dmath}
  \textit{und es existiert ein $\alpha>0$ mit}
  \begin{dmath}[number={\textit{koerziv}}]
    \Re a(x,x)\geq \alpha\norm x^2\condition{\textit{für alle} $x\in H$}.
  \end{dmath}
  \textit{Dann existiert für alle $f\in H'=\L(H,\K)$ genau ein $u_f\in H$ mit}
  \begin{dmath*}
    f(x)=a(x,u_f)\condition{\textit{für alle} $x\in H$}.
  \end{dmath*}}
  Ferner ist $\norm{u_f}_H\leq\frac 1\alpha\norm f_{H'}$.
\end{theorem}

\begin{proof}
  Sei $u\in H$ beliebig. Dann ist $[x\mapsto a(x,u)]\in\L(H,\K)=H'$. Nach Theorem~\ref{theorem:7.8} existiert dann genau ein $Su\in H$ mit
  \begin{dseries}
    \label{eq:7.1}
    \begin{math}
      a(x,u)=(x\vert Su)\condition*{x\in H}
    \end{math}
    und
    \begin{math}
      \norm{Su}=\norm{a(\cdot, u)}_H\underset{\scriptsize\text{stetig}}\leq c\norm u_H
    \end{math}.
  \end{dseries}
  Sei $S:H\ra H$ mit $u\mapsto Su$. Wegen Bemerkung~\ref{bem:7.9} (\ref{bem:7.9-2}) ist $S$ linear, also $S\in\L(H,H)=\L(H)$.

  Nun ist
  \begin{dmath}
    \label{eq:7.2}
    \alpha\norm u^2\underset{\scriptsize\text{koerziv}}\leq\Re a(u,u)\hiderel\leq\abs{a(u,u)}\hiderel=(u\vert Su)
    \underset{\scriptsize\text{Cauchy-Schwartz}}\leq\norm u\norm{Su}.
  \end{dmath}
  Somit ist $\norm{Su}\geq\alpha\norm u$.

  Sei $(u_j)$ eine Folge in $H$ mit $Su_j\ra v$ in $H$. Dann ist
  \begin{dmath*}
    \norm{u_j-u_k}\underset{\scriptsize\eqref{eq:7.2}}\leq\frac 1\alpha\norm{Su_j-Su_k}
    \xrightarrow{j,k\ra\infty}0.
  \end{dmath*}
  Da $(u_j)$ eine Cauchy-Folge ist, existiert ein $u\in H$ mit $u_j\ra u$ in $H$. Da $S$ stetig ist, gilt $Su_j\ra Su$ und $Su_j\ra v$. Also ist $v=Su\in\im(S)$. Somit ist $M:=\im(S)$ ein abgeschlossener Untervektorraum von $H$. Nach Theorem~\ref{theorem:7.6} ist dann $H=M\oplus M^\perp$.

  Sei $w\in M^\perp$. Dann ist $0=\abs{(w\vert Sw)}=\abs{a(w,w)}\geq\alpha\norm w^2$, da $a$ koerziv ist. Es muss dann $w=0$ bzw.\ $M^\perp=\{0\}$ sein. Aus Korollar~\ref{kor:7.7} (~\ref{kor:7.7-3}) folgt $H=\bar M$. Da $M$ abgeschlossen ist, gilt außerdem $M=\bar M$. Somit ist $S\in\L(H)$ bijektiv.

  Sei $f\in H'$ beliebig. Dan existert laut Theorem~\ref{theorem:7.8} genau ein $J(f)\in H$ mit $f(x)=(x\vert J(f))$ für $x\in H$. Wir setzen $u_f:=S^{-1}J(f)\in H$. Dann ist
  \[ a(x,u_f)\overset{\scriptsize\eqref{eq:7.1}}=(x\vert Su_f)=(x\vert J(f))=f(x)\in H. \]
  Um die Eindeutigkeit zu zeigen sei $a(x,y_1)=a(x,y_2)$ für alle $x\in H$. Mit $x:=y_1-y_2$ erhalten wir $a(y_1-y_2,y_1-y_2)=0$. Da $a$ koerziv ist, muss $y_1-y_2=0$ bzw.\ $y_1=y_2$ sein.

  Ferner ist 
  \[ 
  \norm{u_f}=\norm{S^{-1}J(f)}\overset{\scriptsize\eqref{eq:7.2}}\leq\frac 1\alpha\norm{J(f)}
  \overset{\scriptsize\text{Riesz}}=\frac 1\alpha\norm f_{H'}.
  \]
\end{proof}

\subsection*{Orthonormalbasen}

Wir setzen nun voraus, dass $\dim H=\infty$ ist.

\begin{defi}
  Sei $\Phi\subset H$.
  \begin{enumerate}[\rm(i)]
  \item $\Phi$ heißt \idx{Orthogonalsystem} (OS), wenn $\phi\perp\psi$ für alle $\phi,\psi\in\Phi$ und $\phi\neq\psi$.
  \item $\Phi$ heißt \idx{Orthonormalsystem} (ONS), wenn $\Phi$ ein Orthogonalsystem und $\norm\phi=1$ für alle $\phi\in\Phi$.
  \item $\Phi$ heißt \idx{Orthonormalbasis} (ONB), wenn $\Phi$ ein Orthonormalsystem und $\Phi^\perp=\{0\}$ ist.
  \end{enumerate}
\end{defi}

\begin{lemma}
  \label{lemma:7.11} Sei $\Phi:=\{\phi_j\with j\in\N\}$ ein Orthogonalsystem. Dann gilt:
  \begin{enumerate}[\rm(i)]
  \item $\Phi\setminus\{0\}$ ist linear unabhängig.
  \item $\sum\phi_j$ konvergiert in $H$ genau dann, wenn $\sum\norm{\phi_j}^2<\infty$ ist. Dann ist $\Norm{\sum\phi_j}^2=\sum\norm{\phi_j}^2$ eine Verallgemeinerung des Satzes von Pythagoras. 
  \end{enumerate}
\end{lemma}

\begin{proof}
  \begin{enumerate}[(i)]
  \item Seien $\alpha_0,\ldots,\alpha_m\in\K$. Dann ist
    \begin{dmath*}
      \Norm{\sum_{j=0}^m\alpha_j\phi_j}^2=\left(
        \sum_{j=0}^m\alpha_j\phi_j\;\vrule\;\sum_{k=0}^m\alpha_k\phi_k
      \right)
      \hiderel=\sum_{j,k=0}^m\alpha_j\overline{\alpha_k}\underbrace{(\phi_j\vert \phi_k)}_{=\delta_{jk}}
      \hiderel=\sum_{j=0}^m\abs{\alpha_j}^2.
    \end{dmath*}
  \item Mit $S_N:=\sum_{j\leq N}\phi_j$ erhalten wir
    \[ \norm{S_N-S_M}^2=\Norm{\sum_{j=M+1}^N\phi_j}^2=\sum_{j=M+1}^N\norm{\phi_j}^2. \]
  \end{enumerate}
\end{proof}

\begin{defi}
  Sei $\Phi:=\{\phi_j\with j\in \N\}$ ein Orthonormalsystem und $x\in H$. $(x\vert\phi_j)$ heißt \idx{Fourierkoeffizient} von $x$ bezüglich $\phi_j$. $\sum_j(x\vert\phi_j)\phi_j$ heißt \idx{Fourierreihe} von $x$ bezüglich $\Phi$.
\end{defi}

\begin{theorem}
  \label{theorem:7.12} Sei $\Phi:=\{\phi_j\with j\in\N\}$ ein Orthonormalsystem. Dann gilt:
  \begin{enumerate}[\rm(a)]
  \item \label{theorem:7.12-1} Für alle $x\in H$ konvergiert die Fourierreihe $\sum(x\vert \phi_j)\phi_j$ unbedingt. Ferner gilt für $N:=\spn\Phi=\Phi^{\perp\perp}$:
    \[ P_N(x)=\sum(x\vert \phi_j)\phi_j\, . \]
  \item \label{theorem:7.12-2} Es sind folgende Aussagen äquivalent:
    \begin{enumerate}[\rm(i)]
    \item \label{theorem:7.12-2-1} $\Phi$ ist eine Orthonormalbasis.
    \item \label{theorem:7.12-2-2} Für alle $x\in H$ gilt: $x=\sum(x\vert \phi_j)\phi_j$.
    \item \label{theorem:7.12-2-3} Für alle $x\in H$ gilt: $\norm x^2=\sum\abs{(x\vert\phi_j)}^2$.
    \end{enumerate}
  \end{enumerate}
\end{theorem}

\begin{proof}
  \begin{enumerate}[(a)]
  \item
    \begin{align*}
      0&\  \leq\Norm{x-\sum_{j\leq N}(x\vert \phi_j)\phi_j}^2 \\
      &\underset{\scriptsize\text{ONS}}=\norm x^2-2\Re\sum_{j\leq N}
      \underbrace{\overline{(x\vert \phi_j)}(x\vert\phi_j)}_{=\abs{(x\vert\phi_j)}^2}
      +\sum_{j,k\leq N}\underbrace{(x\vert\phi_j)\overline{(x\vert\phi_k)}}_{\underset{j=k}=\abs{(x\vert\phi_j)}^2}
      \underbrace{(\phi_j\vert\phi_k)}_{\delta_{jk}}  \\
      & \ =\norm x^2-\sum_{j\leq N}\abs{(x\vert\phi_j)}^2
    \end{align*}
    Damit ist 
    \begin{dmath*}
      \sum_{j\leq N}\abs{(x\vert\phi_j)}^2\leq\norm x^2
      \condition{für alle $N$}.
    \end{dmath*}
    Also konvergiert $\sum(x\vert\phi_j)\phi_j$ für alle $x\in H$. Für alle $\phi_k\in\Phi$ gilt
    \begin{dmath*}
      \left(\sum (x\vert \phi_j)\phi_j-x\Big\vert\phi_k\right)
      =\sum(x\vert\phi_j)\underbrace{(\phi_j\vert\phi_k)}_{\delta_{jk}}-(x\vert \phi_k)\hiderel=0\, .
    \end{dmath*}
    Mit Korollar~\ref{kor:7.5} folgt
    \[ P_N(x)=\sum(x\vert\phi_j)\phi_j \]
    und damit die unbedingte Konvergenz.

  \item
    (i) $\Rightarrow$ (ii)
      Für alle $\phi_k\in\Phi$ ist
      \[ \Big(\underbrace{x-\sum(x\vert\phi_j)\phi_j}_{=:y}\Big\vert\phi_k\Big)=0\, . \]
      Damit ist $y\in\Phi^\perp=\{0\}$, da $\Phi^\perp$ eine Orthonormalbasis ist.
      
   (ii) $\Rightarrow$ (i)
         Sei $x\in\Phi^\perp$. Dann ist $(x\vert\phi_j)=0$ für alle $j$. Wegen (ii) ist dann $x=0$.
      
      (ii) $\Lra$ (iii) Es gilt
        \begin{dmath*}
          \Norm{x-\sum(x\vert\phi_j)\phi_j}^2
          \underset{\
            \begin{subarray}{c}
              \text{ONS} \\
              \text{vgl. (a)}
            \end{subarray}
          }=\norm x^2-\sum\abs{(x\vert\phi_j)}^2,
        \end{dmath*}
        damit folgt die Behauptung. \qedhere
  \end{enumerate}
\end{proof}

\begin{bem}
  \label{bem:7.13}
  Jeder Hilbert-Raum besitzt eine Orthonormalbasis. Allerdings muss diese nicht notwendigerweise abzählbar sein. Jedes Orthonormalsystem kann zu einer Orthonormalbasis ergänzt werden (Gram-Schmidt). Es gilt: hat $H$ genau dann eine abzählbare Orthonormalbasis, wenn $H$ separabel ist, d.h. es existiert eine abzählbare dichte Teilmenge (z.B.\ ist $L_s(\Omega)$ separabel). Lemma~\ref{lemma:7.11} bzw.\ Theorem~\ref{theorem:7.12} gelten auch dann, wenn $\Phi$ nicht abzählbar ist.
\end{bem}

\section{Schwache Lösungen des Laplace-Dirichlet-Pro-blems}

Wir setzen voraus, dass $\Omega\subset\R^n$ ein $C^2$-Gebiet ist.

Wir betrachten das \idx{Dirichlet-Problem}
\begin{align*}
  -\Delta u& =f\condition{in $\Omega$}, \\ u\rvert_{\partial\Omega}&=g
\end{align*}
mit $f\in C(\bar\Omega)$ und $g\in C(\partial\Omega)$. Eine klassische Lösung $u$ liegt in $C^2(\Omega)\cap C(\bar\Omega)$.

\begin{bem}
  \label{bem:7.14}
  \begin{enumerate}[(a)]
  \item \label{bem:7.14-1} Für $n\geq2$ hat das Dirichlet-Problem im Allgemeinen keine klassische Lösung $u\in C^2(\Omega)\cap C(\bar\Omega)$.
    \begin{proof}
      Sei $n=2, \Omega:=\frac 12\B(0,1)\subset\R^2$ und 
       \begin{align*}
        f(x):=\begin{cases}
           \frac{-x_1x_2}{\abs x^2}\left(\frac 4{\log\abs x}-\frac 1{(\log\abs x)^2}\right) &,x\neq 0 \\
           \qquad 0 &,x=0
         \end{cases}.
       \end{align*}
       Dann ist $f\in C(\bar\Omega)$. Weiter sei $u:=x_1x_2(\log\abs{\log\abs x}-\log\log 2)$ mit $x\in\Omega$. Dann ist $u\in C(\bar\Omega), u\rvert_{\partial\Omega}=0, \partial_ju\in C(\Omega), \partial_j^2 u\in C(\bar\Omega)$ und $-\Delta u=f$ in $\Omega$. Jedoch ist
       \begin{dmath*}
         \partial_1\partial_2u(x)=\log\abs{\log\abs x}-\log\log 2-
         \left(\frac{x_1x_2}{\abs x^2}\right)^2
         \left(\frac1{\log\abs x^2}+\frac2{\log\abs x}\right)
         =\log\abs{\log\abs x}+O(1).
       \end{dmath*}
       Also ist $\partial_1\partial_2u\notin C(\Omega)$, d.h.\ $u\notin C^2(\Omega)$.
   

     Annahme: Es existiere ein $v\in C^2(\Omega)\cap C(\bar\Omega)$ mit  $-\Delta v=f$ in $\Omega$ und $v\rvert_{\partial\Omega}=0$. Es sei $w:=u-v$. Da $w$ harmonisch ist, ist $-\Delta w=0$ in $\Omega$ mit $w\rvert_{\partial\Omega}=0$ und $w\in C^\infty(\Omega)$. Damit ist
     \[
      0=\int w\underbrace{\Delta w}_{=0}\d x\underset{\scriptsize\text{Gauß}}=\int_{\partial\Omega}\underbrace{w}_{=0}\partial_\nu w\d\sigma-\int_\Omega\abs{\nabla w}^2\d x\, .
      \]
      Somit ist $\abs{\nabla w}^2=0$ in $\Omega$ und damit $w\equiv\text{const}$. Also ist $u=v\in C^2(\Omega)$. Dies ist ein Widerspruch.
        \end{proof}

  \item \label{bem:7.14-2} Abhilfemöglichkeiten:
    \begin{enumerate}[(i)]
    \item Verlange mehr Regularität: $a>0$, $f\in C^\infty(\Omega)$. Ist $\Omega\in C^3$, so existiert dann genau eine klassische Lösung $u\in C^2(\Omega)\cap C(\bar\Omega)$. Das ist die "`klassische"' oder "`Schaudersche Theorie"'. \index{Theorie|Schaudersche}\index{Theorie|klassische}
      
    \item Distributionelle Lösungen: $-\Delta u=f$ in $D'(\Omega)$. Dann ist $u=\mathcal{N}\ast f$ mit dem Newtonpotential $\mathcal{N}$ und es ist $f\in\Lloc(\Omega)$. Es ergibt sich aber das Problem, wie $u\rvert_{\partial\Omega}$ zu interpretieren ist.


    \item \label{bem:7.14-2-3} Variationeller Zugang: \\
      Es sei $f\in L_1(\Omega)$ und $u(x)$ sei die Auslenkung einer homogenen elastischen Membran unter dem Einfluss der Kraftdichte $f$, wobei für $f\equiv 0$ gerade $\Omega$ ausgefüllt wird. Die Membran sei fest eingespannt auf $\partial\Omega$. Wir nehmen an, dass die innere Spannungsenergie $E$ proportional zum Flächenzuwachs ist.
      \begin{figure}[ht!]
        \centering
        \begin{pspicture}(-3,-3)(3,3)
          \psset{Beta=15}
          
          % Membran
          \pstThreeDCircle(0,0,0)(2,2,0)(-2,2,0)
          \pstThreeDEllipse[beginAngle=0,endAngle=180](0,0,0)(2,-2,0)(0,0,-2)
          \pstThreeDLine[arrows=|-|](2.5,-2.5,0)(2.5,-2.5,-2)
          \pstThreeDPut[origin=rt](2.9,-2.9,-1){\Large $u(x)$}
          \pstThreeDPut(-2.4,2.4,0){\Large$\Omega$}

          % Kraft
          \pstThreeDLine[arrows=->,arrowscale=3](0,0,2)(0,0,0)
          \pstThreeDPut[origin=lt](-0.2,0.2,1.3){\Large $f$}
        \end{pspicture}
        \caption{Membran}
      \end{figure}
      Der Flächeninhalt ist
      \[ \int_\Omega\sqrt{1+\abs{\nabla u}^2}\d x \, . \]
      Damit ist
      \[ E=c\cdot\int_\Omega(\sqrt{1+\abs{\nabla u}^2}-1)\d x\, . \]
      Nun ist $\sqrt{1+\xi}=1+\frac 12 \xi+\sigma(\xi)$ mit $\xi\ra 0$ (mit  Taylor für $\sqrt x$ in $x_0 = 1$). Also gilt für kleine Auslenkungen
      \[ E\approx \frac c2\int_\Omega\abs{\nabla u}^2\d x\, . \]
      Außerdem ist 
      \[ E_{\text{pot}}=c\cdot\int_\Omega\left(\frac 12\abs{\nabla u}^2-f\cdot u\right)\d x. \]
      Es ergibt sich das \idx{Variationsproblem}: Gesucht ist $u\in C^1(\Omega)$ mit $u(\partial\Omega)=0$ und
      \begin{align}
      \begin{aligned}
        \label{eq:7.VP}
        J(u):=\int_\Omega&\left(\frac 12\abs{\nabla u}^2-f\cdot u\right)\d x\hiderel \leq J(v)\, , \\
        &%\textnormal{für alle } 
        \forall \, v\in C^1(\bar\Omega) \textnormal{ mit } v\rvert_{\partial\Omega}=0\, . 
      \end{aligned}
      \end{align}
     
      Wir machen nun folgende Annahme: $u$ sei eine Lösung des Variationsproblems und $M:=\{v\in C^1(\bar\Omega)\with v\rvert_{\partial\Omega}=0\}$. Für alle $v\in M$ sei $\varphi(t):=J(u+tv)$, $t\in\R$. Dann hat $\varphi:\R\ra\R$ in $t=0$ ein Minimum, d.h.\ $\dot\varphi(0)=0$. Wegen $\abs{\nabla(u+tv)}^2=\abs{\nabla u}^2+2t\nabla u\cdot\nabla v+t^2\cdot \abs{\nabla v}^2$ ist  
      \begin{align}
        \label{eq:7.plus}
        \begin{aligned}
        \varphi(t)&=J(u+tv) \\
        & =J(u)+t\cdot\int_\Omega(\nabla u\cdot\nabla
        v-fv)\d x + \frac{t^2}2\cdot\int_\Omega\abs{\nabla v}^2\d x\, .
      \end{aligned}
      \end{align}
      Wegen 
      \[ 0=\dot\varphi(0)=\int_\Omega(\nabla u\cdot\nabla v-fv)\d x\quad\fa \, v\in C^1(\bar\Omega)\, ,\quad v\rvert_{\partial\Omega}=0  \]
    ist
    \begin{align}
      \label{eq:7.3}
      \int_\Omega\nabla u\cdot\nabla v\d x=\int_\Omega fv\d x\quad\fa\,  v\hiderel \in C^1(\bar\Omega)\, 
      \condition{$v\rvert_{\partial\Omega}=0$}\, .
    \end{align}

    Andererseits folgt aus \eqref{eq:7.plus}, dass $u\in M$ Lösung von \eqref{eq:7.3} ist. Damit ist $u$ auch Lösung von \eqref{eq:7.VP}.


    Wir nehmen nun an, dass $u$ eine Lösung von \eqref{eq:7.VP} mit $u\in C^2(\bar\Omega)$ sei. Dann ist
    \begin{align}
      \label{eq:7.4}
      \nabla u\cdot\nabla v=\div(v\cdot\nabla u)-v\cdot\Delta u
    \end{align}
    und damit
    \begin{align}
      \label{eq:7.5}
      \begin{aligned}
      \int_\Omega fv\d x
     & \  \,  =\ \, \int_\Omega\nabla u\cdot\nabla v\d x \\
      & \underset{\scriptsize\text{Gauß}}=\int_{\partial\Omega}\underbrace{v}_{=0}\partial_\nu u\d\sigma
      -\int_\Omega v\Delta u\d x \, ,
      \end{aligned}
    \end{align}
    für alle $v\in M$. Also ist
    \begin{dmath*}
      \int_\Omega(-\Delta u-f)v\d x=0
      \condition{für alle $v\in M\subset \D(\Omega)$}\, .
    \end{dmath*}
    Mit Theorem~\ref{theorem:3.6} ist dann
    \begin{dmath*}
      -\Delta u=f\text{ in }\Omega
      \condition{$u\rvert_{\partial\Omega}=0$ mit $f\in L_1(\Omega)$}\, ,
    \end{dmath*}
    d.h.\ $u\in C^2(\bar\Omega)$ ist eine klassische Lösung.

    Andererseits ist $u\in C^2(\Omega)\cap C(\bar\Omega)$ eine klassische Lösung von
    \begin{dmath*}
      -\Delta u=f\text{ in }\Omega\condition{$u\rvert_{\partial\Omega}=0$}\, .
    \end{dmath*}
    Dann gilt für alle $v\in\D(\Omega)$
    \begin{align*}
      \int_\Omega fv\d x = \int_\Omega -\Delta u\cdot v\d x
      \underset{\scriptsize\text{Gauß},~\eqref{eq:7.4}}=
      \int_\Omega\nabla u\cdot\nabla v\d x\, ,
    \end{align*}
    d.h.\ $u$ löst \eqref{eq:7.3}, also \eqref{eq:7.VP}.
    \end{enumerate}
  \end{enumerate}
\end{bem}

\begin{defi}
  $u$ heißt \idx{schwache Lösung} des Laplace-Dirichlet-Problems
  \begin{align}
    \label{eq:7.DP}
    -\Delta u=f\text{ in }\Omega\condition{$u\rvert_{\partial\Omega}=0$} \tag{LDP}
  \end{align}
  genau dann, wenn $u\in \mathring H^1(\Omega)=\mathring W^1_2(\Omega)$ mit
  \begin{dmath*}
    \int_\Omega\nabla u\cdot\nabla\varphi\d x=\int_\Omega f\cdot\varphi\d x
    \condition{für alle $\varphi\in\D(\Omega)$}\, .
  \end{dmath*}
\end{defi}

\begin{bem}
  \label{bem:7.15}
  \begin{enumerate}[(a)]
  \item \[ \delta J(u)v:=\lim_{t\ra 0}\frac 1t\cdot(J(u+tv)-J(v)) \]
    heißt erste Variation von $J$ im Punkt $u$ in Richtung $v$. Damit ist
    \[ \delta  J(u) v=\int_\Omega(\nabla u\cdot\nabla v-fv)\d x\, . \]
    Es gilt: $\delta J(u)\varphi=0$ für alle $\varphi\in\D(\Omega)$ genau dann, wenn $u$ eine schwache Lösung von \eqref{eq:7.DP} ist. Das Minimieren des Funktionals $J:M\ra \R$ heißt \idx{Dirichletsches Prinzip}.
    \begin{proof}
      Bemerkung~\ref{bem:7.14} (b) (iii).
    \end{proof}
  \item
    \begin{itemize}
    \item Ist $u\in C^2(\bar\Omega)$ eine schwache Lösung von \eqref{eq:7.DP}, dann ist $u$ eine klassische Lösung.
    \item Ist $u\in C^2(\Omega)\cap C(\bar\Omega)$ eine klassische Lösung, dann ist $u$ schwache Lösung.      
    \end{itemize}
    \begin{proof}
      Siehe auch Bemerkung~\ref{bem:7.14} (b) (iii).
    \end{proof}
  \item Ist $u$ schwache Lösung von \eqref{eq:7.DP}, dann ist $u$ distributionelle Lösung von $-\Delta u=f$ (d.h.\ in $\D'(\Omega)$).
    \begin{proof}
      Definition der distributionellen Ableitung.
    \end{proof}
  \item Ist $u\in\mathring H^1(\Omega)$, so folgt aus Bemerkung~\ref{bem:6.17}~(\ref{bem:6.17-1}) dass $\spur(\gamma_0 u)=0$ (d.h.\ $u$ erfüllt die Randbedingungen $u\rvert_{\partial\Omega}=0$ im Sinne der Spur).
  \end{enumerate}
\end{bem}

Wir verfolgen nun folgende Strategie: (zum Lösen elliptischer Probleme)
\begin{enumerate}[(1)]
\item Konstruiere schwache Lösung.
\item Regularität $\ra$ Klassische Lösung (z.B.\ $u\in H^2(\Omega)\cap\mathring H^1(\Omega)$, $-\Delta u=f$ in $L_p(\Omega)$).
\end{enumerate}

\begin{theorem}
 \label{theorem:7.16}
 Sei $\Omega\subset\R^n$ offen und beschränkt $($in einer Richtung$)$. Dann existiert für alle $f\in L_2(\Omega)$ genau eine schwache Lösung $u\in\mathring H^1(\Omega)$ von $-\Delta u=f$ in $\Omega$, $u\rvert_{\partial\Omega}=0$.
\end{theorem}

\begin{proof}
   Aus Korollar~\ref{kor:6.10} folgt, dass $(u,v)\mapsto(\nabla u\vert\nabla v)_{L_2}$ ein äquivalentes Skalarprodukt auf $\mathring H^1(\Omega)$ definiert. Sei $F(v):=(v\vert \bar f)_{L_2(\Omega)}$ mit $v\in\mathring H^1(\Omega)$ für $f\in L_2(\Omega)$ fest. Nun ist $F:\mathring H^1(\Omega)\ra\K$ linear und es gilt
   \begin{dmath*}
     \abs{F(v)}\leq\norm v_{L_2(\Omega)}\norm {\bar f}_{L_2(\Omega)}
     \leq c\cdot\norm f_{L_2(\Omega)}\cdot\norm v_{\mathring H^1(\Omega)}\, .
   \end{dmath*}
   Also ist $F\in\L(\mathring H^1(\Omega), \K)=\mathring H^1(\Omega)'$. Nach Theorem~\ref{theorem:7.8} existiert genau ein $u\in\mathring H^1(\Omega)$ mit $F(v)=(u|v)_{\mathring H^1(\Omega)}=(\nabla v\vert\nabla u)_{L_2(\Omega)}$ für alle $v\in\mathring H^1(\Omega)$. Wegen $\D(\Omega)\subset\mathring H^1(\Omega)$ ist
   \begin{dmath*}
     \int_\Omega\nabla v\cdot\nabla \bar u\d x=(\nabla v\vert\nabla u)_{L_2}
     \hiderel =F(v)\hiderel =(v\vert f)_{L_2}
     = \int_\Omega f\bar v\d x
     \condition{für alle $v\in\D(\Omega)$}\, .
   \end{dmath*}
   Damit ist
  \begin{dmath*}
    \int_\Omega\nabla\bar v\cdot\nabla u\d x=\int_\Omega\nabla f\d x
    \condition{für alle $\bar v\in\D(\Omega)$}\, .
  \end{dmath*}
\end{proof}

\begin{bem}
  \label{bem:7.17}
  \begin{enumerate}[(a)]
  \item Theorem~\ref{theorem:7.16} bleibt richtig für $f\in\mathring H^1(\Omega)$.
  \item Regularität: Man kann folgendes zeigen für die schwache Lösung $u$:
    \begin{itemize}
    \item Sind $m\in\N$, $f\in H^m(\Omega)$, dann ist $u\in H^{m+2}(\Omega)$, d.h. $u$ ist eine klassische Lösung.
    \item Speziell: Ist $f\in C^\infty(\bar \Omega)$, so ist $u\in C^\infty(\bar\Omega)$.
    \end{itemize}
    Literatur: Evans, Jost.
  \item $C^1(\bar\Omega)$ ist kein Hilbertraum, d.h. Riesz ist nicht anwendbar.
  \end{enumerate}
\end{bem}

\section{Lineare elliptische Differentialoperatoren 2. Ordnung}

Im folgenden setzen wir voraus, dass $\Omega\subset\R^n$ offen, beschränkt und glatt ist und für $a_{jk}=a_{kj}\hiderel\in C^1(\bar\Omega, \R)$, $b_j,c\in C(\bar\Omega,\R)$ ist
\begin{align*}
  A(x,D)u\coloneqq -\sum_{j,k=1}^n\partial_k(a_{jk}(x)\partial_ju)
  + \sum_{j=1}^nb_j(x)\partial_ju+c(x)u\, .
\end{align*}

Für Elliptizität muss ein $\alpha>0$ mit
\begin{dmath*}
  \sum_{j,k=1}^na_{jk}(x)\xi^j\xi^k\geq\alpha\abs\xi^2
  \condition{$\, \fa\, \xi=(\xi^1,\ldots,\xi^n)\in\R^n\,  \fa\,  x\in\bar\Omega$}
\end{dmath*}
existieren.

Beachte: $a(x):=[a_{jk}(x)]_{1\leq j,k\leq n}\in\R^{n\times n}$ ist symmetisch und $b(x):=[b_j(x)]_{1\leq j\leq n}\in\R^n$. Weiter ist
\begin{align*}
  A(x,D)u\coloneqq\underbrace{-\div (a(x)\cdot\nabla u)}_{\text{"`Hauptteil"'}}
    +\underbrace{b(x)\cdot\nabla u}_{\text{"`Transport"'}}+c(x)u\, .
\end{align*}
Für die Elliptizität muss $a(x)\xi\cdot\xi\geq\alpha\abs\xi^2$ für alle $\xi\in\R^n$ und $x\in\bar\Omega$ gelten. D.h.\ $a$ ist positiv definit.

$A(\cdot, D)$ ist ein gleichmäßig stark elliptischer linearer Differentialoperator zweiter Ordnung mit Hauptteil in Divergenzform.

Seien $a_{jk}=\delta_{jk}$, $b_j\equiv 0$ und $c\equiv 0$. Dann ist $A(x,D)=-\Delta$.

Im Folgenden nehmen wir an, $u$ sei eine "`glatte"' Lösung (d.h.\ zum Beispiel sei $u\in C^2(\Omega)\cap C(\bar\Omega)$ oder $u\in H^2(\Omega)\cap \mathring H^1(\Omega)$). Dann ist für alle $v\in D(\Omega)$
\begin{dmath*}
  \int_\Omega\bar fv\d x=\bar\lambda\int\bar u\cdot v\d x
  -\underbrace{\int_\Omega v\cdot\div(a(x)\nabla\bar u)\d x}_{ 
    \underset{\text{Gauß}}=\int_\Omega\nabla v\cdot a(x)\nabla \bar u\d x+\int_{\partial\Omega}0
  }
  +\int_\Omega vb(x)\cdot\nabla\bar u\d x+\int_\Omega vc(x)\bar u\d x\, ,
\end{dmath*}
d.h.\
\begin{dmath*}
  \bar\lambda(v\vert u)_{L_2}+(\nabla u\vert a(\cdot)\nabla u)_{L_2}
  +(v\vert b(\cdot)\cdot\nabla u+cu)_{L_2}
  = (v\vert f)_{L_2}
\end{dmath*}
für alle $v\in D(\Omega)$ mit $(v\vert\omega)_{L_2}:= \int_\Omega v\bar\omega\d x$.

\begin{defi}
  Es gelten die obigen Voraussetzungen und weiter seien $u,v\in H^1(\Omega)$. Dann ist
  \begin{align*}
    a(v,u)&\coloneqq (\nabla v\vert a(\cdot)\cdot\nabla u)_{L_2}+(v\vert b(\cdot)\cdot\nabla u+cu)_{L_2} \\
    & =\sum_{j,k=1}^n\int_\Omega\partial_jv\, a_{jk}(x)\partial_k\bar u\d x
    +\sum_{j=1}^n\int_\Omega v\,  b_j(x)\partial_j\bar u\d x
    +\int_\Omega v\, c(x)u\d x\, .
  \end{align*}
  Speziell ist für $a_{jk}=\delta_{jk}$, $c\equiv 0$ und $b\equiv 0$
  \[ a(v,u)=\int_\Omega\nabla v\cdot\nabla\bar u\d x\, , \]
  und wir definieren
  \[ a_\lambda(v,u):=a(v,u)+\bar\lambda(v\vert u)_{L_2}\, . \]
\end{defi}

\subsection{Dirichletproblem}

\begin{defi}
  Sei $f\in L_2(\Omega)$ und $\lambda\in\K$. Dann heißt $u$ "`schwache Lösung"' von
  \begin{align}
    \label{eq:7.D}
    (\lambda+A(\, \cdot\, ,D))u=f\quad\text{in}\;\Omega
    \condition{$u\rvert_{\partial\Omega}=0$}\, , \tag{DP}
  \end{align}
  wenn $u\in\mathring H^1(\Omega)$ mit $a_\lambda(v,u)=(v\vert f)_{L_2(\Omega)}$ für alle $v\in\D(\Omega)$.

  Beachte: ist $u\in\mathring H^1(\Omega)$, so ist  $\spur\gamma_0 u=0$.
\end{defi}

Im Folgenden verwenden wir die Notationen $c^+:=\max\{c, 0\}$, $c^-:=\max\{-c, 0\}\geq0$ und $c=c^+-c^-$ für ein $c\in\R$.

\begin{theorem}
  \label{theorem:7.18} Es gelten die obigen Voraussetzungen und es sei $f\in L_2(\Omega)$. Dann ist für alle $\lambda\in\K$ mit $\Re\lambda\geq\norm{c^-}_\infty+\frac1{2\alpha}\norm{b}^2_\infty$ besitzt \eqref{eq:7.D} eine eindeutige schwache Lösung $u=u(f)$. Ferner ist $[f\mapsto u(f)]\in\L(L_2(\Omega),\mathring H^1(\Omega))$.
\end{theorem}

\begin{proof}
  Mit der Cauchy-Schwartzschen Ungleichung erhalten wir
  \begin{dmath*}
    \abs{a_\lambda(v,u)}\leq\norm{a}_\infty\norm{\nabla v}_{L_2}\norm{\nabla u}_{L_2}+\norm{b}_\infty\norm{v}_{L_2}+\norm{\nabla u}_{L_2}+\norm{c}_\infty\norm{v}_{L_2}\norm{u}_{L_2}+\abs\lambda\norm{v}_{L_2}\norm{u}_{L_2}
  \leq\norm{u}_{\mathring H^1}\norm{v}_{\mathring H^1}.
  \end{dmath*}
  Also ist $a_\lambda:\mathring H^1(\Omega)\times\mathring H^1(\Omega)\ra\K$ eine stetige Sesquilinearform für alle $\lambda\in\K$.

  Ferner ist für $u=u_1+iu_2$
  \begin{align}
  \label{eq:7.6}
  \begin{aligned}
    \sum_{j,k=1}^n\Re a_{jk}\partial_k\bar u\partial_j u
   &\  \stackrel[a_{jk}\in \mathbb{R}]{}= \ \sum_{j,k=1}^na_{jk}(\partial_ku_1\partial_ju_1+\partial_ku_2\partial_ju_2)\\
  & \underset{\scriptsize\text{elliptisch}}\geq \alpha\abs{\nabla u_1}^2+\alpha\abs{\nabla u_2}^2-\alpha\abs{\nabla u}^2.
    \end{aligned}
  \end{align}
  Damit ist
  \begin{align*}
    \Re a_\lambda(u,u)& \ = \ \sum_{j,k=1}^n\Re (\partial_ju\vert a_{jk}\partial_k u)_2+\Re(u\vert b\cdot\nabla u)_2
    +\Re(u\vert (\lambda+c)u)_2 \\
 &   \underset{\scriptsize\eqref{eq:7.6}}\geq\alpha\norm{\nabla u}_{L_2}^2
    -\underbrace{\norm b_\infty\norm u_{L_2}\norm{\nabla u}_{L_2}}_{
      \underset{\text{Young}}=\frac \alpha2\norm{\nabla u}^2_{L_2}+\frac1{2\alpha}\norm{b}_\infty^2\norm{u}_{L_2}^2
    }+(\Re\lambda-\norm{c^-}_\infty)\norm u_{L_2}^2  \\
    & \ \geq\ \frac 12\alpha\norm{\nabla u}_{L_2}^2-\left(
      \frac 1{2\alpha}\norm b_\infty^2+\norm{c^-}_\infty-\Re\lambda
    \right)\norm u^2.
  \end{align*}
  Korollar~\ref{kor:6.10} liefert uns nun
  \[ \Re a_\lambda(u,u)\geq\frac 12\alpha\norm{\nabla u}_{L_2}^2\geq c_1\norm u_{L_2}^2 \]
  mit $c_1>0$ für alle $u\in\mathring H^1(\Omega)$. Damit ist $a_\lambda:\mathring H^1(\Omega)\times\mathring H^1(\Omega)\ra\K$ koerziv. Ferner ist $(\, \cdot\, \vert f)_{L_2}\in\L(\mathring H^1(\Omega),\K)=(\mathring H^1(\Omega))'$.

  Mit Theorem~\ref{theorem:7.10} folgt dann die Behauptung.
\end{proof}

\begin{bsp*}
  Das Problem
  \begin{dmath*}
    (\lambda-\Delta)u=f\quad\text{in}\;\Omega
    \condition{$u\rvert_{\partial u}=0$}
  \end{dmath*}
  hat für alle $\Re\lambda\geq0$ eine eindeutige schwache Lösung.
\end{bsp*}

\subsection{Neumannproblem}

Es sei $\nu_a(x):=a(x)\nu(x)$ mit $a=[a_{jk}]$. Dann ist 
\[ \nu_a\cdot\nu\geq\alpha\abs\nu^2=\alpha>0\, , \]
d.h.\ $\nu_a$ ist nirgends tangential an $\partial\Omega$. $\partial_{\nu_a}u:=\nabla u\cdot\nu_a$ wird "`Konormalen-ableitung"' \index{Konormalenableitung} genannt.

\begin{figure}[h]
  \centering
  \begin{pspicture}(-2,-2)(2,2)
    % Omega
    \psccurve(-2,.5)(0,1.2)(1,.7)(1.5,0)(1,-2)(-.5,-.5)
    \rput[l](0,0){$\Omega$}

    % x
    \psdot(1,.7)
    \rput[tr](.9,.6){$x$}
    \psline{->}(1,.7)(1.8,1.5)
    \rput[r](.9,1.4){$\nu_a$}
    \psline{->}(1,.7)(1,1.7)
    \rput[l](1.6,1.1){$\nu(x)$}
  \end{pspicture}
  \caption{Konormalenableitung}
\end{figure}

\begin{defi}
  Es sei $f\in L_2(\Omega)$ und $\lambda\in\K$. Dann heißt $u$ schwache Lösung des Neumannproblems
  \begin{align}
    \label{eq:7.N}
    (\lambda-A(\, \cdot \, ,D))u=f\;\text{in}\;\Omega
    \condition{$\partial_{\nu_a}u=0$ auf $\partial\Omega$} \tag{NP}
  \end{align}
  genau dann, wenn $u\in\mathring H^1(\Omega)$ und $a_\lambda(v,u)=(v\vert f)_{L_2}$ für alle $v\in H^1(\Omega)$.
\end{defi}

\begin{theorem}
  \label{theorem:7.19} Es gelten die obigen Voraussetzungen und es sei $f\in L_2(\Omega)$. Dann besitzt das Problem \eqref{eq:7.N} für alle $\lambda\in\K$ mit $\Re\lambda>\norm{c^-}_\infty+\frac 1{2\alpha}\norm b_\infty^2$ eine eindeutige schwache Lösung $u=u(f)$. Ferner ist $[f\mapsto u(f)]\in\L(L_2(\Omega),$ $H^1(\Omega))$.
\end{theorem}

\begin{proof}
  Vgl. Beweis zu Theorem~\ref{theorem:7.18}: $a_\lambda:H^1(\Omega)\times H^1(\Omega)\ra\K$ ist eine stetige Sesquilinearform mit
  \begin{dmath*}
    \Re a_\lambda(u,u)\geq\frac 12\alpha\norm{\nabla u}_{L_2}^2+
    \underbrace{(\Re\lambda-\norm{c^-}_\infty+\frac1{2\alpha}\norm b_\infty^2)}_{=:q>0} \norm{u}^2_{L_2}
    \geq\min\{\frac 12\alpha,q\}(\underbrace{\norm{\nabla u}_{L_2}^2+\norm u_{L_2}^2}_{\norm u_{H^1(\Omega)}^2})
  \end{dmath*}
  für alle $u\in H^1(\Omega)$. Damit folgt analog zu Theorem 7.18 die Behauptung.
\end{proof}

\begin{bsp*}
  Das Problem
  \begin{dmath*}
    (\lambda-\Delta)u=f\;\text{in}\;\Omega
    \condition{$\partial_{\nu_a} u=0$ auf $\partial\Omega$}
  \end{dmath*}
  hat für alle $\Re\lambda>0$ und $f\in L_2(\Omega)$ eine eindeutige schwache Lösung.

  Beachte: für $\lambda=0$ und $f\equiv0$ ergibt sich das Problem
  \begin{dmath*}
    -\Delta u=0\;\text{in}\;\Omega
    \condition{$\partial_\nu u=0$ auf $\partial\Omega$}\, .
  \end{dmath*}
  Dies hat die Lösungen $u=\R\cdot\mathds{1}$.
\end{bsp*}

\begin{bem}
  \label{bem:7.20}
  \begin{enumerate}[(a)]
  \item \label{bem:7.20-1} Sei $u\in C^2(\Omega)\cap C(\bar\Omega)$ (bzw.\ $u\in H^2(\Omega)$ + Randbed.) eine klassische Lösung der Probleme \eqref{eq:7.D} bzw.\ \eqref{eq:7.N}. Dann ist $u$ ebenfalls eine schwache Lösung.
    \begin{proof}
      Übung.
    \end{proof}
  \item Die Matrix $a$ habe $C^\infty(\bar\Omega)$-Koeffizienten und $u$ sei schwache Lösung von \eqref{eq:7.D} bzw.\ \eqref{eq:7.N}. Dann ist $u$ auch distributionelle Lösung.
  \begin{proof}
  Übung.
  \end{proof}
  \item Ist $u$ schwache Lösung von \eqref{eq:7.D} bzw.\ \eqref{eq:7.N} und ist $f\in L_2(\Omega)$, so ist $u\in H^2(\Omega)$. Ist $f\in C^\infty(\bar\Omega)$ mit $C^\infty(\bar\Omega)$-Koeffizienten von $a$, so ist $u\in C^\infty(\bar\Omega)$ auch klassische Lösung.
    \begin{proof}
      Vgl.\ Evans.
    \end{proof}
  \end{enumerate}
\end{bem}

\begin{kor}
  \label{kor:7.21} Es gelten die Voraussetzung von Theorem~\ref{theorem:7.18} bzw.\ Theorem~\ref{theorem:7.19} und $u(f)$ sei die entsprechende Lösung. Dann ist $[f\mapsto u(f)]\in\mathscr{K}(L_2(\Omega))$, d.h.\ $[f\mapsto u(f)]\in\L(L_2(\Omega)):=\L(L_2(\Omega),L_2(\Omega))$ und beschränkte Mengen $($in $L_2(\Omega))$ werden auf relativ kompakte Mengen $($in $L_2(\Omega))$ abgebildet.
\end{kor}

\begin{proof}
  Es ist
  \[ 
  \mathring H^1(\Omega)
  \underset{\scriptsize\text{Def.}}\hookrightarrow H^1(\Omega) 
  \underset{\scriptsize\text{Bem.~\ref{bem:6.14}}}\hhookrightarrow L_2(\Omega)
  \]
  und
  \[ [f\mapsto u(f)]\in\L(L_2(\Omega),H^1(\Omega))\subset\mathscr{K}(L_2(\Omega),L_2(\Omega))=\mathscr K(L_2(\Omega))\, . \qedhere \]
\end{proof}

\subsection{Variationeller Zugang}

\begin{lemma}
  \label{lemma:7.22} Es sei $H$ ein reeller Hilbertraum, $a:H\times H\ra\R$ eine stetige koerzive Bilinearform, $L\in H'=\L(H,\R)$ und $V$ ein abgeschlossener Untervektorraum von $H$. Dann existiert genau ein $u_0\in V$ mit
  \begin{dmath*}
    a(u_0,\varphi)+a(\varphi,u_0)+L\varphi=0
    \condition{für alle $\varphi\in V$}.
  \end{dmath*}
\end{lemma}

\begin{proof}
  $(u,v)\mapsto a(u,v)+a(v,u)$ ist eine stetig, koerziv und bilinear auf dem Hilbertraum $V$. Nun ist $-L\in V'=\L(V,\R)$. Die Behauptung folgt dann aus Theorem~\ref{theorem:7.10} (Lax-Milgram).
\end{proof}

Falls $a$ symmetrisch ist, so ist $a(\varphi,u_0)=-\frac 12L\varphi$ für alle $\varphi\in V$ (z.B.\ schwache Lösung von $(\lambda+A)u=f$, falls $b\equiv0)$.

\begin{kor}[Dirichletsches Prinzip]
  \label{kor:7.23} Es gelten die Voraussetzungen von Lemma~\ref{lemma:7.22} und es sei $J(u):=a(u,u)+Lu$ mit $u\in V$. Dann existiert genau ein $u_0\in V$ mit $J(u_0)=\inf_{v\in V}J(v)$.
\end{kor}

\begin{proof}
  Sei $u_0$ wie in Lemma~\ref{lemma:7.22}. Dann ist
  \begin{dmath*}
    J(u_0+t\varphi)=J(u_0)+\underbrace{t^2 a(\varphi,\varphi)}_{\geq0 \, \text{(koerziv)}}
    +t(\underbrace{a(u_0,\varphi)+a(\varphi,u_0)+L\varphi}_{=0})
    \geq J(u_0)
    \condition{$\fa\,  t\in\R, \varphi\in V$} .
  \end{dmath*}
  Damit ist $u_0$ ein Minimum.

  Zum Beweis der Eindeutigkeit sei $v_0\in V$ ebenfalls Minimum. Dann ist
  \[ 
  0=\left.\diff{t}J(v_0+t\varphi)\right\vert_{t=0}
  =a(v_0,\varphi)+a(\varphi,v_0)+L\varphi\, .
  \]
  Aus Lemma~\ref{lemma:7.22} folgt dann, dass $v_0=u_0$ ist.
\end{proof}

\begin{theorem}
  \label{theorem:7.24} Es seien $H$ ein Hilbertraum, $a:H\times H\ra\R$ stetig, koerziv und bilinear, $L\in H'$ und $J(v):=a(v,v)+L(v)$ für $v\in H$. Außerdem sei $(V_n)_{n\in\N}$ eine Folge von abgeschlossenen Untervektorräumen von $H$ mit $V_n\subset V_{n+1}$ und
  \[ \fa \, v\in H\;\fa\, \delta>0\;\exists\,  n\in\N\;\exists\,  v_n\in V_n:\norm{v-v_n}<\delta\, . \]
  Sei nun $u_n\in V_n$ die eindeutige Lösung von $J(u)=\inf_{v\in V_n}J(v)$ gemäß Korollar~\ref{kor:7.23}. Dann existiert ein $u_*\in H$ mit $u_n\ra u_*$ in $H$ und $u_*$ löst das Variationsproblem $J(u_*)=\inf_{v\in H}J(v)$.
\end{theorem}

\begin{proof}
  \begin{enumerate}[(i)]
  \item Für alle $v\in H$ gilt
    \begin{dmath*}
      J(v)=a(v,v)+Lv\geq\alpha\norm v^2-\norm L_{H'}\norm v\geq -\frac{\norm L_{H'}^2}{4\alpha}\, .
    \end{dmath*}
    Also ist $J$ nach unten beschränkt.
  \item Sei $\kappa:=\inf_{v\in H} J(v)\in\R$. Wir nehmen nun an, es existiere ein $\epsilon>0$ mit $J(u_n)\geq\kappa+\epsilon$ für alle $n\in\N$. Dann gibt es ein $u_0\in H$ mit $J(u_0)<\kappa+\frac\epsilon 2$. Da $J$ stetig ist, existiert ein $\delta>0$ mit $\abs{J(u_0)-J(v)}<\frac\epsilon2$ für alle $v\in H$ mit $\norm{v-u_0}<\delta$. Damit existieren nach Voraussetzung ein $n\in\N$ und ein $v_n\in V_n$ mit $\norm{v_n-u_0}<\delta$. Also ist
    \begin{dmath*}
      J(v_n)\leq\abs{J(v_n)-J(u_0)}+J(u_0)\leq\frac\epsilon2+\kappa+\frac\epsilon2\leq J(u_n)\, ,
    \end{dmath*}
    was ein Widerspruch ist. Somit gilt $J(u_n)\ra\kappa$ und ist eine Minimalfolge.
  \item $(u_n)$ ist eine Cauchy-Folge, denn
   \begin{dmath*}
   0 \leq    \alpha\frac14\norm{u_n-u_m}^2\leq\frac14a(u_n-u_m,u_n-u_m)
      =\frac12J(u_n)+\frac12J(u_m)-J\left(\frac{u_m-u_n}{2}\right)
      \xrightarrow[n,m\ra\infty]{}0\, .
    \end{dmath*}
    Damit existiert ein $u_*\in H$ mit $u_n\ra u_*$. Da $J$ stetig ist, ist $J(u_*)=\lim_{n\ra \infty} J(u_n)=\kappa\leq J(v)$ für alle $v\in H$.\qedhere
  \end{enumerate}
\end{proof}

\begin{bem*}
Theorem~\ref{theorem:7.24} liefert eine konstruktive Methode zur approximativen Lösung des Variationsproblems $J(u) = \inf_{v \in H} J(v)$, wenn die (endliche-dimensionalen) Untervektorräume $V_n$ geeignet gewählt werden (z.B. Polynome).

Ist $\{\varphi^n_1, \ldots, \varphi_{N_n}^n\}$ eine Basis von $V_n$, so genügt es die endlich vielen Gleichungen $a(u_n, \varphi^n_j) + a(\varphi^n_j, u_n) + L\varphi_j^n = 0, j = 1, \ldots, N_n$ zu lösen (vgl. Lemma~\ref{lemma:7.22}).
\end{bem*}

\textit{Beachte.} Die Dirichletformen $a_\lambda$ aus Theorem~\ref{theorem:7.18} bzw. Theorem~\ref{theorem:7.19} erfüllen die Bedingungen von Theorem~\ref{theorem:7.24}.

\section{Schwache Lösungen für nichtlineare Probleme}

Wir setzen in diesem Abschnitt voraus, dass $\Omega \in\R^n$  offen, beschränkt und $C^\infty$ ist, sowie $g \in C(\partial \Omega)$ und eine \idx{Lagrangefunktion} $L\in C^\infty (\R^n \times \R \times \bar\Omega, \R)$ ist.

\begin{notation}
Für die Lagrangefunktion notieren wir $L = L(p,z,x), (p,z,x)\in \R^n\times \R\times \bar \Omega$, sowie $L_p = (L_{p_1}, L_{p_2}, \ldots, L_{p_n})$ für die partiellen Ableitungen (analog für $L_z, L_x$).
\end{notation}

Wir betrachten das Variationsproblem
\[
	J(u) := \int_\Omega L(\nabla u(x), u(x), x) \d x \longrightarrow \min
\]
für $u : \bar\Omega \ra \R$ unter der Nebenbedingung $u \vert_{\partial \Omega} = g$, dabei ist $J$ ein Energiefunktional.

\subsection*{Euler-Lagrange-Gleichung}

Es sei $V:=\{u \in C^\infty (\bar\Omega) \with u |_{\partial \Omega} = g \}$. Wir nehmen an,  $\exists \, u \in V: J(u) = \inf_{v \in V} J(v)$, dann folgt
\[
	\fa \, v \in \D (\Omega) : u + tv \in V \, ,  \quad t \in \R \, .
\]
Wir betrachten die 1. \idx{Variation} von $J$ im Punkt $u$ in Richtung $v$ definiert als
\[
	\delta J(u) v := \lim_{t\ra 0} \frac 1 t (J(u+tv)-J(u)) ,
\]
dann ist $\delta J(u) v = 0 \, \fa \, v \in \D(\Omega)$ (bzw. $\varphi'_v (0) = 0$ mit $\varphi(t) := J(u+tv)$), wenn $u$ ein Minimum von $J$ auf $V$ ist, d.h.
\begin{align*}
&\frac \d {\d t} \varphi_v (t)  = \frac \d{\d t} \int_\Omega L(\nabla u + t \nabla v, u+tv, x) \d x \\
			 \stackrel{\parbox{1.077cm}{\scriptsize$u,v$, \\ $L \in C^\infty$}}=& \int_\Omega \frac \d{\d t} L(\nabla u + t \nabla v, u+tv, x) \d x \\
			=\quad & \int_\Omega \Bigr(\sum_{j=1}^n L_{p_j} (\nabla u + t \nabla v, u+tv, x) v_{x_j} + L_z(\nabla u + t \nabla v, u+tv, x) v \Bigr) \d x \, ,
\end{align*}
also:
\begin{align*}
& \ \,  0  =  \varphi'_v(0) = \int_\Omega \Bigr( \sum_{j=1}^n L_{p_j} (\nabla u, u , x) v_{x_j} + L_z (\nabla u, u, x) v\Bigr) \d x \\
&\stackrel{\parbox{1.2cm}{\center\scriptsize part. Int. \\ $v|_{\partial\Omega} = 0$}}= \int_\Omega \Bigr(-\sum_{j=1}^n(L_{p_j} (\nabla u, u, x))_{x_j} + L_z(\nabla u, u, x)\Bigr) v \d x \quad \fa \, v \in \D(\Omega) \, .
\end{align*}
Damit folgt aus Theorem~\ref{theorem:3.6}
\begin{align*}
\label{eq:ELG}
\tag{ELG} -\sum_{j=1}^n(L_{p_j} (\nabla u, u, x))_{x_j} + L_z(\nabla u, u, x) = 0 \text{ in } \Omega \, ,
\end{align*}
d.h. falls $u \in C^\infty (\bar\Omega)$ das Energiefunktional $J$ minimiert, so erfüllt $u$ die \idx{Euler-Lagrange-Gleichung}.

\begin{bsp}
\begin{enumerate}[(a)]
\item Wir betrachten
\[
	L(p,z,x) := \frac 12\sum_{j,k=1}^n a_{jk} (x) p_j p_k - z f(x)
\]
mit $a_{jk} = a_{kj}$. Dann folgt
\[
	L_{p_j} = \sum_{k=1}^n a_{jk} p_k \, , \quad L_z = -f(x)
\]
und das Energiefunktional $J$ lautet
\[
	J(u)= \int_\Omega \Bigr( \frac 1 2 \sum_{j,k=1}^n a_{jk} (x) \partial_j u \partial_k u - u(x) f(x)\Bigr) \d x \, .
\]
Damit ist die Euler-Lagrange-Gleichung
\[
	-\sum_{j,k=1}^n \partial_j (a_{jk} (x) \partial_k u) = f \text{ in } \Omega \, .
\]
Speziell gilt dann also für $a_{jk} = \delta_{jk}$
\[
	J(u) = \int_\Omega \Bigr( \frac 12 \abs{\nabla u}^2 - fu\Bigr) \d x \, , 
\]
damit ergibt sich als Euler-Lagrange-Gleichung die Laplace-Gleichung $-\Delta u = f$ in $\Omega$ \text{(vgl. Abschnitt 7.2)}.
\item Sei nun
\[
	L(p,z,x) := \frac 1 2 \abs p^2 - F(z) \, , \quad F(z) := \int_0^z f(r) \d r \, .
\]
Dann ist das Energiefunktional
\[
	J(u) = \int_\Omega \Bigr(\frac 1 2 \abs{\nabla u}^2 - F(u) \Bigr) \d x \, .
\]
Damit folgt die Euler-Lagrange-Gleichung $-\Delta u = f(u)$ in $\Omega$, also erhalten wir eine semilineare Laplace-Gleichung, d.h. diese ist nicht lösbar mit Lax-Milgram, da $f$ von $u$ abhängt.
\item Mit $L(p,z,x) = \sqrt{1+\abs p^2}$ ergibt sich
\[
	J(u) = \int_\Omega \sqrt{1+\abs{\nabla u}^2} \d x \, , 
\]
d.h. $J$ berechnet die Fläche des Graphen $u:\Omega \ra \R$ (also wird das $u$ mit minimaler Oberfläche gesucht). Damit erhalten wir die Euler-Lagrange-Gleichung
\[
	-\underbrace{\sum_{j=1}^n \partial_j \left( \frac{\partial_j u}{\sqrt{1+\abs{\nabla u}^2}}\right)}_{\text{Krümmung}} = 0 \text{ in } \Omega \, .
\]
\end{enumerate}
\end{bsp}

\begin{lemma}
\label{lemma:7.26}
Sei $X \subset \R^n$ beschränkt und messbar, sowie $\varphi_j, \varphi \in L_{\infty}(X)$, $\varphi_j \ra \varphi$ gleichmäßig auf $X$. Sei weiter $1\leq q < \infty, \omega_j, \omega \in L_q(X), \omega_j \ra \omega$ in $L_q(X)$, dann folgt
\[
	\int_X \varphi_j \omega_j \d x \longrightarrow \int_X \varphi \omega \d x \, .
\]
\end{lemma}

\begin{proof}
Wegen $\omega_j \ra \omega$ in $L_q (X)$ folgt aus Bemerkung~\ref{bem:6.19} (a) $$\sup_j \norm{\omega_j}_{L_q(X)}  <\infty \, .$$ Da $X$ beschränkt ist, ist $L_q(X) \hookrightarrow L_1(X)$ und damit $\norm{\omega_j}_{L_1(X)} \leq c < \infty \, \fa \, j \in \N$. Also gilt
\begin{dmath*}
\Abs{\int_X \varphi_j \omega_j \d x - \int_X \varphi \omega \d x} \leq \overbrace{\norm{\varphi_j-\varphi}}^{\longrightarrow 0} \ \!\!\!\!_{L_\infty(X)}\overbrace{\norm\omega}^{\leq c} \!\!_{L_1(X)} + \underbrace{\Abs{\int_X \varphi(\omega_j-\omega) \d x}}_{\longrightarrow 0}  .
\end{dmath*}
Damit folgt die Behauptung.
\end{proof}

\begin{satz}\label{satz:7.27}
Sei $1 < q < \infty$ und $L$ nach unten beschränkt. Weiter sei $L(\, \cdot \, , z, x)$ konvex für alle $(z,x) \in \R \times \Omega$. Dann folgt $f: W^1_q(\Omega) \ra (-\infty, \infty]$ ist schwach folgenunterhalbstetig, d.h.
\[
	J(u) \leq \liminf_{j \ra \infty} J(u_j) \quad \fa \, \text{Folgen } (u_j) \text{ in } W^1_q(\Omega) \text{ mit } u_j \longrightarrow u \text{ in } W^1_q(\Omega) \, .
\]
\end{satz}

\begin{proof}
Sei $u_j \in W^1_q (\Omega), u_j \rightharpoonup u$ in $W^1_q (\Omega)$, d.h. $u_j \rightharpoonup u, \partial_k u_j \rightharpoonup \partial_k u$ in $L_q(\Omega) \, \fa \, k = 1, \ldots, n$. Wir setzen $\kappa := \liminf_{j\ra \infty} J(u_j)$. Wir wählen eine Teilfolge, so dass o.B.d.A. $\kappa = \lim_{j\ra \infty} J(u_j)$. Wegen der schwachen Konvergenz und Bemerkung~\ref{bem:6.19} (d) ist $(u_j)$ beschränkt in $W^1_q(\Omega) \hhookrightarrow L_q(\Omega)$ (wegen Bemerkung~\ref{bem:6.14} und Redlich Konarachov). Aufgrund von Bemerkung~\ref{bem:6.19} (e) sind schwache Grenzwerte eindeutig und mit der Wahl einer Teilfolge gilt dann auch
\begin{align}
\label{eq:7.9}
u_j \longrightarrow u \quad \text{in } L_q(\Omega) \text{ (fast überall)} \, .
\end{align}
Sei $\epsilon > 0$ beliebig. Dann gilt nach Egoroff (Maßtheorie), 
\begin{align}\label{eq:here}\exists \text{ messbare Menge } E_\epsilon \text{ mit }\lambda_n (\Omega\setminus E_\epsilon) < \epsilon, u_j \ra u\text{ glm. auf }E_\epsilon\, . \end{align}
Wir definieren $F_\epsilon := \{ x \in \Omega \with \abs{u(x)} + \abs{\nabla u(x)} < \frac 1\epsilon\}$
\begin{align}
\label{eq:7.11}
\Longrightarrow F_\epsilon \text{ messbare Menge mit } \lambda_n(\Omega \setminus F_\epsilon) \xrightarrow{\epsilon \ra 0} 0 \, .
\end{align}
Setze $G_\epsilon := E_\epsilon \cap F_\epsilon$.
\begin{align}
\label{eq:7.12}
\stackrel{\scriptsize(7.10), \eqref{eq:7.11}}\Ra 0 \leq \lambda_n(\Omega \setminus G_\epsilon) \leq \lambda_n(\Omega \setminus E_\epsilon) + \lambda_n (\Omega \setminus F_\epsilon) \xrightarrow[\epsilon \ra \infty]{} 0
\end{align}
Sei $L$ nach unten beschränkt, dann folgt o.B.d.A. $L\geq 0$ (sonst $\tilde L := L+\beta , \beta \gg0$). Da $L(\, \cdot \, z,x)$ konvex ist, folgt
\begin{align*}
\tag{$\ast$}
\begin{aligned}
 L(p_1, z, x) \geq L(p_2, z, x) + L_p (p_2, z,x) \cdot (p_1-p_2) \quad & \fa \, p_1,p_2 \in \R^2,\\
 & (z,x)\in\R\times \Omega
 \end{aligned}
 \end{align*}
 \begin{align}
 \label{eq:7.13}
\Ra \ \ \, & J(u_j)  = \int_\Omega L(\nabla u_j, u_j, x) \d x \stackrel{L\geq 0}\geq \int_{G_\epsilon} L(\nabla u_j, u, x) \d x \notag\\
\stackrel{\scriptsize\text{konvex}, (\ast)}\geq &\int_{G_\epsilon} L(\nabla u, u_j, x) \d x + \int_{G_\epsilon} L_p(\nabla u, u_j, x) \cdot (\nabla u_j-\nabla u) \d x \, .
\end{align}
Da $L$ stetig ist, folgt mit $(7.10), \eqref{eq:7.11}$
\begin{align*}
L(\nabla& u(x), u_j(x), x) \longrightarrow L(\nabla u(x),u(x),x) \quad \fa \, x \in G_\epsilon \; , \\
&\abs{L(\nabla u(x), u_j(x), x)} \leq c(\epsilon) \quad \fa \, j \in \N, x \in G_\epsilon
\end{align*}
\begin{align}
\label{eq:7.14}
\stackrel{\scriptsize\text{Lebesgue}}\Ra \int_{G_\epsilon} L(\nabla u, u_j, x)\d x \longrightarrow \int_{G_\epsilon} L(\nabla u, u , x) \d x \, .
\end{align}
Analog erhalten wir $L_p (\nabla u(x), u_j(x), x) \rightarrow L_p(\nabla u(x),u(x),x)$ gleichmäßig bzgl. $x \in G_\epsilon$. Wegen der schwachen Konvergenz gilt $\nabla u_j \rightharpoonup \nabla u$ in $L_q(G_\epsilon, \R^n)$.
\begin{align}
\label{eq:7.15}
&\, \stackrel{\scriptsize\text{Lemma }\ref{lemma:7.26}}\Ra  \int_{G_\epsilon} \underbrace{L_p(\nabla u, u_j, x)}_{\text{konv. glm.}} \cdot \underbrace{(\nabla u_j - \nabla u)}_{\rightharpoonup 0} \d x \stackrel[j \ra \infty]{}\longrightarrow 0 \\
&\stackrel{\scriptsize\eqref{eq:7.13}-\eqref{eq:7.15}} \Ra \kappa = \lim_{j\ra \infty} J(u_j) \geq \int_{G_\epsilon} L(\nabla u, u, x) \d x \notag
\end{align}
\begin{align*}
\stackrel{\epsilon >0}\Ra \liminf_{j} J(u_j) =& \kappa \geq \liminf_{\epsilon \ra 0} \int_\Omega \underbrace{\chi_{G_\epsilon} (x) L(\nabla u, u, x)}_{\geq 0} \d x \\
&\! \! \! \!\stackrel[\scriptsize\text{Faton}]{\scriptsize\text{Lemma v.}}\geq l \int_\Omega \underbrace{\lim_{\epsilon\ra 0} \chi_{G_\epsilon} (x)}_{\ra \chi_\Omega (x)} L(\nabla u, u, x) \d x \\
&\ \ =J(u)\qedhere
\end{align*}
\end{proof}

\begin{notation}
Wir schreiben
\[
	W := \{u \in W^1_q(\Omega) \with \gamma_0 u = g\} \, ,
\]
wobei $\gamma_0$ der Spuroperator ist. Man beachte, dass $W = \mathring W^1_q(\Omega)$ ist, wenn $g \equiv 0$ (dies folgt aus Bemerkung~\ref{bem:6.17}).

Man kann zeigen: Ist $g$ genügend glatt, z.B. $g \in C^1(\partial \Omega)$, so ist $W \neq \emptyset$ (hierzu: $\gamma_0$ surjektiv auf "`geeigneten"' Räumen).
\end{notation}

\begin{theorem}[Existenz von Minima]
\label{theorem:7.28}
Sei $1 < q < \infty$ und $W\neq \emptyset$. Ferner sei $L(\, \cdot , z, x)$ konvex für $(z,x) \in \R \times \Omega$ und erfülle die Koerzivitätsbedingung
\[
	L(p, z, x) \geq \alpha \abs p^q-\beta \, , \quad \forall \, p \in \R^n, z\in  \R, x \in \bar\Omega
\]	
für $\alpha > 0, \beta \geq 0$. Dann folgt, dass $J$ ein Minimum auf $W$ annimmt, d.h. $\exists \, u_\ast \in W$ mit $J(u_\ast) = \inf_{w \in W} J(w)$.
\end{theorem}

\begin{proof}
Ohne Beweis (siehe  ungeteXtes Skript).
\end{proof}

\begin{bem}
\label{bem:7.29}
Es gelten die Voraussetzungen von Theorem~\ref{theorem:7.28}.
\begin{enumerate}[(a)]
\item Gilt ferner, dass $L = L(p,x)$ unabhängig von $z$ ist und genügt $L$ der Elliptizitätsbedingung (mit $\alpha \geq 0$)
\[
	\sum_{j,k=1}^n L_{p_jp_k} (p,x) \xi^j\xi^k \geq \alpha \abs \xi^2 , \quad p, \xi \in \R^n, x \in \bar \Omega \, ,
\]
so ist das Minimum $u_\ast$ eindeutig. (ohne Beweis)
\item Genügt $L$ den Wachstumsbedingungen
\begin{align*}
\abs{L(p,z,x)} & \leq c(1+\abs p^q + \abs z^q) \\
\abs{L_p(p,z,x)}+ \abs{L_z(p,z,x)} & \leq c(1+\abs p^{q-1} + \abs z^{q-1})
\end{align*}
für alle $p \in \R^n, z \in \R, x \in \bar\Omega$, so ist das Minimum $u_\ast$ von $J$ auf $W$ eine schwache Lösung der Euler-Lagrange-Gleichungen
\[
	- \sum_{j=1}^n (L_{p_j} (\nabla u, u, x))_{x_j} + L_z (\nabla u, u, x) = 0 \, , \quad u |_{\partial \Omega} = g
\]
in dem Sinne, dass
\[
	\int_\Omega\left( \sum_{j=1}^n L_{p_j} (\nabla u, u, x) v_{x_j} + L_z (\nabla u, u, x) v\right) \d x = 0 \quad \fa \, v \in \mathring W^1_q (\Omega) \, .
\]
\begin{proof}
Übung (vgl. die Herleitung der Euler-Lagrange-Gleichungen und verwende die Wachstumsbedingungen).
\end{proof}
\end{enumerate}
\end{bem}

\begin{bsp}
\begin{enumerate}[(a)]
\item Semilineare Laplacegleichung: Sei $f \in L_1(\R) \cap L_q(\R)$ und 
\[ F(z) :=\int_0^z f(r) \d r \, . \]
Wir betrachen $L(p,z,x) := \frac 1 2 \abs p^2 - F(z)$, damit ergibt sich das Energiefunktional
\[ J(u) := \int_\Omega \left(\frac 1 2 \abs{\nabla u}^2 - F(u) \right) \d x \, ,\]
dann folgt mit Theorem~\ref{theorem:7.28} und Bemerkung~\ref{bem:7.29} (b)
\[
	\exists \, u_\ast \in \mathring W^1_2 (\Omega) : J(u_\ast) = \inf_{v \in \mathring W^1_2 (\Omega)} J(v)
\]
und $u_\ast$ ist schwache Lösung von $-\Delta u = f(u)$ in $\Omega, u|_{\partial\Omega} = 0$.
\item Wir betrachten
\begin{align*}
\label{eq:stern}
\tag{$\ast$}
\begin{aligned}
	-u_{xx}& = \frac 1 2 \frac{h'(u)}{h(u)} u_x^2 \, , \quad x \in (0,1) =: \Omega \\
	u(0) &= g_0 \, , \quad u(1) = g_1
\end{aligned}
\end{align*}
mit $g_0, g_1 \in\R, h \in C^1(\R), h'\in L_\infty (\R), h(z) \geq \alpha > 0, z \in \R$. Dann ist $L(p,z,x):=h(z) p^2$ und damit folgt
\[
	L_p (p,z,x) = 2 h(z) p \, , \quad L_z = h'(z) p^2 \, .
\]
Wir erhalten mit Euler-Lagrange
\begin{dmath*}
0 = -(2h(u)u_x)_x + h'(u) u_x^2 = -2h'(u) u_x^2 - 2h(u) u_{xx} + h'(u) u_x^2
\end{dmath*}
und nach Umformen folgt die DGL aus \eqref{eq:stern}. Mit den Voraussetzungen gilt, dass $L$ koerziv ($q=2$) und konvex in $p$ ist.
\begin{align*}
	\stackrel{\scriptsize \text{Theorem}~\ref{theorem:7.28}} \Ra \exists \, u \in W :=& \{ w \in W^1_2 ((0,1)) \with w(0) = g_0, w(1) = g_1 \} \neq \emptyset \\ &\text{ mit } J(u) = \inf_W J
\end{align*}
Mit Sobolev-Morrey (Theorem~\ref{theorem:6.15}) und $n=1$ ist
\begin{align}
\label{eq:7.16}
	W^1_2((0,1)) \hookrightarrow BUC((0,1)) \stackrel ! = C([0,1])\, .
\end{align}
Man beachte, dass Bemerkung~\ref{bem:7.29} (b) hier nicht angewendet werden kann, da die Wachstumsbedingungen nicht erfüllt sind (z.B. $L_z$ ist quadratisch in $p$).

\textit{Behauptung.} $u$ ist schwache Lösung von \eqref{eq:stern}.
\begin{proof}
Sei $v \in \mathring W^1_2((0,1))$ beliebig, $\varphi(t) := f(u+tv), t \in \R$. Für $\abs t \leq 1$ gilt
\begin{dmath*}
	\frac 1 t (\varphi(t) - \varphi(0)) = \int_0^1 \frac 1 t (h(u+tv) (u_x + tv_x)^2 - h(u) u_x^2) \d x
	= \int_0^1 \underbrace{\frac{h(u+tv)-h(u)}t (u_x+tv_x)^2}_{\stackrel{t\ra 0}\longrightarrow h'(u) vu_x^2 \text{ punktweise}} \d x + \int_0^1 \underbrace{h/(u) \frac{(u_x+tv_x)^2-u_x^2}t}_{\stackrel{t\ra 0}\longrightarrow h(u) 2 u_xv_x \text{ punktweise}} \d x \, .
\end{dmath*}
Dann folgt aus Hölder und \eqref{eq:7.16}, dass die Integranden Majoranten in $L_1((0,1))$ besitzen und damit gilt
\[
	\underbrace{\frac 1 t (\varphi(t) - \varphi(0))}_{\longrightarrow \varphi'(0)} \xrightarrow{\scriptsize\text{Lebesgue}} \int_0^1 h'(u) u_x^2 v \d x + \int_0^1 2h(u) u_xv_x \d x \, .
\]
Da $u$ ein Minimum von $f$ ist, folgt $\varphi'(0) = 0$ und somit
\[
	 0 =  \int_0^1 ( h'(u) u_x^2 v  +  2h(u) u_xv_x) \d x \quad \fa \, v \in \mathring W^1_2(\Omega) \,  ,
\]
d.h. $u$ ist schwache Lösung von \eqref{eq:stern}.
\end{proof}
Man kann sogar zeigen, dass $u \in C^\infty ((0,1))$, d.h. $u$ ist klassische Lösung von \eqref{eq:stern}.
\end{enumerate}
\end{bsp}

\section{Eigenwertprobleme}

Wir betrachten in diesem Abschnitt das Problem
\begin{align*}
	-\Delta u = \lambda u\quad \text{in } \Omega \, , \quad u|_{\partial\Omega} = 0 \, , 
\end{align*}
d.h. $(-\lambda - \Delta)u = 0, u|_{\partial\Omega} = 0$. Jede Lösung $u \not\equiv 0$ ist eine Eigenfunktion von $-\Delta_D$ und $\lambda$ ist Eigenwert von $-\Delta_D$.

\begin{notation}
Mit $-\Delta_D$ bezeichnen wir den Laplace-Operator mit Dirichlet-Nebenbedingung.
\end{notation}

\textit{Ziel.} Wir wollen Eigenwerte des Laplace-Operators bestimmen.

Wegen Theorem~\ref{theorem:7.18} gilt für alle Eigenwerte $\lambda$, dass $\Re \lambda \geq 0$. Abstrakt erhalten wir
\[
	Au = \mu u
\]
mit $\mu := \frac 1 \lambda, A :=(-\Delta_D)^{-1} : L_2(\Omega)\ra L_2(\Omega)$ kompakt und beschränkt (vgl. Theorem~\ref{theorem:7.18} und Korrolar~\ref{kor:7.21}), d.h. man sucht Eigenwerte und Eigenfunktionen von einem kompakten, beschränkten Operator in einem Hilbertraum.

Als Generalvoraussetzung soll gelten, dass $(H,(\, \cdot\, \vert \, \cdot \, ))$ ein $\C$-Hilbertraum (oder Komplexifizierung) ist und $A \in \mathcal L(H):=\mathcal L(H,H)$.

\begin{defi}
$\mu \in \C$ heißt \idx{Eigenwert} von $A : \Longleftrightarrow \, \exists \, \varphi \in H \setminus \{0\} : A \varphi = \mu \varphi$ (beachte: $\abs \mu \leq \norm A$). Dann heißt $\varphi$ \idx{Eigenvektor}.

$A$ heißt \index{Operator!symmetrisch}symmetrisch $: \Longleftrightarrow (Ax | y ) = (x| Ay) \, \fa \, x,y \in H$.
\end{defi}

\begin{bsp}\label{bsp:7.31}
\begin{enumerate}[(a)]
\item Eine symmetrische Matrix in $\R^{n\times n}$ definierten symmetrische kompakte Abbildungen auf $H = \C^n$.
\item Integraloperatoren: Es sei $H:= L_2(\Omega), \Omega \subset \R^n$ offen, beschränkt, $K \in L_2(\Omega \times \Omega)$ ("`Kern"') reelwertig und symmetrisch, d.h. $K(x,y) = K(y,x)$ $ \fa \, x,y \in \Omega \times \Omega$.
\[
	(Af)(x) := \int_{\Omega} \underbrace{K(x,y)}_{\scriptsize\text{Greensche Fkt.}} f(y) \d y \, , \quad x \in \Omega, f \in L_2(\Omega)
\]
Dann folgt, $A \in \mathcal L(L_2(\Omega))$ ist symmetrisch und kompakt.
\begin{proof}
Übung.
\end{proof}
\end{enumerate}
\end{bsp}

\begin{satz}\label{satz:7.32}
Sei $A \in \mathcal L(H)$ symmetrisch, dann folgt: Alle Eigenwerte sind reell und Eigenvektoren zu verschiedenen Eigenwerten sind orthogonal.
\end{satz}

\begin{proof}
Es seien $\lambda, \mu \in \C, \varphi, \psi \in H\setminus\{0\}$ mit $A\varphi = \lambda \varphi, A\psi = \mu \psi$. Dann gilt
\begin{align*}
\lambda (\varphi\vert\psi) = (\lambda \varphi \vert \psi) = (A \varphi \vert \psi) \stackrel{\scriptsize\text{symm.}}=(\varphi\vert A\psi)=(\varphi\vert\mu\psi) = \bar\mu(\varphi\vert\psi) \, ,
\end{align*}
d.h. wenn $\varphi = \psi$ ist, so gilt $\lambda = \bar\lambda$ und bei $\lambda \neq \mu$ ist $\varphi \perp \psi$. Damit folgt die Behauptung.
\end{proof}

\begin{satz}\label{satz:7.33}
Sei $A \in \mathcal L(H)$ symmetrisch, dann folgt
\[
	\norm A = \sup_{\norm x = 1} \abs{(Ax \vert x)} \, .
\]
\end{satz}

\begin{proof}
\begin{enumerate}[(i)]
\item Sei $d := \sup_{\norm x = \norm y = 1} \abs{(Ax\vert y)}$. Für alle $\norm x = \norm y = 1$ gilt wegen Cauchy-Schwarz
\begin{dmath*}
	\abs{(Ax\vert y)} \leq \norm{Ax} \norm y \leq \norm A \norm x \norm y = \norm A\, , 
\end{dmath*}
also gilt $d \leq \norm A$. Sei $(x_n)$ eine Folge mit $\norm{x_n} = 1, \norm{Ax_n} \ra \norm A$. Sei o.B.d.A. $A\neq 0$,
\[
	y_n := \frac 1{\norm{Ax_n}} Ax_n \Longrightarrow \norm{y_n} = 1 \, .
\]
Damit gilt
\begin{dmath*}
	d \geq \abs{(Ax_n \vert y_n)} = \frac 1{\norm{Ax_n}} \norm{Ax_n}^2 = \norm{Ax_n} \xrightarrow{n \ra \infty} \norm A \, , 
\end{dmath*}
d.h. $d = \norm A$.
\item Es gilt mit Cauchy-Schwarz (analog zu oben)
\[
	r := \sup_{\norm x = 1} \abs{(Ax\vert x)} \leq \norm A\, .
\]
Man beachte, dass
\begin{align*}
	4 (Ax\vert y) \stackrel{\scriptsize\text{symm.}}= & (A(x+y)\vert x+y) - (A(x-y)\vert x-y) \\
	= \ \, \,   & 4 \abs{(Ax\vert y)} \\
	\leq \ \, \, & \abs{(A(x+y)\vert x+y)} + \abs{(A(x-y)\vert x-y)} \\
	\leq \ \, \, & r \norm{x+y}^2 + r \norm{x-y}^2 \\
	= \ \, \, & 2r (\norm x^2 + \norm y^2)
\end{align*}
gilt, daraus folgt $d \leq r$ und damit gilt mit (i)
\[
	\norm A = d \leq r \leq \norm A \, . \qedhere
\]
\end{enumerate}
\end{proof}

\begin{satz}\label{satz:7.34}
$A \in \mathcal L(H)$ sei symmetrisch und kompakt, dann gilt, $\norm A$ oder $-\norm A$ ist Eigenwert von $A$.
\end{satz}

\begin{proof}
Wegen Satz~\ref{satz:7.33} gilt, es existiert eine Folge $(x_n)$ mit $\norm{x_n} = 1$ und $\abs{(Ax_n\vert x_n)} \ra \norm A$. Wähle o.B.d.A. eine Teilfolge von $(x_n)$, dann 
\[
	\exists \, \lambda \in \C : (Ax_n \vert x_n) \longrightarrow \lambda \, , \quad \abs\lambda = \norm A \, .
\]
\begin{dmath*}
	\norm{Ax_n - \lambda x_n}^2 = \norm{Ax_n}^2 - 2 \Re (Ax_n\vert \lambda x_n) + \norm{\lambda x_n}^2 \leq \underbrace{\norm A^2}_{=\abs\lambda^2} - 2 \Re \bar\lambda \underbrace{(Ax_n \vert x_n)}_{\ra \lambda} + \abs \lambda^2
\end{dmath*}
Damit folgt 
\begin{align}
\label{eq:7.17}	Ax_n - \lambda x_n \longrightarrow 0 \, .
\end{align}
Da $(x_n)$ eine beschränkte Folge, $A$ kompakter Operator ist und wegen der gewählten Teilfolge folgt o.B.d.A., dass $(Ax_n)$ konvergiert.
\[
	\stackrel{\scriptsize\eqref{eq:7.17}}\Longrightarrow	 \, \exists \, x \in H : x_n \longrightarrow x \, , \quad \norm x = 1 
\]
Da $A$ stetig ist, folgt $Ax_n - \lambda x_n \ra Ax - \lambda x$, also $Ax = \lambda x$, d.h. $\lambda$ ist  Eigenwert. Wegen Satz~\ref{satz:7.32} ist $\lambda \in \R$, d.h. $\lambda \in \{-\norm A, \norm A\}$.
\end{proof}


\begin{theorem}\label{theorem:7.35}
Es sei $A \in \mathcal L(H)$ symmetrisch, kompakt, sowie $H$ separabel. Dann gilt:
\begin{enumerate}[\rm (i)]
\item $A$ hat abzählbar viele Eigenwerte $(\mu_j)$ mit $0$ als einzig möglichen Häu-fungspunkt.
\item Jeder Eigenwert $\mu_j \neq 0$ hat endliche Vielfachheit, d.h. $\dim (\ker (A-\mu_j)) < \infty$ $($somit: o.B.d.A. $\abs{\mu_1} \geq \abs{\mu_2} \geq \ldots \geq 0)$.
\item Die zugehörigen normierten Eigenvektoren $\{\varphi_j\}$ $($gemäß Vielfachheit$)$ bilden eine ONB von $H$.
\item $\fa \, x \in H : Ax = \sum_j \mu_j (x\vert \varphi_j) \varphi_j$.
\end{enumerate}
\end{theorem}

\begin{proof}
Sei o.B.d.A. $A \neq 0$. (Satz~\ref{satz:7.34}) Sei $0 \neq \mu_1 \in \{\norm A, - \norm A\}$ ein Eigenwert mit Eigenvektor $\varphi_1 \in H\setminus\{0\}, \norm{\varphi_1} =1$. Dann ist $H_2:=\{\varphi_1\}^T$ ein abgeschlossener Unterraum von $H$, also ein Hilbertraum.
\begin{align*}
\fa \, x \in H_2 : (Ax \vert \varphi_1) \stackrel{\scriptsize A\text{ symm.}}= (x \vert A\varphi_1) = \mu_1 (x \vert \varphi_1) = 0
\end{align*}
Daraus folgt, $Ax \in H_2 \, \fa \, x \in H_2 \Ra A_2 := A\vert_{H_2} \in \mathcal L(H_2)$ ist symmetrisch. Da $H_2$ abgeschlossen  in $H$ und $A$ kompakt ist, folgt $A_2 \in \mathcal L(H_2)$ symmetrisch. Daraus folgt aus Satz~\ref{satz:7.34} und Induktion, dass Eigenwerte $\mu_1, \mu_2, \ldots , \mu_n \neq 0$ existieren (nicht notwendigerweise verschieden) mit zugehörigen paarweise verschiedenen Eigenvektoren $\varphi_1, \ldots, \varphi_n, \norm{\varphi_j} = 1$. Dann ist $H_{j+1} := \operatorname{span} \{\varphi_1 , \ldots , \varphi_n\}^\perp$ abgeschlossener Untervektorraum, also Hilbertraum. Weiter ist $A_{j+1} := A\vert_{H_{j+1}} \in \mathcal L(H_{j+1}), 1 \leq j \leq n$ und $\abs{\mu_j} = \norm{A_j}_{\mathcal L(H_j)}, 1 \leq j \leq n+1$.
\begin{enumerate}
\item[\underline{1. Fall:}] Die Folge $(\mu_j)$ bricht ab, d.h. $\mu_{n+1} = 0$. Dann gilt $A_{n+1} = 0$ und Null ist ein Eigenwert von $A$ (mit möglicherweise unendlicher Vielfachheit). Dann ist $H_{n+1} \subset \ker (A)$ abgeschlossener Unterraum von $H$, also separabler Hilbertraum. Damit folgt mit Bemerkung~\ref{bem:7.13}, $H_{j+1}$ besitzt eine abzählbare ONB $\{\varphi_{n+1}, \varphi_{n+2}, \ldots \}$, also
\[
	\{\varphi_1, \ldots , \varphi_n, \varphi_{n+1}, \ldots\} \text{ ist ONB von } H .
\]
\item[\underline{2. Fall:}] Die Folge $(\mu_j)$ bricht nie ab, d.h. $\mu_j \neq 0 \, \fa \, j \in \N$. Da $A$ kompakt ist, hat $(Ax_j)$ eine kompakte Teilfolge.
\begin{align*}
	\norm{A\varphi_j - A\varphi_k}^2 & \ \, \,  =\ \, \,  \norm{A\varphi_j}^2 - 2\Re (A\varphi_j\vert A\varphi_k) + \norm{A\varphi_k}^2 \\
	& \stackrel{\scriptsize\text{EW}\in\R}= \abs{\mu_j}^2 + \abs{\mu_k}^2
\end{align*}
Daraus folgt
\begin{align*}
	 & \mu_j \longrightarrow 0 \Ra \, \fa \, \mu_j \neq 0 : \mu_j \text{ hat endliche Vielfachheit,}
\end{align*}
daraus folgt 0 ist einzig möglicher Häufungspunkt.

Sei $x \in H$ beliebig und
\[
	x_n := x  - \sum_{j=1}^n (x\vert \varphi_j) \varphi_j \in H_{n+1} = \operatorname{span} \{\varphi_1,\ldots, \varphi_n\}^\perp .
\]
(Denn: $(\varphi_l \vert x_n) =( \varphi_l\vert x) - (\varphi_l\vert x) = 0$.) Daraus folgt
\begin{dmath*}
 \norm{Ax_n} = \norm{A_{n+1}x_n}_{H_{n+1}} \leq \norm{A_{n+1}}_{\mathcal L(H_{n+1})} \underbrace{\norm{x_n}_{H_{n+1}}}_{=\norm{x_n}} = \abs{\mu_{n+1}} \norm{x_n} \stackrel{\scriptsize\text{Def.}}\leq \abs{\mu_{n+1}} \norm x\, .
\end{dmath*}
Mit $\mu_j \ra 0$ gilt dann $Ax_n \ra 0$,
\begin{align}
\label{eq:7.18}
\Ra: Ax = \sum_j \mu_j (x\vert \varphi_j)\varphi_j \quad \forall \, x \in H \, .
\end{align}
\textit{Behauptung.} $\Phi := \{\varphi_j\}$ ist ONB in $N:= (\ker A)^\perp$.

Aus Korollar~\ref{kor:7.7} folgt, dass $N$ abgeschlossener Unterraum von $H$, also Hilbertraum, ist.
\begin{align*}
	\fa \, x \in \ker A : (\varphi_j \vert x) & \ \,  \, \ = \frac 1{\mu_j} (\mu_j \varphi_j\vert x) 
	= \frac 1{\mu_j} (A\varphi_j \vert x) \\
	&  \stackrel{\scriptsize A\text{ symm.}}= \frac 1{\mu_j} (\varphi_j \vert \underbrace{Ax}_{=0}) = 0 \, , 
\end{align*}
d.h. $\varphi_j \perp \ker A \, \fa \, j$, damit ist $\Phi \subset N$.

Sei $x \in \Phi^\perp = \{ x \in N \with x \perp \Phi\}$.
\begin{align*}
 & \Ra (x \vert \varphi_j) = 0 \, \fa \, j \stackrel{\scriptsize\eqref{eq:7.18}}\Ra x \in \ker A \\ 
 & \Ra: x \in \ker A \cup (\ker A)^\perp = \{0\} \Ra: \Phi^\perp = \{ 0\} \, ,
\end{align*}
somit ist $\Phi$ ONB in $N = (\ker A)^\perp$.

$\ker A$ ist ein separabler Hilbertraum, daher folgt aus Bemerkung~\ref{bem:7.13}, dass eine abzählbare ONB von $\ker A$ existiert. Daraus folgt, $\Phi = \{\varphi_j\}$ lässt sich zu einer ONB von $H =(\ker A)^\perp \otimes \ker A$ ergänzen.\qedhere
\end{enumerate} 
\end{proof}

\subsubsection{Anwendung auf den Laplace-Operator}

\underline{Voraussetzungen:} $\Omega \subset \R^n$ sei ein beschränktes Gebiet und $C^\infty$. $G$ sei die Greensche Funktion für $\Omega$ (vgl. Bemerkung~\ref{bem:5.3} (d)).

Wir definieren
\[
	(Af)(x) := \int_\Omega G(x,y) f(y) \d y \, , \quad x \in \Omega, f \in L_2(\Omega) \, .
\]
Aus Theorem~\ref{theorem:5.7} folgt dann, ist $f\in C^\alpha (\bar\Omega), \alpha >0,$ so ist $u := Af \in C^2(\Omega ) \cap C(\bar\Omega)$ die eindeutige Lösung von $-\Delta u = f$ in $\Omega, u\vert_{\partial \Omega} = 0$.

\begin{lemma}\label{lemma:7.36}
Sei $n \in \{2,3\}$ und $A$ wie oben definiert. Dann ist $A \in \mathcal L(L_2(\Omega))$ symmetrisch und kompakt.
\end{lemma}

\begin{proof}
Sei $R>0 : \Omega \subset \B (0,R).$ Sei $G_R$ die Greensche Funktion für $\B(0,R)$ (vgl. Satz~\ref{satz:5.9}, Bemerkung~\ref{bem:5.10}).
\begin{align}\label{eq:7.19}
\stackrel{\scriptsize\text{Lemma}~\ref{lemma:5.6}}\Longrightarrow 0 \leq G(x,y) \leq G_R(x,y) \, , \quad (x,y) \in \Omega \times \Omega
\end{align}
Weiter folgt aus Satz~\ref{satz:5.9} und Bemerkung~\ref{bem:5.10}, dass $G_R$ eine Singularität hat von der Form
\begin{align*} 
	\begin{cases}
		\log \abs{x-y} & \text{falls } n=2 \\
		\abs{x-y}^{-n+2} & \text{falls } n\geq 3
	\end{cases} \, .
\end{align*}
In Polarkoordinaten folgt $G_R \in L_2(\B(0,R) \times \B(0,R))$ und damit mit \eqref{eq:7.19} $G \in L_2(\Omega \times \Omega)$. Wegen Satz~\ref{satz:5.4} gilt, $G$ ist symmetrisch und insgesamt folgt dann mit Beispiel~\ref{bsp:7.31} (b) $A \in \mathcal L(L_2(\Omega))$ symmetrisch und kompakt.
\end{proof}

\begin{lemma}
Sei $n = 2,3, A$ wie oben, dann folgt $Af \in C(\bar\Omega) \, \fa \, f \in L_2(\Omega)$.
\end{lemma}

\begin{proof}
Sei $\eta \in C^1 (\R), 0 \leq \eta \leq 1, \eta(t) = 0, t \leq 1$ und $\eta(t) = 1 , t \geq 2$.

Sei $f \in L_2(\Omega)$ fest,
\[
	\omega_\epsilon (x) := \int_\Omega G(x,y) \eta \left(\frac{\abs{x-y}}\epsilon\right) f(y) \d y \, , \quad x \in \Omega \, ,
\]
daraus folgt $\omega_\epsilon \in C(\bar\Omega) \, \fa \, \epsilon >0$ und
\begin{dmath*}
\abs{(Af)(x) - \omega_\epsilon(x)} \leq \int_\Omega G(x,y) \Abs{1-\eta\left(\frac{\abs{x-y}}\epsilon\right)} \abs{f(y)} \d y  \\
\stackrel{\scriptsize\text{CS}}\leq \norm f_{L_2(\Omega)} \left(\int_\Omega G(x,y)^2 \Abs{1-\eta\left(\frac{\abs{x-y}}\epsilon\right)}^2 \d y \right)^{\frac 1 2} 
= \norm f_{L_2(\Omega)} \left( \int_{[\abs{x-y}\leq 2 \epsilon]} G(x,y)^2 \d y \right)^{\frac 1 2} .
\end{dmath*}
Sei $n = 3: G(x,\, \cdot \,) = G(x,\, \cdot \, ) - \mathcal N(x-\cdot) + \mathcal N(x-\cdot)$.
\begin{align*}
	\Ra: \int_{[\abs{x-y} \leq 2\epsilon]} G(x,y)^2 \d y & \leq c \int_{[\abs{x-.y} \leq 2 \epsilon]} \frac 1{\abs{x-y}^2} \d y \\
	& = c \int_{\B(0,2\epsilon)} \frac 1{\abs y^2} \d y = c \int_0^{2\epsilon}  r^{-2} r^{3-1} \\
	& = 2c\epsilon \\
	\Ra \abs{(Af)(x) - \omega_\epsilon (x)} & \leq c_1 \sqrt \epsilon \xrightarrow{\epsilon \ra 0} 0
\end{align*}
Somit ist $\omega_\epsilon \in C(\bar\Omega), \omega_\epsilon \xrightarrow{\epsilon\ra 0} Af$ gleichmäßig in $\bar\Omega$. Da $Af$ ein gleichmäßiger Grenzwert einer stetigen Funktion ist, folgt
$Af\in C(\bar\Omega)$. Analog gilt dies für $n=2$.
\end{proof}

\begin{theorem}[Eigenwerte des Laplace-Dirichlet-Operators] 
\label{theorem:7.37}
Sei $\Omega \subset \R^n$ beschränktes $C^\infty$-Gebiet, $n = 2, 3$. Dann besitzt das Eigenwertproblem
\[
	- \Delta u = \lambda u \quad\text{in } \Omega \, , \quad u|_{\partial \Omega} = 0 
\]
eine Folge $(\lambda_k)$ positiver Eigenwerte $0 < \lambda_1 \leq \lambda_2 \leq \ldots\leq \lambda_k\xrightarrow{k\ra\infty} \infty$. Jeder Eigenwert hat endliche Vielfachheit und die zugehörigen $($in $L_2(\Omega)$ normierten$)$ Eigenfunktionen $\{\varphi_n\}$ sind aus $C^2(\Omega)\cap C^1(\bar\Omega)$ und bilden eine ONB von $L_2(\Omega)$.

Ferner: $\left\{\frac 1{\sqrt{1+\lambda_k}} \varphi_k\right\}$ ist eine ONB in $\mathring W^1_2(\Omega)$.
\end{theorem}

\begin{proof}
Die Idee ist, dass wir $A=(-\Delta_D)^{-1}$ betrachten. Wegen Theorem~\ref{theorem:7.35} und Lemma~\ref{lemma:7.36} folgt ($L_2(\Omega)$ separabel), es existiert eine ONB $\{\varphi_j\}$ von $L_2(\Omega)$ bestehend aus Eigenfunktionen mit zugehörigen Eigenwerten $\mu_j \neq 0$ des Operators $A$, definiert durch
\[
	(Af)(x) := \int_\Omega G(x,y)f(y) \d y \, , \quad f\in L_2(\Omega) , x \in \Omega 
\] 
und $\mu_j \ra 0$. Man beachte, dass aus $A\varphi_j = \mu_j \varphi_j$ folgt
\begin{align*}
\label{eq:7.20}
\Ra \, \, \quad  & \varphi_j = \frac 1{\mu_j} A\varphi_j \in C(\bar \Omega) \\
\stackrel{\scriptsize\text{Theorem}~\ref{theorem:5.7}}\Ra & \varphi_j = \frac 1{\mu_j} A\varphi_j \in C^1(\bar\Omega) \qquad (\text{impliziert Hölder-stetig})  \\
\stackrel{\scriptsize\text{Theorem}~\ref{theorem:5.7}}\Ra & \varphi_j \in C^2(\Omega) \cap C^1(\bar\Omega)  \\
\end{align*}
löst
\begin{align}
\begin{aligned}
-\Delta \varphi_j & = \frac 1{\mu_j} \varphi_j \quad \text{in } \Omega \, , \\
\varphi_j|_{\partial \Omega}& = 0 \, .
\end{aligned}
\end{align}
Setze $\lambda_j := \frac 1{\mu_j}$, dann folgt $\abs{\lambda_j} \xrightarrow{j\ra\infty} \infty$. Wegen Theorem~\ref{theorem:7.16} wissen wir, dass $\lambda = 0$ kein Eigenwert sein kann. Wir betrachten
\begin{align*}
	\lambda_j\underbrace{(\varphi_j \vert \varphi_j)}_{=1} & \stackrel{\scriptsize\eqref{eq:7.20}}= - \int_\Omega \underbrace{\Delta \varphi_j \bar \varphi_j}_{\parbox{1.9cm}{\scriptsize$=\div (\bar\varphi_j \nabla \varphi_j)\\ - \nabla \varphi_j \nabla \bar\varphi_j$}} \d x \\
	&\,  \stackrel{\scriptsize\text{Gauß}}= - \underbrace{\int_{\partial\Omega} \underbrace{\bar\varphi_j}_{\stackrel{\eqref{eq:7.20}}= 0} \nabla \varphi_j \cdot \nu \d \sigma(x)}_{=0} + \underbrace{\int_\Omega \abs{\nabla \varphi_j}^2 \d x}_{>0} \, ,
\end{align*}
daraus folgt, $\lambda_j > 0 \, \fa \, j \geq1$, also o.B.d.A. $0 < \lambda_1 \leq \lambda_2 \leq \ldots \leq \lambda_j \ra \infty$.

Es bleibt also noch zu zeigen, dass $\left\{\frac 1{\sqrt{1+\lambda_k}} \varphi_k\right\}$ eine ONB von $\mathring W^1_2(\Omega)$ ist. Es gilt
\begin{align*}
(\varphi_j \vert \varphi_k)_{\mathring W^1_2} & = \underbrace{(\varphi_j \vert \varphi_k)_{L_2} }_{= \delta_{jk}}+ (\nabla\varphi_j \vert \nabla\varphi_k)_{L_2}  \\
&= \delta_{jk} + \int_\Omega \nabla \varphi_j \overline{\nabla\varphi_k} \d x \\
&= \delta_{jk} - \int_\Omega \underbrace{\Delta\varphi_j}_{\stackrel{\eqref{eq:7.20}}=-\lambda_j \varphi_j} \bar \varphi_k \d x \\
&= \delta_{jk} + \lambda_j \underbrace{(\varphi_j \vert \varphi_k)_{L_2}}_{= \delta_{jk}} = (1+\lambda_j) \delta_{jk} \, ,
\end{align*}
d.h. $\left\{\frac1{\sqrt{1+\lambda_j}} \varphi_j\right\}$ ist ONS in $\mathring W^1_2 (\Omega)$.

Sei $g \in (\operatorname{span} \{\varphi_j \with j \geq 1\})^\perp$ in $\mathring W^1_2(\Omega)$, d.h. $g \in \mathring W^1_2(\Omega)$ und $(g \vert \varphi_j)_{\mathring W^1_2} = 0 \, \fa \, j \geq 1$.
\begin{align*}
 \Ra : 0& = (g \vert \varphi_j)_{L_2} + \underbrace{(\nabla g \vert \nabla \varphi_j)_{L_2}}_{\parbox{2.6cm}{\scriptsize $\, \, \,  = \int_\Omega \nabla g \overline{\nabla\varphi_j} \d x \\ \stackrel{\text{Gauß}}= - \int_\Omega g \underbrace{\Delta \bar \varphi_j}_{=-\lambda_j \bar \varphi_j}$}} \\
 \Ra : 0 &= (g \vert \varphi_j)_{L_2} + \lambda_j (g\vert \varphi_j)_{L_2} \\
 & = \underbrace{(1+\lambda_j)}_{>0} (g \vert \varphi_j)_{L_2}\\
 \Ra (g \vert \varphi_j)_{L_2}& = 0 \quad \fa \, j \geq 1 \stackrel[\scriptsize\text{in } L_2(\Omega)]{\scriptsize\{\varphi_j\} \text{ ONB}}\Ra g = 0 \, ,
\end{align*}
d.h. $(\operatorname{span} \{\varphi_j\})^\perp = \{0\}$ in $\mathring W^1_2(\Omega)$.
\end{proof}

\begin{bem}
\label{bem:7.38}
\begin{enumerate}[(a)]
\item Theorem~\ref{theorem:7.37} gilt auch für $n \neq 2,3$.
\begin{proof}
E. Di Benedetto.
\end{proof}
\item $\lambda_1 > 0$ heißt Haupteigenwert von $-\Delta_D$. Es gilt
\begin{enumerate}[(i)]
\item $\lambda_1$ ist einfach (d.h. $0 < \lambda_1 < \lambda_2 \leq \ldots$),
\item $\lambda_1^{-1} = \sup_{\norm \varphi_{L_2} = 1} \abs{(A\varphi\vert\varphi)}$ und die zugehörige Eigenfunktion $\varphi_1$ ist strikt positiv, d.h. $\varphi_1 (x) > 0, x \in \Omega, \varphi_1(x) = 0, x \in \partial\Omega$.
\end{enumerate} 
\item Neumannproblem: $-\Delta u = \lambda u$ in $\Omega, \partial_\nu u = 0$ auf $\partial \Omega$ hat Eigenwerte 
\[
	0 = \lambda_1 < \lambda_2 \leq \lambda_3 \leq \ldots \leq \lambda_j \xrightarrow{j\ra\infty} \infty
\]
mit Eigenfunktionen $\varphi_j \in C^2(\Omega) \cap C^1(\bar\Omega)$. Es gilt $\varphi_1 = \frac 1{\abs\Omega^{\frac 1 2}}$.
\item Theorem~\ref{theorem:7.37} gilt für allgemeine Operatoren der Form
\[
	Lu = - \sum_{j,k=1}^n \partial (a_{jk}(x) \partial_k u)
\]
mit $a_{jk}(x) = a_{kj}(x) > 0, x \in \bar\Omega$, d.h. $L$ ist gleichmäßig elliptisch.
\end{enumerate}
\end{bem}


%%% Local Variables: 
%%% mode: latex
%%% TeX-master: "Skript"
%%% End: 

\input{Kapitel08}
\newchapter{Wellengleichung}

Wir betrachten  in diesem Kapitel die homogene \idx{Wellengleichung} in $\Omega$
\[
	\underbrace{\partial_t^2u}_{\parbox{1.3cm}{\centering\scriptsize Beschleu-  nigung}} - \underbrace{\Delta_x u}_{\scriptsize\text{Kraft}} = 0\, , \quad t \in \R, x \in \Omega
\]
mit $u = u(t,x)$.

Hintergrund: $\Omega\subset \R^n$ ist ein elastisches Medium und $u(t,x)$ die Ausrenkung des Mediums in einer festen Richtung zur Zeit $t$ am Ort $x$.

Als Randbedingung kann z.B. durch $u(t,x) = 0, t \in \R, x \in \partial \Omega$ gegeben sein. Die Wellengleichung ist ein Prototyp einer hyperbolischen Gleichung.

\section{Wellengleichung auf $\Omega=\R^n$}

\begin{theorem}
\label{theorem:9.1}
Sei $u \in C^2 ([0,T]\times \R^n)$ Lösung der homogenen Wellengleichung
\[
	\partial_t^2 u - \Delta_x u = 0 \, , \quad t \in (0,T), x \in \R^n .
\]
Ferner sei $(t_0,x_0) \in (0,T)\times \R^n$ mit $u(0,x)=0=\partial_t u(0,x)$ auf $\abs{x-x_0} \leq t_0$. Dann folgt $u \equiv 0$ auf dem Kegel $K:=\{(t,x) \with 0 \leq t \leq t_0, \abs{x-x_0} \leq t_0 - t\}$.
\begin{figure}[ht!]
    \centering<
    \begin{pspicture}(-3,-1)(3,2.5)
	 \psaxes[linewidth=1pt,labels=none]{->}(-2,-0.5)(-3,-1)(3,2.5)
	 \rput(2.95,-.75){$t$}
	 \rput(-1.75,2.5){$x$}
	 \rput(-2.7,0){$x_0-t_0$}
	 \rput(-2.7,2){$x_0+t_0$}
	 \rput(1,-.78){$t_0$}
	 \psline[linewidth=2pt](-2,0)(-2,2)
	  \psline[linewidth=1.3pt](-2,0)(1,1)
	   \psline[linewidth=1.3pt](-2,2)(1,1)
    \end{pspicture}
    \caption{Kegel}
    \label{fig:4.1}
  \end{figure}
Störungen außerhalb von $\{(0,x) \with \abs{x-x_0} \leq t_0\}$ $($zur Zeit $t = 0)$ haben keinen Einfluss auf die Lösung innerhalb des Kegels $K$ $($endliche Ausbreitungsgeschwindigkeit$)$.
\end{theorem}

\begin{proof}
Es sei o.B.d.A. $u$ reellwertig. Wir definieren
\[
	B_t:=\{ x\in\R^n \with \abs{x-x_0} \leq t_0 - t\} = \bar\B_{\R^n} (x_0,t_0-t) 
\]
und betrachten das Energiepotential
\[
	E(t) := \frac 1 2 \int_{B_t} \left((\partial_t^2 u)^2 + \abs{\nabla_x u}^2\right) \d x \, .
\]
Man kann zeigen, dass gilt (s. Skript)
\begin{align}
\label{eq:9.1}
	\frac{\d}{\d r} \int_{\B (x_0,r)} f(x) \d x = \int_{\partial \B(x_0,r)} f(x) \d \sigma(x) \, .
\end{align}
Also gilt
\begin{align}\label{eq:9.2}
	\dot E(t) \stackrel{\scriptsize\eqref{eq:9.1}}= \, & \int_{B_t} (\partial_t^2 u \, \partial_t u + \underbrace{\nabla u\,  \partial_t u}_{\scriptsize\parbox{2cm}{\centering$=\div_x (\partial_t u \nabla u)$\\ $-\partial_t u \Delta_x u$}} ) \d x - \frac 1 2 \int_{\partial B_t }((\partial_t u)^2 + \abs{\nabla u}^2) \d \sigma(x) \notag \\
	\stackrel{\scriptsize\text{Gauß}}= & \int_{B_t} \partial_t u \, (\underbrace{\partial_t^2 u - \Delta_x u}_{=0}) \d x + \int_{\partial B_t} \partial_t u\, \partial_\nu u \d \sigma (x) \notag \\ 
	& - \frac 12 \int_{\partial B_t} \left( (\partial_tu)^2+\abs{\nabla u}^2\right) \d \sigma(x) \, .
\end{align}
Man beachte, dass gilt
\begin{align}
\label{eq:9.3}
	\abs{\partial_t u \, \partial_\nu u} \stackrel{\scriptsize\text{C.S.}}\leq & \abs{\partial_t u}\,  \abs{\nabla u}	\leq \frac 12 \abs{\partial_t u}^2 + \frac 12 \abs{\nabla u}^2 \, .
\end{align}
Dann folgt aus \eqref{eq:9.2} und \eqref{eq:9.3}
\begin{align*}
	& \dot E(t) \leq 0 \Ra 0 \leq E(t) \leq E(0) = 0 \, , \quad 0 \leq t \leq t_0 \\
	\Ra\,  & \partial_t u \equiv 0 \, , \quad \nabla_xu \equiv 0 \quad \text{in } K  \\
	\Ra \, & u \equiv \text{const.} \quad \text{in } K \stackrel{\scriptsize\text{Vor.}}\Ra u \equiv 0 \quad \text{in } K \qedhere
\end{align*}
\end{proof}

\subsubsection{Die \idx{Wellengleichung} im 1-dimensionalen Fall}

Wir betrachten $u_{tt} - u_{xx} = 0, t \in \R, x\in \R$, daraus folgt per Definition der Wellenoperator
\[
	0 = Lu := (\partial_t -\partial_x)(\partial_t + \partial_x) u \, .
\]
Wir zerlegen also diesen in zwei Gleichungen 1. Ordnung
\begin{align}
	\label{eq:9.4}
	v :=& (\partial_t + \partial_x) u \\
	\label{eq:9.5}
	0 = & (\partial_t-\partial_x) v 
\end{align}
und aus \eqref{eq:9.5} erhalten wir mit Charakteristiken
\begin{align}\label{eq:9.6}
	 & v(t,x) = f(x+t) \quad \text{mit } f \in C^1 (\R) \\
	 \label{eq:9.7}
	\stackrel{\scriptsize\eqref{eq:9.4}}\Ra \, & (\partial_t +\partial_x) u(t,x) = f(x+t)  \, .
\end{align}
Verwenden wir wiederum Charakteristiken, so erhalten wir: Sei $x = \xi + t, w_\xi := u(t,\xi+t)$, dann folgt aus \eqref{eq:9.7} $\dot w_\xi (t) = f(\xi + 2t)$. Sei $F \in C^2(\R)$ mit $F' = \frac 1 2 f$. Dann folgt
\[
	\frac\d{\d t} F(\xi + 2t) = f(\xi + 2t) \, ,
\] 
also $w_\xi (t) = F(\xi + 2t) + G(\xi)$ mit $G(\xi)$ geeignet gewählt.
\begin{align}\label{eq:9.8}
	\stackrel{x=\xi+t}\Ra u (t,x) = \underbrace{F(x+t)}_{\parbox{1.45cm}{\scriptsize Welle nach links}} + \underbrace{G(x-t)}_{\parbox{1.4cm}{\scriptsize Welle nach rechts}} \, , 
\end{align}
dabei ist $u$ also eine Überlagerung von Wellen. Jede Funktion $u$ mit $F,G \in C^2 (\R)$ beliebig ist Lösung der 1-dimensionalen Wellengleichung. $F$ und $G$ sind willkürlich, d.h. es lassen sich also zwei Bedingungen zur Bestimmung einer eindeutigen Lösung vorschreiben.

\subsubsection{Das Cauchy-Problem für die 1-dimensionale Wellengleichung}

Wir betrachten das \idx{Cauchy-Problem} in $\R$
\begin{align*}
	u_{tt} - u_{xx} &= 0 \, , \quad \qquad \! \! t \in \R, x \in \R \\
	u(0,x) & = \varphi (x) \, , \quad x \in \R \\
	\partial_t u(0,x)& = \psi(x) \, , \quad x \in \R
\end{align*}
wobei $\varphi \in C^2(\R), \psi \in C^1(\R)$ fest gegeben. Wegen \eqref{eq:9.8} gilt $F(x) + G(x) = \varphi(x)$ und $F'(x) -G'(x) = \psi(x)$, also bleibt folgendes System zu lösen:
\begin{align*}
	F(x) & = \frac 1 2 \varphi (x) + \frac 1 2 \int_0^x \psi (s) \d s + c \\
	G(x) & = \frac 1 2 \varphi (x) - \frac 1 2 \int_0^x \psi(s) \d s - c
\end{align*}
mit $c \in \R$.

\begin{theorem}
\label{theorem:9.2}
Seien $\varphi \in C^2(\R)$ und $\phi \in C^1(\R)$. Dann ist die eindeutige $C^2$-Lösung des Cauchy-Problems für die eindimensionale \idx{Wellengleichung} gegeben durch
\[
	u(t,x) = \frac 1 2 (\varphi(x+t) + \varphi(x-t)) + \frac 1 2 \int_{x-t}^{x+t} \psi (s) \d s \, .
\]
\end{theorem}

\begin{proof}
Existenz: Nachrechnen. Eindeutigkeit: s.o.
\end{proof}

\begin{bem}
\label{bem:9.3}
\begin{enumerate}[(a)]
\item $u$ ist im Punkt $(t,x)$ durch die Werte von $\varphi$ und $\psi$ auf $[x-t,x+t]$ bestimmt (vgl. Theorem~\ref{theorem:9.1}/Wärmeleitungsgleichung in Kapitel 10).
\item Die Lösung $u$ ist nicht regulärer als der Anfangswert $\varphi$ (keine Regularisierung wie z.B. bei Wärmeleitungsgleichung/vgl. auch Bemerkung \ref{bem:9.7}).
\item Für die Eindeutigkeit müssen $u|_{t = 0} $ und $\partial_t u|_{t = 0}$ vorgeschrieben werden.
\end{enumerate}
\end{bem}

\subsubsection{Höhere Dimensionen}

Die Idee hier ist ähnlich wie bei der Laplacegleichung: Mittelwertbildung reduziert die Wellengleichung für $n>1$ auf \idx{Darbouxgleichung} (hyperbolische Gleichung, 1-dimensional $\rightsquigarrow$ 1-dimensionale \idx{Wellengleichung}).

\begin{defi}
Wir definieren das sphärische Mittel\index{sphärisches Mittel} mit $\omega_n := \vol (\partial \B(x,1)),$ $ \vol (\partial \B (x,r)) = r^{n-1} \omega_n, \varphi \in C(\R^n), r>0, x \in \R^n$ durch
\begin{align*}
	M_\varphi (x,r)  := & \frac 1{r^{n-1} \omega_n} \int_{\partial \B (x,r)} \varphi (y) \d \sigma(y) \\
	 = & \frac 1{\omega_n} \int_{\partial\B(0,1)} \varphi(x+ry) \d \sigma(y) \, .
\end{align*}
\end{defi}

\begin{satz}[\idx{Darbouxgleichung}]
\label{satz:9.4}
Für $\varphi \in C^2(\R^n)$ gilt:
\[
	\left(\partial_r^2 + \frac{n-1}r \partial_r\right) M_\varphi (x,r) = \Delta_x M_\varphi (x,r) \, , \quad r \neq 0 , x \in \R^n .
\]
\end{satz}

\begin{proof}
Es sei o.B.d.A. $r > 0$. Dann gilt mit $y$ als äußere Einheitsnormale
\begin{align*}
	\partial_r M_\varphi(x,r)& \ \, =\ \, \frac 1{\omega_n} \int_{\partial \B(0,1)} \nabla_y \varphi (x+ry) y \d \sigma(y) \\
	& \stackrel{\scriptsize\text{Gauß}}= \frac 1{\omega_n} \int_{\B(0,1)} (\Delta_y)(x+ry) r \d y \\
	& \stackrel{z:=ry}= \frac 1{\omega_n r^{n-1}} \int_{\B(0,1)} \Delta \varphi(x+z) \d z \\
	& \stackrel{\scriptsize\parbox{0.8cm}{Polar-\\koord.}}= \frac 1{\omega_n r^{n-1}} \int_0^r\int_{\abs y = 1} \Delta \varphi (x+\rho y) \rho^{n-1} \d \sigma(y) \d \rho \\
\end{align*}
\begin{align*}
	\Ra \, \partial_r [r^{n-1} \partial_r M_\varphi(x,r)] &\  =\, \, \frac 1{\omega_n} \int_{\abs y = 1} \Delta_x \varphi (x+ry) r^{n-1} \d \sigma(y) \\
	& \stackrel{\scriptsize\text{Def.}}= r^{n-1} \Delta_x M_\varphi (x,r) 
\end{align*}
Man beachte, dass $\partial_r(r^{n-1} \partial_r) = r^{n-1} \partial_r^2 + \frac{n-1}r r^{n-1} \partial_r$ gilt und daraus folgt die Behauptung.
\end{proof}

\begin{kor}
\label{kor:9.5}
Sei $u = u(t,x)\in C^2 (\R^+ \times \R^n)$ und $M_u (t,x,r):=M_{u(t,\cdot)} (x,r)$ aus sphärischen Mittel von $u(t,\cdot)$. Dann sind äquivalent:
\begin{enumerate}[\rm(i)]
\item $u$ löst die Wellengleichung $\partial_t^2 - \Delta_x u = 0, t > 0, x \in \R^n$.
\item $ \left[ \partial_r^2 + \frac{n-1}r \partial_r\right] M_u(t,x,r) = \partial_t^2 M_u(t,x,r) , t > 0, x \in \R^n \setminus\{0\}$.
\end{enumerate}
\end{kor}

\begin{proof}
Es gilt $\partial_t^2 M_u = M_{\partial^2_tu}$ und $\Delta_x Mu = M_{\Delta_x u}$ und mit Satz~\ref{satz:9.4} folgt dann die Behauptung.
\end{proof}

\subsubsection{Lösung der Wellengleichung im $\R^3$}

Wir nehmen an $u \in C^2(\R^+ \times \R^3)$ ist Lösung des Cauchy-Problems
\begin{align*}
	\partial^2_t u - \Delta_x u &= 0 \, , \quad \qquad \! \! t >0, x \in \R^3 \\
	u(0,x) & = \varphi (x) \, , \quad x \in \R^3 \\
	\partial_t u(0,x)& = \psi(x) \, , \quad x \in \R^3 .
\end{align*}
Wir setzen $W(t,r) := M_u (t,x,r)$ mit $x \in \R^3$ fest.
\[
	\stackrel{\scriptsize\text{Kor.}~\ref{kor:9.5}}\Ra \left\{
	\begin{aligned}
		W_{tt}&= W_{rr} + \frac 2 r W_r \\
		W(0,r) & = M_u(0,x,r) = M_\varphi (x,r) \\
		W_t (0,r) & = \partial_t M_u (0,x,r) = M_\psi (x,r)
	\end{aligned}
	\right.
\]
Setze $V(t,r) := r \, W(t,r)$, dann folgt $V_{tt} = r(W_{rr} + \frac 2 r W_r) = V_{rr}$, d.h. $V$ löst die 1-dimensionale Wellengleichung
\[
	\left\{
	\begin{aligned}
		V_{tt} - V_{rr} &= 0 \, , \quad t > 0, r \in\R \\
		V(0,r) &= rM_\varphi (x,r) \\
		V_t (0,r) & = r M_\psi (x,r)
	\end{aligned}
	\right. \, .
\]
Dann folgt aus Theorem~\ref{theorem:9.2}
\begin{align}
\label{eq:9.9}
	\begin{aligned}
	V(t,r) =& \frac 12 \left[  (r+t) M_\varphi (x,r+t) + (r-t) M_\psi (x,r-t)  \right]\\
	& + \frac 1 2 \int_{r-t}^{r+t} s M_\varphi (x,s) \d s \, .
	\end{aligned}
\end{align}
Aus der Definition von $M_u$ und $u$ stetig folgt
\begin{align}
\label{eq:9.10}
	\lim_{r \ra 0^+} M_u (t,x,r) = u(t,x)
\end{align}
\begin{align*}
	\stackrel{V=rM_u}\Ra u(t,x) & \stackrel{\scriptsize\eqref{eq:9.10}}= \lim_{r\ra 0^+} \frac 1 r V(t,r) \\
	&\hspace{0.2em} \stackrel{\scriptsize\eqref{eq:9.9}}= \hspace{0.2em} \partial_t (tM_\varphi (t,x)) + t M_\psi (x,t) \\
	&\hspace{0.6em}  = \hspace{0.6em}M_\varphi(t,x) + tM_\varphi (t,x) + t \partial_t \underbrace{\frac 1{\omega_3} \int_{\partial\B (0,1)} \varphi(x+ty) \d \sigma(y)}_{=M_\varphi (t,x)} \\
	& \hspace{0.6em}=\hspace{0.6em} M_\varphi (t,x) + t M_\psi(t,x) + \frac t{\omega_3} \int_{\partial \B(0,1)} \nabla \varphi(x+ty) y \d \sigma(y) \\
	& \hspace{0.6em}=\hspace{0.6em} M_\varphi(t,x) + t M_\psi(t,x) + \frac 1{t\omega_3} \int_{\partial\B(x,t)} \nabla \varphi (\bar y) \cdot \frac{\bar y- x}t \d \sigma(\bar y)
\end{align*}
Somit gilt, jede $C^2$-Lösung der 3-dimensionalen Wellengleichung lässt sich schreiben als
\[
	u(t,x) = \frac 1{t^2\omega_3} \int_{\partial\B (x,t)} [\varphi(y)+t \psi(y) + \nabla \varphi(y) \cdot (y-x)] \d \sigma(y) \, ,
\]
wobei $\omega_3 = 4\pi$ ist. Dies nennen wir die \idx{Kirchhoffsche Formel}.

\begin{theorem}
\label{theorem:9.6}
Sei $n=3, \varphi \in C^3(\R^3), \psi \in C^2(\R^3)$. Dann wird die eindeutige Lösung $u \in C^2([0,\infty) \times \R^3)$ für das Cauchy-Problem der 3-dimensionalen Wellengleichung
\begin{align*}
	\partial^2_t u - \Delta_x u &= 0 \, , \quad \qquad \! \! t >0, x \in \R^3 \\
	u(0,x) & = \varphi (x) \, , \quad x \in \R^3 \\
	\partial_t u(0,x)& = \psi(x) \, , \quad x \in \R^3 .
\end{align*}
durch die Kirchhoffsche Formel gegeben.
\end{theorem}

\begin{proof}
Existenz: Nachrechnen (Darbeoux, Kapitel 5).

Eindeutigkeit: Siehe oben.
\end{proof}

\begin{bem}
\label{bem:9.7}
Lösungen der Wellengleichung (Prototyp für hyperbolische Gleichungen) können für strikt positive Zeiten weniger regulär sein als die Anfangswerte (im Gegensatz zu parabolischen Gleichungen wie z.B. die Wärmeleitungsgleichung, vgl. nächstes Kapitel), d.h. beispielsweise $\varphi \in C^k, \psi \in C^{k-1} \Ra u \in C^{k-1}$ (Regularitätsverlust).
\end{bem}

\subsubsection{Lösung der Wellengleichung im $\R^2$}

Problem: Es gibt keine Transformation, die die 2-dimensionale Gleichung auf eine 1-dimensionale reduziert. 

Ausweg: Wir betrachten das 2-dimensionale Problem als 3-dimensionales, in dem wir $x_3 := 0$ setzen (in der \index{Kirchhoffsche Formel}Kirchhoffschen Formel).

\begin{figure}[ht!]
      \centering
      \begin{pspicture}(-3,-1)(3,3.5)
      \pscircle(0,1){2cm}
      \psellipse(0,1)(2,0.6)
      \psline(0,1)(-1,1)
      \psline(0,1)(-1,2.73)
      \psline(-1,1)(-1,2.73)
      \pswedge(-1,1){0.3}{0}{90}
      \psdot[dotsize=1.2pt](-0.89,1.11)
      \psdot(-1,2.73)
      \rput(-1.4,3.1){$(y,T(y))$}
      \rput(-1,0.8){$y$}
      \rput(-0.3,2.1){$t$}
      \rput(0.2,1){$\bar x$}
      \rput(3,-0.7){$\partial\B_{\R^2}(x,t)$}
      \rput(3.4,2){$\partial \B_{\R^3} (\bar x,t)$}
      \psline{->}(2.1,-0.6)(1,0.45)
      \psline{->}(2.5,2)(1.9,1.8)
      \end{pspicture}
     \caption{Transformation vom $\R^2$ in den $\R^3$}
\end{figure}

Es sei $\bar x :=(x_1,x_2,0)$ mit $x = (x_1,x_2) \in \R^3$, beachte
\begin{align*}
	\int_{\partial\B_{\R^3} (\bar x,t)} h(y) \d \sigma(y) \stackrel[T(y):=\sqrt{t^2 -\abs{x-y}^2}]{\scriptsize\text{Transformation}}= 2 \int_{\B_{\R^2} (x,t)} h(y) \sqrt{1+\abs{\nabla y T(y)}^2} \d \sigma(y) \, .
\end{align*}
Daraus folgt mit Nachrechnen (vgl. Kirchhoffsche Formel), dass die Lösung der 2-dimensionalen Wellengleichung wie folgt lautet:
\[
	u(t,x) = \frac 1{2\pi t^2} \int_{\B_{\R^2} (x,t)} \frac{t \varphi(y) + t^2 \psi (y) + t \nabla \varphi (y) \cdot (y-x)}{\sqrt{t^2-\abs{x-y}^2}} \d y \, .
\]
Dies ist die \idx{Poisson-Formel}.

\begin{bem}
\label{bem:9.8}
\begin{enumerate}[(a)]
\item Die Wellengleichung lässt sich in beliebigen Dimensionen $n>3$ lösen (Fallunterscheidung: $n \in 2\N +1$ bzw. $ n \in 2\N$, vgl. Evans).
\item Für $n = 1, 3$ (bzw. allgemein $n \in 2\N +1$) hängt die Lösung $u(t,x)$ nur von den Anfangsdaten $\varphi, \psi$ auf der Oberfläche $\partial \B(x,t)$ ab. (\idx{Huygens-Prinzip})

Für $n =2$ (bzw. allgemein $n \in 2\N$) hängt $u(t,x)$ von den Anfangsdaten $\varphi, \psi$ auf ganz $\B(x,t)$ ab.
\end{enumerate}
\end{bem}

\subsubsection{Cauchy-Problem für die inhomogene Wellengleichung}

Wir betrachten
\begin{align*}\label{eq:IW}\tag{IW}
\begin{aligned}
	\partial^2_t u - \Delta_x u &= f(t,x) \, , \, \, \,  t >0, x \in \R^n \\
	u(0,x) & = \varphi (x) \, , \quad x \in \R^n \\
	\partial_t u(0,x)& = \psi(x) \, , \quad x \in \R^n .
\end{aligned}
\end{align*}
Sei $u_1$ Lösung für $f\equiv 0$ (vgl. vorher) und $u_2$ Lösung für $\varphi \equiv 0, \psi \equiv 0$. Dann folgt $u_1+u_2 =: u$ löst \eqref{eq:IW}. 

\begin{satz}
\label{satz:9.9}
Sei $f\in C^{\lfloor \frac n2\rfloor +1} (\R^+ \times \R^n)$. Für $s>0$ sei $v = v(s,t,x)$ die $C^2$-Lösung von
\[
	\partial_t^2 v-\Delta_x v = 0\, ,\quad v(s,0,x) = 0\, ,\quad \partial_t v(s,0,x) = f(s,x)\, .
\]
Dann löst
\[
	u_2 (t,x) := \int_0^t v(s,t-s,x) \d s
\]
das Problem \eqref{eq:IW} für $\varphi\equiv 0, \psi \equiv 0$.
\end{satz}

\begin{proof}
Es gilt $u_2 (0,x) = 0$.
\begin{align}\label{eq:9.11}
	\partial_t u_2(t,x) = \underbrace{v(t,0,x)}_{=0} + \int_0^t \partial_t v(s,t-s,x)\d s
\end{align}
Dann folgt $\partial_t u_2(0,x) = 0$.
\begin{align*}
	\stackrel{\scriptsize\eqref{eq:9.11}}\Ra \partial_t^2 u_2(t,x) & = \underbrace{\partial_t v(t,0,x)}_{=f(t,x)} + \int_0^t \underbrace{\partial_t^2 v(s,t-s,x)}_{=\Delta_x v(s,t-s,x)} \d s \\
	& = f(t,x) + \Delta_x u_2 (t,x) \qedhere
\end{align*}
\end{proof}

\section{Wellengleichung auf beschränktem Gebiet $\Omega$}

\underline{Idee:} Wir wollen versuchen die Wellengleichung auf einem beschränkten Gebiet durch Variablenseparation zu lösen, d.h. $u(t,x) = w(t) v(x)$.

 \begin{figure}[ht!]
      \centering
      \begin{pspicture}(-3,-1)(3,2.5)
      		\psccurve(-2,1)(0.6,2)(1.2,0.8)(2,-0.5)
		\rput(1,0){$\Omega$}
      \end{pspicture}
      \caption{Beschränktes $C^\infty$-Gebiet $\Omega$}
 \end{figure}

Sei $\Omega \subset \R^n$ ein beschränktes $C^\infty$-Gebiet. Wir betrachten das Cauchy-Problem für die homogene Wellengleichung in $\Omega$.
\begin{align*}
	&\partial^2_t u - \Delta_x u  = 0 \, , \quad t > 0, x \in \Omega \\
	&\hspace{2em} u(t,x)  = 0 \, , \quad t > 0, x \in \partial \Omega \quad\text{Dirichletrandbedinung} \\
	&\hspace{0.6em} \left.\begin{aligned}
		u(0,x) & = u^0(x) \, , \quad x \in \Omega \\
		\partial_t u(0,x) & = u^1(x) \, , \quad x \in \Omega  
	\end{aligned}\quad \, \right\} \text{ Anfangswerte}
\end{align*}
Abstrakte Formulierung: Wir definieren $A: \mathring W_2^2 (\Omega) \rightarrow L_2(\Omega), W \mapsto -\Delta_x W$. Daraus folgt $A \in \mathcal L(\mathring W_2^2 (\Omega) , L_2(\Omega))$.

Man beachte, dass die Randbedingung $u|_{\partial \Omega}=0$ wird in den Definitionsbereich $\operatorname{dom} (A) = \mathring W^2_2 (\Omega)$ (vgl. Kapitel 6: $\gamma_0 \omega = 0 \, \fa \, \omega \in \mathring W^2_2 (\Omega)$).

Wir betrachten die abstrakte Wellengleichung mit $\dot u := \partial_t u, \ddot u :=\partial_{tt} u$ als Gleichung im Hilbertraum $H=L_2(\Omega)$:
\begin{align*}
	\ddot u + Au & = 0 \, ,\quad t >0 \\
	u(0) & = u^0 \\
	\dot u(0) & = u^1
\end{align*}
mit (gesuchter) Funktion $u \in C^2([0,\infty),H), u(t) \in \operatorname{dom} (A) = \mathring W^2_2(\Omega)$. Somit haben wir eine gewöhnliche Differentialgleichung im Hilbertraum $H=L_2(\Omega)$.

\begin{vor}
\begin{enumerate}[(a)]
\item $H$ ist ein seperabler, komplexer Hilbertraum mit $\dim H = \infty$, Skalarprodukt $( \cdot | \cdot )$ und Norm $\norm{\, \cdot \, }$.
\item $\operatorname{dom}(A)$ ist Unterraum von $H, A \in \mathcal L(\operatorname{dom}(A),H)$.
\item $A$ ist selbstadjungierter Operator (Notation: $A = A^\ast$), d.h. $(Au | v) = (u | Av) \, \fa \, u,v \in \operatorname{dom} (A)$.
\item $A$ ist abgeschlossenen (Notation: $A \in \mathcal A(H)$), d.h. $\fa $ Folgen $(u_n)$ in $\operatorname{dom}(A)$ mit $u_n \rightarrow u$ in $H$ und $Au_n \rightarrow v$ in $H$ gilt: $u \in \operatorname{dom}(A)$ und $Au = v$.
\item Es existiert eine Orthonormalbasis $\{\varphi_j \with j \in \N\}$ in $H$ bestehend aus Eigenvektoren aus $A$ mit entsprechenden Eigenwerten $\{\lambda_j \with j \in \N\}$, d.h. $A\varphi_j = \lambda_j \varphi_j$ und $\lambda_1 \leq \lambda_2 \leq \ldots \leq \lambda_j \xrightarrow[j\ra \infty]{} \infty$.
\end{enumerate}
\end{vor}

\begin{bsp}
\label{bsp:9.10}
Sei $\Omega \subset \R^n$ ein beschränktes $C^\infty$-Gebiet.
\begin{enumerate}[(a)]
\item Sei $H:= L_2(\Omega), \operatorname{dom}(A) :=\mathring W^2_2 (\Omega), Aw := -\Delta_x w, w \in \mathring W^2_2 (\Omega)$. Es ist zu zeigen, dass die obigen Voraussetzungen erfüllt sind.
\begin{proof}
Es ist noch zu zeigen, dass $A \in \mathcal A(L_2(\Omega))$. Es ist $A=A^\ast$, da mit Gauß gilt
\[
	\int_\Omega -\Delta_x u \bar v \d x = \int_\Omega u(-\Delta_x \bar v )\d x \quad \fa \, u,v \in \mathring W^2_2(\Omega) \, .
\]
Wegen Theorem~\ref{theorem:7.37}, Bemerkung~\ref{bem:7.38} müssen wir nur zeigen, dass $A \in \mathcal A(H)$. Seien $w_j \in \mathring W^2_2(\Omega)$ mit $w_j \rightarrow w$ in $L_2 (\Omega), -\Delta_x w_j \rightarrow v$ in $L_2(\Omega)$.

Es ist zu zeigen, dass $w \in \mathring W^2_2(\Omega), -\Delta_x w = v$.

Es ist $\mathring W^2_2(\Omega) = \operatorname{cl}_{W^2_2(\Omega)} \D(\Omega)$ und es sei $\tilde e_\Omega$ die triviale Fortsetzung. 
\begin{align}
	\Ra \ & \tilde e_\Omega \in \mathcal L(\mathring W^2_2(\Omega), W^2_2(\Omega)), W^2_2 (\R^n) \stackrel[\scriptsize\text{Thm.}~\ref{theorem:8.20}]{}\cong H^2(\R^n) \notag \\
	\Ra : & \tilde w_j := \tilde e_\Omega w_j \longrightarrow \tilde e_\Omega w =: \tilde w \text{ in } L_2(\R^n) \notag \\
	& - \Delta_x \tilde w_j \longrightarrow \tilde e_\Omega v =: \tilde v \text{ in } L_2(\R^n) \notag\\
	\Ra \ & (1-\Delta_x) \tilde w_j \longrightarrow \tilde w + \tilde v \text{ in } L_2(\R^n) \notag \\
	\intertext{mit Fouriertransformation und Plancherel folgt}
	&1-\Delta_x \in \mathcal L(W^2_2 (\R^n), L_2(\R^n)) \text{ mit Inversen }\notag \\
	& (1-\Delta_x)^{-1} \in \mathcal L(L_2(\R^n), W^2_2(\R^n)) \notag \\
	\Ra \ & \underbrace{\tilde w_j}_{\scriptsize\parbox{1.4cm}{\centering$\longrightarrow \tilde w$  \\ in $L_2(\R^n)$}}\longrightarrow (1-\Delta_x)^{-1} (\tilde w + \tilde v) \text{ in } W^2_2 (\R^n) \label{eq:9.12} \\
	\Ra \ & (1-\Delta_x)^{-1} \tilde w + \tilde v = \tilde w \in H^2(\R^n) = W^2_2(\Omega) \text{, da $L_2$ Hausdorffsch} \notag \\
	\Ra \ & (1-\Delta_x) \tilde w = \tilde w + \tilde v \label{eq:9.13} \\
	& - \Delta_x \tilde w = \tilde v \label{eq:9.14} 
\end{align}
Aus Fouriertransformation und Plancherel folgt mit Theorem~\ref{theorem:8.21}, dass $1-\Delta : H^2 (\R^n) \rightarrow L_2(\R^n)$  ein isometrischer Isomorphismus ist. Also gilt
\begin{align*}
	&\hspace{1.07cm} w = r_\Omega \tilde w \stackrel{\scriptsize\eqref{eq:9.13}}\in W^2_2(\Omega) \text{ (vgl. Beweis von Theorem~\ref{theorem:6.12})} \\
	&\,   -\Delta w \stackrel{\scriptsize\eqref{eq:9.14}}= v \text{ (also $Aw = v$)} \\
& \left.
\begin{aligned}
	\stackrel{\scriptsize\eqref{eq:9.12}}\Ra & w_j = r_\Omega \tilde w_j \longrightarrow r_\Omega \tilde w = w \text{ in } W^2_2(\Omega) \\
	& w_j \in \mathring W^2_2(\Omega), \mathring W^2_2(\Omega) \text{  abg. UVR von } W^2_2(\Omega)
\end{aligned}\, 
\right\} \Ra w \in \mathring W^2_2 (\Omega) \, ,
\end{align*}
d.h. $w \in \operatorname{dom}(A)$.
\end{proof}
\item Sei $H:=L_2(\Omega), \operatorname{dom}(A) :=\{w \in W^2_2(\Omega) \with \partial_j w = 0 \text{ auf } \partial \Omega\}, Aw := -\Delta_x w$. Daraus folgt, dass die Generalvoraussetzungen von oben erfüllt sind (ohne Beweis).
\end{enumerate}
\end{bsp}

\subsubsection{Produktansatz für "`wellenförmige"' Lösungen}

Es sei $0 \neq v \in H, w : \R^+ \rightarrow \C$ mit $u(t) := w(t) v$ (Separation der Variablen). Dann ist der Ansatz: $u_{tt} = u_{xx}, u = w(t) g(x)$, damit erhalten wir
\[
	w_{tt} g = w g_{xx} \Ra \frac{w_{tt}}{w}=\frac{g_{xx}}g = \lambda \, ,
\]
also zwei Eigenwertprobleme. (Diesen Ansatz verallgemeinern wir in Banachräumen.)

Es sei $u(t)$ Lösung von
\begin{align*}
	& \ddot u + A u = 0 \, , \quad t > 0 \text{ (in } H) \\
	\stackrel{\scriptsize\text{formal}}\Ra \, & \ddot w (t) v + w(t) Av = 0 \\
	\Longleftrightarrow\ & Av = -\frac{\ddot w(t)}{w(t)} v \quad \fa \, t > 0 \\
	&  - \frac{\ddot w(t)}{w(t)} \equiv \text{const.} = \lambda \in \C \, .
\end{align*}
Somit ist folgendes Problem zu lösen.
\[
	\left| \, 
	\begin{aligned}
		-\ddot w & = \lambda w \, , \quad t > 0 \\
		Av & = \lambda v \, , \quad \text{also } \lambda = \lambda_j \text{ für ein } j \in\N, v = \varphi_j
	\end{aligned}
	\right.
\]
Aus der gewöhnlichen Differentialgleichung folgt $w_j(\, \cdot \,, \alpha_j, \beta_j)$ mit $\alpha_j, \beta_j \in \R$ und
\[
	w_j(\, \cdot\, , \alpha_j, \beta_j) = \begin{cases}
							\alpha_j e^{t\sqrt{\abs{\lambda_j}}} + \beta_j e^{-t \sqrt{\abs{\lambda_j}}} & , \lambda_j < 0 \\
							\alpha_j + t \beta_j & , \lambda_j = 0 \\
							\alpha_j \cos(\sqrt{\lambda_j} t) + \beta_j \sin(\sqrt{\lambda_j} t) & , \lambda_j > 0
						\end{cases}.
\]
Dann ist $u_j(t) := w_j(t,\alpha_j,\beta_j)\varphi_j$ eine Lösung von $\ddot u + Au = 0$.

\begin{bem*}
Für $\lambda_j > 0$ heißt $u_j$ Eigenschwingung mit der Eigenfrequenz $v_j = \frac{\sqrt{\lambda_j}}{2\pi}$.
\end{bem*}

Als Ansatz wählen wir nun
\[
	u(t) := \sum u_j (t,\alpha_j,\beta_j)\varphi_j
\]
ist Lösung von
\begin{align}
\label{eq:9.15}
\begin{aligned}
\ddot u + Au = 0 \,   \\
u(0) = u^0 \, , \\
\dot u (0) = u^1 \, .
\end{aligned}
\end{align}
\begin{align*}
\Ra u^0 &= u(0) = \sum_j w_j (0,\alpha_j,\beta_j) \varphi_j \\
u^1 &= \dot u(0) = \sum_j \dot w_j (0,\alpha_j,\beta_j) \varphi_j \\
\stackrel{\scriptsize\text{ONB}}\Ra (u^0|\varphi_j) & = w_j(0,\alpha_j,\beta_j) \, , \\
(u^1|\varphi_j) & = \dot w_j(0,\alpha_j,\beta_j) \quad \fa \, j  
\end{align*}
Löse das lineare Gleichungssystem für $t = 0$ (indem wir in die gewöhnliche DGL einsetzen)
\begin{align}
\label{eq:9.16}
\begin{aligned}
	\Ra \alpha_j & = \begin{cases}
					\frac 1 2 (u^0|\varphi_j) + \frac 1{2\sqrt{\abs{\lambda_j}}} (u^1|\varphi_j) & , \lambda_j < 0 \\
					(u^0|\varphi_j) & , \lambda_j \geq 0 
				\end{cases} \\
	\beta_j & = \begin{cases}
				\frac 12 (u^0|\varphi_j) - \frac 1{2\sqrt{\abs{\lambda_j}}} (u^1|\varphi_j) & , \lambda_j < 0 \\
				(u^1|\varphi_j) & , \lambda_j = 0 \\
				\frac 1{\sqrt{\lambda_j}}(u^1|\varphi_j) & , \lambda_j > 0
			\end{cases}
\end{aligned}
\end{align}
Formal gilt: \eqref{eq:9.15} besitzt eine eindeutige Lösung. Sie wird durch
\[
	u(t) = \sum_j w_j (t,\alpha_j,\beta_j) \varphi_j
\]
mit $\alpha_j, \beta_j$ aus \eqref{eq:9.16} gegeben. Hierbei tritt jedoch das Problem auf, ob die Reihe überhaupt absolut konvergiert.

\begin{defi}
Für alle $s \geq 0$ definieren wir
\[
	H_A^s := \left( \Big\{x \in H \with \sum_j  \abs{\lambda_j}^{2s} \abs{(x|\varphi_j)}^2 < \infty \Big\}, (\cdot | \cdot )_{H_A^s} \right)
\]
mit $(x|y)_{H_A^s} := (x|y) + \sum\limits_j \abs{\lambda_j}^{2s} (x|\varphi_j) \overline{(y |\varphi_j)}$.
\end{defi}

\begin{satz}
\label{satz:9.11}
Für alle $s \geq 0$ gilt:
\begin{enumerate}[\rm(i)]
\item $H_A^s$ ist ein Hilbert-Raum, $\mathring H_A^s = H$.
\item Für $0 \leq t\leq s$ gilt: $H_A^s \stackrel{d}\hookrightarrow H_A^t$.
\item $A\in \mathcal L(H_A^{s+1},H_A^s)$. 
\end{enumerate}
\end{satz}

\begin{proof}
Siehe Skript.
\end{proof}

\begin{kor}
\label{kor:9.12}
$\operatorname{dom}(A) = H_A^1$ und $Ax = \sum_j \lambda_j (x|\varphi_j) \varphi_j, \, \fa\, x \in \operatorname{dom}(A)$.
\end{kor}

\begin{proof}
Aus Satz~\ref{satz:9.11} folgt $H_A^1 \subset \operatorname{dom}(A)$. Sei $x \in \operatorname{dom}(A)$, dann folgt $Ax \in H$.
\begin{align}
\label{eq:9.17}
\stackrel{\scriptsize\text{ONB}}\Ra & Ax = \sum_j (Ax | \varphi_j) \varphi_j \stackrel{A=A^\ast}= \sum_j \lambda_j (x | \varphi_j) \varphi_j \\
\stackrel{\scriptsize\text{ONB}}\Ra & \norm x^2_{H^1_A} = \norm x^2 + \sum_j \abs{\lambda_j}^2 \abs{(x|\varphi_j)}^2 \stackrel{\scriptsize\eqref{eq:9.17}}= \norm x^2 + \norm{Ax}^2 < \infty \, ,
\notag
\end{align}
d.h. $x \in H^1_A$.
\end{proof}

\begin{theorem}
\label{theorem:9.13}
Es gelte die Generalvoraussetzung von oben $($Voraussetzungen$)$. Dann gilt: Für alle $(u^0,u^1) \in H_A^1 \times H^{\frac 1 2}_A \, \exists ! $ Lösung $u$ des Anfangswertproblems 
\[
	\ddot u + Au = 0 \, , \quad t > 0 , u(0) = u^0, \dot u (0 )= u^1
\]
und es gilt
\[
	u \in C(\R^+,H^1_A) \cap C^1\Big(\R^+,H^{\frac 1 2}_A\Big) \cap C^2 (\R^+, H) \, .
\]
Sie wird gegeben durch die Reihe
\[
	u(t) = \sum_j w_j(t) \varphi_j
\]
mit $t > 0$ $($oder $t\in \R)$ mit $\ddot w_j + \lambda_j w_j = 0, w_j(0) = (u^0|\varphi_j),\dot w_j(0) = (u^1|\varphi_j) \, \fa \, j \in \N$.
\end{theorem}


\newchapter{Wärmeleitungsgleichung}

\section{Die Wärmeleitungsgleichung auf dem $\R^n$}

Wir betrachten das folgende Problem.
\begin{align}
\label{eq:10.1}
	\left.
	\begin{aligned}
		\partial_t u - \Delta_x u & = 0 \, , \qquad \ \, t > 0 , x \in \R^n \\
		u(0,x) & = \varphi (x) \, , \quad x \in \R^n
	\end{aligned}
	\, \right\}
\end{align}
mit $\varphi: \R^n \rightarrow \C$ (Anfangsverteilung) gegeben und $u :(0,\infty) \times \R^n \rightarrow \C$ (Verteilung von bspw. Wärme oder Individuen im Raum) gesucht.

Die Wärmeleitungsgleichung ist ein Prototyp einer parabolischen Gleichung. Wir nehmen an $\varphi \in \mathcal S(\R^n), u $ sei Lösung von \eqref{eq:10.1} mit $u(t,\cdot) \in \mathcal S(\R^n)$ mit $\partial_t \hat u(t,\cdot ) = \widehat{\partial_t u} (t,\cdot)$.

Man wende nun Fouriertransformation auf \eqref{eq:10.1} an.
\begin{align*}
	\Ra \, & \partial_t u + \abs\xi^2 \hat u = 0 \, , \quad \hat u (0,\cdot) = \hat \varphi \\
	\intertext{Dies ist eine gewöhnliche Differentialgleichung mit $\xi$ als Parameter.}
	\Ra \, & \hat u (t, \xi) = e^{-t\abs{\xi}^2}\hat\varphi (\xi) \, , t \geq 0, \xi \in \R^n \\
	\intertext{mit $e^{-t\abs{\, \cdot\, }^2} \in \mathcal O_M \, \fa \, t \geq 0$}
	\Ra \, & u(t,\cdot) = \F^{-1} \Big( \underbrace{e^{-t\abs{\, \cdot \, }^2}}_{=\F (\F^{-1} e^{-t \abs{\, \cdot \, }^2})} \hat \varphi \Big) \\
	&\hspace{0.7em} \stackrel[\scriptsize\text{Satz}~\ref{satz:8.1}]{\scriptsize\text{Faltungssatz}}= \underbrace{(2\pi)^{-\frac n2} \F^{-1} \Big(e^{-t\abs{\, \cdot\, }^2}\Big)}_{=: K_t(\cdot)} \ast \varphi
\end{align*}
$K_t$ heißt \idx{Gaußkern} (\idx{Wärmeleitungskern}), wobei
\[
	K_t \stackrel{\parbox{1.5cm}{\centering\scriptsize Thm.~\ref{theorem:8.7}, Satz~\ref{satz:8.1}, Lemma~\ref{lemma:8.6}}}= (4\pi t)^{-\frac n2} e^{-\frac{\abs{\cdot }^2}{4t}} , \quad t > 0 \, .
\]
Dann ist $u(t,\cdot) = K_t \ast \varphi, t > 0, K_t \in \mathcal S(\R^n) \, \fa \, t > 0$. Die Existenz und Eindeutigkeit folgt, wenn $\varphi$ schnellfallend.

\begin{lemma}
\label{lemma:10.1}
\begin{enumerate}[\rm(i)]
\item $\fa \, t > 0, 1\leq p \leq \infty: \norm{K_t}_{L_p(\R^n)} = p^{-\frac n2} (4\pi t)^{\frac n2 (\frac 1p -1)}$.
\item $\fa \, t > 0, 1 \leq q,r\leq \infty: (f \mapsto K_t \ast f) \in \mathcal L(L_q(\R^n), L_r(\R^n))$ und 
\[
	\norm{K_t \ast }_{\mathcal L(L_q,L_r)} \leq c(q,r) t^{-\frac n2 (\frac 1q-\frac 1r)}
\]
mit $c(q,q) = 1$.
\end{enumerate}
\end{lemma}

\begin{proof}
\begin{enumerate}[(i)]
\item Wir rechnen nach mit Transformation
\begin{align*}
	z := \sqrt{\frac{p}{2t}}x & \Ra \d x = \left( \frac p{2t}\right)^{-\frac n2} \d z\, . \\
	\norm{K_t}_{L_p(\R^n)}^p & = (4\pi t)^{-\frac{np}2} \int_{\R^n} e^{-\frac{p\abs{x}^2}{4t}} \d x \\
	& = (4\pi t)^{-\frac{np}2} \left(\frac p{2t}\right)^{-\frac n2} \underbrace{\int_{\R^n} e^{\frac{\abs z^2}2} \d z}_{\parbox{2.8cm}{\scriptsize $= (2\pi)^{\frac n2} \F(e^{-\frac{\abs\cdot^2}2} )(0)$ \\ $= (2\pi)^{\frac n2}$ (Lem.~\ref{lemma:8.6})}}
\end{align*}
\item Die Behauptung folgt aus (i) und Young (Lemma~\ref{lemma:3.1}).\qedhere
\end{enumerate}
\end{proof}

\begin{theorem}
\label{theorem:10.2}
Sei $1\leq p < \infty, \varphi \in \L_p(\R^n),  u(t,x) := (K_t\ast \varphi)(x), t > 0, x \in \R^n$. Dann folgt $($per Definition$)$
\begin{enumerate}[\rm(i)]
	\item $u \in C^\infty((0,\infty)\times\R^n)$ ist Lösung von $\partial_t u - \Delta_x u = 0 ,  t > 0, x \in \R^n$,
	 \item $ u(t,\cdot) \xrightarrow{t\rightarrow 0^+} \varphi \text{ in } L_p(\R^n)$.
	 \item$ \varphi \in BC(\R^n) \cap L_p(\R^n) \Ra u \in C([0,\infty)\times \R^n),  u(0,\cdot) = \varphi$.
\end{enumerate}
\end{theorem}

\begin{proof}
Erinnerung:
\[
	(f\ast g) (x) = \int_{\R^n} f(x-y) g(y) \d y \, .
\]
\begin{enumerate}[(i)]
\item Sei $[(t,x) \mapsto K_t(x) ] \in C^\infty ((0,\infty)\times \R^n)$. Dann folgt wegen der Differenzierbarkeit von Parameterintegralen, dass $u \in C^\infty ((0,\infty)\times \R^n)$. Nachrechnen: $ (\partial_t - \Delta_x) K_t(x) = 0, t>0, x \in \R^n$.
\[
	\Ra \, (\partial_t - \Delta_x)u \stackrel{\scriptsize\text{vgl. Kapitel 3}}= ((\partial_t-\Delta_x)K_t)\ast \varphi = 0 
\]
\item Aus Lemma~\ref{lemma:10.1} folgt
\[
	\int_{\R^n} K_t (x) = 1 \, , \quad t > 0 \, .
\]
Also ist
\begin{align*}
	& \norm{u(t,\cdot) - \varphi}_{L_p(\R^n)} \\
	  =\hspace{0.75em} & \Norm{\int_{\R^n} (\varphi (x-y ) - \varphi (x)) (4\pi t)^{-\frac n2} e^{-\frac{\abs y}{4t}} \d y}_{L_p(\R^n)} \\
	\stackrel{z := \frac y{\sqrt t}}= & \Norm{\int_{\R^n} (\varphi (x-\sqrt t z) - \varphi(x)) e^{-\frac{\abs z^2}4} \d z (4\pi)^{-\frac n2}}_{L_p(\R^n)}  \\
	\leq \hspace{0.65em}& c \int_{\R^n} \underbrace{\norm{\varphi (\cdot - \sqrt t z)-\varphi(\cdot)}_{L_p(\R^n)}}_{\xrightarrow{t\rightarrow 0} 0} e^{-\frac{\abs z^2}4} \d z \xrightarrow[t \rightarrow 0^+]{\scriptsize\text{Lebesgue}} 0 \, , 
\end{align*}
weil wir eine $L_1$-Majorante gibt: $\abs{\,\cdot\,} \leq 2 \norm \varphi_{L_p(\R^n)} e^{-\frac{\abs\cdot^2}4} \in L_1(\R^n)$.
\item Sei $\varphi \in BC(\R^n)$. Mit derselben Transformation wie in (ii) folgt
\begin{dmath*}
	u(t,x) = \int_{\R^n} \varphi(x-y) (4\pi t)^{-\frac n2} e^{-\frac{\abs y^2}{4t}} \d y 
	= \int_{\R^n} \underbrace{\varphi(x-\sqrt t z) e^{-\frac{\abs z^2}4}}_{\abs{\, \cdot\,} \leq \norm\varphi_\infty e^{-\frac{\abs z^2}4} \in L_1(\R^n)} \d z (4 \pi)^{-\frac n2}\\ 
	\xrightarrow[t\rightarrow 0^+]{\scriptsize\text{Lebesgue}} \int_{\R^n}\varphi(x) e^{-\frac{\abs z^2}4} \d z (4\pi)^{-\frac n2} = \varphi(x) \text{ (wegen Lemma~\ref{lemma:8.6})},
\end{dmath*}
d.h. $u$ kann in $t=0$ stetig fortgesetzt werden. \qedhere
\end{enumerate}
\end{proof}

Damit hat man die Wärmeleitungsgleichung auf $\R^n$ gelöst.

\begin{bem}
\label{bem:10.3}
\[
	u(t,x) = (K_t \ast \varphi)(x) = (4\pi t)^{-\frac n2} \int_{\R^n} \varphi(y) e^{-\frac{\abs{x-y}^2}{4t}} \d y
\]
\begin{enumerate}[(a)]
\item Regularisierung: $u(0,\cdot) := \varphi \in L_p(\R^n) \Ra u \in C^\infty ((0,\infty)\times \R^n)$ (Regularisierungsgewinn).
\item Unendliche Ausbreitungsgeschwindigkeit:
\[
	u(0,\cdot) = \varphi \text{ mit }\varphi \geq 0, \varphi\not\equiv 0 \Ra u(t,x) >0 \, \fa \, t> 0\, \fa \, x \in \R^n,
\]
d.h. die Temperatur ist zu jeder Zeit $t>0$ an jedem Ort $x \in\R^n$ strikt positiv im Gegensatz zur Wellengleichung.
\item Die Temperatur $u$ wird im Raumpunkt $x \in \R^n$ von den Anfangswerten in allen (noch so weit entfernten) Punkten $y$ beeinflusst (obwohl der Einfluss exponentiell abnimmt mit zunehmender Distanz).
\item Maximumsprinzip: $\inf \varphi \leq u(t,x) \leq \sup \varphi, t > 0, x \in \R^n$.
\item Eindeutigkeit: Das Anfangswertproblem \eqref{eq:10.1} besitzt höchstens eine Lösung $u$ mit
\[
	\abs{u(t,x)} \leq  M e^{\alpha\abs x^2} , \quad t > 0, x \in \R^n
\]
mit $\alpha, M > 0$.

Speziell: $\varphi \in BC(\R^n) \stackrel{\scriptsize\text{(d)}}\Ra$ Lösung von Theorem~\ref{theorem:10.2} ist eindeutig.
\end{enumerate}
\begin{proof}
Jost.
\end{proof}
\end{bem}

\begin{satz}
\label{satz:10.4}
Sei $1\leq p < \infty, \varphi \in L_p(\R^n)\cap L_\infty (\R^n)$ und $u(t,\cdot) := K_t \ast \varphi, t >0$. Dann folgt
\[
	\exists c \in \K : u(t,x) \xrightarrow{t\rightarrow \infty} c \quad \fa \, x \in \R^n ,
\]
wobei $c = 0$ ist, falls $\supp \varphi \Subset \R^n$.
\end{satz}

\begin{proof}
Sei $\mathscr S := \supp \varphi \Subset \R^n$. Dann ist
\begin{align*}
u(t,x) & = (4\pi t)^{-\frac n2} \int_{\R^n} \varphi (y) e^{-\frac{\abs{x-y}^2}{4t}} \d y  = \int_{\mathscr S} \varphi (y) e^{-\frac{\abs{x-y}^2}{4t}} \d y  \\
\Ra \abs{u(t,x)} & \leq \underbrace{(4\pi t)^{-\frac n2} e^{-\frac{\operatorname{dist} (x,S)^2}{4t}}}_{\xrightarrow{t\rightarrow\infty} 0} \underbrace{\int_{\mathscr S} \abs{\varphi (y)} \d y}_{< \infty} \, .
\end{align*}
Für $\varphi$ allgemein gilt
\[
	\abs{u(t,x)-u(t-z)} \leq \norm \varphi_\infty \underbrace{(4\pi t)^{-\frac n2} \int_{\R^n} \Abs{e^{-\frac{\abs{x-y}^2}{4t}} e^{-\frac{\abs{z-y}^2}{4t}} }}_{\xrightarrow{t\rightarrow \infty} 0 \quad \fa \, x,z \in \R^n} \d y \, .
\]
\end{proof}

\begin{theorem}
\label{theorem:10.5}
Es sei $1\leq p < \infty, U(t) \ast \varphi := K_t \ast \varphi , t > 0$ und $U(0)  \varphi:=\varphi$. Dann gilt:
\begin{enumerate}[\rm(i)]
\item $\fa \, t \geq 0 : U(t) \in \mathcal L(L_p(\R^n)), \norm{U(t)}_{\mathcal L(L_p)} \leq 1$, d.h. $U$ ist eine Kontraktion, und $U(0) = \id_{L_p}$.
\item $\fa \, t,s \geq 0: U(t+s) = U(t)U(s)$ $($kommutieren!$)$.
\item $\fa \, \varphi \in L_p(\R^n): U(t) \varphi \xrightarrow{t\ra 0^+} \varphi$ in $L_p(\R^n)$, d.h. $U$ ist stark stetig.
\end{enumerate}
Das heißt zusammen, $\{U(t) \with t \geq 0\}$ ist eine stark stetige Kontraktionsgruppe auf $L_p(\R^n)$.
\end{theorem}

\begin{proof}
\begin{enumerate}[(i)]
\item Lemma~\ref{lemma:10.1}.
\item Übung (nachrechnen).
\item Theorem~\ref{theorem:10.2} (ii).\qedhere
\end{enumerate}
\end{proof}

\begin{kor}
\label{kor:10.6}
Sei $1\leq p < \infty, \varphi \in L_p(\R^n)$, dann folgt 
\[
	U(\cdot) \varphi \in C([0,\infty),L_p(\R^n)) \, .
\]
\end{kor}

\begin{proof}
Es sei o.B.d.A. $t > 0$ ($t = 0$ folgt direkt aus Theorem~\ref{theorem:10.5} (iii)), dann ist
\[
	\fa \, t,h > 0 : U(t+h)\varphi = U(h)U(t) \varphi \xrightarrow[\scriptsize\text{Thm.\ref{theorem:10.5} (iii)}]{h \ra 0^+\scriptsize\text{ in } L_p} U(t) \varphi \, .
\]
Sei $0 < h < t$, dann folgt
\begin{align*}
	\norm{U(t-h)\varphi - U(t) \varphi}_{L_p(\R^n)} & \stackrel{\scriptsize\text{Thm.\ref{theorem:10.5}(ii)}}= \norm{U(t-h)(\varphi-U(h) \varphi)}_{L_p(\R^n)} \\
	& \hspace{0.2em} \stackrel{\scriptsize\text{Thm.\ref{theorem:10.5}(i)}}\leq\hspace{0.05em} \norm{\varphi-U(h) \varphi}_{L_p (\R^n)} \xrightarrow[\scriptsize\text{Thm.\ref{theorem:10.5} (iii)}]{h \ra 0^+} 0\, .
\end{align*}
\end{proof}

\begin{bem}[inhomogener Fall]
\label{bem:10.7}
Wir betrachten 
\begin{align*}
	\partial_t u - \Delta_x u & = f(t,x) \, , \quad t > 0, x \in \R^n \\
	u(0,x) & = \varphi (x) \\
	\stackrel[\scriptsize\text{trafo}]{\scriptsize\text{Fourier-}}\Ra \partial_t \hat u + \abs \xi^2 \hat u & = \hat f(t,\cdot) \\
	\hat u(0,\cdot) & = \hat \varphi \, .
\end{align*}
Dann folgt nach Lösen der gewöhnlichen DGL mit Parameter $\xi$ und dem Anfangswert von $\F^{-1}$
\[
	u(t,\cdot) = K_t \ast \varphi +\int_0^t K_{t-s} \ast f(s,\cdot) \d s
\]
mit der Notation aus Theorem~\ref{theorem:10.5}:
\[
	u(t) = U(t) \varphi + \int_0^t U(t-s) f(s) \d s \, , \quad t \geq 0
\]
heißt Variation-der-Konstanten-Formel\index{Variation der Konstanten}.
\end{bem}

\section{Wärmeleitungsgleichung auf einem Gebiet $\Omega$}

Als Generalveraussetzung fordern wir, dass $\Omega\subset \R^n$ ein $C^\infty$-Gebiet ist.
Wir betrachten zunächst das Cauchy-Problem für die Wärmeleitungsgleichung
\begin{align}
\label{eq:10.2}\left.
\begin{aligned}
	\partial_t u -\Delta_x u & = 0 \, , \qquad t > 0 , x \in \Omega \\
	u(t,x) &= 0 \, , \qquad t > 0, x \in \partial \Omega \\
	u(0,x) & = u^0 (x) \, , \quad x \in \Omega
\end{aligned}
\quad \right|
\end{align}
Abstrakte Formulierung (vgl. Wellengleichung):
\[
	A : \mathring W^2_2(\Omega) \longrightarrow L_2(\Omega) \, , \quad w \mapsto -\Delta_x w
\]
Dann ist \eqref{eq:10.2} äquivalent zu
\[
	\dot u + Au = 0 \, , \quad t > 0\, , \qquad u (0 ) = u^0 \text{ in } L_2(\Omega)
\]
und dabei handelt es sich um eine gewöhnliche Differentialgleichung im Hilbertraum.

\begin{vor}
\begin{enumerate}[(a)]
\item $H$ ist ein $\C$-Hilbertraum mit  $\dim H = \infty$, Skalarprodukt $( \cdot | \cdot  )$ und Norm $\norm{\,\cdot \,}$.
\item $\operatorname{Dom}(A)$ ist Unterraum von $H, A \in \mathcal L(\dom(A),H)\cap \mathcal A(H), A^\ast = A$.
\item Es existiert eine ONB $\{\varphi_j\}$ von $H$ bestehend aus Eigenvektoren von $A$ mit Eigenwerten $\lambda_1 \leq \lambda_2 \leq \ldots \leq \lambda_j \xrightarrow[j \ra \infty]{} \infty$.
\end{enumerate}
\end{vor}
Diese Voraussetzungen sind mit dem oberen Operator bei \eqref{eq:10.2} erfüllt. Wir betrachten das Cauchy-Problem
\[
	\dot u + A u = 0 \, , \quad t > 0 \, , \qquad u(0) = u^0 \text{ in } H
\]
mit $u^0 \in H$ gegeben und $u:(0,\infty) \ra \dom(A)$ gesucht.

Idee: Produktsatz $u(t) = w(t) v, w(t) \in \C, v \in \dom (A)$. (Im eindimensionalen ist die Idee analog zur Wellengleichung.)

\begin{notation}
$H_A^s$ ist wie in Kapitel 9 definiert. Weiterhin definieren wir
\[
	H^\infty_A := \bigcap_{s\geq 0} H_A^s
\]
und $f \in C^{\infty}(J,H^\infty_A) :\Longleftrightarrow f \in C^\infty(J,H^s_A)\, \fa \, s \geq 0$. 
\end{notation}

\begin{theorem}
\label{theorem:10.8}
Es gelten die Voraussetzungen von oben. Dann gilt,
	$\fa \, u^0 \in H \, \exists! \, \text{Lösung } u \in C^\infty ((0,\infty),H^\infty_A) \cap C([0,\infty),H)$ von
\begin{align}
\label{eq:CP}
	\dot u +Au = 0 \, , \quad  t>0\, , \qquad u(0) = u^0  . \tag{CP}
\end{align}
Sie wird gegeben durch die Reihe
\begin{align}
\label{eq:10.3}
	u(t) = \sum_j e^{-\lambda_j t} (u^0|\varphi_j) \varphi_j \, , \quad t \geq 0 \, .
\end{align}
\end{theorem}

\begin{bem*}Regularisierung bzgl. $x$:
$$u^0 \in H = H^0_A \Ra \, \fa \, t > 0: u(t) \in \bigcap_{s\ge 0} H^s_A \, .$$
\end{bem*}

\begin{proof}
Sei $u^0 \in H$ beliebig, definiere $u$ wie in \eqref{eq:10.3}. Dann folgt
\[
	u(0) = \sum_j (u^0|\varphi_j)\varphi_j \stackrel{\scriptsize\text{ONB}}= u^0.
\]
Aus $\lambda_j \nearrow \infty$ folgt o.B.d.A. $\lambda_j > 0 \, \fa \, j$. Für $t \geq \epsilon > 0, s  > 0$ gilt
\begin{align*}
	\sum_j \lambda_j^{2s} e^{-\lambda_j t}\abs{(u^0|\varphi_j)}^2 & \leq \sum_j \underbrace{\lambda_j^{2s} e^{-\lambda_j \epsilon}}_{\leq c_s} \abs{(u^0|\varphi_j)}^2 \\
	& \leq  c_s \sum_j\abs{(u^0|\varphi_j)}^2 \stackrel{\scriptsize\text{ONB}}= c_s \norm{u^0}^2 < \infty \, .
\end{align*}
Aus Weierstraß und der ONB folgt: Die Reihe $u(t)$ konvergiert absolut und lokal gleichmäßig bzgl. $t \geq \epsilon > 0$ in $H^s_A ,\, \fa \, s >0$.

Die Summanden sind stetig in $t$:
\[
	\Ra u \in C((0,\infty),H_A^s) \quad \fa \, s \geq 0 \, .
\]
Mit $s = 0$ ist dies analog: $C([0,\infty),H)$. Analog oben: Die gliedweise differenzierte Reihe
\[
	\sum_j (-\lambda_j) e^{-\lambda_jt} (u^0|\varphi_j)\varphi_j
\]
konvergiert absolut und lokal gleichmäßig bzgl. $t\geq \epsilon > 0$ in $H^s_A$. 

Mit dem Satz über die gliedweise Differentiation von Reihen ist $u \in C^1((0,\infty),H^s_A)$ und
\[
	\dot u (t) = \sum_j (-\lambda_j) e^{-\lambda_jt} (u^0|\varphi_j)\varphi_j\, , \quad t>0 \, \fa \, s \geq 0
\]
folgt induktiv $u \in C^\infty ((0,\infty),H^s_A) \, \fa \, s \geq 0$ und damit  $u \in C^\infty ((0,\infty),H^\infty_A)$.

Ferner: $u (t) \in H^1_A = \dom (A) , t > 0$ (vgl. Korollar~\ref{kor:9.12}) und
\[
	Au(t) = \sum_j e^{\lambda_j t}(u^0|\varphi_j)\underbrace{A\varphi_j}_{=\lambda_j \varphi_j} =-\dot u(t) \, , \quad t >0 \, .
\]
Daraus folgt per Definition die Existenz und Regularität. Zur Eindeutigkeit: Es sei $\dot w + Aw = 0, t >0, w(0) = 0$ mit $w = u-v$, wobei $u, v$ zwei Lösungen von \eqref{eq:CP} sind. Dann folgt
\[
	0 = (\dot w(t)|\varphi_j) + (Aw(t)|\varphi_j) \stackrel[\lambda_j \in \R]{A = A^\ast}= (\dot w(t)+ \lambda_j w(t)|\varphi_j) \quad \fa \, j \, .
\]
Setze $\alpha_j (\cdot) := (w(\cdot)|\varphi_j) \in C^1((0,\infty),\C)$.
\[
	\Ra : \, \dot \alpha_j + \lambda_j \alpha_j = 0 \, , \qquad t > 0 \, , \quad\alpha_j(0) = 0 \stackrel{\scriptsize\text{gew. DGL}}\Ra \alpha_j \equiv 0 \, ,
\]
jedoch ist
\[
	w(t) \stackrel{\scriptsize\text{ONB}}= \sum_j \underbrace{(w(t)|\varphi_j)}_{\alpha_j(t)=0} \varphi_j = 0\, .
\]
\end{proof}

\begin{kor}
\label{kor:10.9}
Sei $\Omega \subset \R^n$ ein beschränktes $C^\infty$-Gebiet. Dann folgt, für alle $u^0 \in L_2(\Omega)\, \exists ! \, u \in C^\infty((0,\infty),\mathring W^2_2(\Omega)) \cap C([0,\infty),L_2(\Omega))$ für die Wärme-leitungsgleichung
\begin{align*}
	\partial_t u -\Delta_x u & = 0 \, , \qquad t > 0 , x \in \Omega \\
	u(t,x) &= 0 \, , \qquad t > 0, x \in \partial \Omega \\
	u(0,x) & = u^0 (x) \, , \quad x \in \Omega \, .
\end{align*}
\end{kor}

\begin{proof}
Theorem~\ref{theorem:10.8}, Beispiel~\ref{bsp:9.10} (a), Theorem~\ref{theorem:7.37}, Bemerkung~\ref{bem:7.38}.
\end{proof}

\begin{bem*}
Es gilt $u \in C^\infty((0,\infty),C^\infty(\bar\Omega))$.
\end{bem*}

\begin{bem}
\label{bem:10.10}
Eine analoge Aussage gilt für die Neumannrandbedingungen
\[
	\partial_\nu u (t,x) = 0 \, , \quad t > 0, x \in \partial\Omega
\]
bzw. für allgemeine gleichmäßige elliptische Differentialpoeratoren zweiter Ordnung
\[
	\partial_t u - \sum_{j,k=1}^n a_{jk}(x) \partial_j\partial_k u + \sum_{j=1}^n b_j \partial_j u + c(x) u = 0
\]
mit $a_{jk} (x) = a_{kj} (x) \geq \alpha > 0 \, \fa \, x \in \bar \Omega$.
\end{bem}

\begin{theorem}
\label{theorem:10.11}
Es gelten die Voraussetzungen von oben. Sei
\[
	e^{-tA} u^0 := u(t) = \sum_j e^{-\lambda_j t} (u^0|\varphi_j)\varphi_j \, , \quad t \geq 0 , u^0 \in H
\]
die eindeutige Lösung von \eqref{eq:CP} $\dot u + Au = 0, t>0, u^0 = u(0)$. Dann ist $\{e^{-tA} \with t\geq 0\}$ eine starkstetige Halbgruppe auf $H$, d.h. es gelten
\begin{enumerate}[\rm(i)]
\item $\fa \, t \geq 0: e^{-tA} \in \mathcal L(H), e^{-0A} = id_H$.
\item $\fa \, t,s \geq 0: e^{-sA} e^{-tA} = e^{-(t+s)A}$.
\item $\fa \, u^0 \in H: e^{-tA} u^0 \xrightarrow{t\ra 0^+} u^0$ in $H$.
\end{enumerate}
Ferner gelten folgende Eigenschaften:
\begin{enumerate}[\rm(iv)]
\item $\norm{e^{-tA}}_{\mathcal L(H)} \leq e^{-\lambda_1 t} , t \geq 0$.
\item[\rm(v)] $(t\mapsto e^{-tA} u^0) \in C^\infty ((0,\infty), H) \, \fa \, u^0 \in H$ $($d.h. die Halbgruppe ist analytisch$)$ mit 
\[
	\frac\d{\d t} e^{-tA} u^0 = -A e^{-tA} u^0 .
\]
\end{enumerate}
\end{theorem}

\begin{proof}
\begin{enumerate}[(i)]
\item Es ist klar, dass $e^{-tA}: \ra H$ linear ist $\fa \, t\geq 0$.
\begin{align*}
	\norm{e^{-tA} u^0}^2 & \stackrel{\scriptsize\text{ONB}}=\sum_j e^{-2\lambda_j t} \abs{(u^0|\varphi_j)}^2 
	 \leq e^{-2\lambda_j t} \underbrace{\sum_j \abs{(u^0|\varphi_j)}^2}_{=\norm{u^0}^2} 
\end{align*}
\[
	\Ra e^{-tA} \in \mathcal L(H)\, , \quad \norm{e^{-tA}}_{\mathcal L(H)} \leq e^{-\lambda_1 t} , \quad t \geq 0
\]
Und damit ist klar, dass $e^{-0A} u^0 = u^0$ ist.
\item Wir rechnen einfach nach.
\begin{dmath*}
	e^{-sA} e^{-tA} u^0 = \sum_j e^{-\lambda_j s} (\underbrace{e^{-tA} u^0}_{=\sum_k e^{-\lambda_k t} (u^0|\varphi_k)\varphi_k} |\varphi_j)\varphi_j 
	= \sum_{j,k} e^{-\lambda_j s}e^{-\lambda_k t} (u^0 | \varphi_k) \underbrace{(\varphi_k |\varphi_j)}_{=\delta_{jk}} \varphi_j \\
	\stackrel{\scriptsize\text{ONB}}= \sum_j e^{-\lambda_j (s+t)} (u^0|\varphi_j)\varphi_j
	= e^{-(t+s)A} u^0
\end{dmath*}
\item Vgl. Theorem~\ref{theorem:10.8}.
\item Siehe (i).
\item Theorem~\ref{theorem:10.8}.\qedhere
\end{enumerate}
\end{proof}

Halbgruppen: Wir betrachten $\dot u + A u = 0, t>0, u(0)= u^0$ als gewöhnliche DGL in Banachraum $E$ mit $A\in\mathcal L(\dom(A),E), \dom(A)$ Unterraum von $E, u^0\in E$ gegeben und $u:(0,\infty) \ra \dom(A)$ ist gesucht.
\[
	\ddot w + \mathbb{A} w = 0 \, , \quad u := \begin{pmatrix} w \\ \dot w \end{pmatrix}  \Ra \dot u + A u = 0
\]

\begin{kor}
\label{kor:10.12}
Es gilt $\dom(A) = H^1_A$ und
\[
	Ax = \sum_j \lambda_j (x|\varphi_j)\varphi_j \quad  \fa \, x \in \dom(A) \, .
\]
\end{kor}

\begin{proof}
Aus dem Beweis von Theorem~\ref{theorem:10.11} folgt $H^1_A \subset \dom(A)$. Es sei $x \in \dom (A)$, dann ist $Ax \in H$.
\begin{align}
	\label{eq:10.4}
	\stackrel{\scriptsize\text{ONB}}\Ra \, & Ax = \sum_j (Ax|\varphi_j)\varphi_j \stackrel[A=A^\ast]{}= \sum_{j} \lambda_j (x|\varphi_j)\varphi_j \\
	\stackrel{\scriptsize\text{ONB}}\Ra \, & \norm x^2_{H^1_A} = \norm x^2 + \sum_j \abs{\lambda_j}^2\abs{(x|\varphi_j)}^2 \stackrel[\scriptsize\eqref{eq:10.4}]{}= \norm x^2 + \norm{Ax}^2 < \infty \,  , \notag
\end{align}
d.h. $x \in H^1_A$.
\end{proof}

\begin{theorem}
\label{theorem:10.13}
Es seien die Voraussetzungen von oben erfüllt. Dann gilt, $\fa \, (u^0,u^1)\in H^1_A \times H^{\frac 12 }_A \, \exists!$ Lösung $u$ des Anfangswertproblems
\[
	\ddot u + A u = 0 \, , \quad t > 0 \, , \qquad u(0) = u^0 \, , \quad \dot u (0) = u^1
\]
und es gilt
\[
	u \in C(\R^+,H^1_A) \cap C^1(\R^+,H_A^{\frac 12})\cap C^2(\R^+,H) \, .
\]
Sie wird gegeben durch
\[
	u(t) = \sum_j w_j(t) \varphi_j \, , \quad t \geq 0 \ (\text{oder } t \in \R)
\]
mit $\ddot w_j + \lambda_j w_j = 0, w_j(0) = (u^0|\varphi_j),\dot w_j(0) = (u^1|\varphi_j), j \in \N$.
\end{theorem}

\begin{proof}
\begin{enumerate}[(i)]
\item \underline{Existenz und Regularität:} Setze
\begin{align*}
	u(t) & := \sum_j w_j(t) \varphi_j \, , \quad
	u_1(t)  := \sum_j \dot w_j(t) \varphi_j \, , \quad
	u_2(t)  := \sum_j \ddot w_j(t) \varphi_j\, .
\end{align*}
Die Konvergenz wegen der ONB genau dann, wenn
\[
	\sum_j \abs{w_j(t)}^2 < \infty \, , \quad \sum_j \abs{\dot w_j(t)}^2 <\infty \, , \quad \sum_j \abs{\ddot w_j(t)}^2 <\infty \, . 
\]
Es sei o.B.d.A. $\lambda_j \geq 1 \, \fa \, j$, dann folgt
\begin{align}\label{eq:10.5}
	\Ra w_j (t) & = (u^0|\varphi_j) \cos (t\sqrt{\lambda_j}) + (u^1 |\varphi_j) \frac 1{\sqrt{\lambda_j}} \sin(t\sqrt{\lambda_j}) \notag \\
	\abs{w_j(t)}^2 & \leq 2 \abs{(u^0|\varphi_j)}^2 + 2 \frac 1{\lambda_j} \abs{(u^1|\varphi_j)}^2 \\
	\Ra \sum_j \abs{w_j(t)}^2 & \stackrel[\lambda_j\geq 1]{}\leq 2\norm{u^0}^2 + 2 \norm{u^1}^2 < \infty \quad \fa \, t \notag \, .
\end{align}
Also folgt mit Weierstraß, dass $\sum_j w_j(\cdot) \varphi_j$ konvergiert absolut und gleichmäßig bzgl. $t$ in $H$. Aus $w_j(\cdot) \varphi \in C(\R, H)$ folgt wegen der gleichmäßigen Konvergenz, dass
\[
	u  = \sum_j w_j(\cdot) \varphi \in C(\R,H) \, .
\]
Mit $(u(t)|\varphi_j) = w_j(t) \, \fa \, j,t$ folgt
\begin{align*}
	\norm{u(t)}^2_{H_A^1} & \, \,\,  =\, \, \,  \norm{u(t)}^2 + \sum_j \abs{\lambda_j}^2 \abs{w_j(t)}^2 \\
	&\stackrel[\scriptsize\eqref{eq:10.5}]{}\leq \norm{u(t)}^2 + 2\norm{u^0}^2_{H_A^1} + 2\norm{u^1}^2_{H_A^{\frac 12}} < \infty \quad \fa \, t
\end{align*}
und damit ist mit Weierstraß $u \in C(\R,H^1_A)$. Dann ist
\[
	u(0) = \sum_j w_j(0) \varphi_j = \sum_j (u^0|\varphi_j)\varphi_j \stackrel[\scriptsize\text{ONB}]{}= u^0 \, .
\]
Analog: $u_1 \in C(\R,H^{\frac 12}_A), u_1(0) = u^1$ und $u_2\in C(\R,H)$. Da $H^1_A \hookrightarrow H_A^{\frac 12}$, folgt 
\begin{align*}
	u = \sum_j w_j(\cdot) \varphi_j \in C^1(\R,H^{\frac 12}_A) \text{ und} \\
	\dot u = \left( \sum_j w_j \varphi_j\right)^\cdot = \sum_j \dot w_j \varphi_j = u_1
\end{align*}
mit dem Satz über gliedweise Differentiation von Reihen. Analog: $u \in C^2(\R,H)$ und $\ddot u = u_2$. Ferner gilt
\[
	u(t)  \in H^1_A \stackrel[\scriptsize\text{Kor.}\ref{kor:10.12}]{}= \dom (A)
\]
und $A \in \mathcal L(\dom (A),H)$. Damit ist
\begin{align*}
	Au(t) &= A \sum_j w_j (t) \varphi_j = \sum_j w_j(t) \lambda_j \varphi_j \\
	&=-\sum_j \ddot w_j(t) \varphi_j = -\ddot u(t) \, , \quad t \in \R \, .
\end{align*}
Also ist
\begin{align}
\label{eq:10.6}
	\ddot u + Au = 0, u(0) = u^0, \dot u(0) = u^1.
\end{align}
\item \underline{Eindeutigkeit:} Es seien $u,v$ Lösungen von \eqref{eq:10.6}, dann löst $w:=u-v$ das Problem
\begin{align}
\label{eq:10.7}
	& \ddot w + Aw = 0 \, , \qquad w(0) = 0 \, , \quad \dot w (0) = 0 \notag \\
	\Ra \,\, \,  & 0 = (\ddot w(t)|\varphi_j) + (Aw(t)|\varphi_j) \stackrel[A=A^\ast]{}= (\ddot w(t)  + \lambda_j w(t) |\varphi_j ) \, .
\end{align}
Setze $\alpha_j := (w_j(\cdot)|\varphi_j) \in C^2(\R,\C)$, d.h. $\ddot \alpha_j = (\ddot w_j |\varphi_j)$.
\begin{align*}
	\stackrel[\scriptsize\eqref{eq:10.7}]{}\Ra \hspace{1.4em}& \ddot \alpha_j + \lambda_j \alpha_j = 0 \, , \qquad \alpha_j(0) = 0 \, , \quad \dot \alpha_j (0 ) = 0 \\
	\stackrel[\scriptsize\text{Eind.g.DGL}]{}\Ra \, \, & \alpha_j \equiv 0 \, ,
\end{align*}
jedoch gilt $w(t) \stackrel[\scriptsize\text{ONB}]{}= \sum_j \alpha_j(t) \varphi_j = 0, $  also ist $u=v$.\qedhere
\end{enumerate}
\end{proof}

\begin{kor}
\label{kor:10.14}
Es sei $\Omega \subset \R^n$ ein beschränktes $C^\infty$-Gebiet. Dann folgt, $\fa \, (u^0,u^1) \in \mathring W^2_2(\Omega) \times \mathring W^1_2(\Omega)$ besitzt
\[
	\left.
	\begin{aligned}
		\partial^2_t u - \Delta_x u & = 0 \, , \quad t > 0, x \in \Omega \\
		u(t,x) & = 0 \, , \quad t > 0, x \in \partial \Omega \\
		u(0,x) & = u^0(x) \, , \quad x \in \Omega \\
		\partial_t u(0,x) & =u^1(x) \, , \quad x \in \Omega
	\end{aligned}
	\quad  \right|
\]
eine eindeutige Lösung $u \in C(\R^+,\mathring W^2_2(\Omega))\cap C^1(\R^+,\mathring W^1_2(\Omega))\cap C^2(\R^+,L_2(\Omega))$.
\end{kor}

\begin{proof}
Es sei $H:=L_2(\Omega), \dom(A) := \mathring W^2_2 (\Omega), Aw := -\Delta_x w$, dann folgt aus Bemerkung~\ref{bem:10.10} (a), Theorem~\ref{theorem:10.13}, Theorem~\ref{theorem:7.37} und Bemerkung~\ref{bem:7.38}: Es genügt zu zeigen, dass $H^{\frac 12}_A \dot = \mathring W^1_2(\Omega)$ bzgl. der ONB $\{\varphi_j\}$ in $L_2(\Omega)$ mit Eigenwerten $0< \lambda_1 < \lambda_2 \leq \ldots \leq \lambda_j \ra \infty$. Man beachte
\begin{align}
\label{eq:10.9}
	\Phi_j := \frac 1{\sqrt{1+\lambda_j}}\varphi_j \Ra \{\Phi_j\} \text{ ONB von } \mathring W^1_2 (\Omega) \, .
\end{align}
Vergleiche Beweis zu Theorem~\ref{theorem:7.37}: Es folgt mit Gauß
\begin{align}
\label{eq:10.10}
	\begin{aligned}
	(w |\varphi_j)_{\mathring W^1_2} & = (1+\lambda_j) (w|\varphi_j)_{L_2} \\
	& = (w|\varphi_j)_{L_2} + (\nabla w |\nabla \varphi_j)_{L_2} \quad \fa \, w \in \mathring W^1_2(\Omega) \, .
	\end{aligned}
\end{align}
Sei $w \in \mathring W^1_2(\Omega) = \dom(A) \stackrel{\scriptsize\text{Kor.\ref{kor:10.12}}}= H^1_A \hookrightarrow \mathring W^1_2 (\Omega) \cap H^{\frac 1 2}_A$.
\begin{align*}
	\norm w^2_{\mathring W^1_2} \stackrel[\scriptsize\text{Parseval}]{\scriptsize\eqref{eq:10.9}}= \sum_j \abs{(w|\Phi_j)_{\mathring W^1_2}}^2 
	\stackrel[\scriptsize\text{Def.}\Phi_j]{\scriptsize\eqref{eq:10.10}}= \sum_j (1+\lambda_j) \abs{(w |\varphi_j)_{L_2}}^2 \stackrel[\scriptsize\text{Parseval}]{}= \norm w^2_{H^{\frac 12}_A}
\end{align*}	
\begin{align}
\label{eq:10.11}
	\Ra (\dom(A), \norm{\, \cdot \,}_{\mathring W^1_2}) \, \dot = \, (\dom(A), \norm{\, \cdot \,}_{H^{\frac 12}_A})
\end{align}
Sei $w \in \mathring W^1_2(\Omega)$ beliebig, $\mathring W^2_2 = \dom(A) \stackrel{\scriptsize d}\subset \mathring W^1_2$
\begin{align*}
	\Ra \hspace{1.1em}& \, \exists \, w_j \in \dom (A): w_j \longrightarrow w \text{ in } \mathring W^1_2 \hookrightarrow L_2 = H \\
	\stackrel{\scriptsize\eqref{eq:10.11}}\Ra \hspace{0.9em} & (w_j ) \text{ ist Cauchy-Folge in } H^{\frac 12 }_A \\
	\stackrel[H_A^{\frac 12 }\scriptsize\text{vollst.}]{\scriptsize\text{Thm.}\ref{theorem:10.11}}\Ra & \, \exists \, z \in H^{\frac 12}_A : w_j \longrightarrow z \text{ in } H_A^{\frac 12} \hookrightarrow H=L_2 \\
	\Ra  \hspace{1em}& w = z \in H^{\frac 12}_A \, , 
\end{align*}
d.h. $\mathring W^1_2(\Omega) \subset H_A^{\frac 12}$, analog $H^{\frac 12}_A \subset \mathring W^1_2(\Omega)$ und mit \eqref{eq:10.11} folgt die Behauptung für "`$\dot =$"'.
\end{proof}




\backmatter

% Literatur
\addcontentsline{toc}{chapter}{Literaturverzeichnis}
\begin{thebibliography}{}
\bibitem{evans} L.C. Evans. \emph{Partial Differential Equations.} American Mathematical Society.
\bibitem{arendt} W. Arendt, K. Urban. \emph{Partielle Differentialgleichungen.} Spektrum Akademischer Verlag 2010.
\bibitem{jost} J. Jost. \emph{Partielle Differentialgleichungen.} Springer 1998.
\bibitem{gilbarg} D. Gilbarg, N. Trudinger. \emph{Elliptic Partial Differential Equations of Second Order.} Springer.
\end{thebibliography}
\newpage

% Index
\addcontentsline{toc}{chapter}{Index}
\printindex

\end{document}

%%% Local Variables: 
%%% mode: latex
%%% TeX-master: t
%%% End: 
