
\newchapter{Gleichungen 1. Ordnung: Methode der Charakteristiken}

\section{Motivation}

Für $(x,y)\in\Omega\subset\R^2$ offen betrachten wir
\[
  \left.
  \begin{aligned}
      a(x,y)\cdot u_x+b(x,y)\cdot u_y & = c(x,y,u), &(x,y)\in\Omega \\
    \text{mit}\qquad  u(x,y) &= \varphi(x,y), &(x,y)\in\Gamma
  \end{aligned}
  \quad
  \right\}
  \eq{eq:charakteristik}
\]
wobei $a,b:\Omega\ra\R$, $c:\Omega\times\R\ra\R$, $\Gamma$ eine 1-dimensionale Kurve in $\Omega$ und $\varphi:\Gamma\ra\R$ gegeben sind. 

Wir suchen nun ein $u=u(x,y)$, das \eqref{eq:charakteristik} erfüllt.
Die linke Seite von \eqref{eq:charakteristik} ist die Richtungsableitung von diesem $u(x,y)$ in Richtung $(a(x,y),b(x,y))$. Es ist nun naheliegend, dass wir diejenigen Kurven $(x,y)=(x(t),y(t))$ durch einen Punkt $(x_0,y_0)\in\Gamma$ betrachen, deren Tangentialvektor $(\dot x,\dot y)$ in Richtung von $\left(a(x(t),y(t)), b(x(t),y(t))\right)$ zeigt. Also erhalten wir
\begin{eqnarray*}
  \dot x = a(x,y), & x(0)=x_0 \\
  \dot y = b(x,y), & y(0)=y_0\;.
\end{eqnarray*}
Entlang einer charakteristischen Kurve genügt $u=u(x(t),y(t))$ einer gewöhn\-lichen Differentialgleichung, denn:
\[
\left.
  \begin{aligned}
    \diff t u(t) &= \dot x u_x+\dot y u_y=a(x,y)u_x+b(x,y)u_y \\
    &= c(x(t),y(t),u(t))
  \end{aligned}
  \quad
\right\vert
\eq{eq:gewDgl_charakteristik}
\]
mit $u(0)=u(x_0,y_0)=\varphi (x_0,y_0)$. Damit haben wir eine Reduktion des Problems auf gewöhnliche Differentialgleichungen erreicht, und $u$ ist durch $\varphi(x_0,y_0)$ entlang der gesamten charakteristischen Kurve durch $(x_0,y_0)\in\Gamma$ eindeutig bestimmt.
\begin{itemize}
\item[$\Ra$] $u(x,y)$: Projektionen der charakteristischen Kurven durch $(x_0,y_0)\in\Gamma$
\end{itemize}

Es liegt nun folgende Vermutung nahe: Mittels Gleichung \eqref{eq:gewDgl_charakteristik} lässt sich eine eindeutige Lösung von \eqref{eq:charakteristik} in dem Gebiet bestimmen, das durch die charakteristischen Kurven abgedeckt wird.

Allerdings darf der Vektor  $(a(x_0,y_0),b(x_0,y_0))$ in keinem Punkt $(x_0,y_0)\in\Gamma$ tangential an $\Gamma$ stehen! Genau dann nennen wir $\Gamma$ nicht-charakteristisch.

\begin{satz}
  \label{satz:2.1}
  Sei $f\in C^1(\R\times\R^n,\R^n)$, $x_0\in\R^n$. Dann existiert eine Umgebung $(\alpha,\beta)\times\Omega$ von $(0,x_0) \in \R\times\R^n$, sodass das Anfangswertproblem
  \[ \dot x=f(t,x),\quad x(0)=\xi \]
  für jedes $\xi\in\Omega$ eine eindeutige Lösung
  \[ w=w(\;\cdot\;;\xi)\in C^1((\alpha,\beta),\R^n) \]
  hat. 

  Ferner gilt: $w\in C^1((\alpha,\beta)\times\Omega,\R^n)$.
\end{satz}

\begin{proof}
  Siehe Analysis II (Picard-Lindelöf).
\end{proof}

\section{Quasilineare Gleichungen 1. Ordnung}

Wir betrachten
\begin{align*}
\left.
  \begin{aligned}
    a(x,y,u)u_x+b(x,y,u)u_y & =c(x,y,u), &(x,y)\in\Omega \\
    u(x,y) & =\varphi(x,y), &(x,y)\in\Gamma
  \end{aligned}
  \quad
\right\rvert
\eq{eq:3}
\end{align*}
wobei $\Gamma$ eine $C^1$-Kurve in $\Omega\subset\R^2$ mit Parameterdarstellung $\gamma:[a,b]\ra\R^2$ ist, und $\varphi\in C^1(\Gamma)$, $a,b,c\in C^1(\Omega\times\R)$. Ist $u\in C^1(\Omega)$ eine Lösung von \eqref{eq:3}, so gilt
\[
\vec A :=
\begin{pmatrix}
  a(x,y,u(x,y)) \\
  b(x,y,u(x,y)) \\
  c(x,y,u(x,y))
\end{pmatrix}
\perp
\begin{pmatrix}
  u_x(x,y) \\
  u_y(x,y) \\
  -1
\end{pmatrix}
=: \vec u.
\]
Das heißt, $\vec A=\vec A(x,y,u)$ liegt tangential auf dem Graphen $\{(x,y,z)\with z=u(x,y)\}$.

\begin{figure}[ht!]
  \centering
  \begin{pspicture}(-1,-2)(5,6)
    \psset{Alpha=155,Beta=12}
    \pstThreeDCoor[linewidth=.8pt,linecolor=black,xMax=6, yMax=8]
    
    % Weg am Boden
    \listplotThreeD[plotstyle=curve]{
      1.5  3  0
      2  4  0
      3  3  0
      4  4  0
    }

    % Weg auf Fläche
    \listplotThreeD[plotstyle=curve]{
      1.5 3 2
      2 4 2.1
      3 3 2.2
      4 4 1.5
    }
    \pstThreeDLine[linestyle=dotted](1.5, 3, 0)(1.5, 3, 2)
    \pstThreeDLine[linestyle=dotted](4, 4, 0)(4, 4, 1.5)

    % Weg Beschriftungen
    \pstThreeDPut[origin=lt](4.3, 4, 0){$\Gamma$}
    \pstThreeDPut[origin=lb](4.3, 4.2, 1.9){$\graph(\varphi)$}
   
    % A und u
    \pstThreeDLine[arrows=->](2, 4, 2.1)(3, 4, 2.1)
    \pstThreeDPut[origin=b](2, 4, 3.3){$\vec u$}
    \pstThreeDLine[arrows=->](2, 4, 2.1)(2, 4,3.1)
    \pstThreeDPut(3.1, 4, 2.3){$\vec A$}

    % Graph von u
    \listplotThreeD[plotstyle=curve]{
      1 1 2
      3 1 2.2
      5 1 1.5
    }
    \listplotThreeD[plotstyle=curve]{
      1 6 2
      3 6 2.2
      5 6 1.5
    }
    \pstThreeDLine(1, 1, 2)(1, 6, 2)
    \pstThreeDLine(5, 1, 1.5)(5, 6, 1.5)

    % Graph Beschriftung
    \pstThreeDPut[origin=lb](3, 6, 2.6){$\graph(u)$}
  \end{pspicture}
  \caption{Weg $\Gamma$}
\end{figure}

\begin{defi}
  Der Graph $\Sigma$ einer Funktion $u\in C^1(\Omega)$ heißt \idx{Lösungsfläche} für Gleichung \eqref{eq:3}, falls in jedem $P=(x,y,z)\in\Sigma$ der Vektor $$
\vec A =
\begin{pmatrix}
  a(P) \\ b(P) \\ c(P)
\end{pmatrix}
$$ tangential in $\Sigma$ liegt.
\end{defi}

\begin{defi}
  Lösungen $(t\mapsto (x(t),y(t),z(t))$ von
  \begin{eqnarray*}
    \dot x=a(x,y,z), & x(0)=x_0 \\
    \dot y=b(x,y,z), & y(0)=y_0 \\
    \dot z=c(x,y,z), & z(0)=z_0
  \end{eqnarray*}
  heißen Charakteristiken\index{Charakteristik} von \eqref{eq:3} durch $(x_0,y_0,z_0)$.
\end{defi}

\begin{lemma}
  Sei $u\in C^1(\Omega)$, $\Sigma=\graph(u)$. $\Sigma$ ist genau dann eine Lösungs-fläche wenn es die Vereinigung von Charakteristiken ist.
\end{lemma}

\begin{proof}
  Es sei $\Sigma:=\{(\bar x, \bar y, \bar z)\in\Omega\times\R\with \bar z=u(\bar x,\bar y)\}$, $t\mapsto(x(t),y(t),z(t))$ Charakteristik durch einen beliebigen Punkt $(x_0,y_0,z_0)\in\Sigma$ und $w(t):=z(t)-u(x(t),y(t))$. Damit ist
  \[
  \eq{eq:5}
  w(0)=z_0-u(x_0,y_0)=0,  
  \]
  da $(x_0,y_0,z_0)\in\Sigma$.

  Ferner ist 
  \[
  \left.
    \begin{aligned}
      \dot w(t)=\;&\dot z-\dot xu_x-\dot yu_y=c(x,y,z)-a(x,y,z)u_x-b(x,y,z)u_y \\
      =\;& c(x,y,w+u(x,y))-a(x,y,w+u(x,y))u_x \\
      &-b(x,y,w+u(x,y))u_y
    \end{aligned}
    \quad
  \right\vert
  \eq{eq:6}
  \]
  und $w$ ist durch \eqref{eq:5}, \eqref{eq:6} eindeutig bestimmt.
  
  \begin{itemize}
  \item["`$\Rightarrow$"']  Sei $\Sigma$ eine Lösungsfläche. Damit gilt $au_x+bu_y=c$. Aus \eqref{eq:5}, \eqref{eq:6} folgt $\dot w\equiv 0$, und da $w(0)=0$ ist, gilt $w\equiv 0$. Die Charakteristik $(t\mapsto (x(t),y(t),z(t)))$ liegt also ganz in $\Sigma$. Nun ist aber $(x_0,y_0,z_0)\in\Sigma$ beliebig gewählt. Somit ist $\Sigma$ eine Vereinigung von Charakteristiken.

  \item["`$\La$"'] Es sei $\Sigma$ Vereinigung von Charakteristiken. Es ist also $w\equiv 0$ und mit Gleichung \eqref{eq:6} folgt 
    \[
    c(x_0,y_0,\underbrace{u(x_0,y_0)}_{z_0})-a(x_0,y_0,z_0)u_x
    -b(x_0,y_0,z_0)u_y=0.
    \]
    Das heißt, $
      \begin{pmatrix}
        a(x_0,y_0,z_0) ,
        b(x_0,y_0,z_0) ,
        c(x_0,y_0,z_0)
      \end{pmatrix}^T
      $ steht tangential an $\Sigma$. Da $(x_0,y_0,z_0)$ beliebig ist, ist $\Sigma$ eine Lösungsfläche.
  \end{itemize}
  Damit folgt die Behauptung.
\end{proof}

\begin{folge}
  Lösungsflächen lassen sich durch Vereinigung von Charakteristiken durch jeden $(\gamma(s),\varphi(\gamma(s))), s\in [\alpha,\beta]$ (mit $\gamma : \R \ra \R^2$) konstruieren.
\end{folge}

\begin{einschr}
  Der Vektor $(a(\gamma(s),\varphi(\gamma(s))), b(\gamma(s),\varphi(\gamma(s))))$ darf in keinem Punkt $\gamma(s)=(\gamma_1(s),\gamma_2(s))\in\Gamma$ tangential sein an $\Gamma$. Der Weg $\Gamma$ ist also genau dann nicht-charakteristisch, wenn
  \begin{align*} \tag{$\ast$}
    a\left(\gamma(s),\varphi(\gamma(s))\right)\dot\gamma_2(s)
    -b\left(\gamma(s),\varphi(\gamma(s))\right)\dot\gamma_1(s)\neq 0\quad \fa s\in
    [\alpha,\beta]\;
  \end{align*}
  gilt.
  \index{Weg!nicht-charakteristisch}
\end{einschr}

\begin{theorem}
  Seien $a,b,c,\varphi$ reellwertige Charakteristiken und $\,\Gamma$ eine nicht-charakteristische $C^1$-Kurve. Dann besitzt das quasilineare Problem 1. Ordnung
  \[
  a(x,y,u)u_x+b(x,y,u)u_y=c(x,y,u),\quad u\vert_\Gamma=\varphi
  \]
  in einer Umgebung von $\Gamma$ eine eindeutige $C^1$-Lösung $u$.
\end{theorem}

\begin{proof}
  \begin{description}
  \item[Existenz:] Für $s\in [\alpha,\beta]$ berechnet $x(t,s),y(t,s),z(t,s)$ durch:
    \[
    \left.
     \begin{aligned}
       \pdiff t x(t,s)=a(x,y,z), &\quad x(0,s)=\gamma_1(s) \\
       \pdiff t y(t,s)=b(x,y,z), &\quad y(0,s)=\gamma_2(s) \\
       \pdiff t z(t,s)=c(x,y,z), &\quad z(0,s)=\varphi(\gamma(s))
    \end{aligned}
    \quad
    \right\rbrace
    \eq{eq:star1}
    \]
    (Charakteristik durch $(\gamma(s), \varphi(\gamma(s)))$)

    Nach Satz~\ref{satz:2.1} exitiert eine eindeutige Lösung $(x,y,z)$ und hängt $C^1$ von $s$ ab.

    Für $t=0$:
    \begin{align*}
      \det
      \begin{pmatrix}
        \partial_t x(0,s) & \partial_sx(0,s) \\
        \partial_t y(0,s) & \partial_sy(0,s)
      \end{pmatrix}
      =\det
      \begin{pmatrix}
        a(\gamma(s),\varphi(\gamma(s))) & \dot\gamma_1(s) \\
        b(\gamma(s),\varphi(\gamma(s))) & \dot\gamma_2(s)
      \end{pmatrix}
      \stackrel{(\ast)}{\neq} 0
     \end{align*}
     Nach dem Satz über Umkehrabbildungen ist die Abbildung $((t,s)\mapsto(x,y) = (x(s,t),y(s,t)))$ umkehrbar in einer Umgebung der Anfangskurve $t=0$. Also existieren  $t=t(x,y)$ und $s=s(x,y)$ in einer Umgebung $D\subset\R^2$ von $\Gamma$ und sind dort $C^1$. Setze $$u(x,y):=z(t(x,y),s(x,y)),  \quad(x,y)\in D.$$
Nun ist einerseits
\begin{align*}
  \pdiff t u(x,y)= \ &u_x(x,y)\cdot \pdiff t x+u_y(x,y)\pdiff t y \\
  \overset{\scriptsize\eqref{eq:star1}}=& a(x,y,u(x,y))u_x(x,y)+b(x,y,u(x,y))u_y(x,y) \eq{eq:7} \\
  \intertext{und andererseits}
  \pdiff t u(x,y)=&\pdiff t z(t,s)\overset{\scriptsize\eqref{eq:star1}}=
  c(x,y,z)=c(x,y,u(x,y)). \eq{eq:8}
\end{align*}
         
Aus den Gleichungen \eqref{eq:7} und \eqref{eq:8} folgt, dass $u$ eine Lösungsfläche über $D$ definiert.

       \item[Eindeutigkeit:] Es sei nun $\bar u$ eine weitere Lösung. Wir setzen $\bar z(t,s):=\bar u(\bar x(t,s),\bar y(t,s))$, wobei
    \begin{eqnarray*}
      \partial_t\bar x(t,s)=a(\bar x,\bar y,\bar u(\bar x,\bar y)), &
      \bar x(0,s)=\gamma_1(s), \\
      \partial_t\bar y(t,s)=b(\bar x,\bar y,\bar u(\bar x,\bar y)), &
      \bar y(0,s)=\gamma_2(s).
    \end{eqnarray*}
Dann ist
    \begin{align*}
      \pdiff t \bar z(t,s) = \ & a(\bar x,\bar y,\bar u)\bar u_x
      + b(\bar x,\bar y,\bar u)\bar u_y = c(\bar x,\bar y,\bar u) \\
      \underset{\scriptsize\text{Def.}}=&c(\bar x,\bar y,\bar z).
    \end{align*}
Nun ist
\[
\bar z(0,s)=\bar u(\gamma_1(s),\gamma_2(s))\overset{\bar u\; \scriptsize\text{Lsg.}}=\varphi(\gamma(s))
\]
und  $(\bar x,\bar y,\bar z)$ löst \eqref{eq:star1}. Der Satz von Picard-Lindelöf liefert die Eindeutigkeit der Lösung, d.h. $(x,y,z)=(\bar x,\bar y,\bar z)$. Also ist $u=\bar u$.
  \end{description}
  
  Damit folgt komplett die Behauptung.
\end{proof}

\begin{bem}
  \begin{enumerate}[(a)]
  \item Der Beweis ist konstruktiv!
  \item Der Beweis lässt sich auf $n>2$ verallgemeinern.
  \end{enumerate}
\end{bem}

\begin{bsp}
  $\Gamma:=\{(x,1)\with x\in\R\}$, $\alpha\in\R$, $\varphi\in C^1(\R)$. Wir betrachten die lineare Differentialgleichung erster Ordnung
  \begin{align*}
    \left.
      \begin{aligned}
        xu_x+yu_y=&\;\alpha u\\
        u\rvert_\Gamma=&\;\varphi
      \end{aligned}
      \quad\right\rvert
  \end{align*}
  Damit ist $
  \begin{pmatrix}
    a(x,y,u) \\
    b(x,y,u) \\
    c(x,y,u)
  \end{pmatrix}
  =
  \begin{pmatrix}
    x \\
    y \\
    \alpha u
  \end{pmatrix}$, und für $\Gamma$ gilt wegen $
  \begin{pmatrix}
    a(x,y,u) \\
    b(x,y,u)
  \end{pmatrix}
  =
  \begin{pmatrix}
    x \\
    1
  \end{pmatrix}$, dass es nicht-charakteristisch ist, da $(a,b)^T$ für $y = 0$ tangential wäre.

  \begin{figure}[ht!]
    \centering
    \begin{pspicture}(-4,-1)(4,3)
      \psaxes{->}(0,0)(-4,-1)(4,3)
      \psline[linewidth=1.6pt](-4,1)(4,1)
      \rput[tl](1.1,.9){$\Gamma$}
      \psline{->}(1,1)(2,2)
      \psline{->}(1.5,1)(3,2)
      \psline{->}(2,1)(4,2)
      \psline{->}(0.5,1)(1,2)
      \psline{->}(0,1)(0,2)
      \psline{->}(-0.5,1)(-1,2)
      \psline{->}(-1,1)(-2,2)
      \psline{->}(-1.5,1)(-3,2)
      \psline{->}(-2,1)(-4,2)
    \end{pspicture}
    \caption{Weg $\Gamma$ mit Vektorfeld $(x,1)$}
  \end{figure}
  Wir treffen die Vereinbarung, dass die zeitliche Ableitung durch einen Punkt dargestellt wird, also $\dot{\hspace{1em}} := \partial_t$. Mit dieser Notation ergibt sich aus den obigen Gleichungen
  \begin{eqnarray*}
    &\dot x = x, & x(0,s)=s \\
    &\dot y = y, & y(0,s)=1 \\
    &\dot z = \alpha \cdot z, & z(0,s)=\varphi(s).
  \end{eqnarray*}
  Wir erkennen sofort, dass $x=x(t,s)=s\cdot e^t$ und $y=y(t,s)=e^t$ sein muss. Zusätzlich ergibt sich $z=z(t,s)=\varphi(s)e^{\alpha t}$. Durch Auflösen der Gleichungen für $x$ und $y$ nach $t,s$, erhalten wir $t=\ln y$ und $s=\frac x y$. Schließlich erhalten wir      
  \[
  \underline{u(x,y)}=\varphi\left(\frac x y\right)e^{\alpha\ln
    y}=\underline{\varphi\left(\frac x y\right)y^\alpha}
  \]
  mit $x,y\in\R$ und $y>0$.
\end{bsp}

\begin{bsp}[nicht-viskose Burgersgleichung]
  \index{Burgersgleichung}
  \label{bsp:6}
  Wir betrachten die quasilineare Differentialgleichung
  \begin{align*}
    \left.
      \begin{aligned}
        u_x+uu_y&=0,  &&x>0,y\in\R \\
        u(0,y)&=h(y), && y\in\R
      \end{aligned}\quad
    \right\vert\tag{B}\label{eq:B}
  \end{align*}
  \[
  \begin{pmatrix}
    a(x,y,u) \\
    b(x,y,u)
  \end{pmatrix}
  =
  \begin{pmatrix}
    1 \\
    u
  \end{pmatrix}
  \]
  \begin{figure}[ht!]
    \centering
    \begin{pspicture}(-1,-2)(3,2)
      \psaxes{->}(0,0)(-1,-2)(3,2)
      \psline[linewidth=1.6pt](0,-2)(0,2)
      \rput[bl](0.1,1.1){$\Gamma$}
    \end{pspicture}
    \caption{Weg $\Gamma$ entlang der y-Achse}
  \end{figure}
  Dies führt auf das Gleichungssystem
  \begin{eqnarray*}
    \dot x = 1, &x(0,s)=0 \\
    \dot y=z, &y(0,s)=s \\
    \dot z=0, &z(0,s)=h(s),
  \end{eqnarray*}
  mit folgender Lösung:
  \begin{align*}
    \left.
      \begin{aligned}
        x=x(t,s)=t \\
        y=y(t,s)=s+h(s)t \\
        z=z(t,s)=h(s)
      \end{aligned}\quad
    \right\}\eq{eq:star2}
  \end{align*}
  Wir erkennen, dass $u$ entlang der Geraden durch $(0,s)$ mit der Steigung $h(s)$ den konstanten Wert $h(s)$ hat. Also haben wir
  \[ u(t,s+h(s)t)=h(s).\eq{eq:star3} \]
  Wir wissen $(t,s)\mapsto (x,y)$ ist \underline{lokal} ein Diffeomorphismus. Es ergibt sich die implizite Gleichung 
  \[
  \boxed{
    u(x,y) \stackrel{\scriptsize\eqref{eq:star3}}= h(s)
    \stackrel{\scriptsize\eqref{eq:star2}}=h(y-h(s)x)=h(y-u(x,y)x)
  }\,.
  \]
  Global kann aber ein Problem entstehen: Die Geraden
  \[
  \{(t,s+h(s)t\with t\geq0\} \text{ und }
  \{(t,\bar s+h(\bar s)t\with t\geq 0\}
  \]
  können sich schneiden. Im Schnittpunkt müsste $u$ beide Werte
  $h(s)$ bzw. $h(\bar s)$ annehmen. Dies ist ein Widerspruch.
  \begin{figure}[ht!]
    \centering
    \begin{pspicture}(-1,-1)(4,3)
      \psaxes[labels=none,ticks=none]{->}(0,0)(-1,-1)(3,3)
      \psline(-1,0)(3,2)
      \rput[l](3.1,2){$u=h(s)$}
      \rput[br](-.1,.5){$s$}
      \psline(-1,1.5)(3,1.5)
      \rput[l](3.1,1.5){$u=h(\bar s)$}
      \rput[br](-.1,1.6){$\bar s$}
      \pscircle(2,1.5){.3}
    \end{pspicture}
    \caption{Schnitt der Geraden}
  \end{figure}

  Wir betrachten nun $u(x,y)=h(y-u(x,y)x)$. Ableiten nach $y$ ergibt
  \begin{align*}
    u_y(x,y)=&\underbrace{h'(y-u(x,y)x)}_{=h'(s)}\cdot(1-u_y(x,y) x)
    \end{align*}
    und somit
    \begin{align*}
    u_y(x,y)=&\frac{h'(s)}{1+h'(s)x} \, .
  \end{align*}
  Ist nun $h'(s)<0$, so geht $u_y(x,y) \ra -\infty$ für $x\ra\frac {-1}
  {h'(s)}$. Dieses Verhalten nennt man "`Schock"' (der Gradient "`explodiert"').
  \begin{figure}[ht!]
    \centering
    \begin{pspicture}(-1,-1)(4,4)
      \psaxes[labels=none,ticks=none]{->}(0,0)(-1,-1)(4,4)
      \psline(0,0.5)(1.5,1.25)
      \psline(0,1.5)(3,3)
      \psline[linestyle=dotted](1.5,0)(1.5,3.5)
      \psline[linestyle=dotted](3,0)(3,3.5)
      \psdot(3,3)
      \psdot(1.5,1.25)
      \rput[r](-.1,.5){$s_0$}
      \rput[t](1.5,-.1){$-\frac 1{h'(s_0)}$}
    \end{pspicture}
    \caption{Test}
  \end{figure}

  Ist also allgemeiner $h'(s_0)=\min h'<0$, so gibt es keine $C^1$-Lösung in einer beidseitigen Umgebung der Geraden $x=\frac{-1}{h'(s_0)}$. Dieser "`\idx{blow-up}"' ist "`typisch"' für nicht-lineare Gleichungen.
  Als Ausweg verwenden wir einen "`schwächeren"' Lösungsbegriff (nicht $C^1$). Seien $R,S\in C^1(\R),S'(w)=wR'(w)$, $R'(w)\neq0\, \fa\,  w\in\R$. Sei $u$ eine $C^1$-Lösung von \eqref{eq:B}. Wir betrachten dafür die Divergenzform
  \begin{align*}
    \partial_xR(u)+\partial_yS(u)=&R'(u)\cdot u_x+S'(u)u_y \\
    =& R'(u)(u_x+uu_y)=0.
  \end{align*}
  Integrieren wir diese bezüglich $y$, erhalten wir
  \[ \diff x
  \int\limits_a^bR(u(x,y))dy+S(u(x,b))-S(u(x,a))=0\eq{eq:I}
  \]
  $\quad\fa\, x>0,\fa\, a<b$. Jede Funktion $u$, die \eqref{eq:I} löst, heißt \idx{Integrallösung} von \eqref{eq:B}. Somit gilt \eqref{eq:I} für eine größere Klasse von Funktionen. Wenn allerdings $u$ eine Integrallösung und $C^1$ ist, dann differenziere \eqref{eq:I} nach \eqref{eq:B}, und wir erhalten
  \begin{align*}
    \underbrace{R'(u(x,b))}_{\neq 0}(u_x(x,b)+u(x,b)\cdot u_y(x,b))=0
    \end{align*}
    {und schließlich ist}
    \begin{align*}
    u_x(x,b)+u(x,b)\cdot u_y(x,b)=0
  \end{align*}
  mit $x>0$ und $b\in\R$. Also ist $u$ Integrallösung und $C^1$ bzw. $u$ ist $C^1$-Lösung von \eqref{eq:B}.
\end{bsp}

Betrachten wir nun folgende Situation: Ein Gebiet $\Omega\subset\R^2$ wird durch eine glatte Kurve $y=\xi(x)$ in zwei Gebiete $\Omega_+$ und $\Omega_-$ zerlegt und $u\in C^1$ sei eine auf $\Omega_+$ bzw. $\Omega_-$ von beiden Seiten bis zum Rand $y=\xi(x)$ stetige Integrallösung von \eqref{eq:I}. Allerdings sei $u$ einem Sprung auf $y=\xi(x)$ versehen. Wir setzen
\[ u^\pm(x):=\lim_{y\ra\xi^\pm(x)}u(x,y). \]

\begin{figure}[ht!]
  \centering
  \begin{pspicture}(-1,-1)(6,5)
    \psaxes[labels=none,ticks=none]{->}(0,0)(-.5,-.5)(5.5,4.5)
    \rput[tl](5.5,-.1){$x$}
    \rput[br](-.1,4.5){$y$}

    %Omega
    \psccurve(1,1)(2.5,.7)(4,.7)(4,3)(1,3)
    \rput[tl](4,.7){$\Omega$}
    \rput(3.5,1.5){$\Omega_-$}
    \rput(1.8,2.5){$\Omega_+$}

    % Kurve
    \pscurve(.5,.5)(2,1.5)(3,2.5)(4.5,3.5)
    \rput[tl](4.5,3.5){$y=\xi(x)$}
  \end{pspicture}
  \caption{Zerlegung von $\Omega$ in zwei Gebiete}
\end{figure}
Unser Ziel ist eine Charakterisierung der \idx{Schockkurve} $y=\xi(x)$. Für $a<\xi(x)<b$:
\begin{align*}
  0=&S(a(x,b))-S(u(x,a))+\diff x\Big( 
    \int_a^{\xi(x)}R(u(x,y))\d y+\int\limits_{\xi(x)}^bR(u(x,y))\d y 
  \Big) \\
  =& S(u(x,b))-S(u(x,a))+\xi'(x)R(u^-(x))-\xi'(x)R(u^+(x)) \\
  &+\Big(
    \int_b^{\xi(x)}\underbrace{\partial_x R(u(x,y))}_{=-\partial_yS(u(x,y))}\d y
    +\int_{\xi(x)}^b\underbrace{\partial_x R(u(x,y))}_{=-\partial_yS(u(x,y))}\d y
  \Big) \\
  =& \xi'(x)\left(
    R(u^-(x))-R(u^+(x))
  \right) - S(u^-(x))+S(u^+(x)).
\end{align*}

Falls $R(u^-)\neq R(u^+)$, so gilt:
\[
\left. 
  \xi'(x)=\frac{S(u^-(x))-S(u^+(x))}{R(u^-(x))-R(u^+(x))}\quad
\right\vert\qquad\text{"`Schockrelation"'}
\]

\begin{bsp}[vergleiche Beispiel~\ref{bsp:6}]
  Wir betrachten wieder die Burgersgleichung mit einem konkreten $h$
  \begin{align*}
    u_x+uu_y&=0,\quad x>0,y\in\R \\
    u(0,y)=h(y)&:=
    \begin{cases}
      1 &,y<0 \\
      1-y &,0\leq y\leq 1 \\
      0 &, y>1
    \end{cases}
  \end{align*}

  Aus Beispiel~\ref{bsp:6} wissen wir, dass $u(x,y)$ für $s\in\R$ entlang der Geraden $(x,y)=(t,s+h(s)t)$ den konstanten Wert $h(s)$ hat. Ist beispielsweise $s<0$ (und damit $h(s)=1)$, so ist $u(x,y)=1$. Entlang $(x,y)=(t,s+t)$ ist dann $u(x,y)=1$ für alle $0< x, y< x$. Analog lässt sich folgern, dass
  \[
  u(x,y)=
  \begin{cases}
    1 &,0< x\leq 1,y< x \\
    \frac{1-y}{1-x} &, 0<x\leq y\leq 1 \\
    0 &, 0< x\leq1,y>1
  \end{cases}
  \]
  eine stetige und stückweise $C^1$-Lösung auf $x \in (0,1]$ ist. Was ist jedoch für $x>1$?

  Wir wählen $R(u):=u$ und $S(u):=\frac 12 u^2$. Dann ist die Schockrelation zwischen $\Omega_-=[u=1]$ und $\Omega_+=[u=0]$ 
  \[ \frac{S(1)-S(0)}{R(1)-R(0)}=\frac 12. \]
  Die Schockkurve $y=\xi(x)$ beginnt im Punkt $(1,1)$ und erfüllt $\xi'(x)=\frac 12$ und $\xi(1)=1$, also $\xi(x)=\frac{x+1}2$. Daher definieren wir  mit $x>1$
  \[
  u(x,y):=
  \begin{cases}
    1 &, y<\xi(x) \\
    0 &, y>\xi(x)
  \end{cases}.
  \]
  
  \begin{figure}[ht!]
    \centering
    \begin{pspicture}(-1,-1)(5,4)
      \psaxes[labels=none,ticks=none]{->}(0,0)(-.5,-.5)(4.5,3.5)
      \rput[tl](4.5,-.1){$x$}
      \rput[br](-.1,3.5){$y$}
      \psline(0,0)(3,3)
      \psline(0,0.4)(2,2)
      \psline(0,0.8)(2,2)
      \psline(0,1.2)(2,2)
      \psline(0,1.6)(2,2)
      \psline(0,2)(2,2)

      \rput(3,1){$u=1$}
      \rput(1.5,3){$u=0$}
      \rput[tl](3.1,3){$y=\xi(x)$}
    \end{pspicture}
    \caption{Schockkurve}
  \end{figure}

  Die so definierte Funktion ist eine Integrallösung, denn
  \[
  \diff x\int\limits_a^bu(x,y)\d y+\frac 12(u(x,b)^2-u(x,a)^2)=0.
  \]
\end{bsp}


%%% Local Variables: 
%%% mode: latex
%%% TeX-master: "Skript"
%%% End: 
