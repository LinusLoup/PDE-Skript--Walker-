\newchapter{Sobolevräume}

Auch in diesem Kapitel sei $\emptyset\neq\Omega\subset\R^n$ offen.

\begin{defi}
  Seien $1\leq p\leq\infty$ und $m\in\N$. Die Menge
  \[
  W_p^m(\Omega):=\left(
    \{u\in L_p(\Omega)\with\partial^\alpha u\in L_p(\Omega)\,\fa\,\abs a\leq m\}
    , \norm \cdot_{W_p^m}
  \right)
  \]
  heißt \idx{Sobolevraum} der Ordnung $m$. Dabei ist 
  \[
  \norm u_{W_p^m}:=\norm u_{W_p^m(\Omega)}:=
  \left(
    \sum_{\abs\alpha\leq m}\norm{\partial^\alpha u}_{L_p}^p
  \right)^{\frac 1p},
  \]
  wenn $1\leq p<\infty$. Im Fall $p=\infty$ ist $\norm u_{W_p^m}:=\max_{\abs\alpha\leq m}\norm{\partial^\alpha u}_\infty$.
\end{defi}

\begin{bem}
  \label{bem:6.1}
  \begin{enumerate}[\rm(a)]
  \item Es ist $W_p^0(\Omega)=L_p(\Omega)$.
  \item Seien $u\in W_p^m(\Omega)$, $\varphi\in\D(\Omega)$ und $\alpha\in\N^n$ mit $\abs\alpha\leq m$. Dann gilt:
    \[
    \begin{split}
      \int_\Omega\partial^\alpha u\cdot\varphi\d x
      &=\<\partial^\alpha u,\varphi\>_{\D(\Omega)}
      =(-1)^{\abs\alpha}\<u,\partial^\alpha\varphi\>_{\D(\Omega)} \\
      &=(-1)^{\abs\alpha}\int_\Omega u\cdot\partial^\alpha\varphi\d x\, .
    \end{split}
    \]
    $v = \partial^\alpha u$ bezeichnen wir dann als schwache Ableitung von $u$.
  \end{enumerate}
\end{bem}

\begin{defi}
  Seien $E,F$ normierte Vektorräume.
  \begin{enumerate}[\rm(a)]
  \item Dann ist $T\in\L(E,F)$ genau dann, wenn $T:E\ra F$
    linear und stetig ist und wir definieren
    \[
    \norm T_{\mathcal{L}(E,F)}:=\sup_{\norm e_E=1}\norm{T(e)}_F=\sup_{
      \begin{subarray}{c}
        e\in E \\
        e\neq 0
      \end{subarray}}\frac{\norm{T(e)}_F}{\norm e_E}\, .
    \]
    Damit ist $\norm{T(e)}_F\leq\norm T_{\mathcal{L}(E,F)}\cdot\norm
    e_E$, d.h. $T$ ist  ein beschränkter Operator.
  \item Wir schreiben $E\hookrightarrow F$, wenn $E$ \idx{Untervektorraum} von $F$ ist und $i:[e\mapsto e]\in\L(E,F)$ (d.h.\ es existiert ein $c>0$, so dass $\norm e_F\leq c\cdot\norm e_E$ für alle $e\in E$).
  \item Wir schreiben $E\xhookrightarrow{d}F$ für $E\hookrightarrow F$ und $E \overset d\subset F$.
  \item $E\hhookrightarrow F$ bedeutet $E\hookrightarrow F$ und beschränkte Mengen in $E$ sind relativ kompakt in $F$, d.h. wenn $\bar E$ in $F$ kompakt ist.
  \end{enumerate}
\end{defi}

\begin{theorem}
  \label{theorem:6.2} 
  Es seien $1\leq p\leq\infty$ und $m\in\N$. Dann gilt:
  \begin{enumerate}[\rm(a)]
  \item $W_p^m(\Omega)$ ist ein \idx{Banachraum}.
  \item $H^m:=W_2^m(\Omega)$ ist ein \idx{Hilbertraum} mit Skalarprodukt 
    \begin{align*}
      (u\vert v)_{H^m}&:=\sum_{\abs\alpha\leq m}(\partial^\alpha u\vert\partial^\alpha v)_{L_2},\quad u,v\in H^m,
      \intertext{wobei}
      (u\vert v)_{L_2}&:=\int_\Omega u\bar v\d x\, .
    \end{align*}
  \item $W_p^m(\Omega)\hookrightarrow W_p^k(\Omega)$ mit $k\in\N$ und $k\leq m$.
  \item Es ist $\partial^\alpha\in\L(W_p^m(\Omega),W_p^{m-\abs{\alpha}}(\Omega))$ für alle $\abs\alpha\leq m$.
  \end{enumerate}
\end{theorem}

\begin{proof}
  Übung.
\end{proof}

\begin{satz}
  \label{satz:6.3}
  Es seien $1\leq p\leq\infty$ und $m\in\N$. Dann gilt:
  \begin{enumerate}[\rm(a)]
  \item \label{satz:6.3-1} Ist $\{\varphi_\epsilon\with\epsilon>0\}$ ein Mollifier, so gilt
    \[ \varphi_\epsilon\ast u\xrightarrow[\epsilon\ra0]{}u\ \text{ in }\  W_p^m(\Omega)\, . \]

  \item Es ist $\D(\R^n)\overset d\subset W_p^m(\R^n)$.
  \end{enumerate}
\end{satz}

\begin{proof}
  \begin{enumerate}[\rm(a)]
  \item Satz~\ref{satz:3.4} liefert uns, dass $\varphi_\epsilon\ast u\ra u$ in $L_p(\Omega)$ gilt und
    \[ \partial^\alpha(\varphi_\epsilon\ast u)=\varphi_\epsilon\ast\partial^\alpha u \ra\partial^\alpha u\ \text{ in }\ L_p(\Omega) \]
    mit $\abs\alpha\leq m$. Damit folgt $\varphi_\epsilon\ast u\ra u$ in $W_p^m(\Omega)$.
  \item Sei $u\in W_p^m(\R^n)$ und $\chi\in\D(\R^n)$ mit $0\leq\chi\leq1$, $\chi\rvert_{\bar\B(0,1)}=1$. Wir definieren
    \[ u_\epsilon:=\chi(\epsilon\, \cdot)u\, , \]
    und aus den Sätzen von Lebesgue und Leibniz folgt $u_\epsilon\in W_p^m(\Omega)$ mit $u_\epsilon\xrightarrow[\epsilon\ra0^+]{}u$ in $W_p^m(\Omega)$ (vgl.\ Beweis zu Satz~\ref{satz:3.4}).

    Sei $\eta>0$ beliebig. Dann existiert ein $\epsilon>0$ mit $\norm{u_\epsilon-u}_{W_p^m}<\frac\eta2$. Aus Satz~\ref{satz:6.3} (a) folgt dann, dass ein $\delta>0$ mit 
    \[ \norm{\underbrace{\varphi_\epsilon}_{\in\D(\R^n)}\ast u_\epsilon-u}_{W_p^m}<\frac\eta2 \]
    existiert. \qedhere
  \end{enumerate}
\end{proof}

\begin{defi}
  Seien $m\in\N$, $u\in\D'(\R^n)$ und $\varphi\in\D(\Omega)$. Wir definieren
  \[ \<r_\Omega u,\varphi\>_{\D(\R^n)}:=\<u,\varphi\>_{\D(\Omega)} \]
  mit der \idx{Restriktion} $r_\Omega u:=u\rvert_\Omega$ für alle $u\in\Lloc(\Omega)$. Es ist $r_\Omega\in\L(W_p^m(\R^n),\allowbreak W_p^m(\Omega))$. Aus Theorem~\ref{theorem:3.6} folgt, dass $r_\Omega u\in\D'(\Omega)$.

  Wir schreiben $\Omega\in\Ext^m$, wenn $\Omega$ die $C^m$-Erweiterungseigenschaft hat. Dann existiert ein $e_\Omega\in\L(W_p^k(\Omega),W_p^k(\R^n))$ mit $r_\Omega\circ e_\Omega=\id_{W_p^m(\Omega)}$ für alle $k\in\{0,\ldots,m\}$ und $1\leq p<\infty$.
\end{defi}

\begin{bem}
  \label{bem:6.4} Es sei $\Omega\in\Ext^m$. Dann existiert ein $c=c(\Omega,m)>0$ mit
  \[ \norm{e_\Omega u}_{W_p^k(\R^n)}\leq c\norm{u}_{W_p^k(\Omega)} \]
  für alle $u\in W_p^k(\Omega)$ und
  \[ \norm{r_\Omega v}_{W_p^k(\Omega)}\leq\norm v_{W_p^k(\R^n)} \]
  mit $v\in W_p^k(\R^n)$.
\end{bem}

\begin{theorem}
  \label{theorem:6.5}
  Es sei $\R^n\in\Ext^m$ für alle $m\in\N$. Wir definieren
  \[ \H :=\R^{n-1}\times(0,\infty)\in\Ext^m \]
  für alle $m\in\N$. Sind nun $\Omega\in C^m$ und $\partial\Omega$ kompakt, dann ist $\Omega\in\Ext^m$.
\end{theorem}

\begin{proof}
  Literatur.
\end{proof}

\begin{kor}
  \label{kor:6.6} Sei $\Omega\in\Ext^m$. Dann ist $\D(\R^n)\overset{d}\subset W_p^k(\Omega)$ für $1\leq k\leq m$ und $1\leq p<\infty$.

  Speziell gilt: Ist $\Omega\in C^m$ beschränkt, so ist $C^\infty(\bar\Omega)\overset{d}\subset W_p^k(\Omega)$ für $0\leq k\leq m$ und $1\leq p<\infty$.
\end{kor}

\begin{proof}
  Seien $u\in W_p^k(\Omega)$ und $\epsilon>0$. Dann ist $e_\Omega u\in W_p^k(\R^n)$. Nach Satz~\ref{satz:6.3} existiert ein $v\in\D(\R^n)$ mit $\norm{v-e_\Omega u}_{W_p^k(\R^n)}<\epsilon$. Dann ist 
  \[
  \norm{r_\Omega v-\underbrace{r_\Omega e_\Omega u}_{=u}}_{W_p^k(\Omega)}\leq\norm{v-e_\Omega u}_{W_p^k(\R^n)}<\epsilon\, .
  \]
\end{proof}

% ----------------- 10.05.2011 ----------------------------

\begin{defi}
  Seien $m\in\N$ und $1\leq p\leq\infty$. Wir definieren
  \[ \mathring{W}_p^m(\Omega):=\operatorname{cl}_{W_p^m(\Omega)}\D(\Omega) \, , \]
  d.h. der Abschluss von $\D (\Omega)$ bzgl. $\norm{\cdot}_{W_p^m(\Omega)}$.
\end{defi}

\begin{bem}
  \begin{enumerate}[\rm(a)]
  \item $\mathring{W}_p^m(\Omega)$ ist ein abgeschlossener Untervektorraum von $W_p^m(\Omega)$, also ein Banachraum. 
  \item $\mathring{W}_p^m(\R^n)=W_p^m(\R^n)$ für $p<\infty$.
    \begin{proof}
      Satz~\ref{satz:6.3}.
    \end{proof}
  \end{enumerate}
\end{bem}

\begin{theorem}[\idx{Friedrichsche Ungleichung}]
  \label{theorem:6.8} Sei $\Omega$ beschränkt und $1<p<\infty$. Dann existiert ein $c:=c(\Omega,p)$ mit
  \[ \norm u_{L_p(\Omega)}\leq c\norm{\nabla u}_{L_p(\Omega)} \]
  für $u\in\mathring{W}_p^1(\Omega)$.
\end{theorem}

\begin{proof}
  Sei $R>0$ so, dass $\Omega\subset[-R,R]^n$ ist. Seien außerdem $u\in\D(\Omega)$ und $x=(y,t)\in\R^{n-1}\times\R$. Nach dem Mittelwertsatz ist
  \[ u(x)=\int_{-R}^t\partial_u u(y,\tau)\d\tau \]
  für $x\in\Omega$. Damit ist
  \[
  \abs{u(x)}^p\leq\left(
    \int_{-R}^t 1\cdot\abs{\partial_u u(y,\tau)}\d\tau
  \right)^p
  \stackrel[\scriptsize\text{Hölder}]{1=\frac 1p+\frac 1{p'}}\leq
  (2R)^{\frac p{p'}}\int_{-R}^t\abs{\partial_u u(y,\tau)}^p\d\tau\, .
  \]
  Schließlich erhalten wir
  \[
  \begin{split}
    \int_\Omega\abs{u(x)}^p\d x&\leq c\int_{[-R,R]^n}\int_{-R}^t\abs{\partial_u u(y,\tau)}^p\d\tau\d(y,t) \\
    &\stackrel[\d(y,t)=\d y\d\tau]{ \scriptsize\text{Fubini}}{\leq}
    c\cdot 2R\int_\Omega\underbrace{\abs{\partial_u u(x)}^p}_{\abs{\nabla u(x)}^p}\d x
    \leq c\int_\Omega\abs{\nabla u(x)}^p\d x\, .
  \end{split}
  \]
  Somit erhalten wir $\norm u_{L_p(\Omega)}\leq c\norm{\nabla u}_{L_p(\Omega)}$ für alle $u\in\D(\Omega)\overset d\subset\mathring{W}_p^1(\Omega)$.
\end{proof}

\begin{bem}
  \label{bem:6.9} Die Friedrichsche Ungleichung bleibt richtig, wenn $\Omega$ nur in einer Richtung (nicht notwendigerweise in Koordinatenrichtung) beschränkt ist.
\end{bem}

\begin{kor}
  \label{kor:6.10} Es sei $\Omega$ beschränkt $($zumindest in einer Richtung$)$ und $1<p<\infty$. Dann ist $[u\mapsto\norm{\nabla u}_{L_p(\Omega)}]$ eine äquivalente Norm auf $\mathring{W}_p^1(\Omega)$. Für $p=2$ definiert $(\nabla u\vert\nabla v)$ ein normerzeugendes Skalarprodukt auf $\mathring W_2^1(\Omega)=:\mathring H^1(\Omega)$.
\end{kor}

\begin{proof}
  Für alle $u\in\mathring W_p^1(\Omega)$ gilt
  \[
  \begin{split}
    \norm{\nabla u}_{L_p(\Omega)}^p&\ \ \quad\leq \ \ \quad\norm u^p_{L_p(\Omega)}+\norm{\nabla u}^p_{L_p(\Omega)}=\norm u^p_{\mathring W_p^1(\Omega)} \\
    &\underset{\text{\scriptsize Theorem~\ref{theorem:6.8}}}\leq(c(\Omega,p)+1)\norm{\nabla u}^p_{L_p(\Omega)}. \qedhere
  \end{split}
  \]
\end{proof}

\begin{lemma}[Sobolevsche Ungleichung]
  \label{lemma:6.11} Für $1\leq p<n$ sei
  \[ \frac 1{p\ast}:=\frac 1p-\frac 1n \]
  der \idx{Sobolev-Exponent} mit $p\ast>p$. Dann existiert ein $c:=c(n,p)>0$ mit
  \[ \norm u_{L_{p\ast}(\R^n)}\leq c\norm{\nabla u}_{L_p(\R^n)} \]
  für alle $u\in W_p^n(\R^n)$.
\end{lemma}

\begin{proof}
  Übung.
\end{proof}

\begin{theorem}[Sobolevscher Einbettungssatz]
  \label{theorem:6.12} Seien $k,m\in\N^\ast$ mit $k \leq m$ und $  1\leq p,q<\infty$ mit
  \[
  \eq{eq:6.star}\frac 1p\geq\frac 1q\geq\frac 1p-\frac{m-k}n>0.
  \]
  Dann gilt:
  \begin{enumerate}[\rm(a)]
  \item \label{theorem:6.12-1} $\mathring W_p^m(\Omega)\overset d\hookrightarrow\mathring W_q^k(\Omega)$
  \item \label{theorem:6.12-2} Ist $\Omega\in\Ext^m$, so ist $W_p^m(\Omega)\overset d\hookrightarrow W_q^k(\Omega)$.
  \end{enumerate}
\end{theorem}

\begin{proof} Wir beweisen die Aussagen in umgekehrter Reihenfolge.
  \mbox{}
  \begin{enumerate}[(b)]
  \item Sei $\Omega\in\Ext^m$.
    \begin{enumerate}[\rm(i)]
    \item \label{proof:6.12-1} Sei $n>p$ und $\frac 1p\geq\frac 1q\geq\frac 1{p\ast}$. Wir erhalten das Diagramm
      \[
      \begin{diagram}
        \node{W_p^1(\Omega)} \arrow{s,l}{e_\Omega} 
        \arrow[2]{e,J} \node{} \node{L_q(\Omega)} \\
        \node{W_p^1(\R^n)} 
        \arrow{e,b,J}{\text{Lemma~\ref{lemma:6.11}}} 
        \node{L_p(\R^n)\cap L_{p\ast}(\R^n)}
        \arrow{e,b,J}{p\leq q\leq p\ast}
        \node{L_q(\R^n)} \arrow{n,r}{r_\Omega}
      \end{diagram}
      \]
    \item \label{proof:6.12-2} Sei $k=m-1$ und $u\in W_p^m(\Omega)$. Für alle $\alpha\in\N^n$ mit $\abs\alpha\leq m-1=k$ ist $\partial^\alpha u\in W_p^1(\Omega)$. Nach (i) ist $\partial^\alpha u\in L_q(\Omega)$ und es existiert ein $c>0$ mit
      \[
      \norm{\partial^\alpha u}_{L_q(\Omega)}\leq c\norm{\partial^\alpha u}_{W_p^1(\Omega)}\leq c\norm u_{W_p^m(\Omega)}
      \]
      für alle $\abs\alpha\leq m-1$. Also ist $W_p^m\hookrightarrow W_p^{m-1}$ für die Bedingungen aus \eqref{eq:6.star}.
    \item \label{proof:6.12-3} Sei $k=m-2$ und damit wegen \eqref{eq:6.star} $p<\frac n2$. Wegen (ii) ist dann
      \[
      \eq{eq:6.12-1}
      \left.
      \begin{split}
        W_p^m&\hookrightarrow W_r^{m-1} &&, \frac 1p\geq\frac 1r\geq\frac 1p-\frac 1n \\
        W_r^{m-1}&\hookrightarrow W_p^{m-2} &&, \frac 1r\geq\frac 1q\geq\frac 1r-\frac 1n
      \end{split}\qquad
      \right\}
      \]
      Nach \eqref{eq:6.star} existiert ein $r$ mit \eqref{eq:6.12-1}. Es gilt also $W_p^m\hookrightarrow W_q^{m-2}$ für $\frac 1p\geq\frac 1q\geq\frac 1p-\frac 2n=\frac 1p-\frac{m-k}n$.
    \end{enumerate}
    Der Rest ergibt sich durch Induktion. Die Dichtheit folgt aus Korollar~\ref{kor:6.6}.

  \item[(a)] Sei $\Omega$ beliebig und offen in $\R^n$. Es sei
    \[ 
    \tilde e_\Omega:=
    \begin{cases}
      u(x) &, x\in\Omega \\
      0 &, x\in\R^n\setminus\Omega
    \end{cases}
    \]
    die \idx{triviale Fortsetzung} von $u$ in $\R^n$. Es ist $\tilde e_\Omega\in\L(W_p^m(\Omega), W_p^m(\R^n))$ und $r_\Omega\circ\tilde e_\Omega=\id_{W_p^m(\Omega)}$. Wir nun haben die folgenden Beziehungen:
    \[
    \begin{diagram}
      \node{\mathring W_p^m(\Omega)}
      \arrow{s,l}{\tilde e_\Omega} \arrow{e,J} 
      \node{W_q^n(\Omega)} \\
      \node{W_p^m(\R^n)} \arrow{e,b,J}{\text{(\ref{theorem:6.12-2})}}
      \node{W_q^n(\R^n)} \arrow{n,r}{r_\Omega}
    \end{diagram}
    \]
    Damit erhalten wir $\mathring W_p^m(\Omega)\overset d\hookrightarrow\mathring W_q^k(\Omega)$.\qedhere
  \end{enumerate}
\end{proof}

\begin{theorem}[Rellich-Kondrachov]
  \index{Theorem~von!Rellich-Kondrachov}
  \label{theorem:6.13}
  Sei $\Omega$ beschränkt und $C^1$. Dann gilt
  \[ W_p^1(\Omega)\hhookrightarrow L_q(\Omega) \]
  für $1\leq p<n$ und $1\leq q<p\ast$.
\end{theorem}

\begin{proof}
  Theorem~\ref{theorem:6.12} und Theorem~\ref{theorem:6.5} liefern uns $W_p^1(\Omega)\hookrightarrow L_q(\Omega)$.

  Sei $B\subset W_p^1(\Omega)$ beschränkt. Dann genügt zu zeigen, dass $B$ relativ kompakt in $L_q(\Omega)$ ist. Mit Theorem~\ref{theorem:6.5} können wir o.B.d.A. annehmen, dass jedes $u\in B$ auf ganz $\R^n$ definiert ist und einen Träger in $V$ mit $U\Subset V=\mathring V\Subset\R^n$ hat. Außerdem ist $B$ beschränkt in $W_p^1(\Omega)$.

  Sei $\{\varphi_\epsilon\with\epsilon>0\}$ ein Mollifier und $u\in B$, $x\in\Omega$. Dann ist
  \begin{dmath}
    %\eq{eq:6.13-1}
    \label{eq:6.13-1}
    \abs{\varphi_\epsilon\ast u(x)}\leq \int_{\epsilon\B^n}\abs{u(x-y)}\varphi_\epsilon(y)\d y 
    \underset{z=\frac y\epsilon}{=}\int_{\B^n}\abs{u(x-\epsilon z)}\varphi(z)\d z 
    \leq\norm\varphi_\infty\norm u_{L_1(\Omega)}\leq c(B)\, ,
  \end{dmath}
  da $W_p^1(V)\hookrightarrow L_p(V)\hookrightarrow L_1(V)$. Weiter ist
  \begin{dmath}
    % \eq{eq:6.13-2}
    \label{eq:6.13-2}
    \abs{\partial_j(\varphi_\epsilon\ast u)(x)}
    \leq\int_{\epsilon\B^n}\abs{u(x-y)}\abs{\underbrace{\partial_j\varphi_\epsilon(y)}_{\frac 1\epsilon(\partial_j\varphi)(\frac y\epsilon)}}\d y
    \leq\frac 1\epsilon\norm{\nabla\varphi}_\infty\norm u_{L_1(V)}\leq c(B)\epsilon^{-1}.
  \end{dmath}
  Der Mittelwertsatz liefert uns
  \[
  \eq{eq:6.13-3}
  \abs{\varphi_\epsilon\ast u(x)-\varphi_\epsilon\ast u(y)}
  \underset{\scriptsize ~\eqref{eq:6.13-2}}\leq\epsilon^{-1}c(B)\abs{x-y}
  \]
  mit $x,y\in\Omega$, $\epsilon>0$ und $u\in B$. 

  Wir definieren $B_\epsilon:=\{\varphi_\epsilon\ast u\with u\in B\}$. Nach \eqref{eq:6.13-1}, \eqref{eq:6.13-3} und dem Satz von Arzela-Ascoli ist $B_\epsilon$ kompakt in $C(\bar\Omega)\hookrightarrow L_1(\Omega)$, d.h.
  \[
  \eq{eq:6.13-4}
 B_\epsilon\;\text{ist relativ kompakt in}\;L_1(\Omega).
 \]

  Sei $u\in\D(\R^n)$. Dann ist
  \begin{equation}
  \begin{split}
    \abs{u(x)-\varphi_\epsilon\ast u(x)}&\underset{z=\frac y\epsilon}\leq
    \int_{\B^n}\abs{u(x)-u(x-\epsilon z)}\varphi(z)\d z \\
    &\underset{\scriptsize\text{MWS}}=\int_{\B^n}\Abs{\int_0^1\nabla u(x-t\epsilon z)\cdot z\d t}\varphi(z)\d z \\
    &\stackrel[\scriptsize\text{Schwartz}]{\scriptsize\text{Cauchy-}}\leq
    \epsilon\norm\varphi_\infty\int_{\B^n}\int_0^1
    \abs{\
      \nabla u(\underbrace{x\;-\;t\;\epsilon\;z}_{\parbox{1.89cm}{\centering\scriptsize$\in V\;\text{für}\;\epsilon\;\text{klein},$ \\ $\; x\in\Omega$}})
    }\d t\d z\, .
  \end{split}
  \end{equation}
  Also ist
  \[
  \norm{u-\varphi_\epsilon\ast u}_{L_1(\Omega)}\leq c(\varphi)\epsilon\norm{\nabla u}_{L_1(V)}
  \]
  mit $u\in\D(\R^n)$. Aus Korollar~\ref{kor:6.6} und der Tatsache, dass $r_\Omega\D(\R^n)\overset d\subset W_p^1(\Omega)$ ist, folgt
  \[
  \eq{eq:6.13-5}
  \norm{u-\varphi_\epsilon\ast u}_{L_1(\Omega)}\leq c(\varphi, B)\epsilon
  \]
  für alle $u\in B$.

  Sei nun $r>0$ beliebig. Dann existiert ein $\epsilon_0>0$ mit $c_0(\varphi, B)\epsilon_0<\frac r2$. Aus \eqref{eq:6.13-4} folgt dann, dass $w_1,\ldots,w_l\in B_{\epsilon_0}$ existieren mit
  \[
  \eq{eq:6.13-6}
  B_{\epsilon_0}\subset\bigcup_{j=1}^l\B_{L_1(\Omega)}\left(w_j,\frac r2\right).
  \]
  Für ein beliebiges $u\in B$ ist wegen \eqref{eq:6.13-5} $\norm{u-\varphi_{\epsilon_0}\ast u}_{L_1(\Omega)}<\frac r2$. Mit \eqref{eq:6.13-6} folgt die Existenz eines $j\in\{1,\ldots,l\}$ mit $\varphi_{\epsilon_0}\ast u\in\B_{L_1(\Omega)}(w_j,\frac r2)$. Also ist auch $u\in\B_{L_1(\Omega)}(w_j,\frac r2)$. Da $u$ beliebig ist, ist 
  \[ B\subset\bigcup_{j=1}^l\B_{L_1(\Omega)}(w_j,r)\, . \]
  Da $r>0$ beliebig war, ist $B$ total beschränkt, also relativ kompakt. Somit ist 
  \[ \eq{eq:6.13-7} W_p^1(\Omega)\hhookrightarrow L_1(\Omega)\, . \]
  
  Seien nun $1<q<p\ast$ und $p<n$. Dann existiert ein $\Theta\in(0,1)$ mit
  \[ \frac 1q=\frac \Theta1+\frac{1-\Theta}{p\ast}. \]
  Die Hölder-Ungleichung liefert damit
  \begin{dmath}
    %\eq{eq:6.13-8}
    \label{eq:6.13-8}
    \norm u_{L_q(\Omega)}\leq\norm u_{L_1(\Omega)}^\Theta\norm u_{L_{p\ast}(\Omega)}^{1-\Theta}
    \overset{\scriptsize\text{Lem.~\ref{lemma:6.11}}}\leq c\norm u_{L_1(\Omega)}^\Theta
    \underbrace{\norm u_{L_q(V)}^{1-\Theta}}_{\leq c(B)}
    \leq c(B)\norm u_{L_1(\Omega)}^\Theta\, .
  \end{dmath}
  Aus \eqref{eq:6.13-7}, \eqref{eq:6.13-8} und der Cauchy-Schwartzschen Ungleichung folgt dann $W_p^1(\Omega)\hhookrightarrow L_q(\Omega)$.
\end{proof}

\begin{bem}
  \label{bem:6.14}
  \begin{enumerate}[\rm(a)]
  \item \label{bem:6.14-1} Es sei $\Omega$ beschränkt und $C^1$. Dann ist $W_p^1(\Omega)\hhookrightarrow L_q(\Omega)$ für $1\leq p<\infty$.
    \begin{proof}
      Aus $p<n$ folgt $p<p\ast$. $p\geq n(n-\epsilon)\ast\nearrow_{\epsilon\ra0}\infty$
    \end{proof}
  \item \label{bem:6.14-2} Sei $\Omega\subset\R^n$ offen und beschränkt. Dann ist $\mathring W_p^1(\Omega)\hhookrightarrow L_p(\Omega)$.
    \begin{proof}
      (\ref{bem:6.14-1}) $+$ triviale Fortsetzung.
    \end{proof}
  \end{enumerate}
\end{bem}

% -------------- 12.05.2011 --------------------------

\begin{defi}
  Es sind für $0<\nu<1$
  \begin{align*} 
      BUC^\nu(\Omega):=\left\{ u:\Omega\ra\K \with \norm
        u_{BUC^\nu(\Omega)}< \infty \right\}
        \intertext{mit}
        \norm
        u_{BUC^\nu(\Omega)}:=\sup_{x\in\Omega}\abs{u(x)} +\sup_{\
          \begin{subarray}{c}
            x,y\in\Omega \\
            x\neq y
          \end{subarray}
        } \frac{\abs{u(x)-u(y)}}{\abs{x-y}^\nu} 
\end{align*}
und
\begin{align*}
      BUC^{k+\nu}(\Omega):=\left\{ 
        u\in BUC^k(\Omega)\with \partial^\alpha u\in BUC^\nu(\Omega)\, \fa\, \abs\alpha \leq k \right 
      \} .
  \end{align*}
\end{defi}

\begin{theorem}[Sobolev-Morrey]
  \label{theorem:6.15}
  \index{Theorem~von!Sobolev-Morrey}
  Es seien $m\in\N$, $1\leq p<\infty$ und $0\leq\nu\leq m-\frac np$ mit $\nu<m-\frac np$, falls $m-\frac np\in\N$. Dann gilt:
  \begin{enumerate}[\rm(i)]
  \item \label{theorem:6.15-1} $\mathring W_p^m(\Omega)\hookrightarrow BUC^\nu(\Omega)$
  \item Ist $\Omega\in\Ext^m$, dann ist $W_p^m(\Omega)\hookrightarrow BUC^\nu(\Omega)=BUC^\nu(\bar\Omega)$.
  \end{enumerate}
\end{theorem}

\begin{proof}
  Adams ("`Sobolev Spaces"'), Triebel.
\end{proof}

\begin{bem*}
  $-\Delta u=f$ in $\Omega$, $u\rvert_{\partial\Omega}=0$ und für $f\in C(\bar\Omega)$ gibt es im Allgemeinen keine Lösung $u\in C^2(\Omega)\cap C(\bar\Omega)$. Deshalb betrachten wir $u\in W_p^2(\Omega)$. Es ergibt sich dabei allerdings die Frage, wie wir $u\rvert_{\partial\Omega}=0$ interpretieren.
\end{bem*}

\begin{theorem}[\idx{Spursatz}]
  \label{theorem:6.16}
  Es sei $\Omega\subset\R^n$ und $C^2$. Dann gilt für $1<p<\infty$, dass ein $c>0$ existiert mit
  \[ \norm{u\rvert_{\partial\Omega}}_{L_p(\partial\Omega)}\leq c\norm u_{W_p^1(\Omega)} \]
  für alle $u\in C^\infty(\bar\Omega)$.

  Die Restriktion $[u\mapsto u\rvert_{\partial\Omega}]$ lässt sich also eindeutig stetig zum Spuroperator $\gamma_0\in\L(W_p^1(\Omega),L_p(\partial\Omega))$ fortsetzen.
\end{theorem}

\begin{proof}
  Sei $\nu$ die äußere Einheitsnormale an $\partial\Omega$ mit $\norm{\nu(x)}=1$ mit $x\in\partial\Omega$. $\nu$ kann zu einem $C^2$-Vektorfeld auf $\bar\Omega$ erweitert werden. Für $u\in C^\infty(\bar\Omega)$ ist
  \begin{dmath*}
    \int_{\partial\Omega}\abs u^p\d\sigma
    =\int_{\partial\Omega}(\abs u^p\nu)\cdot\nu\d\sigma
    \hiderel{\underset{\scriptsize\text{Gauß}}=}\int_\Omega\div(\abs u^p\nu)\d x
    \leq\int_\Omega\sum_{j=1}^n\left(
      p\abs u^{p-1}\cdot\abs{\partial_j u}\abs{\nu^j}
      +\abs u^p\cdot\abs{\partial_j\nu^j}
    \right)\d x 
    \leq c\cdot\left(\norm\nu_{C^1(\bar\Omega)}\int_\Omega\sum_{j=1}^n\left(
      p\abs u^{p-1}\cdot\abs{\partial_j u}+\abs u^p
    \right) \d x\right)
    \mkern-14mu\underset{\scriptsize\text{Young}}\leq c\cdot\int_\Omega\sum_{j=1}^n\left(
      \abs u^p+\abs{\partial_j u}+\abs u^p
    \right) \d x\, .
  \end{dmath*}
  Damit ist 
  \[
  \norm {u\rvert_{\partial\Omega}}_{L_p(\Omega)}\leq c\cdot\norm u_{W_p^1(\Omega)}
  \]
  für alle $u\in C^\infty(\bar\Omega)$. Mit Korollar~\ref{kor:6.6} folgt damit $C^\infty(\bar\Omega)\overset d\subset W_p^1(\Omega)$.

  In den Übungen werden wir zeigen, dass eine Erweiterung $\gamma_0\in \L(W_p^1(\Omega),\allowbreak L_p(\partial\Omega))$ existiert mit $\gamma_0(u)=u\rvert_{\partial\Omega}$ für $u\in C\infty(\bar\Omega)$. Damit folgt die Behauptung.
\end{proof}

\begin{bem}
  \label{bem:6.17}
  Es seien $1<p<\infty$, $m\in\N^*$ und $\Omega\in C^2$ beschränkt. Dann gelten:
  \begin{enumerate}[\rm(i)]
  \item \label{bem:6.17-1} Sei
    \begin{dseries*}
      \begin{math}
        \mathring W_p^m(\Omega)=\operatorname{cl}_{W_p^m(\Omega)}\D(\Omega)
      \end{math},
      \begin{math}
        \partial^\alpha u\vert_{\partial\Omega}=0
      \end{math}
    \end{dseries*}
    für alle $u\in\D(\Omega)$ und $\alpha\in\N^m$. Dann ist
    \begin{dmath*}
      \gamma_0(\partial^\alpha u)=0
      \condition{für alle $u\in\mathring W_p^m(\Omega)$, $\abs\alpha\leq m-1$}.
    \end{dmath*}
  \item \label{bem:6.17-2} Gauß:
    \begin{dmath*}
      \int_\Omega\div(u)\d x = \int_{\partial\Omega}\gamma_0(u)\cdot \nu \d\sigma
      \condition{für alle $u\in W_p^1(\Omega,\K^n)=(W_p^1(\Omega))^n$}.
    \end{dmath*}
  \end{enumerate}
\end{bem}

\begin{erinnerung}
\[ -\Delta u = f \quad \text{in } \Omega \, , \quad u\vert_{\partial \Omega} = 0\]
hat im Allgemeinen keine Lösung $u \in C^2(\Omega)$, falls $f \in C(\Omega)$. Nun muss also $u \in W_p^2(\Omega)$ sein, jedoch stellt sich dann die Frage, wie $u|_{\partial\Omega}$ zu interpretieren ist.

Die Idee ist nun
\[
	u \stackrel{!}\in \mathring W^2_p = \{u \in W^2_p \with \gamma_0 u = 0\}
\]
und wenn $f\in L_p(\Omega)$, dann existiert genau ein $u \in \mathring W^2_p(\Omega)$ für $-\Delta u = f(u)$.
\end{erinnerung}

\begin{defi}
Es sei $m\in \N, 1\leq p < \infty, 1 = \frac 1 p + \frac 1 {p'}$.
\begin{enumerate}[(a)]
\item Eine Folge $(u_j)$ in $L_p$ konvergiert schwach gegen $u \in L_p(\Omega)$
\begin{align*}
	&: \Longleftrightarrow u_j \rightharpoonup u \text{ in } L_p(\Omega) \\
	& : \Longleftrightarrow \, \fa \, v \in L_{p'}(\Omega) : \int_\Omega u_j v \d x \longrightarrow \int_\Omega uv \d x \text{ in } \K \, .
\end{align*}
\item Eine Folge $(u_j) \in W^m_p(\Omega)$ konvergiert schwach gegen $u \in W_p^m (\Omega)$
\begin{align*}
	& : \Longleftrightarrow u_j \rightharpoonup u \text{ in } W_p^m(\Omega) \\
	& : \Longleftrightarrow \partial^\alpha u_j \rightharpoonup \partial^\alpha u \text{ in } L_p(\Omega) \, \forall \, \abs\alpha \leq m \, .
\end{align*}
\end{enumerate}
\end{defi}

\begin{bem}
Man kann "`schwache Konvergenz"' (bzw. "`schwache Topologie"') für allgemeine (z.B. Banach-/lokal konvexe Räume) definieren (s. z.B. Rudin). Wir haben in der Definition benutzt, dass gilt
\[
	(L_p(\Omega))' := \mathcal L(L_p(\Omega),\K) \, , \quad (L_p(\Omega))' \cong L_{p'} (\Omega) \, ,
\]
denn ist $v \in L_{p'} (\Omega)$, so ist
\[
	\langle v ,u\rangle := \int_\Omega vu \d x \, , \quad u \in L_p(\Omega) 
\]
und damit folgt mit Hölder, dass $\langle v, \, \cdot \, \rangle \in \mathcal L(L_p(\Omega), \K)$ gilt.

Speziell gilt also $(L_2(\Omega))' \cong L_2(\Omega)$.
\end{bem}

\begin{bem}
\label{bem:6.19}
Sei $1\leq p < \infty, m \in \N$, dann ist:
\begin{enumerate}[(a)]
\item Ist $u_j \ra u$ in $W^m_p(\Omega)$, dann folgt $u_j \rightharpoonup u$ in $W^m_p(\Omega)$, d.h. "`starke Konvergenz ist stärker als schwache Konvergenz"'.
\begin{proof}
$\fa \, v \in L_{p'}(\Omega), \abs \alpha \leq m$ gilt
\[
	\Abs{\int_\Omega (\partial^\alpha u_j - \partial^\alpha u) v \d x} \stackrel[\scriptsize\text{Hölder}]{}\leq \norm v_{L_{p'}(\Omega)} \norm{\partial^\alpha u_j -\partial^\alpha u }_{L_p(\Omega)} \longrightarrow 0 \, .
\] 
\end{proof}
\item Sei $1 < p < \infty, (u_j)\subset W^m_p(\Omega)$ beschränkt (bzgl. $\norm{\, \cdot\, }_{W^m_p}$), dann folgt, dass eine Teilfolge $(u_{j'})$ und ein $u \in W^m_p(\Omega)$ existiert, so dass $u_j \rightharpoonup u$ in $W^m_p(\Omega)$, d.h. beschränkte Folgen sind relativ schwach kompakt.
\begin{proof}
Vgl. Rudin.
\end{proof}
\item Sei $M\subset W^m_p(\Omega)$ konvex und abgeschlossen (bzgl. $\norm{\, \cdot\, }_{W^m_p}$), sowie $(u_j) \in M$ mit $u_j \rightharpoonup u $ in $W^m_p(\Omega)$, dann ist $u \in M$, d.h. "`abgeschlossene konvexe Mengen sind schwach abgeschlossen"' (Theorem von Mazun; ohne Beweis, vgl. Rudin).
\item Es sei $u_j \rightharpoonup u $ in $W^p_m (\Omega)$, dann folgt $(u_j)$ ist beschränkt in $W^m_p(\Omega)$ (bzgl. $\norm{\, \cdot\, }_{W^m_p}$), d.h. "`schwach konvergente Folgen sind beschränkt"'.
\begin{proof}
Theorem von Mackey, vgl. Rudin.
\end{proof}
\item $u_ j \rightharpoonup u $ in $W^m_p(\Omega), u_j \rightharpoonup v$ in $W^m_p(\Omega)$, dann gilt $u=v$, d.h. "`Grenzwerte von schwach konvergenten Folgen sind eindeutig"'.
\begin{proof}
Aus Theorem~\ref{theorem:3.6} folgt die Behauptung.
\end{proof}
\item Sei $u_j \rightharpoonup u $ in $W^m_p(\Omega)$, dann folgt $\norm u_{W^m_p(\Omega)} \leq \lim \inf \norm{u_j}_{W^m_p(\Omega)}$.
\end{enumerate}
\end{bem}

%%% Local Variables: 
%%% mode: latex
%%% TeX-master: "Skript"
%%% End: 
